% abcat0a
% !TEX encoding = UTF-8 Unicode
\documentclass[12pt]{article} 
\pagestyle{empty} 
\addtolength{\parskip}{.3\baselineskip} 
\usepackage[a4paper]{geometry} 
%\usepackage[a4paper,hmargin=3cm,vmargin=3.5cm]{geometry}%\usepackage{comment}%\usepackage{bm}
\usepackage{amssymb,amsmath} 
\usepackage[T1]{fontenc} 
\usepackage[utf8]{inputenc} 
\usepackage{tikz-cd}%\usepackage{tikz} 
\usepackage{hyperref} 
\usepackage{datetime} 
%\usepackage{comment}
\usepackage{amsthm} 
\newtheorem{thm}{Theorem} 
\newtheorem{lem}[thm]{Lemma} 
\newtheorem{prop}[thm]{Proposition} 
\newtheorem{cor}[thm]{Corollary} 
\newtheorem{df}[thm]{Definition}%\newtheorem{defn}[thm]{Definition} 
\theoremstyle{remark}%\newtheorem{cm}[thm]{Comment} 
\newtheorem{rk}[thm]{Remark}%\theoremstyle{definition}\newtheorem{defn}{Definition}
%\newcommand{\bu}{\bullet} 
\newcommand{\bu}{\cdot} 
\newcommand{\n}{\noindent} 
%
\newcommand{\cc}{\mathcal} 
\newcommand{\bb}{\mathbb} 
\newcommand{\A}{\mathcal A}
\newcommand{\B}{\mathcal B}
\newcommand{\C}{\mathcal C}
\newcommand{\F}{\mathcal F}
\newcommand{\G}{\mathcal G}
\newcommand{\J}{\mathcal J}
\newcommand{\M}{\mathcal M} 
\newcommand{\SSS}{\mathcal S}
\newcommand{\U}{\mathcal U}
\newcommand{\V}{\mathcal V}
\newcommand{\Set}{\textbf{Set}} 
%\newcommand{\Set}{\boldmath{\mathrm{Set}}}%\newcommand{\Set}{\pmb{Set}} 
\newcommand{\Cat}{\textbf{Cat}} 
\newcommand{\CCat}{\textbf{CCat}} 
\newcommand{\e}{\varepsilon} 
\newcommand{\epi}{\twoheadrightarrow} 
\newcommand{\mono}{\rightarrowtail}
\newcommand{\m}{\rightarrowtail} 
%\newcommand{\op}{\text{op}}
\newcommand{\p}{\varphi} 
\newcommand{\pa}{\rightrightarrows} 
\newcommand{\pt}{\{\text{pt}\}} 
\newcommand{\xl}{\xleftarrow} 
\newcommand{\xr}{\xrightarrow} 
\newcommand{\be}{\begin{equation}} 
\newcommand{\ee}{\end{equation}} 
\newcommand{\cd}{commutative diagram} 
\newcommand{\ccd}{the comment containing Display} 
\newcommand{\nm}{natural morphism}
\newcommand{\pr}{Proposition} 
\newcommand{\sts}{t suffices to show} 
\newcommand{\rw}{[This is a rewriting of a previous comment. The new version below has already been incorporated into the main text.]} 
\newcommand{\cn}{(See (\ref{convnot}) p.~\pageref{convnot} for an explanation of the notation.) }
%
% LIMITS
% old
\newcommand{\colim}{\operatornamewithlimits{\underset{\longrightarrow}{lim}}} 
\newcommand{\ilim}{\operatornamewithlimits{\underset{\longrightarrow}{lim}}} 
\newcommand{\plim}{\operatornamewithlimits{\underset{\longleftarrow}{lim}}} 
% new
\DeclareMathOperator*{\coli}{colim}
\DeclareMathOperator*{\co}{colim}
\DeclareMathOperator*{\icolim}{``\coli"}
\DeclareMathOperator*{\ic}{``\coli"}
% 
\DeclareMathOperator{\Ad}{Add} 
\DeclareMathOperator{\Coim}{Coim}
\DeclareMathOperator{\Coker}{Coker}
\DeclareMathOperator{\Ima}{Im} 
\DeclareMathOperator{\IM}{IM} 
\DeclareMathOperator{\hy}{h} 
\DeclareMathOperator{\ky}{k} 
\DeclareMathOperator{\id}{id}
\DeclareMathOperator{\Fct}{Fct}
\DeclareMathOperator{\Hom}{Hom}
\DeclareMathOperator{\h}{Hom}
\DeclareMathOperator{\Ind}{Ind}
\DeclareMathOperator{\Ker}{Ker}
\DeclareMathOperator{\Mod}{Mod} 
\DeclareMathOperator{\Mor}{Mor} 
\DeclareMathOperator{\Ob}{Ob} 
\DeclareMathOperator{\op}{op} 
%
\begin{document} 
% 
\n$\bu$ P.~169, Lemma 8.2.3. Here is a statement contained in Lemma 8.2.3:
%
\begin{cor}\label{823}
%
Let $\C$ be a pre-additive category, let $X_1$ and $X_2$ be two objects of $\C$ such that the product $X=X_1\times X_2$ exists in $\C$, let $p_a:X\to X_a$ be the projection, define $i_a:X_a\to X$ by 
$$
p_a\circ i_b=\begin{cases}\id_{X_a}&\text{if }a=b\\0&\text{if }a\not=b.\end{cases}
$$ 
Then $X$ is a coproduct of $X_1$ and $X_2$ by $i_1$ and $i_2$. Moreover we have 
$$
i_1\circ p_1+i_2\circ p_2=\id_{X_1\times X_2}.
$$
\end{cor}

Let us denote the object $X$ above by $X_1\oplus X_2$. The following lemma is implicit in the book. 

\begin{lem}
For $a=1,2$ let $f_a:X_a\to Y_a$ be a morphism in $\C$. Assume that $X_1\oplus X_2$ and $Y_1\oplus Y_2$ exist in $\C$. Then we have $f_1\times f_2=f_1\sqcup f_2$ (equality in $\h_\C(X_1\oplus X_2,Y_1\oplus Y_2)$. 
\end{lem} 

We denote this morphism by $f_1\oplus f_2$.\medskip 

\n{\em Proof.} Put $X:=X_1\oplus X_2,\ Y:=Y_1\oplus Y_2$ and write 
$$
X_a\xr{i_a}X\xr{p_a}X_a,\quad Y_a\xr{j_a}Y\xr{q_a}Y_a
$$ 
for the projections and coprojections. We have $q_a\circ(f_1\times f_2)=f_a\circ p_a$, and we must show $q_b\circ (f_1\times f_2)\circ i_a=q_b\circ j_a\circ f_a$ for all $b$. This follows immediately from Corollary~\ref{823}. q.e.d. 

%% 

\n$\bu$ P.~172, proof of Lemma 8.2.10. Recall the statement: $\C$ is an additive category, $X$ is in $\C$. The claim is that $X$ is an abelian group object. The addition is given by the codiagonal morphism $\sigma:X\oplus X\to X$. This comment is only about the associativity of the addition. This associativity can also be proved as follows. 

Put $X^n:=X\oplus\cdots\oplus X$ ($n$ factors), and let $X\xr{i_a}X^n\xr{\sigma_n}X$ be respectively the $a$-th coprojection and the codiagonal morphism. It clearly suffices to show that the composition 
$$
X^3\xr{\sigma_2\oplus X}X^2\xr{\sigma_2}X
$$ 
is equal to $\sigma_3$. This follows from the fact that the composition 
$$
X\xr{i_a}X^3\xr{\sigma_2\oplus X}X^2
$$ 
is equal to $i_b$ with 
$$
b=\begin{cases}1&\text{if }a=1,2\\2&\text{if }a=3.\end{cases}
$$ 
q.e.d. 

%% 

\n$\bu$ P.~172, Lemma 8.2.11. Here is a minor variant of the statement: 
%
\begin{lem}
Let $F:\C\to\C'$ be a functor between additive categories, let $X$ be in $\C$, and let 
$$
\begin{tikzcd}
F(X\oplus X)\ar[yshift=.7ex]{r}{f}&F(X)\oplus F(X)\ar[yshift=-.7ex]{l}{g}
\end{tikzcd}
$$ 
be the natural morphisms. (More precisely, $f$ and $g$ are respectively obtained by regarding $\oplus$ as a product and as a coproduct.) If $f$ or $g$ is an isomorphism, then the other is its inverse. 
\end{lem}
% 
This follows from Lemma 8.2.3 p.~169 of the book. 

%% 

\n$\bu$ P.~173. \pr s 8.2.12 and 8.2.13 can be stated as follows. 
%
\begin{prop}\label{8212}
%
Let $\C$ be an additive category, let $\operatorname{Add}(\C,\Mod(\mathbb Z))$ and $\operatorname{Prod}(\C,\textbf{\em Set})$ be the category of additive functors from $\C$ to $\Mod(\mathbb Z)$ and the category of finite products preserving functors from $\C$ to $\textbf{\em Set}$, and let $F$ be in $\operatorname{Prod}(\C,\textbf{\em Set})$. Then the composition 
$$
F(X)\times F(X)\xleftarrow\sim F(X\oplus X)\xr{\sigma_X}F(X)
$$ 
defines a structure of abelian group on $F(X)$. This construction defines a functor 
$$
\Phi:\operatorname{Prod}(\C,\textbf{\em Set})\to\operatorname{Add}(\C,\Mod(\mathbb Z)).
$$ 
Let 
$$
\Psi:\operatorname{Add}(\C,\Mod(\mathbb Z))\to\operatorname{Prod}(\C,\textbf{\em Set})
$$ 
be the natural functor. Then $\Phi$ and $\Psi$ are inverse isomorphisms. 
%
\end{prop}

\n$\bu$ P.~173, Theorem 8.2.14. Recall the statement: 

Let $\C$ be an additive category. Then $\C$ has a unique structure of pre-additive category. 

Here is a minor variant of the proof of the existence of such a structure. 

Let $X$ and $Y$ be in $\C$. We define the addition on $\h_\C(X,Y)$ by letting $k:\C\to\C^\vee$ be the Yoneda embedding and observing that, in the notation of \pr\ \ref{8212}, $\h_\C(X,Y)$ is the set underlying the abelian group $\Phi(k(X))(Y)$. In particular, we have $g\circ(f_1+f_2)$ for $X\xr{f_i}Y\xr gZ$. It is easy to conclude from this that the above construction endows $\C$ with a structure of pre-additive category. 

%% 

\n$\bu$ P.~173, \pr\ 8.2.15. %(See also Section \ref{169} p.~\pageref{169}.) 
Recall the setting: $F:\C\to\C'$ is a functor between additive categories, and the claim is: $F$ is additive $\iff$ $F$ commutes with finite products. I think the authors forgot to prove the implication $\implies$. Let us do it. It suffices to show that $F$ commutes with $n$-fold products for $n=0$ or $n=2$. 

Case $n=0$: Put $X:=F(0)$. We must prove $X\simeq 0$. The equality $0=1$ holds in the ring $\Hom_\C(X,X)$ because it holds in the ring $\Hom_\C(0,0)$. As a result, the morphisms $0\to X$ and $X\to 0$ are inverse isomorphisms. 

Case $n=2$: Let $X_1,X_2$ be in $\C$. To check that the natural morphisms 
%
\be\label{173} 
F(X_1\oplus X_2)\rightleftarrows F(X_1)\oplus F(X_2)
\ee 
% 
are inverse isomorphisms, let $p_j:X_1\oplus X_2\to X_j$ and $i_j:X_j\to X_1\oplus X_2$ be the projections and coprojections, and apply Lemma 8.2.3 p.~169 to the morphisms $p_j,i_j,F(p_j),F(i_j)$. 
%
\end{document}
