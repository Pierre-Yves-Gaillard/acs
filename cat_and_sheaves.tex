%\title{Print ACS}
% !TEX encoding = UTF-8 Unicode
% print acs
% old version:
% https://docs.google.com/document/d/1n47WZ7Ka9hSBzIXCBkboFPYzhe6lHJdQZ3QZliwOs_4/edit
\documentclass[12pt]{article}
\newcommand{\nc}{\newcommand}
\nc{\up}{\usepackage}
\up[T1]{fontenc}
\up[utf8]{inputenc}
\up{amssymb,amsmath,amsthm}
\up[a4paper,headheight=14.5pt]{geometry}
%\up{makeidx}
%\up{datetime}
\up{tikz-cd}
\up[pdfusetitle]{hyperref}
%\up{comment}
\up{fancyhdr}
%\up{showkeys}%%\up{showlabels}
%\makeindex
\pagestyle{empty}
%\pagestyle{fancy}
\addtolength{\parskip}{.5\baselineskip}
\nc{\bu}{\bullet}%\nc{\bu}{*}

% BEGIN
\nc{\nt}{\newtheorem}
\nt{thm}{Theorem}
\nt{lem}[thm]{Lemma}
\nt{prop}[thm]{Proposition}
\nt{cor}[thm]{Corollary}
\nt{df}[thm]{Definition}
\nt{nota}[thm]{Notation}

\theoremstyle{remark}
\nt{claim}[thm]{Claim}
\nt{cond}[thm]{Condition}
\nt{conv}[thm]{Convention}
\nt{rk}[thm]{Remark}

\theoremstyle{definition}
\nt{s}[thm]{\S}

\hyphenation{Grothen-dieck mono-mor-phism mono-mor-phisms car-di-nal car-di-nals}

\nc{\dis}{\displaystyle}\nc{\ds}{\displaystyle}
\nc{\nn}{\noindent}
\nc{\mc}{\mathcal}%\nc{\cc}{\mathcal}
\nc{\bb}{\mathbb}
\nc{\oo}{\operatorname}

\nc{\A}{\mc A}
\nc{\B}{\mc B}
\nc{\C}{\mc C}
\nc{\DD}{\mc D}
\nc{\E}{\mc E}
\nc{\F}{\mc F}
\nc{\G}{\mc G}
\nc{\I}{\mc I}
\nc{\J}{\mc J}
\nc{\K}{\mc K}
\nc{\M}{\mc M}
\nc{\N}{\mc N}
\nc{\OO}{\mc O}
\nc{\PP}{\mc P}
\nc{\R}{\mc R}
\nc{\SSS}{\mc S}
\nc{\T}{\mc T}
\nc{\U}{\mc U}
\nc{\V}{\mc V}
\nc{\Cat}{\mathbf{Cat}}
\nc{\Set}{\mathbf{Set}}
\nc{\pt}{\{\text{pt}\}}

\nc{\al}{\alpha}
\nc{\bt}{\beta}
\nc{\ci}{\circ}\nc{\rd}{\circ}
\nc{\DT}{\Delta}
\nc{\dg}{\dagger}\nc{\ddg}{\ddagger}
\nc{\ee}{\varepsilon}
\nc{\epi}{\twoheadrightarrow}
\nc{\fthat}{(f^t)\ \widehat{}\ }
\nc{\HOM}{\cc H\!\mathit{om}}
\nc{\incl}{\hookrightarrow}
\nc{\iso}{\simeq}
\nc{\lb}{\label}
\nc{\ld}{\lambda}
\nc{\mono}{\rightarrowtail}
\nc{\mt}{\mapsto}
\nc{\os}{\overset}
\nc{\parar}{\rightrightarrows}
\nc{\pp}{\varphi}
\nc{\pr}{\pageref}
\nc{\qr}{\eqref}
\nc{\sbs}{\subsection}
\nc{\si}{\Leftarrow}
\nc{\ssi}{\Leftrightarrow}
\nc{\then}{\Rightarrow}
\nc{\tm}{\times}
\nc{\wdg}{\wedge}\nc{\wg}{\wedge}
\nc{\xl}{\xleftarrow}
\nc{\xr}{\xrightarrow}

% LIMITS
% old
\nc{\ilim}{\operatornamewithlimits{\underset{\longrightarrow}{lim}}}
\nc{\plim}{\operatornamewithlimits{\underset{\longleftarrow}{lim}}}
% new
\nc{\DMO}{\DeclareMathOperator}
\DMO*{\colim}{colim}
\DMO*{\col}{colim}
\DMO*{\icolim}{``\!\colim\!"}
\DMO*{\ic}{``\!\colim\!"}

\DMO{\Ad}{Add}
\DMO{\card}{card}
\DMO{\Coim}{Coim}
\DMO{\Coker}{Coker}
\DMO{\D}{D}
\DMO{\Ext}{Ext}
\DMO{\Ima}{Im}
\DMO{\IM}{IM}
\DMO{\hy}{h}
\DMO{\ky}{k}
\DMO{\id}{id}
\DMO{\jj}{j}
\DMO{\Fct}{Fct}
\DMO{\Hom}{Hom}
\DMO{\RHom}{RHom}
\DMO{\Ind}{Ind}
\DMO{\Ker}{Ker}
\DMO{\Mc}{Mc}
\DMO{\Mod}{Mod}
\DMO{\Mor}{Mor}
\DMO{\Ob}{Ob}
\DMO{\op}{op}
\DMO{\PSh}{PSh}
\DMO{\Qis}{Qis}
\DMO{\Sh}{Sh}
%\ar[r,yshift=0.7ex]\ar[r,yshift=-0.7ex]\begin{tikzcd}\end{tikzcd}
% END

\begin{document}

\begin{nota}\lb{nhove}
We denote this morphism of functors by
$$
\begin{pmatrix}
\theta_{11}&\theta_{12}\\ 
\theta_{21}&\theta_{22}
\end{pmatrix}:F_{31}F_{32}\to F_{11}F_{12}.
$$ 
If $\theta_{21}$ and $\theta_{22}$ are identity morphisms, we put 
$$
\theta_{11}*\theta_{12}:=
\begin{pmatrix}
\theta_{11}&\theta_{12}\\ 
\theta_{21}&\theta_{22}
\end{pmatrix}.
$$ 
If $\theta_{12}$ and $\theta_{22}$ are identity morphisms, we put 
$$
\theta_{11}\ci\theta_{21}:=
\begin{pmatrix}
\theta_{11}&\theta_{12}\\ 
\theta_{21}&\theta_{22}
\end{pmatrix}.
$$ 
\end{nota}

\begin{df}[universal element]\lb{ue} 
Let $F:\C^{\op}\to\Set$ be a functor and $X$ an object of $\C$. An $(F,X)$-\emph{universal element}\index{universal element} is an element $u$ of $F(X)$ such that, for all $Y$ in $\C$, the map $\Hom_\C(Y,X)\to F(Y),\ f\mt F(f)(u)$ is bijective. In such a setting, we say that $X$ \emph{represents}\index{represents} $F$, or, when we want to be more precise, that the pair $(X,u)$ \emph{represents} $F$. 
\end{df}

\begin{df}[co-universal element]\lb{ue2} 
Let $F:\C\to\Set$ be a functor and $X$ an object of $\C$. An $(F,X)$-\emph{co-universal element}\index{co-universal element} is an element $u$ of $F(X)$ such that, for all $Y$ in $\C$, the map $\Hom_\C(X,Y)\to F(Y),\ f\mt F(f)(u)$ is bijective. In such a setting, we say that $X$ \emph{co-represents}\index{co-represents} $F$, or, when we want to be more precise, that the pair $(X,u)$ \emph{co-represents} $F$. 
\end{df}

\hrule

\sbs{Partially defined adjoints (Section 1.5, p.~28)}\lb{defat}

P. 27, Section 1.5. The following is implicit in the book.

Let $L:\C\to\C'$ be a functor and $X'$ an object of $\C'$. Let us assume we are given an object of $\C$ which we denote by $R(X')$. (The reason for this seemingly silly notation will be given soon!) We say that ``the value of the right adjoint $R$ to $L$ is defined at $X'$ and isomorphic to $R(X')$'', or, abusing the terminology, that ``$R(X')$ exists'' if there is a morphism $\eta_{X'}:L(R(X'))\to X'$ such that the pair $(R(X'),\eta_{X'})$ represents the functor $\Hom_{\C'}(L(\ ),X'):\C^{\op}\to\Set$ in the sense of Definition~\ref{ue} p.~\pr{ue}. This means that, for all $X\in\C$, the map 
$$
\Hom_\C(X,R(X'))\to\Hom_{\C'}(L(X),X'),\quad f\mt\eta_{X'}\ci L(f)
$$ 
is bijective, or, more naively, that for all $g:L(X)\to X'$ there is a unique $f:X\to R(X')$ such that $\eta_{X'}\ci L(f)=g$: 
$$
\begin{tikzcd}
X\ar[d,dashed,"f"']&L(X)\ar[d,"L(f)"']\ar[r,"g"]&X'\\ 
R(X')&L(R(X')).\ar[ru,"\eta_{X'}"']
\end{tikzcd}
$$ 

Let us spell out the dual statement:

Let $R:\C'\to\C$ be a functor and $X$ an object of $\C'$. Let us assume we are given an object of $\C'$ which we denote by $L(X)$. We say that ``the value of the left adjoint $L$ to $R$ is defined at $X$ and isomorphic to $L(X)$'', or, abusing the terminology, that ``$L(X)$ exists'' if there is a morphism $\ee_X:X\to R(L(X))$ such that the pair $(L(X),\ee_X)$ co-represents the functor $\Hom_\C(X,R(\ )):\C'\to\Set$ in the sense of Definition~\ref{ue2} p.~\pr{ue2}. This means that, for all $X'\in\C'$, the map 
$$
\Hom_{\C'}(L(X),X')\to\Hom_\C(X,R(X')),\quad f\mt R(f)\ci\ee_X
$$ 
is bijective, or, more naively, that for all $g:X\to R(X')$ there is a unique $f:L(X)\to X'$ such that $R(f)\ci\ee_X=g$: 
$$
\begin{tikzcd}
X\ar[r,"\ee_X"]\ar[dr,"g"']&R(L(X))\ar[d,"R(f)"]&L(X)\ar[d,dashed,"f"]\\ 
&R(X')&X'.
\end{tikzcd}
$$ 

\hrule

The definition of projective and inductive limits is a particular case of the situation considered in Section~\ref{defat} p.~\pr{defat}:% added !!!!!!!!!

\begin{df}[projective limit]\lb{plim}
If $\al:I^{\op}\to\C$ is a functor, then a {\em projective limit of}\index{projective limit} $\al$ {\em in} $\C$ is a pair 
$$
(X,p)\in\Ob(\C)\tm\Hom_{\C^{I^{\op}}}(\DT X,\al)
$$
such that $p$ is a $(\Hom_{\C^{I^{\op}}}(\DT(\ ),\al),X)$\--universal element (see Definition~\ref{ue} p.~\pr{ue}). For each $i$ in $I$ the morphism $p_i:X\to\al(i)$ is called the $i$-{\em projection}\index{projection} of $X$. If $\bt:I^{\op}\to\C$ is a functor, we often write $X\iso\lim\bt$ to indicate the fact that $(X,p)$ is a projective limit of $\bt$ in $\C$. (This huge abuse of notation seems unavoidable.) We also often say that $X$ (instead of $(X,p)$) is a projective limit of $\bt$ in $\C$. %However, if $\C=\Set$ and $I$ is small we denote by $\lim\bt$ the subset of $\prod_{i\in I}\bt(i)$ defined in Display (2.1.2) p.~36 of the book.
\end{df}

Recall that the condition that $p$ is a $(\Hom_{\C^{I^{\op}}}(\DT(\ ),\al),X)$\--universal element means that for each $Y$ in $\C$ and each morphism of functors $\theta:\DT Y\to\al$ there is a unique morphism $f:Y\to X$ satisfying $p\ci\DT f=\theta$: 
\begin{equation}\lb{yfy}
\begin{tikzcd}
Y\ar[d,dashed,"f"']&\DT Y\ar[d,"\DT f"']\ar[r,"\theta"]&\al\\ 
X&\DT X.\ar[ru,"p"']
\end{tikzcd}
\end{equation} 
In particular, if $(X',p')$ is another projective limit of $\al$, then there is a unique morphism $f:X\to X'$ such that $p'\ci\DT f=p$; moreover $f$ is an isomorphism.

Let $\al:I^{\op}\to\Set$ be a functor defined on a small category, set 
$$ 
\lim\al:=\left\{x\in\prod_{i\in I}\al(i)\ \bigg|\ x_i=\al(f)(x_j)\ \forall\ f:i\to j\right\}\in\Set,
$$ 
and define $p_i:\lim\al\to\al(i)$ by $p_i(x):=x_i$. It is easily seen that $(\lim\al,p)$ is a projective limit of $\al$. More precisely, the map $f$ in \qr{yfy} is given by 
$$
f(y):=(\theta_i(y))_{i\in I}.
$$
%\begin{df}We say that the above pair $(\lim\al,p)$ is \emph{the} projective limit of $\al$.\end{df}
\begin{conv}\lb{the}
We say that the above pair $(\lim\al,p)$ is \textbf{the} projective limit of $\al$.
\end{conv}

Note that the projective limit of $\al:I^{\op}\to\Set$ does not depend on the universe which makes $I$ a small category.

\begin{df}[inductive limit]\lb{ilim}
If $\al:I\to\C$ is a functor, then an {\em inductive limit}\index{inductive limit} of $\al$ {\em in} $\C$ is a pair 
$$
(X,p)\in\Ob(\C)\tm\Hom_{\C^I}(\al,\DT X)
$$
such that $p$ is a $(\Hom_{\C^I}(\al,\DT(\ )),X)$\--co-universal element (see Definition~\ref{ue2} p.~\pr{ue2}). For each $i$ in $I$ the morphism $p_i:\al(i)\to X$ is called the $i$-{\em coprojection}\index{coprojection} of $X$. If $\al:I\to\C$ is a functor, we often write $\colim\bt\iso X$ to indicate the fact that $(X,p)$ is an inductive limit of $\al$ in $\C$. (This huge abuse of notation seems unavoidable.) We also often say that $X$ (instead of $(X,p)$) is an inductive limit of $\al$ in $\C$. %However, if $\C=\Set$ and $I$ is small we denote by $\colim\al$ the set defined in Proposition 2.4.1 p.~54 of the book.
\end{df}

Recall that the condition that $p$ is a $(\Hom_{\C^I}(\al,\DT(\ )),X)$\--co-universal element means that for each $Y$ in $\C$ and each morphism of functors $\theta:\al\to\DT Y$ there is a unique morphism $f:X\to Y$ satisfying $\DT f\ci p=\theta$: 
\begin{equation}\lb{cue}
\begin{tikzcd}
\al\ar[r,"p"]\ar[dr,"\theta"']&\DT X\ar[d,"\DT f"]&X\ar[d,dashed,"f"]\\ 
&\DT Y&Y.
\end{tikzcd}
\end{equation} 
In particular, if $(X',p')$ is another inductive limit of $\al$, then there is a unique morphism $f:X\to X'$ such that $\DT f\ci p=p'$; moreover $f$ is an isomorphism.

Let $\al:I\to\Set$ be a functor defined on a small category, set 
$$ 
U:=\{(i,x)\in\U\ |\ i\in I,x\in\al(i)\},
$$ 
let $\sim$ be the least equivalence relation on $U$ satisfying $(i,x)\sim(j,\al(f)(x))$ for all morphisms $f:i\to j$, let $\col\al$ be the quotient $U/\!\!\sim$, let $\pi:U\to\col\al$ be the canonical projection, and, for all $i$ in $I$, define $p_i:\al(i)\to\col\al$ by $p_i(x):=\pi(i,x)$. It is easily seen that $(\col\al,p)$ is an inductive limit of $\al$. 
%\begin{df}We say that the above pair $(\lim\al,p)$ is \emph{the} projective limit of $\al$.\end{df}
\begin{conv}
We say that the above pair $(\lim\al,p)$ is \textbf{the} inductive limit of $\al$.
\end{conv}

\hrule

Items (a), (b) and (c) below are particular cases of the situation considered in Section~\ref{defat} p.~\pr{defat}:% added !!!!!!!!!

\nn(a) We say that $\pp^\dg\bt$ exists if there is a $u:\bt\to\pp_*\pp^\dg\bt$ such that the pair $(\pp^\dg\bt,u)$ co-represents the functor $\Hom_{\C^J}(\bt,\pp_*(\ )):\C^I\to\Set$ in the sense of Definition~\ref{ue2} p.~\pr{ue2}. This means that, for all $\al:I\to\C$, the map 
$$
\Hom_{\C^I}(\pp^\dg\bt,\al)\to\Hom_{\C^J}(\bt,\pp_*\al),\quad v\mt(v*\pp)\ci u
$$ 
is bijective, or, more naively, that for all $w:\bt\to\pp_*\al$ there is a unique $v:\pp^\dg\bt\to\al$ such that $(v*\pp)\ci u=w$: 
$$
\begin{tikzcd}
\bt\ar[r,"u"]\ar[dr,"w"']&\pp_*\pp^\dg\bt\ar[d,"v*\pp"]&\pp^\dg\bt\ar[d,dashed,"v"]\\ 
&\pp_*\al&\al.
\end{tikzcd}
$$
(Recall that $v*\pp$ designates the horizontal composition of $v$ and $\pp$; see Notation~ \ref{nhove} p.~ \pr{nhove}.)%(See \S\ref{durl} p.~\pr{durl} below.)%

\nn(b) We say that $\pp^\ddg\bt$ exists if there is a $u:\pp_*\pp^\ddg\bt\to\bt$ such that the pair $(\pp^\ddg\bt,u)$ represents the functor $\Hom_{\C^J}(\pp_*(\ ),\bt):(\C^I)^{\op}\to\Set$ in the sense of Definition~\ref{ue} p.~\pr{ue}. This means that, for all $\al:I\to\C$, the map 
$$
\Hom_{\C^I}(\al,\pp^\ddg\bt)\to\Hom_{\C^J}(\pp_*\al,\bt),\quad v\mt u\ci(v*\pp)
$$ 
is bijective, or, more naively, that for all $w:\pp_*\al\to\bt$ there is a unique $v:\al\to\pp^\ddg\bt$ such that $u\ci(v*\pp)=w$: 
$$
\begin{tikzcd}
\al\ar[d,dashed,"v"']&\pp_*\al\ar[d,"v*\pp"']\ar[r,"w"]&\bt\\ 
\pp^\ddg\bt&\pp_*\pp^\ddg\bt.\ar[ru,"u"']
\end{tikzcd}
$$ 

\end{document}
