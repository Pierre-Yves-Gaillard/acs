% !TEX encoding = UTF-8 Unicode
% about "categories and sheaves"
% https://drive.google.com/open?id=1JeiCW-iya7sAiY69ne4SZlPGzL6iXd6JffKMgXxLyZ8 
% https://goo.gl/ORIE5W
% https://github.com/Pierre-Yves-Gaillard/acs/edit/master/cat_and_sheaves.tex
% pdflatex makeindex pdflatex pdflatex
\documentclass[12pt]{article}
\newcommand{\nc}{\newcommand}
%\nc{\up}{\usepackage}
\usepackage[T1]{fontenc}
\usepackage[utf8]{inputenc}
\usepackage{amssymb,amsmath,amsthm}
\usepackage[a4paper,headheight=14.5pt]{geometry}
\usepackage{makeidx}
\usepackage{datetime}
%\usepackage{wasysym}
\usepackage{tikz-cd}
\usepackage[pdfusetitle]{hyperref}
\usepackage{comment}
\usepackage{fancyhdr}
%\usepackage{showkeys}%%\usepackage{showlabels}
\makeindex
%\pagestyle{empty}
\pagestyle{fancy}
\addtolength{\parskip}{.5\baselineskip}
\nc{\bu}{\bullet}%\nc{\bu}{*}

% BEGIN
\nc{\nt}{\newtheorem}
\nt{thm}{Theorem}
\nt{lem}[thm]{Lemma}
\nt{prop}[thm]{Proposition}
\nt{cor}[thm]{Corollary}
\nt{df}[thm]{Definition}
\nt{nota}[thm]{Notation}

\theoremstyle{remark}
\nt{claim}[thm]{Claim}
\nt{cond}[thm]{Condition}
\nt{conv}[thm]{Convention}
\nt{rk}[thm]{Remark}
\nt{uspb}[thm]{Unsolved Problem}
\nt{warning}[thm]{Warning}

\theoremstyle{definition}
\nt{s}[thm]{\S}

\hyphenation{Grothen-dieck mono-mor-phism mono-mor-phisms car-di-nal car-di-nals rep-re-sen-ta-tion}

\nc{\dis}{\displaystyle}\nc{\ds}{\displaystyle}
\nc{\nn}{\noindent}
\nc{\oo}{\operatorname}
\nc{\bb}{\mathbb}
\nc{\mbf}{\mathbf}
\nc{\mc}{\mathcal}

\nc{\A}{\mc A}
\nc{\B}{\mc B}
\nc{\C}{\mc C}
\nc{\DD}{\mc D}
\nc{\E}{\mc E}
\nc{\F}{\mc F}
\nc{\G}{\mc G}
\nc{\I}{\mc I}
\nc{\J}{\mc J}
\nc{\K}{\mc K}
\nc{\M}{\mc M}
\nc{\N}{\mc N}
\nc{\OO}{\mc O}
\nc{\PP}{\mc P}
\nc{\R}{\mc R}
\nc{\SSS}{\mc S}
\nc{\T}{\mc T}
\nc{\U}{\mc U}
\nc{\V}{\mc V}
\nc{\Cat}{\mathbf{Cat}}
\nc{\Set}{\mathbf{Set}}
\nc{\pt}{\{\text{pt}\}}

\nc{\al}{\alpha}
\nc{\bt}{\beta}
\nc{\ci}{\circ}\nc{\rd}{\circ}
\nc{\DT}{\Delta}
\nc{\dg}{\dagger}\nc{\ddg}{\ddagger}
\nc{\ee}{\varepsilon}
\nc{\epi}{\twoheadrightarrow}
\nc{\fthat}{(f^t)\ \widehat{}\ }
\nc{\HOM}{\mc H\!\mathit{om}}
\nc{\incl}{\hookrightarrow}
\nc{\iso}{\simeq}
\nc{\lb}{\label}
\nc{\ld}{\lambda}
\nc{\mono}{\rightarrowtail}
\nc{\mt}{\mapsto}
\nc{\os}{\overset}
\nc{\parar}{\rightrightarrows}
\nc{\pp}{\varphi}
\nc{\pr}{\pageref}
\nc{\qr}{\eqref}
\nc{\sbs}{\subsection}
\nc{\si}{\Leftarrow}
\nc{\ssi}{\Leftrightarrow}
\nc{\then}{\Rightarrow}
\nc{\tm}{\times}
\nc{\vide}{\varnothing}\nc{\vi}{\varnothing}
\nc{\wdg}{\wedge}\nc{\wg}{\wedge}
\nc{\xl}{\xleftarrow}
\nc{\xr}{\xrightarrow}

% LIMITS
% old
\nc{\ilim}{\operatornamewithlimits{\underset{\longrightarrow}{lim}}}
\nc{\plim}{\operatornamewithlimits{\underset{\longleftarrow}{lim}}}
% new
\nc{\DMO}{\DeclareMathOperator}
\DMO*{\colim}{colim}
\DMO*{\col}{colim}
\DMO*{\icolim}{``\!\colim\!"}
\DMO*{\ic}{``\!\colim\!"}

\DMO{\Ad}{Add}
\DMO{\card}{card}
\DMO{\Coim}{Coim}
\DMO{\Coker}{Coker}
\DMO{\D}{D}
\DMO{\Ext}{Ext}
\DMO{\Ima}{Im}
\DMO{\IM}{IM}
\DMO{\hy}{h}
\DMO{\ky}{k}
\DMO{\id}{id}
\DMO{\jj}{j}
\DMO{\Fct}{Fct}
\DMO{\Hom}{Hom}
\DMO{\RHom}{RHom}
\DMO{\Ind}{Ind}
\DMO{\Ker}{Ker}
\DMO{\Mc}{Mc}
\DMO{\Mod}{Mod}
\DMO{\Mor}{Mor}
\DMO{\Ob}{Ob}
\DMO{\op}{op}
\DMO{\PSh}{PSh}
\DMO{\Qis}{Qis}
\DMO{\Sh}{Sh}
%\textsf{happly}\ar[yshift=0.7ex]{r}\ar[yshift=-0.7ex]{r}
% END

\title{About \em{Categories and Sheaves}}
\author{Pierre-Yves Gaillard}%\footnote{pierre.yves.gaillard at gmail.com}
\date{\today, \currenttime}

\begin{document}

\maketitle

\nn The last version of this text is available at\medskip

\centerline{\href{http://iecl.univ-lorraine.fr/~Pierre-Yves.Gaillard/DIVERS/KS/}{http://iecl.univ-lorraine.fr/$\sim$Pierre-Yves.Gaillard/DIVERS/KS/}}

My email is pierre.yves.gaillard at gmail.com.

\tableofcontents\newpage

\nn The purpose of this text is to make a few comments about the book 

\textbf{Categories and Sheaves} by Kashiwara and Schapira, Springer 2006, 

\nn referred to as ``the book'' henceforth. 

An important reference is

\nn[GV] Grothendieck, A. and Verdier, J.-L. (1972). Préfaisceaux. In Artin, M., Grothendieck, A. and Verdier, J.-L., editors, Théorie des Topos et Cohomologie Étale des Schémas, volume 1 of Séminaire de géométrie algébrique du Bois-Marie, 4, pages 1-218. Springer. 

\nn{\href{http://www.cmls.polytechnique.fr/perso/orgogozo/SGA4/01/01.pdf}{http://www.cmls.polytechnique.fr/perso/orgogozo/SGA4/01/01.pdf}}

Here are two useful links:

\nn Schapira's Errata:\\ \href{https://webusers.imj-prg.fr/~pierre.schapira/books/Errata.pdf}{https://webusers.imj-prg.fr/$\sim$pierre.schapira/books/Errata.pdf}, 

\nn nLab entry: \href{http://ncatlab.org/nlab/show/Categories+and+Sheaves}{http://ncatlab.org/nlab/show/Categories+and+Sheaves}. 

The tex and pdf files for this text are available at
 
\nn\href{http://iecl.univ-lorraine.fr/~Pierre-Yves.Gaillard/DIVERS/KS/}{http://iecl.univ-lorraine.fr/$\sim$Pierre-Yves.Gaillard/DIVERS/KS/}

The tex file is available at

\nn\href{https://github.com/Pierre-Yves-Gaillard/acs}{https://github.com/Pierre-Yves-Gaillard/acs}

\nn\href{https://goo.gl/eJxVyj}{https://goo.gl/eJxVyj}

%\nn\href{https://app.box.com/s/ktfju6mts4bq3loknnrt}{https://app.box.com/s/ktfju6mts4bq3loknnrt}

%\nn\href{http://goo.gl/klKgiW}{http://goo.gl/klKgiW}

%\nn\href{https://www.mediafire.com/folder/am7yqw1whitdg/}{https://www.mediafire.com/folder/am7yqw1whitdg/}

%\nn\href{https://mega.co.nz/#F!udI0CahS!YSL4YDDiougL0svrOHlyRA}{https://mega.co.nz/\#F!udI0CahS!YSL4YDDiougL0svrOHlyRA}

More links are available at \href{http://goo.gl/df2Xw}{http://goo.gl/df2Xw}. 

I have rewritten some of the proofs in the book. Of course, I'm not suggesting that my wording is better than that of Kashiwara and Schapira! I just tried to make explicit a few points which are implicit in the book. 

I adhere to Bourbaki's\index{Bourbaki} set theory as expounded in the book \textbf{Théorie des ensembles}, N. Bourbaki, Hermann, Paris, 1970. (I'm ignoring the ``Fascicule de résultats'' in the above book because I don't understand it.) 

The notation of the book will be freely used. We will sometimes write $\B^\A$ for $\Fct(\A,\B)$, $\al_i$ for $\al(i)$, $fg$ for $f\ci g$, and some parenthesis might be omitted. We write $\bigsqcup$\index{$\bigsqcup$} instead of $\coprod$\index{$\coprod$} for the coproduct\index{coproduct}. 

Following a suggestion of Pierre Schapira's, we shall denote projective limits\index{projective limit} by $\lim$\index{$\lim$} instead of $\plim$\index{$\plim$}, and inductive limits\index{inductive limit} by $\col$\index{$\col$} instead of $\ilim$\index{$\ilim$}. 

Thank you to Pierre Schapira for his interest!

%%%

\section{U-categories and U-small Categories\index{$\U$-category}\index{$\U$-small category}}\lb{ucat}

Here are a few comments about the definition of a $\U$-category on p.~11 of the book. Let $\U$ be a universe. Recall that an element of $\U$ is called a $\U$-set\index{$\U$-set}. The following definitions are used in the book: 

\begin{df}[$\U$-category]\lb{ucatg} 
A $\U$-{\em category} is a category $\C$ such that, for all objects $X,Y$, the set $\Hom_\C(X,Y)$ of morphisms from $X$ to $Y$ is equipotent to some $\U$-set. 
\end{df} 

\begin{df}[$\U$-small category] 
The category $\C$ is $\U$-{\em small} if in addition the set of objects of $\C$ is equipotent to some $\U$-set. 
\end{df} 

One could also consider the following variant: 

\begin{df}[$\U$-category]\lb{myucat} 
A $\U$-{\em category} is a category $\C$ such that, for all objects $X,Y$, the set $\Hom_\C(X,Y)$ is a $\U$-set. 
\end{df} 

\begin{df}[$\U$-small category]\lb{myuscat}
The category $\C$ is $\U$-{\em small}\index{$\U$-small category} if in addition the set of objects of $\C$ is a $\U$-set. 
\end{df} 

Note that a category $\C$ is a $\U$-category in the sense of Definition~\ref{ucatg} if and only if there is a $\U$-category in the sense of Definition~\ref{myucat} which is isomorphic to $\C$, and similarly for $\U$-small categories. 
\begin{center}\fbox{In this text we shall always use Definitions \ref{myucat} and \ref{myuscat}.}
\end{center}

We often assume implicitly that a universe $\U$ has been chosen, and we say ``category'' and ``small category'' instead of ``$\U$-category'' and ``$\U$-small category''.

%%%

\section{Typos and Details}

$*$ P.~11, Definition 1.2.1, Condition (b): $\Hom(X,X)$ should be $\Hom_{\C}(X,X)$. 

\nn$*$ P.~14, definition of $\Mor(\C)$. As the hom-sets of $\C$ are not assumed to be disjoint, it seems better to define $\Mor(\C)$ as a category of functors. See \S\ref{d125} p.~\pr{d125}. 

\nn$*$ P.~25, Corollary 1.4.6. Due to the definition of $\U$-small category used in this text (see Section~\ref{ucat} p.~\pr{ucat}), the category $\C_A$ of the corollary is no longer $\U$-small, but only canonically isomorphic to some $\U$-small category. 

\nn$*$ P.~25, Proof of Corollary 1.4.6 (second line): $\hy_\C$ should be $\hy_{\C'}$.

\nn$*$ P.~26, Proposition 1.4.10, end of the proof: $\Hom_\C(Y,X)\to F(X)$ should be $\Hom_\C(Y,X)\to F(Y)$. 

\nn$*$ P.~33, Exercise 1.19: the arrow from $L_1\ci R_1\ci L_2$ to $L_2$ should be $\eta_1\ci L_2$ instead of $\ee_1\ci L_2$. 

\nn$*$ P.~37, Remark 2.1.5: ``Let $I$ be a small set'' should be ``Let $I$ be a small category''.

\nn$*$ P.~41, sixth line: (i) should be (a). 

\nn$*$ P.~52, fourth line: $\Mor(I,\C)$ should be $\Fct(I,\C)$.

\nn$*$ P.~53, Part (i) (c) of the proof of Theorem 2.3.3 (Line 2): ``$\bt\in\Fct(J,\A)$'' should be ``$\bt\in\Fct(J,\C)$''.

\nn$*$ P.~54, second display: we should have $i\to\pp(j)$ instead of $\pp(j)\to i$.

\nn$*$ P.~58, Corollary 2.5.3: The assumption that $I$ and $J$ are small is not necessary. (The statement does not depend on the Axiom of Universes.) 

\nn$*$ P.~58, Proposition 2.5.4: Parts (i) and (ii) could be replaced with the statement: ``If two of the functors $\pp,\psi$ and $\pp\ci\psi$ are cofinal, so is the third one''.

\nn$*$ Pp.~63-64, statement and proof of Corollary 2.7.4: all the $h$ are slanted, but they should be straight.

\begin{s}\lb{27i}
P.~65, Exercise 2.7, Line 3: ``that the functor $\cdot\tm_ZY:\Set_Z\to\Set_Z$'' should be ``that, given $Y\in\Set_Z$, the functor $\cdot\tm_ZY:\Set_Z\to\Set_Y$''. (See also \S\ref{s3cat} p.~\pr{s3cat} below.)
\end{s}

\nn$*$ P.~74, first line of the proof of Theorem 3.1.6: $\ds\ilim$ should be $\ds\ilim_i$. 

\nn$*$ P.~74, last four lines: $\al$ should be $\pp$.

\nn$*$ P.~79, proof of Proposition 3.2.5: the word ``filtrant'' should be replaced with the word ``connected''.

\nn$*$ P.~80, last display: a $\ds\ilim$ is missing.

\nn$*$ P.~83, Statement of Proposition 3.3.7 (iv) and (v): $k$ might be replaced with $R$. 

\nn$*$ Pp 83 and 85, Proof of Proposition 3.3.7 (iv): ``Proposition 3.1.6'' should be ``Theorem 3.1.6''. Same typo on p.~85, Line 6.

\nn$*$ P.~84, Proposition 3.3.13. It is clear from the proof (I think) that the intended statement was the following one: If $\C$ is a category admitting finite inductive limits and if $A:\C^{\op}\to\Set$ is a functor, then we have 
$$
\C\text{ small and }\C_A\text{ filtrant }\then A\text{ left exact }\then\C_A\text{ filtrant}.
$$

\nn$*$ P.~88, Proposition 3.4.3 (i). It would be better to assume that $\C$ admits small inductive limits.

\nn$*$ P.~89, last sentence of the proof of Proposition 3.4.4. The argument is slightly easier to follow if $\psi'$ is factorized as 
$$
(J_1)^{j_2}\xr a(J_1)^{\psi_2(j_2)}\xr b(K_1)^{\psi_2(j_2)}\xr c(K_1)^{\pp_2(i_2)}.
$$ 
Then $a,b$ and $c$ are respectively cofinal by Parts (ii), (iii) and (iv) of Proposition 3.2.5 p.~79 of the book.

\nn$*$ P.~90, Exercise 3.2: ``Proposition 3.1.6'' should be ``Theorem 3.1.6''.

\nn$*$ P.~115, line 4: ``two morphisms $i_1,i_2:Y\to Y\sqcup_XY$'' should be ``two morphisms $i_1,i_2:Y\parar Y\sqcup_XY$''. 

\nn$*$ P.~115, Line 8: $i_1\ci g=i_2\ci g$ should be $g\ci i_1=g\ci i_2$.

\nn$*$ P.~120, proof of Theorem 5.2.6. We define $u':X'\to F$ as the element of $F(X')$ corresponding to the element $(u,u_0)$ of $F(X)\tm_{F(X_1)}F(Z_0)$ under the natural bijection. (Recall $X':=X\sqcup_{X_1}Z_0$.)

\nn$*$ P.~121, proof of Proposition 5.2.9. The fact that, in Proposition 5.2.3 p.~118 of the book, only Part~(iv) needs the assumption that $\C$ admits small coproducts is implicitly used in the sequel of the book.

\nn$*$ P.~128, proof of Theorem 5.3.9. Last display: $\sqcup$ should be $\cup$. It would be simpler in fact to put 
$$
\Ob(\F_n):=\{Y_1\sqcup_XY_2\ |\ X\to Y_1\text{ and }X\to Y_2\text{ are morphisms in }\F_{n-1}\}.
$$ 

\nn$*$ P.~128, proof of Theorem 5.3.9,, just before the ``q.e.d.'': Corollary 5.3.5 should be Proposition 5.3.5.

\nn$*$ P.~132, Line 2: It would be slightly better to replace ``for small and filtrant categories $I$ and $J$'' with ``for small and filtrant categories $I$ and $J$ and functors $\al:I\to\C,\bt:J\to\C$''.

\nn$*$ P.~132, Line 3: $\Hom_\C(A,B)$ should be $\Hom_{\Ind(\C)}(A,B)$.

\nn$*$ P.~132, Lines 4 and 5: \guillemotleft We may replace ``filtrant and small'' by ``filtrant and cofinally small'' in the above definition\guillemotright: see Proposition~\ref{355} p.~\pr{355}.

\nn$*$ P.~132, Corollary 6.1.6: The following fact is implicit. Let $\C\xr{F}\C'\xr{G}\C''$ be functors, let $X'$ be in $\C'$, and assume that $G$ is fully faithful. Then the functor $\C_{X'}\to\C_{G(X')}$ induced by $G$ is an isomorphism.

\nn$*$ P.~133, Proposition 6.1.9. ``There exists a unique functor ...'' should be ``There exists a functor ... Moreover, this functor is unique up to unique isomorphism.''

\begin{s}\lb{s133ii} 
P.~133. In Part (ii) of Proposition 6.1.9 the authors, I think, intended to write 
$$
``\ilim"(IF\ci\al)\xr\sim IF(``\ilim"\al)
$$
instead of 
$$
IF(``\ilim"\al)\xr\sim``\ilim"(IF\ci\al). 
$$ 
\end{s}

\nn$*$ P.~134, proof of Proposition 6.1.12: ``$\C_A\tm\C_{A'}$'' should be ``$\C_A\tm\C'_{A'}$'' (twice).

\nn$*$ P.~135, Corollary 6.1.14: $f=``\ilim"\pp$ should be $f\iso``\ilim"\pp$. (This is an isomorphism in $\Mor(\Ind(\C))$.)

\begin{s}\lb{s135a}
$*$ P.~135, Corollary 6.1.15: $f=``\ilim"\pp$ should be $f\iso``\ilim"\pp$ and $g=``\ilim"\psi$ should be $g\iso``\ilim"\psi$. (See Section~\ref{s135b} p.~\pr{s135b} below.)
\end{s}

\nn$*$ P.~136, proof of Proposition 6.1.16: see \S\ref{cipc} p.~\pr{cipc}.

\nn$*$ P.~136, proof of Proposition 6.1.18. Second line of the proof: ``Corollary 6.1.14'' should be ``Corollary 6.1.15''. 

\nn$*$ P.~136, last line: ``the cokernel of $(\al(i),\bt(i))$'' should be ``the cokernel of $(\pp_i,\psi_i)$''. Moreover, the cokernel in question is denoted by $\ld_i$ on the last line of p.~136 and by $\ld(i)$ on the first line of p.~137.

\nn$*$ P.~138, second line of Section 6.2: ``the functor $``\ds\lim_{\longrightarrow}"$ is representable in $\C$'' should be ``the functor $``\ds\lim_{\longrightarrow}"\al$ is representable in $\C$''. Next line: ``natural functor'' should be ``natural morphism''.

\nn$*$ P.~138, Proposition 6.2.1. The assumption that $I$ is small is not really necessary. (See Section~\ref{138} p.~\pr{138} below.) 

\nn$*$ P.~141, Display (6.3.2): $\neq$ should be $\not\iso$ (see Section \ref{131} p.~\pr{131} below). 

\nn$*$ P.~141, Corollary 6.3.7 (ii): $\id$ should be $\id_\C$. 

\nn$*$ P.~143, third line of the proof of Proposition 6.4.2: $\{Y_i\}_{I\in I}$ should be $\{Y_i\}_{i\in I}$.

\nn$*$ P.~144, proof of Proposition 6.4.2, Step~(ii), second sentence: It might be better to state explicitly the assumption that $X_\nu^i$ is in $\C_\nu$ for $\nu=1,2$. 

\nn$*$ P.~146, Exercise 6.3. ``Let $\C$ be a small category'' should be ``Let $\C$ be a category''.

\nn$*$ P.~146, Exercise 6.8 (ii): $(\Mod(A))_M$  should be $(\Mod(R))_M$.

\nn$*$ P.~150, before Proposition 7.1.2. One could add after ``This implies that $F_{\SSS}$ is unique up to unique isomorphism'': Moreover we have $Q^\dg F\iso F_{\SSS}\iso Q^\ddg F$.

\nn$*$ P.~153, statement of Lemma 7.1.12. The readability might be improved by changing $s:X\to X'\in\SSS$ to $(s:X\to X')\in\SSS$. Same for Line 4 of the proof of Lemma 7.1.21 p.~157.

\nn$*$ P.~156, first line of the first display and first line after the first display: $\C_\SSS$ should be $\C_\SSS^r$.

\nn$*$ P.~160, second line after the diagram: ``commutative'' should be ``commutative up to isomorphism''. Line 7 when counting from the bottom to the top: $F(s)$ should be $Q_\SSS(s)$.

\nn$*$ P.~163, last sentence of Remark 7.4.5: ``right localizable'' should be ``universally right localizable''.

\nn$*$ P.~168, Line 9: ``$f:X\to Y$'' should be ``$f:Y\to X$''.

\nn$*$ P.~170, Corollary 8.2.4. The period at the end of the last display should be moved to the end of the sentence.

\nn$*$ P.~172, proof of Lemma 8.2.10, first line: ``composition morphism'' should be ``addition morphism''.

\nn$*$ P.~179, about one third of the page: ``a complex 
\begin{tikzcd}X\ar{r}{u}&Y\ar[yshift=0.7ex]{r}{v}\ar[yshift=-0.7ex]{r}[swap]{w}&Z\end{tikzcd}'' 
should be ``a sequence 
\begin{tikzcd}X\ar{r}{u}&Y\ar[yshift=0.7ex]{r}{v}\ar[yshift=-0.7ex]{r}[swap]{w}&Z\end{tikzcd}''.

\nn$*$ P.~180, Lemma 8.3.11 (b) (i): $\Coker f\xr\sim\Coker f'$ should be $\Coker f'\xr\sim\Coker f$. Proof of Lemma 8.3.11: The notation $\Hom$ for $\Hom_\C$ occurs eight times. Lemma 8.3.11 is stated below as Lemma~\ref{8311} p.~\pr{8311}. 

\nn$*$ P.~181, Lemma 8.3.13, second line of the proof: $h\ci f^2$ should be $f^2\ci h$. 

\begin{s}\lb{dg2} 
P. 184, Definitions 8.3.21 (v) and (vi). Definition 8.3.21 (vi) says that a full subcategory $\SSS$ of a category $\C$ is \emph{generating} if any object of $\C$ is the target of some epimorphism whose source is in $\SSS$. It seems to me this definition might create confusion with Definition~\ref{dg1} p.~\pr{dg1}. For want of a better idea, I suggest to say that $\C$ is \emph{a-generating}\index{a-generating} if its satisfies the above condition. (The letter a stands for the word ``abelian'', the reason being that this notion seems to be only used for abelian categories.) The notion of co-a-generating is defined in the obvious way.
\end{s} 

\nn$*$ P.~186, Corollary 8.3.26. The proof reads: ``Apply Proposition 5.2.9''. One could add: ``and Proposition 5.2.3 (v)''.

\nn$*$ P.~187, proof of Proposition 8.4.3. More generally, if $F$ is a left exact additive functor between abelian categories, then, in view of the observations made on p.~183 of the book (and especially Exercise 8.17), $F$ is exact if and only if it sends epimorphisms to epimorphisms. (A solution to the important Exercise 8.17 is given in Section~\ref{817} p.~\pr{817}.)

\nn$*$ P.~188. In the second diagram $Y'\os{l'}{\rightarrowtail}Z$ should be $Y'\os{l'}{\rightarrowtail}X$. After the second diagram: ``the set of isomorphism classes of $\DT$'' should be ``the set of isomorphism classes of objects of $\DT$''.

\nn$*$ P.~190, proof of Proposition 8.5.5 (a) (i): all the $R$ should be $R^{\op}$, except for the last one.

\nn$*$ P.~191: The equality $\psi(M)=G\otimes_RM$ is used in the second display, whereas $\psi(M)=M\otimes_RG$ is used in the third display. It might be better to use $\psi(M)=M\otimes_{R^{\op}}G$ both times. 

\nn$*$ P.~191, Proof of Theorem 8.5.8 (iii): ``the product of finite copies of $R$'' should be ``the product of finitely many copies of $R$''.

\nn$*$ P.~196, Proposition 8.6.9, last sentence of the proof of (i)$\then$(ii): ``Proposition 8.3.12'' should be ``Lemma 8.3.12''.

\nn$*$ P.~201, proof of Lemma 8.7.7, first line: ``we can construct a commutative diagram''. I think the authors meant ``we can construct an exact commutative diagram''.

\nn$*$ P.~218, middle of the page: ``$b:=\inf(J\setminus A)$'' should be ``$b:=\inf(J\setminus A')$'' (the prime is missing). 

\nn$*$ P.~218, proof of Lemma 9.2.5, first sentence: ``Proposition 3.2.4'' should be ``Proposition 3.2.2''. 

\nn$*$ P.~220, part (ii) of the proof of Proposition 9.2.9, last sentence of the first paragraph: $s(j)$ should be $\tilde s(j)$. Moreover, in the last two paragraphs of the proof, it would be better to denote $j(u)$ by $i(u)$. 

\nn$*$ P.~221, Lemma 9.2.15. ``Let $A\in\C$'' should be ``Let $A\in\Ind(\C)$''.

\nn$*$ P.~224, proof of Proposition 9.3.2, line 2: ``there exist maps $S\to A(G)\to S$ whose composition is the identity'' should be ``there exist maps $A(G)\to S$ such that the composition $S\to A(G)\to S$ is the identity of $S$''.

\nn$*$ Pp 224-228, from Proposition 9.3.2 to the end of the section. The notation $G^{\sqcup S}$, where $S$ is a set, is used twice (each time on p.~224), and the notation $G^{\coprod S}$ is used many times in the sequel of the section. I think the two pieces of notation have the same meaning. If so, it might be slightly better to uniformize the notation.

\begin{s}\lb{225}
P.~225, line 3: ``Since $N_s$ is a subobject of $A$ and $\card(A(G))<\pi$'' should be ``Since $\card(A(G))<\pi$''.

\nn$*$ P.~225, line~4: ``there exists $i_0\to i_1$ such that $N_{i_1}\to A$ is an epimorphism'' should be ``there exists $s:i_0\to i$ such that $N_s\to A$ is an epimorphism''.
\end{s}

\nn$*$ P.~226, four lines before the end: ``By 9.3.4 (c)'' should be ``By (9.3.4) (c)'' (the parenthesis are missing).

\nn$*$ P.~227. The second sentence uses Proposition~\ref{34i} p.~\pr{34i}.

\nn$*$ P.~228, line~3: $\C$ should be $\C_\pi$.

\nn$*$ P.~228, Corollary 9.3.6: $\ilim$ should be $\sigma_\pi$.

\begin{s}\lb{228}
P.~228: It might be better to state Part~(iv) of Corollary 9.3.8 as ``$G$ is in $\SSS$'', instead of ``there exists an object $G\in\SSS$ which is a generator of $\C$''. (Indeed, $G$ is already mentioned in Condition (9.3.1), which is one of the assumptions of Corollary 9.3.8.)
\end{s}

\nn$*$ P.~229, proof of 9.4.3 (i): it might be better to write ``containing $\SSS$ strictly'' (or ``properly''), instead of just ``containing $\SSS$''. 

\nn$*$ P.~229, proof of 9.4.4: ``The category $\C^X$ is nonempty, essentially small ...'': the adverb ``essentially'' is not necessary since $\C$ is supposed to be small. 

%\begin{s}\lb{964i} P.~237: Corollary 9.6.4 (i): $F:\C\to\Set$ should be $F:\C^{\op}\to\Set$.\end{s}
%\nn$*$ P.~237: Corollary 9.6.4 (i): $F:\C\to\Set$ should be $F:\C^{\op}\to\Set$.

\nn$*$ P.~237: ``Proposition 9.6.3'' should be ``Theorem 9.6.3'' (twice). 

\nn$*$ P.~237, proof of Corollary 9.6.6, first display: ``$\psi:\C\to\C$'' should be ``$\psi:\C\to\I_{inj}$''. 

\nn$*$ P.~237, end of proof of Corollary 9.6.6: it might be slightly more precise to write ``$X\to\iota(\psi(X))=K^{\Hom_\C(X,K)}$'' instead of ``$X\to\psi(X)=K^{\Hom_\C(X,K)}$''.

\nn$*$ P.~244, second diagram: the arrow from $X'$ to $Z'$ should be dotted. (For a nice picture of the octahedral diagram see p.~49 of Mili\v{c}i\'c's text

\href{http://www.math.utah.edu/~milicic/Eprints/dercat.pdf}{http://www.math.utah.edu/$\sim$milicic/Eprints/dercat.pdf}.)

\nn$*$ P.~245, beginning of the proof of Proposition 10.1.13: The letters $f$ and $g$ being used in the sequel, it would be better to write $X\xr fY\xr gZ\to TX$ instead of $X\to Y\to Z\to TX$. 

\nn$*$ P.~245, first display in the proof of Proposition 10.1.13: The subscript $\DD$ is missing (three times) in $\Hom_\DD$.

\nn$*$ P.~250, Line 1: ``TR3'' should be ``TR2''. After the second diagram, $s\ci f$ should be $f\ci s$.

\nn$*$ P.~251, right after Remark 10.2.5: ``Lemma 7.1.10'' should be ``Proposition 7.1.10''.

\nn$*$ P.~252, last five lines:

$\bu$ ``$u$ is represented by morphisms $u':\oplus_i\ X_i\xr{u'}Y'\xl sY$'' should be ``$u$ is represented by morphisms $\oplus_i\ X_i\xr{u'}Y'\xl sY$'',

$\bu$ $v'_i$ should (I believe) be $u'_i$,

$\bu$ $Q(u)$ should be $Q(u')$.

\nn$*$ P.~254. The functor $RF$ of Notation 10.3.4 coincides with the functor $R_{\N Q}F$ of Definition 7.3.1 p.~159 of the book.

\nn$*$ P.~266, Exercise 10.6. I think the authors forgot to assume that the top left square commutes.

\nn$*$ P.~278: The first display should start with $T''(s'')$ instead of $T''(s)$.

\nn$*$ P.~287, first display after Proposition~11.5.4: $v(X^{n,m})$ should be $v(X)^{n,m}$.

\begin{s}\lb{290}
\nn$*$ P.~290, Line 17: as indicated in Pierre Schapira's Errata, one should read 
$$
d^{''n,m}=\Hom_\C((-1)^{m+1}d_X^{-m-1},Y^n).
$$
\end{s}

\nn$*$ P.~290, Line -3: ``We define the functor'' should be ``We define the isomorphisms of functors''.

\nn$*$ P.~303, just after the diagram: ``the exact sequence (12.2.2) give rise'' should be ``the exact sequence (12.2.2) gives rise''.

\nn$*$ P.~320, Display (13.1.2): we have $\oo{Qis}=N^{\oo{ub}}(\C)$.

\nn$*$ P.~321, Line 8: $\widetilde\tau\,{}^{\ge n}(X)\to\widetilde\tau\,{}^{\ge n}(X)$ should be $\widetilde\tau\,{}^{\ge n}(X)\to\tau^{\ge n}(X)$.

\nn$*$ P.~328, Line 8: I think the authors meant ``$X^i\to Z^i$ is an isomorphism for $i>n+d\,$'' instead of ``$i\ge n+d\,$''.

\begin{s}\lb{1341}
P.~337, Theorem 13.4.1. ``Let $\C$ be an abelian category'' should be ``Let $\C$ be an abelian category admitting countable products'', and ``right localizable at $(Y,X)$'' should be ``universally right localizable at $(Y,X)$, and let $\oo{RHom}_\C$ denote its right localization''.
\end{s}

\nn$*$ P.~359, Line 3: $\sigma$ should be sh.

\nn$*$ P.~360, Line 5 of Step~(ii) of the proof of Theorem 14.4.5: ``Then $X''$ is an exact complex in $\oo K^-(\PP)$'' should be (I think) ``Then $X''$ is an exact complex in $\oo K^-(\C)$''.

\nn$*$ P.~364, Step~(g) of the proof of Theorem 14.4.8: $\PP_1=\oo K^-(\C_1)$ should be $\PP_1=\C_1$.

\nn$*$ P.~365, line between the last two displays: ``adjoint'' should be ``derived''.

\nn$*$ P.~392, Lemma 16.1.6 (ii). It would be better to write $v:C\to U$ instead of $u:C\to U$ and $t\ci v$ instead of $t\ci u$.

\nn$*$ P.~396, proof of Lemma 16.2.4 (ii), last sentence of the proof: It would be better (I think) write ``by LE2 and LE3'' instead of ``by Proposition 16.1.11 (ii)''.

\nn$*$ P.~401, Line 6: $B''\to B$ should be $B''\to B'$.

\nn$*$ P.~406, first line of the second display: $(\C_Y)^\wg$ should be $\C_Y$ (twice). (See \S\ref{fhat} p.~\pr{fhat}.)

\nn$*$ P.~409, line 2: $\ld\ci(\oo h_X^t)_A\iso\oo h_A$ should be $\ld\ci(\oo h_X^t)_A\iso\oo h_A^t$. 

\begin{s}\lb{17115typo}
P.~410, Display (17.1.15): instead of 
$$
\Hom_{\PSh(X,\A)}(F,G)\iso\lim_{U\in\C_X}\HOM_{\PSh(U,\A)}(F,G)(U).
$$
we should have
$$
\Hom_{\PSh(X,\A)}(F,G)\iso\lim_{U\in\C_X}\HOM_{\PSh(X,\A)}(F,G)(U).
$$ 
See \S\ref{17115ps} p.~\pr{17115ps}. 
\end{s}
See \S
%

\begin{s}\lb{1722}
P.~412, proof of Lemma 17.2.2 (ii), (b)$\then$(a), Step~(3). ``Since $\fthat(u_V)$ is an epimorphism by (2), $\fthat(u_V)$ is a local isomorphism'' should be ``Since $\fthat(u_V)$ is a local epimorphism by (2), $\fthat(u_V)$ is a local isomorphism''.
\end{s}

\nn$*$ P.~414, line before the last display: $h_X^\ddg F$ should be $\oo h_X^\ddg F$, \emph{i.e.} the h should be straight, not slanted. 

\nn$*$ P.~417, first sentence of the paragraph containing Display (17.4.2): $A,A'\in\C^\wg$ should be $A,A'\in\C_X^\wg$. 

\nn$*$ P.~418, last display: 
$$
\ilim:\ilim_{(B\to A)\in\mc{LI}_A}F(B)\to\ilim_{(B\to A)\in\mc{LI}_A}F^b(B)
$$ 
should be 
$$
\ilim_{(B\to A)\in\mc{LI}_A}:\ilim_{(B\to A)\in\mc{LI}_A}F(B)\to\ilim_{(B\to A)\in\mc{LI}_A}F^b(B).
$$

\nn$*$ P.~419, second line: ``applying Corollary 2.3.4 to $\theta=\id_{\mc{LI}_A}$'' should be ``applying Corollary 2.3.4 to $\pp=\id_{\mc{LI}_A}$''.

\nn$*$ P.~421, Theorem 17.4.7 (i): $(h_X^\ddg F)^b\iso(h_X^\ddg F^a)$ should be $(\oo h_X^\ddg F)^b\iso(\oo h_X^\ddg F^a)$, \emph{i.e.} the h's should be straight, not slanted.

\nn$*$ P.~424, proof of Theorem 17.5.2 (iv). ``The functor $f^\dg$ is left exact'' should be ``The functor $f^\dg$ is exact''. (See \S\ref{fdagger} p.~\pr{fdagger}.) 

\nn$*$ P.~426, Line 5: ``morphism of sites by'' should be ``morphism of sites''.

\nn$*$ P.~428, Notation 17.6.13 (i). ``For $M\in\A$, let us denote by $M_A$ the sheaf associated with the constant presheaf $\C_X\ni U\mt M$'' should be  

``For $M\in\A$, let us denote by $M_A$ the sheaf over $\C_A$ associated with the constant presheaf $\C_A\ni(U\to A)\mt M$''. 

It might also be worth mentioning that $M_A$ is called the \emph{constant sheaf over $A$ with stalk} $M$\index{constant sheaf}. 

\nn$*$ P.~437, Line 3 of Step~(ii) of the proof of Lemma 18.1.5: It might be better to write $``\bigoplus_{s\in A(U)}G(U\xr sA)"$ instead of $``\coprod_{s\in A(U)}G(U\xr sA)"$; indeed $\bigoplus$ is more usual that $\coprod$ to denote the coproduct of $k$-modules. 

\nn$*$ P.~438, right after ``q.e.d.'': ``Notations (17.6.13)'' should be ``Notations 17.6.13'' (no parenthesis). 

\nn$*$ P.~438, bottom: One can add that we have $\HOM_\R(\R,F)\iso F$ for all $F$ in $\PSh(\R)$. 

\nn$*$ P.~439, after Definition 18.2.2: One can add that we have $F\os{\text{\tiny psh}}{\otimes}_\R\R\iso F$ for $F$ in $\PSh(\R)$ and $F\otimes_\R\R\iso F$ for $F$ in $\Mod(\R)$. 

\nn$*$ P.~439, Proposition 18.2.3 (ii). Here is a slightly stronger statement: If $\mc{R,S,T}$ are $k_X$-algebras, if $F$ is a $(\T\otimes_{k_X}\R^{\op})$-module, if $G$ is an $(\R\otimes_{k_X}\SSS)$-module, and if $H$ is an $(\SSS\otimes_{k_X}\T)$-module, then there are isomorphisms 
$$
\Hom_{\SSS\otimes_{k_X}\T}(F\otimes_\R G,H)\iso
\Hom_{\R\otimes_{k_X}\SSS}(G,\HOM_\T(F,H)), 
$$ 
$$
\HOM_{\SSS\otimes_{k_X}\T}(F\otimes_\R G,H)\iso
\HOM_{\R\otimes_{k_X}\SSS}(G,\HOM_\T(F,H)), 
$$ 
functorial with respect to $F,G$ and $H$.

\nn$*$ P.~440, last line of second display: $\Hom_{\R(U)}(G(U)\otimes_kF(U),H(U))$ should be $\Hom_k(F(U)\otimes_{\R(U)}G(U),H(U))$. 

\nn$*$ P.~440, first line of the fourth display, $\os{\text{\tiny psh}}{\otimes}_{\R(V)}$ should be $\otimes_{\R(V)}$. 

\nn$*$ P.~441. The proof of Proposition 18.2.5 uses Display (17.1.11) p.~409 of the book and Exercise 17.5 (i) p.~431 of the book (see \S\ref{175i} p.~\pr{175i}).  

\nn$*$ P.~442, first line of Step~(ii) of the proof of Proposition 18.2.7: $\HOM_\R(\R\otimes k_{XA},F)$ should be $\HOM_\R(\R\otimes_{k_X}k_{XA},F)$. 

\nn$*$ P.~442, Line 3 of last display of Section 18.2: $\jj_{A\to X!}\jj_{A\to X}^{-1}$ should be $\jj_{A\to X}^\ddg\jj_{A\to X*}$. 

\nn$*$ P.~442. Lemma 18.3.1 (i) follows from Proposition 17.5.1 p.~432 of the book. 

\nn$*$ P.~443, first display: On the third and fourth lines, $\HOM_{k_X}$ should be $\HOM_{k_Z}$. 

\nn$*$ P.~443, sentence preceding Lemma 18.3.2: $j_{A\to X}$ should be $\oo j_{A\to X}$ (the slanted j should be straight). 

\nn$*$ Pp 447-8, proof of Lemma 18.5.3: in (18.5.3) $M'|_U$ and $M|_U$ should be $M'(U)$ and $M(U)$, and, after the second display on p.~448, $s_1\in((\R^{\op})^{\oplus m}\otimes_\R P)(U)$ should be $s_1\in((\R^{\op})^{\oplus n}\otimes_\R P)(U)$.

\nn$*$ P.~448, Proposition 18.5.4, Line 3 of the proof: $G^{\oplus I}\epi M$ should be $\G^{\oplus I}\epi M$.

\nn$*$ P.~452, Part (i) (a) of the proof of Lemma 18.6.7. I think that $\OO_U$ and $\OO_V$ stand for $\OO_X|_U$ and $\OO_Y|_V$. (If this is so, it would be better, in the penultimate display of the page, to write $\OO_V$ instead of $\OO_Y|_V$.) 

\nn$*$ P.~452, a few lines before the penultimate display of the page, $f_W^{-1}:\OO_U^{\oplus n}\xr u\OO_U^{\oplus m}$ should be (I think) $f_W^{-1}:\OO_W^{\oplus n}\to\OO_W^{\oplus m}$.

\nn$*$ P.~494, Index. I found useful to add the following subentries to the entry ``injective'': $\F$-injective, 231; $F$-injective, 253, 255, 330.

%%%

\section{About Chapter 1}

\sbs{Universes (p.~9)}

The book starts with a few statements which are not proved, a reference being given instead. Here are the proofs.

A \textbf{universe}\index{universe} is a set $\U$ satisfying 

(i) $\vi\in\U$,

(ii) $u\in U\in\U\then u\in\U$,

(iii) $U\in\U\then\{U\}\in\U$,

(iv) $U\in\U\then\PP(U)\in\U$,

(v) $I\in\U$ and $U_i\in\U$ for all $i$ $\then$ $\bigcup_{i\in I}U_i\in\U$,

(vi) $\mathbb N\in\U$.

\nn We want to prove:

(vii) $U\in\U\then\bigcup_{u\in U}u\in\U$,

(viii) $U,V\in\U\then U\tm V\in\U$,

(ix) $U\subset V\in\U\then U\in\U$,

(x) $I\in\U$ and $U_i\in\U$ for all $i$ $\then$ $\prod_{i\in I}U_i\in\U$.

\nn(We have kept Kashiwara and Schapira's numbering of Conditions (i) to (x).) 

\nn Obviously, (ii) and (v) imply (vii), whereas (iv) and (ii) imply (ix). Axioms (iii), (vi) and (v) imply

(a) $U,V\in\U\then\{U,V\}\in\U$,

\nn and thus

(b) $U,V\in\U\then(U,V):=\{\{U\},\{U,V\}\}\in\U$.

\nn\textbf{Proof of (viii).} If $u\in U$ and $v\in V$, then $\{(u,v)\}\in\U$ by (ii), (b) and (iii). Now (v) yields 
$$
U\tm V=\bigcup_{u\in U}\ \bigcup_{v\in V}\ \{(u,v)\}\in\U.\text{ q.e.d.} 
$$ 

Assume $U,V\in\U$, and let $V^U$ be the set of all maps from $U$ to $V$. As $V^U\in\PP(U\tm V)$, Statements (viii), (iv) and (ii) give

(c) $U,V\in\U\then V^U\in\U$.

\nn\textbf{Proof of (x).} As 
$$
\prod_{i\in I}\ U_i\in\PP\left(\left(\bigcup_{i\in I}U_i\right)^I\right),
$$
(x) follows from (v), (c) and (iv). q.e.d.

%%

\sbs{Brief comments}

\begin{s}\lb{d125}
P.~14, category of morphisms\index{category of morphisms}. Here are some comments about Definition 1.2.5 p.~14:

\begin{nota}\lb{c*}
For any category $\C$ define the category $\C^*$\index{$\C^*$} as follows. The objects of $\C^*$ are the objects of $\C$, the set $\Hom_{\C^*}(X,Y)$ is defined by 
$$
\Hom_{\C^*}(X,Y):=\{Y\}\tm\Hom_{\C}(X,Y)\tm\{X\},
$$
and the composition is defined by 
$$
(Z,g,Y)\ci(Y,f,X):=(Z,g\ci f,X).
$$ 
\end{nota}

Note that there are natural mutually inverse isomorphisms $\C\rightleftarrows\C^*$. 

\begin{nota}\lb{mor}
%
Let $\C$ be a category. Define the category $\Mor(\C)$ \index{$\Mor$} by 
$$
\Ob(\Mor(\C)):=\bigcup_{X,Y\in\Ob(\C)}\Hom_{\C^*}(X,Y),
$$
$\ds\Hom_{\Mor(\C)}((Y,f,X),(V,g,U)):=$\bigskip 

$\hfill\ds\{(a,b)\in\Hom_\C(X,U)\tm\Hom_\C(Y,V)\ | \ g\ci a=b\ci f\},$\bigskip

\nn{\em i.e.} 
$$
\begin{tikzcd}
X\ar{d}[swap]{f}\ar{r}{a}&U\ar{d}{g}\\ 
Y\ar{r}[swap]{b}&V,
\end{tikzcd}
$$ 
and the composition is defined in the obvious way.
\end{nota}

Observe that a functor $\A\to\B$ is given by two maps 
$$
\Ob(\A)\to\Ob(\B),\quad\Ob(\Mor(\A))\to\Ob(\Mor(\B))
$$ 
satisfying certain conditions.

When $\C$ is a small category (see Section~\ref{ucat} p. \pr{ucat}), we assume that the hom-sets of $\C$ are disjoint.
\end{s}

%

\begin{s}\lb{ffc}
P.~16, Definition 1.2.11 (iii). Note that fully faithful functors are conservative\index{conservative}. 
\end{s}

%

\begin{s}
P.~18, Definition 1.2.16. If $F:\C\to\C'$ is a functor and $X'$ an object of $\C'$, then we have natural isomorphisms 
\begin{equation}\lb{opslice}
(\C_{X'})^{\op}\iso(\C^{\op})^{X'},\quad(\C^{X'})^{\op}\iso(\C^{\op})_{X'}.
\end{equation} 
Also note that, if $\Cat$ is the category of small categories, then the formula $X'\mt\C_{X'}$ defines a functor $\C'\to\Cat$, and the formula $X'\mt\C^{X'}$ defines a functor $\C'^{\op}\to\Cat$.
\end{s}

%%

\sbs{Horizontal\index{horizontal composition} and vertical compositions\index{vertical composition} (p. 19)}%\lb{hove1}

For each object $X$ of $\C_3$ the diagram 

$$
\begin{tikzcd}
\C_1&{}&\C_2\ar{ll}{}[near start,swap]{F_{11}}&{}&\C_3\ar{ll}{}[near start,swap]{F_{12}}\\ 
\C_1&\ar{u}{\theta_{11}}&\C_2\ar{ll}{}[near start,swap]{F_{21}}&\ar{u}{\theta_{12}}&\C_3\ar{ll}{}[near start,swap]{F_{22}}\\ 
\C_1&\ar{u}{\theta_{21}}&\C_2\ar{ll}{}[near start]{F_{31}}&\ar{u}{\theta_{22}}&\C_3\ar{ll}{}[near start]{F_{32}}
\end{tikzcd}
$$ 

\nn of categories, functors and morphisms of functors yields the commutative diagram 

$$
\begin{tikzcd}
F_{11}F_{12}X&&F_{21}F_{12}X\ar{ll}[swap]{\theta_{11}F_{12}X}&&F_{31}F_{12}X\ar{ll}[swap]{\theta_{21}F_{12}X}\\ 
F_{11}F_{22}X\ar{u}{F_{11}\theta_{12}X}&&F_{21}F_{22}X\ar{ll}[swap]{\theta_{11}F_{22}X}\ar{u}{F_{21}\theta_{12}X}&&F_{31}F_{22}X\ar{ll}[swap]{\theta_{21}F_{22}X}\ar{u}[swap]{F_{31}\theta_{12}X}\\ 
F_{11}F_{32}X\ar{u}{F_{11}\theta_{22}X}&&F_{21}F_{32}X\ar{ll}{\theta_{11}F_{32}X}\ar{u}{F_{21}\theta_{22}X}&&F_{31}F_{32}X\ar{ll}{\theta_{21}F_{32}X}\ar{u}[swap]{F_{31}\theta_{22}X}
\end{tikzcd}
$$ 

\nn in $\C_1$. So, we get a well-defined morphism in $\C_1$ from $F_{31}F_{32}X$ to $F_{11}F_{12}X$, which is easily seen to define a morphism of functors from $F_{31}F_{32}$ to $F_{11}F_{12}$. 

\begin{nota}\lb{nhove}
We denote this morphism of functors by
$$
\begin{pmatrix}
\theta_{11}&\theta_{12}\\ 
\theta_{21}&\theta_{22}
\end{pmatrix}:F_{31}F_{32}\to F_{11}F_{12}.
$$ 
If $\theta_{21}$ and $\theta_{22}$ are identity morphisms, we put 
$$
\theta_{11}\star\theta_{12}:=
\begin{pmatrix}
\theta_{11}&\theta_{12}\\ 
\theta_{21}&\theta_{22}
\end{pmatrix}.
$$ 
If $\theta_{12}$ and $\theta_{22}$ are identity morphisms, we put 
$$
\theta_{11}\ci\theta_{21}:=
\begin{pmatrix}
\theta_{11}&\theta_{12}\\ 
\theta_{21}&\theta_{22}
\end{pmatrix}.
$$ 
\end{nota}

Let $m,n\ge1$ be integers, let $\C_1,\dots,\C_{n+1}$ be categories, let 
$$
F_{i,j}:\C_{j+1}\to\C_j,\quad1\le i\le m+1,\ 1\le j\le n
$$
be functors, let 
$$
\theta_{i,j}:F_{i+1,j}\to F_{i,j},\quad1\le i\le m,\ 1\le j\le n
$$
be morphisms of functors. For instance, if $m=2,n=4$, then we have 
$$
\begin{tikzcd}
\C_1&{}&\C_2\ar{ll}{}[near start,swap]{F_{11}}&{}&\C_3\ar{ll}{}[near start,swap]{F_{12}}&{}&\C_4\ar{ll}{}[near start,swap]{F_{13}}&{}&\C_5\ar{ll}{}[near start,swap]{F_{14}}\\ 
\C_1&\ar{u}{\theta_{11}}&\C_2\ar{ll}{}[near start,swap]{F_{21}}&\ar{u}{\theta_{12}}&\C_3\ar{ll}{}[near start,swap]{F_{22}}&\ar{u}{\theta_{13}}&\C_4\ar{ll}{}[near start,swap]{F_{23}}&\ar{u}{\theta_{14}}&\C_5\ar{ll}{}[near start,swap]{F_{24}}\\ 
\C_1&\ar{u}{\theta_{21}}&\C_2\ar{ll}{}[near start,swap]{F_{31}}&\ar{u}{\theta_{22}}&\C_3\ar{ll}{}[near start,swap]{F_{32}}&\ar{u}{\theta_{23}}&\C_4\ar{ll}{}[near start,swap]{F_{33}}&\ar{u}{\theta_{24}}&\C_5\ar{ll}{}[near start,swap]{F_{34}}.
\end{tikzcd}
$$ 

The following proposition is clear 

\begin{prop}
The operations $\star$\index{$\star$} and $\ci$ are associative, and, in the above setting, we have the equality 

$$
(\theta_{1,1}\star\cdots\star\theta_{1,n})\ci\cdots\ci(\theta_{m,1}\star\cdots\star\theta_{m,n})
$$ 

$$
=(\theta_{1,1}\ci\cdots\ci\theta_{m,1})\star\cdots\star(\theta_{1,n}\ci\cdots\ci\theta_{m,n}).
$$

\nn between functors from $F_{m+1,1}\cdots F_{m+1,n}$ to $F_{1,1}\cdots F_{1,n}$.
\end{prop}

\begin{nota}\lb{nmat} 
We denote this morphism of functors by
$$
\begin{pmatrix}
\theta_{1,1}&\cdots&\theta_{1,n}\\
\vdots&&\vdots\\ 
\theta_{m,1}&\cdots&\theta_{m,n}
\end{pmatrix}:F_{m+1,1}\cdots F_{m+1,n}\to F_{1,1}\cdots F_{1,n}.
$$ 
\end{nota}

\begin{prop}\lb{pil1}
We have, in the above setting,

$$
(\theta_{1,1}\star\cdots\star\theta_{1,n})\ci\cdots\ci(\theta_{m,1}\star\cdots\star\theta_{m,n})
=\begin{pmatrix}
\theta_{1,1}\star\cdots\star\theta_{1,n}\\
\vdots\\ 
\theta_{m,1}\star\cdots\star\theta_{m,n}
\end{pmatrix}
$$ 

$$
=\begin{pmatrix}
\theta_{1,1}&\cdots&\theta_{1,n}\\
\vdots&&\vdots\\ 
\theta_{m,1}&\cdots&\theta_{m,n}
\end{pmatrix}
$$

$$
=\begin{pmatrix}\theta_{1,1}\\ \vdots\\ \theta_{m,1}\end{pmatrix}\star\cdots\star
\begin{pmatrix}\theta_{1,n}\\ \vdots\\ \theta_{m,n}\end{pmatrix}
=(\theta_{1,1}\ci\cdots\ci\theta_{m,1})\star\cdots\star(\theta_{1,n}\ci\cdots\ci\theta_{m,n}).
$$
\end{prop}

\begin{df}[horizontal and vertical composition, Interchange Law]\lb{dil1} 
We call $\star$\index{$\star$} the {\em horizontal composition}.\index{horizontal composition} We call $\ci$ the {\em vertical composition}.\index{vertical composition} We call the equalities in Proposition~\ref{pil1} the {\em Interchange Law}.\index{Interchange Law}
\end{df}

%%

\sbs{The Yoneda Lemma (p. 24)\index{Yoneda Lemma}}

% previous version
% https://docs.google.com/document/d/1unQaNcPupHgKC1_DCOaFZoJjyQ98zsslv0YVHw6dvaQ/edit

We state the Yoneda Lemma for the sake of completeness:

\begin{thm}[Yoneda's Lemma]\lb{yol}
Let $\C$ be a category.

\nn\emph{(a)} Let $h:\C\to\C^\wg$ be the Yoneda embedding, let $A$ be in $\C^\wg$, let $X$ be in $\C$, and define 
\begin{equation}\lb{axp1}
\begin{tikzcd} 
A(X)\ar[yshift=0.7ex]{r}{\pp}&\Hom_{\C^\wg}(h(X),A)\ar[yshift=-0.7ex]{l}{\psi}
\end{tikzcd}
\end{equation} 
by 
\begin{equation}\lb{yo}
\pp(x)_Y(f):=A(f)(x),\quad\psi(\theta):=\theta_X(\id_X)
\end{equation}
for 
$$
x\in A(X),\quad Y\in\C,\quad f\in\Hom_\C(Y,X),\quad\theta\in\Hom_{\C^\wg}(h(X),A):
$$ 
$$
f\in\Hom_\C(Y,X)\xr{\pp(x)_Y}A(Y)\xl{A(f)}A(X)\ni x.
$$
Then $\pp$ and $\psi$ are mutually inverse bijections. In the particular case where $A$ is equal to $h(Z)$ for some $Z$ in $\C$, we get 
$$
\pp(x)=h(x)\in\Hom_{\C^\wg}(h(X),h(Z)).
$$
This shows that $h$ is fully faithful.

\nn\emph{(b)} Let $k:\C\to\C^\vee$ be the Yoneda embedding, let $A$ be in $\C^\vee$, let $X$ be in $\C$, and define 
\begin{equation}\lb{axp2}
\begin{tikzcd} 
A(X)\ar[yshift=0.7ex]{r}{\pp}&\Hom_{\C^\vee}(A,k(X))=\Hom_{\Set^\C}(k(X),A)\ar[yshift=-0.7ex]{l}{\psi}
\end{tikzcd}
\end{equation}
by \qr{yo} for 
$$
x\in A(X),\quad Y\in\C,\quad f\in\Hom_\C(X,Y),\quad\theta\in\Hom_{\Set^\C}(k(X),A):
$$ 
$$
f\in\Hom_\C(X,Y)\xr{\pp(x)_Y}A(Y)\xl{A(f)}A(X)\ni x.
$$
Then $\pp$ and $\psi$ are mutually inverse bijections. In the particular case where $A$ is equal to $k(Z)$ for some $Z$ in $\C$, we get 
$$
\pp(x)=k(x)\in\Hom_{\C^\vee}(k(Z),k(X)).
$$
This shows that $k$ is fully faithful.

\nn\emph{(c)} The bijections \qr{axp1} and \qr{axp2} are functorial in $A$ and $X$.
\end{thm}

\begin{proof}
(a) We have 
$$
\psi(\pp(x))=\pp(x)_X(\id_X)=A(\id_X)(x)=x
$$ 
and
$$
\pp(\psi(\theta))_Y(f)=A(f)(\psi(\theta))=A(f)(\theta_X(\id_X))=\theta_Y(f),
$$ 
the last equality following from the commutativity of the square 
$$
\begin{tikzcd}
h(X)(Y)\ar{r}{\theta_Y}&A(Y)\\ 
h(X)(X)\ar{u}{h(X)(f)}\ar{r}[swap]{\theta_X}&A(X),\ar{u}[swap]{A(f)}
\end{tikzcd}
$$ 
which is equal to the square 
$$
\begin{tikzcd}
\Hom_\C(Y,X)\ar{r}{\theta_Y}&A(Y)\\ 
\Hom_\C(X,X)\ar{u}{\ci f}\ar{r}[swap]{\theta_X}&A(X).\ar{u}[swap]{A(f)}
\end{tikzcd}
$$ 

\nn(b) The proof of (b) is similar. 

\nn(c) Let $h:\C\to\C^\wg$ be the Yoneda embedding, and, for $X$ in $\C$ and $A$ in $\C^\wg$ let 
$$
\Phi_{X,A}:\Hom_{\C^\wg}(h(X),A)\to A(X),\quad\theta\mt\theta_X(\id_X)
$$ 
be the Yoneda bijection. We shall prove that $\Phi_{X,A}$ is functorial in $X$ and $A$. 

Functoriality in $A$: Let $h(X)\xr\ld A\xr\theta B$ be morphisms of functors. We must show $\ld_X(\Phi_{X,A}(\theta))=\Phi_{X,B}((\ld\ci)(\theta))$: 
$$
\begin{tikzcd}
\Hom_{\C^\wg}(h(X),A)\ar[d,"\ld\ci"']\ar[r,"\Phi_{X,A}"]&A(X)\ar[d,"\ld_X"]\\ 
\Hom_{\C^\wg}(h(X),B)\ar[r,"\Phi_{X,B}"']&B(X).
\end{tikzcd}
$$ 
We have 
$$
\ld_X(\Phi_{X,A}(\theta))=\ld_X(\theta_X(\id_X)),
$$ 
$$
\Phi_{X,B}((\ld\ci)(\theta))=\Phi_{X,B}(\ld\ci\theta)=(\ld\ci\theta)_X(\id_X)=(\ld_X\ci\theta_X)(\id_X)=\ld_X(\theta_X(\id_X)), 
$$ 
where the penultimate equality follows from the definition of vertical composition of morphisms of functors (Definition~\ref{dil1} p.~\pr{dil1}): 
$$ 
\Hom_\C(X,X)\xr{\theta_X}A(X)\xr{\ld_X}B(X).
$$ 

Functoriality in $A$: Let $f:X\to Y$ be a morphism in $\C$ and $\theta:h(Y)\to A$ be a morphism in $\C^\wg$. We must show 
$$
\Phi_{X,A}\Big(\big(\ci h(f)\big)(\theta)\Big)=A(f)\big(\Phi_{Y,A}(\theta)\big):
$$ 
$$
\begin{tikzcd}
\Hom_{\C^\wg}(h(X),A)\ar[r,"\Phi_{X,A}"]&A(X)\\ 
\Hom_{\C^\wg}(h(Y),B)\ar[u,"\ci h(f)"]\ar[r,"\Phi_{X,B}"']&A(Y)\ar[u,"A(f)"'].
\end{tikzcd}
$$ 
We have 
$$
\Phi_{X,A}\Big(\big(\ci h(f)\big)(\theta)\Big)=\Phi_{X,A}(\theta\ci h(f))=(\theta_X\ci h(f)_X)(\id_X)
$$ 
$$
=(\theta_X\ci(f\ci))(\id_X)=\theta_X(f),
$$ 
where the second equality follows from the definition of vertical composition of morphisms of functors: 
$$
h(X)\xr{h(f)}h(Y)\xr\theta A,
$$ 
$$
\Hom_\C(X,X)\xr{f\ci}\Hom_\C(X,Y)\xr{\theta_X}A(X)
$$ 
because $h(f)_X=f\ci$. We also have 
$$
A(f)\big(\Phi_{Y,A}(\theta)\big)=A(f)\big(\theta_Y(\id_Y)\big)=\theta_X(f),
$$ 
where the last equality follows from the naturality of $\theta$: 
$$
\begin{tikzcd}
\Hom_\C(X,Y)\ar[r,"\theta_X"]&A(X)\\ 
\Hom_\C(Y,Y)\ar[u,"\ci f"]\ar[r,"\theta_Y"']&A(Y)\ar[u,"A(f)"'].
\end{tikzcd}
$$ 
\end{proof}

\begin{conv}\lb{term} 
An object $Y$ in a category $\A$ is \emph{terminal} if all $X$ in $\A$ admits a unique morphism $X\to Y$. Let $T_\A$ be the set of terminal objects of $\A$. If $Y,Z\in T_\A$, then there is a unique morphism $Y\to Z$, and this morphism is an isomorphism. For all category $\A$ such that $T_\A\neq\vi$ we choose an element in $T_\A$ and call it \textbf{the} terminal object of $\A$. Let us insist: we make a distinction between ``\emph{a} terminal object of $\A$'' and ``\emph{the} terminal object of $\A$'' (when they exist). Unless otherwise indicated, the choice of \emph{the} terminal object of $\A$ is random (but there will be two exceptions to this rule: see Convention~\ref{thep} p.~\pr{thep} and Convention~\ref{thei} p.~\pr{thei}).
\end{conv} 

\begin{conv}\lb{yoc}
We often identify the source and the target of $\pp$ in \qr{axp1} and \qr{axp2}, and we also often consider $\C$ as a full subcategory of $\C^\wg$ and $\C^\vee$ thanks to the Yoneda embeddings. Let $A$ be in $\C^\wg$ and $(X,x)$, with $X$ in $\C$ and $x:X\to A$ a morphism in $\C^\wg$, an object in the category $\C_A$ (see Definition 1.2.16 p.~18 of the book). Then the Yoneda Lemma (Theorem~\ref{yol} p.~\pr{yol}) implies that $(X,x)$ is terminal if and only if $x$ is an isomorphism. If the category $\C_A$ admits a terminal object, we say that $A$ is \emph{representable}\index{representable}. Let $(X,x)$ be a (resp. the) terminal object of $\C_A$. We say that the couple $(X,x)$, or sometimes just the morphism $x$, is a (resp. the) \emph{representation}\index{representation} of $A$, and that $X$ is a (resp. the) \emph{representative}\index{representative} of $A$. We use a similar terminology if $A$ in is in $\C^\vee$ instead of $\C^\wg$, replacing the words \emph{representable}, \emph{representative}, \emph{representation} with \emph{co-representable}, \emph{co-representative}, \emph{co-representation}. 
\end{conv}

A morphism $x:X\to A$ in $\C^\wg$ with $X$ in $\C$ is a representation of $A$ if and only if any morphism $Y\to A$ with $Y$ in $\C$ factors uniquely through $x$:
$$
\begin{tikzcd}
Y\ar[rr]\ar[dr,dashed]&&A\\ 
&X.\ar[ru,"x"']
\end{tikzcd}
$$ 
A morphism $x:A\to X$ in $\C^\vee$ with $X$ in $\C$ is a co-representation of $A$ if and only if any morphism $A\to Y$ with $Y$ in $\C$ factors uniquely through $x$:
$$
\begin{tikzcd}
A\ar[rr]\ar[dr,"x"']&&Y\\ 
&X.\ar[ru,dashed]
\end{tikzcd}
$$ 

%%

\sbs{Brief comment}

P.~25, Corollary 1.4.7. A statement slightly stronger than Corollary 1.4.7 of the book can be proved more naively:

\begin{prop}\lb{yp}
A morphism $f:A\to B$ in a category $\C$ is an isomorphism if and only if 
$$
\Hom_\C(X,f):\Hom_\C(X,A)\to\Hom_\C(X,B)
$$
is (i) surjective for $X=B$ and (ii) injective for $X=A$.
\end{prop}

\begin{proof} By (i) there is a $g:B\to A$ satisfying $f\ci g=\id_B$, yielding $f\ci g\ci f=f$, and (ii) implies $g\ci f=\id_A$.
\end{proof}

%%

\sbs{Partially defined adjoints\index{partially defined adjoint} (Section 1.5, p.~28)}\lb{defat}

%The following observations about the notion of adjoint functor are implicit in the book.

\begin{s}\lb{rdefat}
Let $L:\C\to\C'$ be a functor and $X'$ an object of $\C'$. If the functor 
$$
\Hom_{\C'}(L(\ ),X'):\C^{\op}\to\Set
$$ 
is representable, we denote its representative by $R(X')$ and its representation by $\eta_{X'}:L(R(X'))\to X'$ (see Convention~\ref{yoc} p.~\pr{yoc}), and we say that 

\nn``the value of the right adjoint $R$ to $L$ at $X'$ is defined and isomorphic to $R(X')$'', 

\nn or, abusing the terminology, that 

\nn``$R(X')$ exists''.  

Concretely this means that, for all $X$ in $\C$ and all $g:L(X)\to X'$ there is a unique $f:X\to R(X')$ such that $\eta_{X'}\ci L(f)=g$: 
\begin{equation}\lb{xfl1}
\begin{tikzcd}
X\ar[d,dashed,"f"']&L(X)\ar[d,"L(f)"']\ar[r,"g"]&X'\\ 
R(X')&L(R(X')).\ar[ru,"\eta_{X'}"']
\end{tikzcd}
\end{equation} 
We call $\eta_{X'}$ the \emph{unit}\index{unit} of the adjunction.
\end{s}

%

\begin{s} 
In the above setting consider the commutative diagram 
\begin{equation}\lb{xfl2}
\begin{tikzcd}
X\ar[d,dashed,"f"']\ar[r,"\theta"]&\Hom_{\C'}(L(\ ),X')\\ 
R(X').\ar[ru,"\eta_{X'}"']
\end{tikzcd}
\end{equation}

Even if it is straightforward, we state and prove formally the fact that the conditions summarized by \qr{xfl1} and \qr{xfl2} are equivalent. We prefer to rewrite \qr{xfl1} as 
\begin{equation}\lb{xfl1b}
\begin{tikzcd}
X\ar[d,dashed,"f"']&L(X)\ar[d,"L(f)"']\ar[r,"\theta_X(\id_X)"]&X'\\ 
R(X')&L(R(X')).\ar[ru,"\eta_{X',R(X')}(\id_{R(X')})"']
\end{tikzcd}
\end{equation} 

\begin{lem}
If $L:\C\to\C'$ is a functor, if $X$ and $R(X')$ are objects of $\C$, if $f:X\to R(X')$ is a morphism in $\C$, and if 
$$
\eta_{X'}:R(X')\to\Hom_{\C'}(L(\ ),X')\quad\text{and}\quad\theta:X\to\Hom_{\C'}(L(\ ),X')
$$ 
are morphisms of functors, then we have 
$$
\eta_{X'}\ci f=\theta\iff\eta_{X',R(X')}(\id_{R(X')})\ci L(f)=\theta_X(\id_X)
$$ 
(see \qr{xfl2} and \qr{xfl1b}).
\end{lem} 

\begin{proof} 
The equalities 
$$
(\eta_{X'}\ci f)_X(\id_X)=\eta_{X',X}(f)=\eta_{X',R(X')}(\id_{R(X')})\ci L(f)
$$ 
are respectively justified by the definition of the vertical composition of morphisms of functors and by the naturality of $\eta_{X'}$. As the Yoneda Lemma (Theorem~\ref{yol} p.~\pr{yol}) implies 
$$
\eta_{X'}\ci f=\theta\iff(\eta_{X'}\ci f)_X(\id_X)=\theta_X(\id_X),
$$ 
the lemma is proved. 
\end{proof} 
\end{s}

%

\begin{s}
Let $T$ be a terminal object of $\Set$. Then a functor $A:\C^{\op}\to\Set$ is representable if and only if the right adjoint of $A^{\op}:\C\to\Set^{\op}$ is defined at $T$. 

Indeed we have 
$$
\Hom_{\Set^{\op}}(A^{\op}(\ ),T)\iso\Hom_\Set(T,A(\ ))\iso A.
$$

\end{s}

\begin{s}\lb{ldefat}
Let us spell out the statement dual to \S\ref{rdefat}:

Let $R:\C'\to\C$ be a functor and $X$ an object of $\C$. If the functor 
$$
\Hom_{\C'}(X,R(\ )):\C\to\Set
$$ 
is co-representable, we denote its co-representative by $L(X)$ and its co-representa\-tion by $\ee_X:X\to R(L(X))$ (see Convention~\ref{yoc} p.~\pr{yoc}), and we say that 

\nn``the value of the left adjoint $L$ to $R$ at $X$ is defined and isomorphic to $L(X)$'', 

\nn or, abusing the terminology, that 

\nn``$L(X)$ exists''.  

\nn Concretely this means that, for all $X'$ in $\C'$ and all $g:X\to R(X')$ there is a unique $f:L(X)\to X'$ such that $R(f)\ci\ee_X=g$: 
$$
\begin{tikzcd}
X\ar[r,"\ee_X"]\ar[dr,"g"']&R(L(X))\ar[d,"R(f)"]&L(X)\ar[d,dashed,"f"]\\ 
&R(X')&X'.
\end{tikzcd}
$$ 
We call $\ee_X$ the \emph{co-unit}\index{co-unit} of the adjunction.
\end{s}

%%

\sbs{Commutativity of Diagram (1.5.6) p.~28}

Let us prove the commutativity of the diagram (1.5.6) p.~28 of the book. Recall the setting: We have a pair $(L,R)$ of adjoint functors: 
$$
\begin{tikzcd}
\C\ar[xshift=-.7ex]{d}[swap]{L}\\ 
\C'.\ar[xshift=.7ex]{u}[swap]{R}
\end{tikzcd}
$$ 
Let us denote the functorial bijection defining the adjunction by 
$$
\ld_{X,X'}:\Hom_\C(X,RX')\to\Hom_{\C'}(LX,X')
$$ 
for $X$ in $\C$ and $X'$ in $\C'$. The diagram (1.5.6) can be written as 
\begin{equation}\lb{156}
\begin{tikzcd}
\Hom_{\C'}(X',Y')\ar{r}{R}\ar{dr}[swap]{\ci\ld_{RX',X'}(\id_{RX'})}&\Hom_\C(RX',RY'),\ar{d}{\ld_{RX',Y'}}\\ 
&\Hom_{\C'}(LRX',Y').
\end{tikzcd}
\end{equation} 

As the diagram 
$$
\begin{tikzcd}
\Hom_\C(RX',RX')\ar{r}{\ld_{RX',X'}}\ar{d}[swap]{R(f)\ci}&\Hom_{\C'}(LRX',X'),\ar{d}{f\ci}\\ 
\Hom_\C(RX',RY')\ar{r}[swap]{\ld_{RX',Y'}}&\Hom_{\C'}(LRX',Y')
\end{tikzcd}
$$ 
commutes for $f$ in $\Hom_{\C'}(X',Y')$, we get in particular
$$
f\ci\ld_{RX',X'}(\id_{RX'})=\ld_{RX',Y'}(R(f)\ci\id_{RX'})=\ld_{RX',Y'}(R(f)).
$$ 
This shows that \eqref{156} commutes, as required.

%%

\sbs{Equalities (1.5.8) and (1.5.9) p.~29}

Warning: many authors designate $\ee$ by $\eta$ and $\eta$ by $\ee$. 

\subsubsection{Statements}

We have a pair $(L,R)$ of adjoint functors: 
$$
\begin{tikzcd}
\C\ar[xshift=-.7ex]{d}[swap]{L}\\ 
\C'.\ar[xshift=.7ex]{u}[swap]{R}
\end{tikzcd}
$$ 
Recall that $\ee_X\in\Hom_\C(X,RLX)$ and $\eta_{X'}\in\Hom_{\C'}(LRX',X')$ for all $X$ in $\C$ and all $X'$ in $\C'$: 
$$
\ee_X:X\to RLX,\quad\eta_{X'}:LRX'\to X'.
$$ 
Using Notation~\ref{nhove} p.~\pr{nhove}, Equalities (1.5.8) and (1.5.9) become respectively 

\begin{equation}\lb{158}
(\eta\star L)\ci(L\star\ee)=L
\end{equation}

\nn and 

\begin{equation}\lb{159}
(R\star\eta)\ci(\ee\star R)=R.
\end{equation}

\subsubsection{Pictures}

Let us try to illustrate these two equalities by diagrams:

Picture of $L\xl{\eta\star L}LRL$:
 
$$
\begin{tikzcd}
%
\C'&{}&\C'\ar{ll}{}[swap]{1}&{}&\C\ar{ll}{}[swap]{L}&{}&\C\ar{ll}{}[swap]{1}\\ 
%
\C'&\ar{u}{\eta}&\C'\ar{ll}{}{LR}&\ar{u}{L}&\C\ar{ll}{}{L}&\ar{u}{1}&\C\ar{ll}{}{1}\\ 
%
&&&=\\ 
%
\C'&&&{}&&&\C\ar{llllll}[swap]{L}\\
%
\C'&&&\ar{u}{\eta\star L}&&&\C\ar{llllll}{LRL}.
%
\end{tikzcd}
$$ 

Picture of $LRL\xl{L\star\ee}L$:
 
$$
\begin{tikzcd}
%
\C'&{}&\C'\ar{ll}{}[swap]{1}&{}&\C\ar{ll}{}[swap]{L}&{}&\C\C\ar{ll}{}[swap]{RL}\\ 
%
\C'&\ar{u}{1}&\C'\ar{ll}{}{1}&\ar{u}{L}&\C\ar{ll}{}{L}&\ar{u}{\ee}&\C\ar{ll}{}{1}\\ 
%
&&&=\\ 
%
\C'&&&{}&&&\C\ar{llllll}[swap]{LRL}\\
%
\C'&&&\ar{u}{L\star\ee}&&&\C\ar{llllll}{L}.
%
\end{tikzcd}
$$ 

Picture of \qr{158}, that is, $(\eta\star L)\ci(L\star\ee)=L$:

$$
\begin{tikzcd}
%
\C'&&&{}&&&\C\ar{llllll}[near start,swap]{L}\\
%
\C'&&&\ar{u}{\eta\star L}&&&\C\ar{llllll}[near start,swap]{LRL}\\
%
\C'&&&\ar{u}{L\star\ee}&&&\C\ar{llllll}[near start]{L}\\ 
%
&&&=\\ 
%
\C'&&&{}&&&\C\ar{llllll}[swap]{L}\\ 
%
\C'&&&\ar{u}{L}&&&\C.\ar{llllll}{L}
%
\end{tikzcd}
$$ 

Picture of $R\xl{R\star\eta}RLR$:
 
$$
\begin{tikzcd}
%
\C&{}&\C\ar{ll}{}[swap]{1}&{}&\C'\ar{ll}{}[swap]{R}&{}&\C'\ar{ll}{}[swap]{1}\\ 
%
\C&\ar{u}{1}&\C\ar{ll}{}{1}&\ar{u}{R}&\C'\ar{ll}{}{R}&\ar{u}{\eta}&\C'\ar{ll}{}{LR}\\ 
%
&&&=\\ 
%
\C&&&{}&&&\C'\ar{llllll}[swap]{R}\\
%
\C&&&\ar{u}{R\star\eta}&&&\C'\ar{llllll}{RLR}.
%
\end{tikzcd}
$$ 

Picture of $RLR\xl{\ee\star R}R$:
$$
\begin{tikzcd}
%
\C&{}&\C\ar{ll}{}[swap]{RL}&{}&\C'\ar{ll}{}[swap]{R}&{}&\C'\ar{ll}{}[swap]{1}\\ 
%
\C&\ar{u}{\ee}&\C\ar{ll}{}{1}&\ar{u}{R}&\C'\ar{ll}{}{R}&\ar{u}{1}&\C'\ar{ll}{}{1}\\ 
%
&&&=\\ 
%
\C&&&{}&&&\C'\ar{llllll}[swap]{RLR}\\
%
\C&&&\ar{u}{\ee\star R}&&&\C'\ar{llllll}{R}.
%
\end{tikzcd}
$$ 

Picture of \qr{159}, that is, $(R\star\eta)\ci(\ee\star R)=R$:

$$
\begin{tikzcd}
%
\C&&&{}&&&\C'\ar{llllll}[near start,swap]{R}\\
%
\C&&&\ar{u}{R\star\eta}&&&\C'\ar{llllll}[near start,swap]{RLR}\\
%
\C&&&\ar{u}{\ee\star R}&&&\C'\ar{llllll}[near start]{R}\\ 
%
&&&=\\ 
%
\C&&&{}&&&\C'\ar{llllll}[swap]{R}\\ 
%
\C&&&\ar{u}{R}&&&\C'.\ar{llllll}{R}
%
\end{tikzcd}
$$

\subsubsection{Proofs}

For the reader's convenience we prove \qr{158} p.~\pr{158} and \qr{159} p.~\pr{159}. It clearly suffices to prove \qr{158}. Recall that \qr{158} claims 
$$
(\eta\star L)\ci(L\star\ee)=L.
$$ 
Let us denote the functorial mutually inverse bijections defining the adjunction by 
%
\begin{equation}\lb{bij}
\begin{tikzcd}
\Hom_\C(X,RX')\ar[yshift=0.7ex]{rr}{\ld_{X,X'}}&&\Hom_{\C'}(LX,X'),\ar[yshift=-0.7ex]{ll}{\mu_{X,X'}}
\end{tikzcd}
\end{equation} 
%
and recall that $\ee_X$ and $\eta_{X'}$ are defined by
%
\begin{equation}\lb{epseta}
\ee_X:=\mu_{X,LX}(\id_{LX}),\quad\eta_{X'}:=\ld_{RX',X'}(\id_{RX'}).
\end{equation}

Equality \qr{158} p.~\pr{158} can be written 
$$
\ld_{RLX,LX}(\id_{RLX})\ci L(\ee_X)=\id_{LX},
$$ 
and we have 
$$
\id_{LX}\os{\text{(a)}}{=}
\ld_{X,LX}\big(\mu_{X,LX}(\id_{LX})\big)\os{\text{(b)}}{=}
\ld_{X,LX}(\ee_X)\os{\text{(c)}}{=}
\big(\ld_{X,LX}\ci(\ci\ee_X)\big)(\id_{RLX})
$$
$$
\os{\text{(d)}}{=}\Big(\big(\ci L(\ee_X)\big)\ci\ld_{RLX,LX}\Big)(\id_{RLX})\os{\text{(e)}}{=}\ld_{RLX,LX}(\id_{RLX})\ci L(\ee_X),
$$ 
the successive equalities being justified as follows:

(a) follows from \qr{bij},

(b) follows from \qr{epseta},

(c) is obvious,

(d) follows from the commutative square 
$$
\begin{tikzcd}
\Hom_\C(RLX,RLX)\ar{d}[swap]{\ci\ee_X}\ar{rr}{\ld_{RLX,LX}}&&\Hom_{\C'}(LRLX,LX)\ar{d}{\ci L(\ee_X)}\\
\Hom_\C(X,RLX)\ar{rr}[swap]{\ld_{X,LX}}&&\Hom_{\C'}(LX,LX),
\end{tikzcd}
$$ 

(e) is obvious.

%%%

\section{About Chapter 2}

\sbs{Definition of limits (\S2.1 p.~36)}\lb{212} 

% previous version
% https://docs.google.com/document/d/1nk0fA1LSkAaOoW1U04-D0BROtph6oaqcIQahx1JYlno/edit

\begin{nota}\lb{diag}
If $I$ and $\C$ are categories, we denote by $\DT$\index{$\DT$} the diagonal functor\index{diagonal functor} from $\C$ to $\C^I$. The categories $I$ and $\C$ shall be explicitly indicated only when they are not clear from the context. Furthermore, we shall often write $\DT X$ for $\DT(X)$. To be more precise, $\DT X$ is the constant functor from $I$ to $\C$ with value $X$.
\end{nota}

\begin{df}[projective limit]\lb{plim} 
Let $\al:I^{\op}\to\C$ be a functor. If the value at $\al$ of the right adjoint $\lim$ to $\DT:\C\to\C^{I^{\op}}$ exists (see \S\ref{rdefat} p.~\pr{rdefat}), we denote it by $\lim\al$ and call it {\em the projective limit}\index{projective limit} of $\al$. Moreover, we say that the unit $p:\DT\lim\al\to\al$ of the adjunction is \textbf{the} \emph{projection}\index{projection}. More generally we say that $q:\DT X\to\al$ (with $X$ in $\C$) is \textbf{a} \emph{projection} if the corresponding morphism 
$$
X\to\Hom_{\Fct(I^{\op},\C)}(\DT(\ ),\al)
$$ 
in $\C^\wg$ (see Convention~\ref{yoc} p.~\pr{yoc}) is an isomorphism.
\end{df}

The characteristic property of the pair $(\lim\al,p)$ can be described as follows: For each $Y$ in $\C$ and each morphism of functors $\theta:\DT Y\to\al$ there is a unique morphism $f:Y\to\lim\al$ satisfying $p\ci\DT f=\theta$: 
\begin{equation}\lb{yfy}
\begin{tikzcd}
Y\ar[d,dashed,"f"']&\DT Y\ar[d,"\DT f"']\ar[rd,"\theta"]\\ 
\lim\al&\DT\lim\al\ar[r,"p"']&\al.
\end{tikzcd}
\end{equation} 

In Convention~\ref{term} p.~\pr{term} we stated a rule and indicated that we would make some exceptions to it. Here is the first such exception:

\begin{conv}\lb{thep} 
If $\al:I^{\op}\to\Set$ is a functor defined on a small category, then we define its projective limit $\lim\al$ by 
$$ 
\lim\al:=\left\{x\in\prod_{i\in I}\al(i)\ \bigg|\ x_i=\al(s)(x_j)\ \forall\ s:i\to j\right\}\in\Set,
$$ 
and we define \emph{the} projection $p:\DT\lim\al\to\al$ by $p_i(x):=x_i$. Then $p$ is \emph{a} projection in the sense of Definition~\ref{plim}. [Indeed, let $\theta$ in \qr{yfy} be given. As often it is better to start by verifying the \emph{uniqueness} of $f$. We must have $p_i(f(y))=\theta_i(y)$ for all $i$ in $I$ and all $y$ in $Y$. This implies $f(y)=(\theta_i(y))_{i\in I}$. It is straightforward to check that the map $f$ defined by the above equality does the job.]
\end{conv}

Note that the projective limit of $\al:I^{\op}\to\Set$ does not depend on the universe which makes $I$ a small category.

\begin{df}[inductive limit]\lb{ilim} 
Let $\al:I\to\C$ be a functor. If the value at $\al$ of the left adjoint $\col$ to $\DT:\C\to\C^I$ exists, we denote it by $\col\al$ and call it {\em the inductive limit}\index{inductive limit} of the $\al$ (see \S\ref{ldefat} p.~\pr{ldefat}). Moreover, we say that the co-unit $p:\al\to\DT\col\al$ of the adjunction is \textbf{the} \emph{coprojection}\index{coprojection}. More generally we say that $q:\al\to\DT X$ (with $X$ in $\C$) is \textbf{a} \emph{coprojection} if the corresponding morphism 
$$
\Hom_{\C^I}(\al,\DT(\ ))\to X
$$ 
in $\C^\vee$ is an isomorphism.
\end{df}

The characteristic property of the pair $(\col\al,p)$ can be described as follows: For each $Y$ in $\C$ and each morphism of functors $\theta:\al\to\DT Y$ there is a unique morphism $f:X\to Y$ satisfying $\DT f\ci p=\theta$: 
\begin{equation}\lb{cue}
\begin{tikzcd}
\al\ar[r,"p"]\ar[dr,"\theta"']&\DT\col\al\ar[d,"\DT f"]&\col\al\ar[d,dashed,"f"]\\ 
&\DT Y&Y.
\end{tikzcd}
\end{equation}

In Convention~\ref{term} p.~\pr{term} we stated a rule and indicated that we would make some exceptions to it. Here is the second such exception:

\begin{conv}\lb{thei}
Let $\al:I\to\Set$ be a functor defined on a small category, set 
$$ 
U:=\{(i,x)\in\U\ |\ i\in I,x\in\al(i)\},
$$ 
and let $\sim$ be the least equivalence relation on $U$ satisfying $(i,x)\sim(j,\al(f)(x))$ for all morphisms $f:i\to j$. Then we define the inductive limit $\col\al$ as the quotient $U/\!\!\sim$. Let $\pi:U\to\col\al$ be the canonical projection, and, for all $i$ in $I$, define $p_i:\al(i)\to\col\al$ by $p_i(x):=\pi(i,x)$. We call the resulting morphism $p:\al\to\col\al$ the coprojection. Then $p$ is \emph{a} coprojection in the sense of Definition~\ref{plim}. [Indeed, given $\theta$ in \qr{cue} let us prove the uniqueness of $f$. Any $x$ in $X$ is of the form $p_i(t)$ for some $i$ in $I$ and $t$ in $\al(i)$, and we must have $f(x)=\theta_i(t)$. This proves the uniqueness. To verify the existence, we must assume $p_i(t)=p_j(u)$ (obvious notation), and derive $\theta_i(t)=\theta_j(u)$. We may assume that there is a morphism $s:i\to j$, and the verification is straightforward.]
\end{conv} 

Note that the inductive limit of $\al:I\to\Set$ does not depend on the universe which makes $I$ a small category.

%%

\sbs{Brief comments}

\begin{s}\lb{sv} 
We shall spell out two wordings of a certain statement about the following setting: $\al:I\to\C$ is a functor and $Z$ is an object of $\C$. %The first wording will be short, but somewhat incomplete, whereas the second wording will be much longer, but more precise.

\nn First wording: Assume that $\col$ exists in $\C$ and, for each $i$ in $I$, let $p_i:\al(i)\to\col\al$ be the corresponding coprojection. Then the map 
$$
\Hom_\C(\col\al,Z)\to\prod_{i\in I}\Hom_\C(\al(i),Z),\quad f\mt(f\ci p_i)_{i\in I}
$$ 
induces a bijection
$$
\Hom_\C(\col\al,Z)\xr\sim\lim\Hom_\C(\al,Z).
$$ 
The proof is left to the reader.

\nn Second wording: Let $X$ be an object of $\C$ and $p:\al\to\DT X$ a coprojection in the sense of Definition~\ref{ilim} p.~\pr{ilim}: 
\begin{equation}\lb{xp}
\begin{tikzcd}
\al\ar[r,"p"]\ar[dr,"\ld"']&\DT X\ar[d,"\DT f"]&X\ar[d,dashed,"f"]\\ 
&\DT Y&Y.
\end{tikzcd}
\end{equation} 
We claim that 
$$
\ci p:\DT\Hom_\C(X,Z)\to\Hom_\C(\al,Z)
$$ 
is a projection in the sense of Definition~\ref{plim} p.~\pr{plim}:
\begin{equation}\lb{zc}
\begin{tikzcd}
S\ar[d,dashed,"g"']&\DT S\ar[d,"\DT g"']\ar[rd,"\mu"]\\ 
\Hom_\C(X,Z)&\DT\Hom_\C(X,Z)\ar[r,"\ci p"']&\Hom_\C(\al,Z).
\end{tikzcd}
\end{equation} 
More precisely, assume we are given $\mu$ as above and $s$ in $S$. Then we set $Y:=Z$ and $\ld_i:=\mu_i(s)$ in \qr{xp}. We get an $f:X\to Z$, and we set $g(s):=f$. We leave it to the reader to check that this process yields a solution to \qr{zc}, and that this solution is unique.
\end{s}

%

% removed:
% https://docs.google.com/document/d/1w4pNu9aysXFDFK3VXsAeiKrAm826_hq8rA21PsqNHH0/edit

%

% previous version:
% https://docs.google.com/document/d/1OdYPoUxq-6Hu8lFAKK4yazvZ15DINxnYgdzoVUVTtPc/edit

\begin{s}\lb{216}
P.~38, Proposition 2.1.6. We want to find a setting where the isomorphism 
$$
(\col\al)(j)\iso\col\al(j)
$$ 
makes sense and is true.% prove that it holds.%Here are more details. Recall the setting: 

Let $\al:I\to\C^J$ be a functor, and let us assume that for each $j$ in $J$ the functor $\al(\ )(j):I\to\C$ admits a coprojection $p_j:\al(\ )(j)\to\DT X_j$ in the sense of Definition~\ref{ilim} p.~\pr{ilim}: 
\begin{equation}\lb{ajpj}
\begin{tikzcd}
\al(\ )(j)\ar[r,"p_j"]\ar[dr]&\DT X_j\ar[d]&X_j\ar[d,dashed]\\ 
&\DT Y&Y.
\end{tikzcd}
\end{equation}

We claim that there is a natural functor $\bt:J\to\C$ satisfying $\bt(j)=X_j$ for all $j$ in $J$. Given $j\to j'$ we define $X_j\to X_{j'}$ as suggested by the commutative diagram 
$$
\begin{tikzcd}
\al(\ )(j)\ar[d]\ar[r,"p_j"]&\DT X_j\ar[d]&X_j\ar[d,dashed]\\ 
\al(\ )(j')\ar[r,"p_{j'}"']&\DT X_{j'}&X_{j'}.
\end{tikzcd}
$$ 
We leave it to the reader to verify that this construction does define our functor $\bt$.

We want to define a morphism $q:\al\to\DT\bt$. Let $i$ be in $I$. We must define $q_i:\al(i)\to\bt$, that is, given $j$ in $J$ we must define $q_{ij}:\al(i)(j)\to\bt(j)$. It suffices to set $q_{ij}:=p_{ji}$. 

\begin{prop}\lb{216p}
In the above setting the morphism $q$ is a coprojection in the sense of Definition~\ref{ilim} p.~\pr{ilim}. 
\end{prop}

\begin{proof}
Let $\gamma:J\to\C$ be a functor and $\ld:\al\to\DT\gamma$ a morphism of functors. We must solve the problem described by the commutative diagram 
$$
\begin{tikzcd}
\al\ar[r,"q"]\ar[dr,"\ld"']&\DT\bt\ar[d,"\DT\mu"]&\bt\ar[d,dashed,"\mu"]\\ 
&\DT\gamma&\gamma.
\end{tikzcd}
$$ 

Note that $\ld$ is given by a family of morphisms $\ld_i:\al(i)\to\gamma$, morphisms given in turn by families $\ld_{ij}:\al(i)(j)\to\gamma(j)$. 

In view of \qr{ajpj} we can define $\mu_j:\bt(j)\to\gamma(j)$ as suggested by the commutative diagram 
$$
\begin{tikzcd}
\al(\ )(j)\ar[r,"p_j"]\ar[dr,"\ld_{\bu j}"']&\DT\bt(j)\ar[d,"\DT\mu_j"]&\bt(j)\ar[d,dashed,"\mu_j"]\\ 
&\DT\gamma(j)&\gamma(j).
\end{tikzcd}
$$ 

It is straightforward to check that the functors $\mu_j:\bt(j)\to\gamma(j)$ give rise to a functor $\mu:\bt\to\gamma$, and that this functor satisfies $\DT\mu\ci q=\ld$, as required. 
\end{proof}
\end{s}

%

\begin{s}\lb{c38}
P.~38, Proposition 2.1.6. Here is an example of a functor $\al:I\to\C^J$ such that $\col\al$ exists in $\C^J$ but there is a $j$ in $J$ such that $\col\ (\rho_j\ci\al)$ does not exist in $\C$. (Recall that $\rho_j:\C^J\to\C$ is the evaluation at $j\in J$.) This example is taken from Section 3.3 of the book \textbf{Basic Concepts of Enriched Category Theory} of G.M. Kelly\index{Kelly}:%\medskip 
%
\begin{center}\href{http://www.tac.mta.ca/tac/reprints/articles/10/tr10abs.html}{http://www.tac.mta.ca/tac/reprints/articles/10/tr10abs.html}
\end{center}

The category $J$ has two objects, 1, 2; it has exactly one nontrivial morphism; and this morphism goes from 1 to 2. The category $\C$ has exactly three objects, 1, 2, 3, and exactly four nontrivial morphisms, $f,g,h,g\ci f=h\ci f$, with 
$$
\begin{tikzcd}
1\ar{r}{f}&2\ar[yshift=.7ex]{r}{g}\ar[yshift=-.7ex]{r}[swap]{h}&3.
\end{tikzcd}
$$ 
Then $\C^J$ is the category of morphisms in $\C$. It is easy to see that the morphism 
\begin{equation}\lb{38}
f\xr{(f,h)}g 
\end{equation}
in $\C^J$ is an epimorphism, and that this implies that the coproduct 
$$
g\sqcup_fg,
$$ 
taken with respect to (\ref{38}), exists and is isomorphic to $g$ (the coprojections being given by the identity of $g$). It is also easy to see that the coproduct $2\sqcup_12$ does not exist in $\C$.
\end{s}

%

\begin{s}\lb{217}
P.~39, Proposition 2.1.7. We want to find a setting where the isomorphism 
$$
\col_{i,j}\al(i,j)\iso\col_i\col_j\al(i,j)
$$ 
makes sense and is true.

Let $\al:I\tm J\to\C$ be a bifunctor, and let $(X_i)_{i\in I}$ be a family of objects of $\C$. Assume that for any $i$ in $I$ there is some morphism $p_i:\al(i,\ )\to\DT X_i$ which is a coprojection in the sense of Definition~\ref{ilim} p.~\pr{ilim}: 
$$
\begin{tikzcd}
\al(i,\ )\ar[r,"p_i"]\ar[dr]&\DT X_i\ar[d]&X_i\ar[d,dashed]\\ 
&\DT Y&Y.
\end{tikzcd}
$$ 
By arguing as in \S\ref{216} p.~\pr{216} we see that there is a natural functor $\bt:I\to\C$ such that $\bt(i)=X_i$ for all $i$. Let $q:\bt\to\DT X$ be a coprojection: 
\begin{equation}\lb{bqx}
\begin{tikzcd}
\bt\ar[r,"q"]\ar[dr]&\DT X\ar[d]&X\ar[d,dashed]\\ 
&\DT Y&Y.
\end{tikzcd}
\end{equation} 

We claim that the obvious morphism of functors $r:\al\to\DT X$ is a coprojection. 

Let $Y$ be in $\C$ and $\theta:\al\to\DT Y$ a morphism of functors. We must solve the problem 
$$
\begin{tikzcd}
\al\ar[r,"r"]\ar[dr,"\theta"']&\DT X\ar[d]&X\ar[d,dashed]\\ 
&\DT Y&Y.
\end{tikzcd}
$$ 
Noting that $\theta$ induces, for all $i$, a morphism of functors $\al(i,\ )\to\DT Y$, we get firstly a morphism $\bt(i)\to Y$:
$$
\begin{tikzcd}
\al(i,\ )\ar[r,"p_i"]\ar[dr]&\DT\bt(i)\ar[d]&\bt(i)\ar[d,dashed]\\ 
&\DT Y&Y,
\end{tikzcd}
$$ 
secondly a morphism of functors $\bt\to\DT Y$, and thirdly a morphism $X\to Y$ by \qr{bqx}. It is straightforward to check that this morphism $X\to Y$ does the job. q.e.d.
\end{s}

%

\begin{s} 
The two propositions below are basic.

\begin{prop}\lb{aioc}
If $\al:I^{\op}\to\C$ is a functor defined on a small category, if $X$ is in $\C$, if $p:\DT X\to\al$ is a morphism in $\C^{I^{\op}}$, and if $h:\C\to\C^\wg$ is the Yoneda embedding, then the following conditions are equivalent:

\nn\emph{(a)} $p$ is a projection in the sense of Definition~\ref{plim} p.~\pr{plim},

\nn\emph{(b)} the morphism $h(p):\DT h(X)\to h\ci\al$ in $(\C^\wg)^{I^{\op}}$ induced by $p$ is a projection,

\nn\emph{(c)} for all $Y$ in $\C$ the morphism $\Hom_\C(Y,p):\DT\Hom_\C(Y,X)\to\Hom_\C(Y,\al)$ in $\Set^{I^{\op}}$ induced by $p$ is a projection.
\end{prop}

Condition (c) is often abridged by 
$$
\Hom_\C(Y,\lim\al)\xr\sim\lim\Hom_\C(Y,\al).
$$

\begin{proof}
Conditions (b) and (c) are equivalent by Proposition~\ref{216p} p.~\pr{216p}. We sketch the proof that (a) and (c) are equivalent. Let us summarize (a) and (c) by the following self-explanatory commutative diagrams:
\begin{equation}\lb{zfz}
\begin{tikzcd}
Z\ar[d,dashed,"f"']&\DT Z\ar[d,"\DT f"']\ar[rd,"\ld"]\\ 
X&\DT X,\ar[r,"p"']&\al
\end{tikzcd}
\end{equation} 

\begin{equation}\lb{sgs}
\begin{tikzcd}
S\ar[d,dashed,"g"']&\DT S\ar[d,"\DT g"']\ar[rrd,"\mu"]\\ 
\Hom_\C(Y,X)&\DT\Hom_\C(Y,X)\ar{rr}[swap]{\Hom_\C(Y,p)}&&\Hom_\C(Y,\al).
\end{tikzcd}
\end{equation} 

To prove (c)$\then$(a), we suppose $Z$ and $\ld$ given in \qr{zfz}, and in \qr{sgs} we let $S$ be a singleton, we set $Y:=Z$, we define $\mu$ by the formula $\mu_i(s):=\ld_i$, we get a $g$ as above, we set $f:=g(s)$, and we check that this works. 

To prove (a)$\then$(c), we suppose $S,Y$ and $\mu$ given in \qr{sgs}, and we let $s$ be in $S$. We must define $g(s):Y\to X$. We set $Z:=Y$ in \qr{zfz}. We must define $\ld:\DT Y\to\al$. Letting $i$ be in $I$, it suffices to define $\ld_i:Y\to\al(i)$. To do this we set $\ld_i:=\mu_i(s)$, we get an $f$ (depending on $s$) as above, we set $g(s):=f$, and we check that this works.
\end{proof}

The proof of the following proposition is similar to the previous one and is left to the reader as an easy exercise.

\begin{prop}\lb{aic}
If $\al:I\to\C$ is a functor defined on a small category, if $X$ is in $\C$, if $p:\al\to\DT X$ is a morphism in $\C^I$, and if $k:\C\to\C^\vee$ is the Yoneda embedding, then the following conditions are equivalent:

\nn\emph{(a)} $p$ is a coprojection in the sense of Definition~\ref{ilim} p.~\pr{ilim},

\nn\emph{(b)} the morphism $k(p):k\ci\al\to\DT k(X)$ in $(\C^\vee)^I$ induced by $p$ is a coprojection,

\nn\emph{(c)} the morphism $k(p):\DT k(X)\to k\ci\al^{\op}$ in $(\Set^\C)^{I^{\op}}$ induced by $p$ is a projection in the sense of Definition~\ref{plim} p.~\pr{plim},

\nn\emph{(d)} for all $Y$ in $\C$  the morphism $\Hom_\C(p,Y):\DT\Hom_\C(X,Y)\to\Hom_\C(\al,Y)$ in $\Set^{I^{\op}}$ is a projection.
\end{prop}

Condition (d) is often abridged by 
$$
\Hom_\C(\col\al,Y)\xr\sim\lim\Hom_\C(\al,Y).
$$ 

In \S\ref{sv} p.~\pr{sv} we proved that (a) implies (d).
\end{s}

% paragraph removed:
% https://docs.google.com/document/d/1Hx-fRBi-yTwTyFKvGi-qFLmgNQwYl7YJvZMd_ngC6nw/edit

\begin{s} 
P.~40, Proposition 2.1.10 (stated below as Corollary~\ref{2.1.10c}). Here is a slightly more general statement.%Proposition~\ref{2.1.10}

\begin{prop}\lb{2.1.10b}
Let $\C\os G\longleftarrow\A\os F\longrightarrow\B$ be functors and $I$ a category. Assume that $\A$ admits inductive limits indexed by $I$, that $G$ commutes with such limits, and that for each $Y$ in $\B$ there is a $Z$ in $\C$ and an isomorphism 
$$
\Hom_\B(F(\ ),Y)\iso\Hom_\C(G(\ ),Z)
$$
in $\A^\wg$. Then $F$ commutes with inductive limits indexed by $I$.
\end{prop}

\begin{proof} 
Let $\theta$ be the isomorphism $\Hom_\B(F(\ ),Y)\xr\sim\Hom_\C(G(\ ),Z)$ and let $p:\al\to\DT\col\al$ be the coprojection. Consider the self-explanatory 
commutative diagrams 
\begin{equation}\lb{faq}
\begin{tikzcd}
F\ci\al\ar[r,"q"]\ar[dr,"F(p)"']&\DT\col F\ci\al\ar[d,"\DT f"]&\col F\ci\al\ar[d,dashed,"f"]\\ 
&\DT F(\col\al)&F(\col\al)
\end{tikzcd}
\end{equation} 
and 
$$
\begin{tikzcd}
G\ci\al\ar[r,"r"]\ar[dr,"G(p)"']&\DT\col G\ci\al\ar[d,"\DT g"]&\col G\ci\al\ar[d,dashed,"g"]\\ 
&\DT G(\col\al)&G(\col\al),
\end{tikzcd}
$$ 
where $q$ and $r$ are the coprojections. Note that $g$ is an isomorphism by assumption. Our goal is to prove that $f$ is an isomorphism too. Let $Y$ be in $\B$. It suffices to show that 
$$
\Hom_\B(f,Y):\Hom_\B(F(\col\al),Y)\to\Hom_\B(\col F\ci\al,Y)
$$ 
is an isomorphism. Form the commutative diagram 
$$
\begin{tikzcd}
\DT\Hom_\B(F(\col\al),Y)\ar{rrr}{\Hom_\B(F(p),Y)}\ar[d,"\sim","\DT\theta_{\col\al}"']&&&\Hom_\B(F\ci\al,Y)\ar[d,"\theta_\al","\sim"']\\ 
\DT\Hom_\C(G(\col\al),Z)\ar{rrr}{\Hom_\C(r,Z)}\ar{d}{\sim}[swap]{\DT\Hom_\C(g,Z)}&&&\Hom_\C(G\ci\al,Z)\ar[d,equal]\\ 
\DT\Hom_\C(\col G\ci\al,Y)\ar{rrr}{\Hom_\C(G(p),Z)}&&&\Hom_\C(G\ci\al,Z)\\ 
\DT\Hom_\B(\col F\ci\al,Y)\ar{rrr}{\Hom_\B(q,Y)}&&&\Hom_\B(F\ci\al,Y).\ar[u,"\sim","\theta_\al"']
\end{tikzcd}
$$ 
The last three horizontal arrows are projections. The bottom horizontal arrow being a projection, there is a unique morphism 
$$
h:\Hom_\C(\col G\ci\al,Y)\to\Hom_\B(\col F\ci\al,Y)
$$ 
making the diagram 
\begin{equation}\lb{dbf}
\begin{tikzcd}
\DT\Hom_\B(F(\col\al),Y)\ar{rrr}{\Hom_\B(F(p),Y)}\ar[d,"\sim","\DT\theta_{\col\al}"']&&&\Hom_\B(F\ci\al,Y)\ar[d,"\theta_\al","\sim"']\\ 
\DT\Hom_\C(G(\col\al),Z)\ar{rrr}{\Hom_\C(r,Z)}\ar{d}{\sim}[swap]{\DT\Hom_\C(g,Z)}&&&\Hom_\C(G\ci\al,Z)\ar[d,equal]\\ 
\DT\Hom_\C(\col G\ci\al,Y)\ar{rrr}{\Hom_\C(G(p),Z)}\ar[d,"\DT h"']&&&\Hom_\C(G\ci\al,Z)\\ 
\DT\Hom_\B(\col F\ci\al,Y)\ar{rrr}{\Hom_\B(q,Y)}&&&\Hom_\B(F\ci\al,Y)\ar[u,"\sim","\theta_\al"']
\end{tikzcd}
\end{equation} 
commute. Moreover $h$ is an isomorphism because $\Hom_\C(G(p),Z)$ is a projection. Define the isomorphism 
$$
k:\Hom_\B(F(\col\al),Y)\to\Hom_\B(\col F\ci\al,Y)
$$ 
by $k:=h\ci\Hom_\C(g,Z)\ci\theta_{\col\al}$. It is enough to check that we have $\Hom_\B(f,Y)=k$. As $\Hom_\B(q,Y)$ is a projection, this equality follows from the commutativity of \qr{faq} and \qr{dbf}.
\end{proof}

\begin{cor}[Proposition 2.1.10 p.~40]\lb{2.1.10c}
Let $F:\A\to\B$ be a functor and $I$ a category. Assume that $\A$ admits inductive limits indexed by $I$ and that $F$ admits a right adjoint. Then $F$ commutes with inductive limits indexed by $I$.
\end{cor}

\begin{proof} 
Let $R:\B\to\A$ be right adjoint to $F$, and in Proposition~\ref{2.1.10b}, let $\C$ be $\A$, $G$ be $\id_\A$ and $Z$ be $R(Y)$. %An important particular case is when $\C$ is $\A$, $G$ is $\id_\A$ and $Z=R(Y)$, where $R:\B\to\A$ is right adjoint to $F$.
\end{proof}
\end{s}

%%

\sbs{Universal limits}

% previous version:
% https://docs.google.com/document/d/1dGilBEaYjcIyx7OvpkC3GqboftMYYM7EyqaVY_5vZkM/edit

\begin{prop}\lb{copr}
If $\al:I\to\C$ is a functor defined on a small category, if $X$ is an object of $\C$, if $p:\al\to\DT X$ is a morphism in $\C^I$ and if $h:\C\to\C^\wg$ is the Yoneda embedding, then the following conditions are equivalent:

\nn\emph{(a)} the obvious morphism $h(p):h\ci\al\to\DT h(X)$ is a coprojection,

\nn\emph{(b)} for all object $Y$ in $\C$ the obvious morphism 
$$
\Hom_\C(Y,p):\Hom_\C(Y,\al)\to\DT\Hom_\C(Y,X)
$$ 
is a coprojection,

\nn\emph{(c)} for any functor $F:\C\to\C'$ the obvious morphism $F(p):F\ci\al\to\DT F(X)$ is a coprojection,

\nn\emph{(d)} for any functor $F:\C\to\C'$ and any object $X'$ in $\C'$ the obvious morphism 
$$
\Hom_{\C'}(F(p),X'):\DT \Hom_{\C'}(F(X),X')\to\Hom_{\C'}(F\ci\al,X')
$$ 
is a projection. 
\end{prop}

\begin{proof}
Propositions~\ref{aioc} and \ref{aic} imply (a) $\ssi$ (b), (c) $\ssi$ (d) and (c) $\then$ (a). To prove (a) $\then$ (d), let $X'$ be in $\C'$ and set $A:=\Hom_{\C'}(F(\ ),X')$. We get the commutative diagram 
$$
\begin{tikzcd}
\DT A(X)\ar[d,"\sim"']\ar[rrr,"A(p)"]&&&A\ci\al\ar[d,"\sim"]\\ 
\DT\Hom_{\C^\wg}(X,A)\ar{rrr}[swap]{\Hom_{\C^\wg}(h(p),A)}&&&\Hom_{\C^\wg}(\al,A),
\end{tikzcd}
$$ 
where the vertical isomorphisms are given by the Yoneda Lemma. The bottom horizontal arrow being a projection by assumption, the top horizontal arrow is also a projection.
\end{proof}

\begin{df}[universal inductive limit]\lb{uil} 
Let $\al:I\to\C$ be a functor defined on a small category, let $X$ be an object of $\C$, and let $p:\al\to\DT X$ be a morphism in $\C^I$ (see Notation~\ref{diag} p.~\pr{diag}). If the the equivalent condition in Proposition~\ref{copr} hold, we say that $(X,p)$ is a {\em universal} inductive limit\index{universal inductive limit} of $\al$, and that $\col\al$ exists {\em universally}\index{universal existence of a limit} in $\C$.
\end{df}

There is of course an analogous notion of universal projective limit. 

Here is the classic example of a non-universal inductive limit. Letting $\al$ be the unique functor from the empty category to $\Set$, we get $\col\al=\vi$. Writing $h:\Set\to\Set^\wg$ for the Yoneda embedding yields $\col h\ci\al=\DT\vi$. But we have, on the one hand $(\DT\vi)(\vi)=\vi$, and on the other hand $(h(\vi))(\vi)\not\iso\vi$, implying 
$$
\col h\ci\al\not\iso h(\col\al).
$$ 
This shows that the inductive limit of $\al$ does not exist universally in $\Set$.

%%

\sbs{Brief comments}

\begin{s} 
P.~40, proof of Lemma 2.1.11 (minor variant).

\begin{lem}\lb{l2111}
If $T$ is an object of a category $\C$, then 
$$
T\text{ is terminal }\ssi\ T\iso\col\id_\C.
$$
\end{lem}

\begin{proof} $\then$: Straightforward.

\nn$\si$: Let $p:\id_\C\to\DT T$ be a coprojection (see Definition~\ref{ilim} p.~\pr{ilim}) and let $X$ be in $\C$. For all morphism of functors $\theta:\id_\C\to\DT X$ there is a unique morphism $f:T\to X$ satisfying $\DT f\ci p=\theta$: %Let us denote this $f$ by $f^\theta$: 
$$
\begin{tikzcd}
\id_\C\ar[r,"p"]\ar[dr,"\theta"']&\DT T\ar[d,"\DT f"]&T\ar[d,dashed,"f"]\\ 
&\DT X&X.
\end{tikzcd}
$$ 
%We claim \begin{equation}\lb{fttt}f^\theta=\theta_T.\end{equation} We have indeed $(\DT\theta_T\ci p)_Y=\theta_T\ci p_Y=\theta_Y=(\DT f^\theta\ci p)_Y$ for any $Y$ in $\C$. This proves \qr{fttt}. 
We claim 
\begin{equation}\lb{idtpt}
\id_T=p_T.
\end{equation} 
We have indeed $(\DT p_T\ci p)_X=p_T\ci p_X=p_X=(\DT\id_T\ci p)_X$. This proves \qr{idtpt}. If $f:X\to T$ is a morphism in $\C$, then we have $f=\id_T\ci f=p_T\ci f=p_X$, the second equality following from \qr{idtpt}. This shows that $T$ is terminal.
\end{proof}

\begin{cor}\lb{c2111}
If $\C$ is a category and $A$ an object of $\C^\wg$, then the following conditions are equivalent:

\nn{\em(a)} $A$ is representable,

\nn{\em(b)} $\C_A$ has a terminal object,

\nn{\em(c)} the identity of $\C_A$ has an inductive limit in $\C_A$.
\end{cor}

\begin{proof}
This follows from Lemma~\ref{l2111} above and Convention~\ref{term} p.~\pr{term}.
\end{proof}
\end{s}

%

\begin{s}
P. 41, Lemma 2.1.12. The following variant will be useful to prove Proposition 2.5.2 p.~57 of the book (see \S\ref{252} p.~\pr{252} below).

\begin{lem}\lb{2112}
If $I$ and $\C$ are categories, if $X$ is in $\C$, if $\DT X:I\to\C$ is the constant functor with value $X$, and if $I$ is connected, then 

\nn\emph{(a)} $\id_{\DT X}:\DT X\to\DT X$ is a coprojection in the sense of Definition~\ref{ilim} p.~\pr{ilim},

\nn\emph{(b)} if $i$ is in $I$, $Y$ in $\C$, $f:X\to Y$ and $\theta:\DT X\to\DT Y$, then the equalities $\DT f=\theta$ and $f=\theta_i$ are equivalent: 
$$
\begin{tikzcd}
\DT X\ar[r,"\id"]\ar[dr,"\theta"']&\DT X\ar[d,"\DT f"]&X\ar[d,dashed,"f"]\\ 
&\DT Y&Y.
\end{tikzcd}
$$
%for all $Y$ in $\C$ the natural map $\DT':\Hom_\C(X,Y)\to\Hom_{\C^I}(\DT X,\DT Y)$ is bijective, and, for all $i$ in $I$, the map $\theta\mt\theta_i$ is the inverse of $\DT'$,

\nn\emph{(c)} we have $\theta_i=\theta_j$ for all $\theta:\DT X\to\DT Y$ with $Y$ in $\C$ and $i,j$ in $I$.
\end{lem}

\begin{proof}
To prove (c) we can assume that there is a morphism $i\to j$, in which case the claim is obvious. Clearly (c) implies (a) and (b).
\end{proof}
\end{s}

%

\begin{s}
P.~42, proof of Lemma 2.1.15. Here are some additional details about the last diagram on p.~42:

To the commutative diagram 
$$
\begin{tikzcd}
i\ar[d,"\id"']\ar[r,"f"]&j\\ 
i\ar[r,"\id"']&i\ar[u,"f"']
\end{tikzcd}
$$ 
in $I$ we attach the commutative diagram 
\begin{equation}\lb{42a}
\begin{tikzcd}
\al(i)\ar[d,"\id"']\ar[r,"\pp(f)"]&\bt(j)\\ 
\al(i)\ar[r,"\pp(\id_i)"']&\bt(i)\ar[u,"\bt(f)"']
\end{tikzcd}
\end{equation} 
in $\C$. Turning \qr{42a} upside down we get 
\begin{equation}\lb{42b}
\begin{tikzcd}
\al(i)\ar[r,"\pp(\id_i)"]&\bt(i)\ar[d,"\bt(f)"]\\ 
\al(i)\ar[u,"\id"]\ar[r,"\pp(f)"']&\bt(j).
\end{tikzcd}
\end{equation} 
To the commutative diagram 
$$
\begin{tikzcd}
i\ar[d,"f"']\ar[r,"f"]&j\\ 
j\ar[r,"\id"']&j\ar[u,"\id"']
\end{tikzcd}
$$ 
in $I$ we attach the commutative diagram 
\begin{equation}\lb{42c}
\begin{tikzcd}
\al(i)\ar[d,"\al(f)"']\ar[r,"\pp(f)"]&\bt(j)\\ 
\al(j)\ar[r,"\pp(\id_j)"']&\bt(j)\ar[u,"\id"']
\end{tikzcd}
\end{equation} 
in $\C$. Splicing \qr{42b} and \qr{42c} we get 
$$
\begin{tikzcd}
\al(i)\ar[r,"\pp(\id_i)"]&\bt(i)\ar[d,"\bt(f)"]\\ 
\al(i)\ar[u,"\id"]\ar[d,"\al(f)"']\ar[r,"\pp(f)"]&\bt(j)\\ 
\al(j)\ar[r,"\pp(\id_j)"']&\bt(j).\ar[u,"\id"']
\end{tikzcd}
$$ 
Reversing the identity arrows, we get 
$$
\begin{tikzcd}
\al(i)\ar[d,"\id"']\ar[r,"\pp(\id_i)"]&\bt(i)\ar[d,"\bt(f)"]\\ 
\al(i)\ar[d,"\al(f)"']\ar[r,"\pp(f)"]&\bt(j)\ar[d,"\id"]\\ 
\al(j)\ar[r,"\pp(\id_j)"']&\bt(j),
\end{tikzcd}
$$ 
as desired.
%The last diagram can also be drawn as follows: $$\begin{tikzcd}\al(i)\ar[dd,"\pp(i\xr\id i)"']\ar[rr,"\al(i\xr\id i)"]&&\al(i)\ar[dd,"\pp(i\to j)"]\ar[rr,"\al(i\to j)"]&&\al(j)\ar[dd,"\pp(j\xr\id j)"]\\ \\ \bt(i)\ar[rr,"\bt(i\to j)"']&&\bt(j)\ar[rr,"\bt(i\xr\id i)"']&&\bt(j).\end{tikzcd}$$
\end{s}

%

\begin{s}
%$\frownie$\footnote{The symbol $\frownie$ (``frowny'') means that I'm not satisfied with this \S, and that I'm planning to rewrite it.} 
Lemma 2.1.15 p.~42. Here is a complement which will be used in \S\ref{17115ps} p.~\pr{17115ps}. 

\begin{thm}\lb{fgd}
Let $I$ be a small category with disjoint hom-sets; let $\al,\bt:I\to\C$ be two functors; for each $i$ in $I$, let $U^i:I^i\to I$ be the forgetful functor; and set 
$$
S:=\Hom_{\Fct(I,\C)}(\al,\bt),\quad T:=\lim_{i\in I^{\op}}\Hom_{\Fct(I^i,\C)}(\al\ci U^i,\bt\ci U^i).
$$ 
Then there is a unique map $f:T\to S$ satisfying $f(t)_i=t_{\id_i}$ for all $t$ in $T$. Moreover $f$ is inverse to the natural map from $S$ to $T$. 
\end{thm} 

\begin{proof}
Left to the reader.
\end{proof}
\end{s} 

% removed 
% https://docs.google.com/document/d/1ydcLZ0o0IQjew7DRTL5J8BFQ_9V9fc56CfpscjLnrUE/edit

\begin{s} 
P. 44, Definition 2.2.2 (iii). A sequence $X\to Y\parar Z$ in a category $\C$ is exact if and only if its image in $\C^\wg$ is exact.
\end{s}

%%

\sbs{Stability by base change}\lb{sbsarsbc}

\begin{s}\lb{parsbc}
Recall the following definition:

\begin{df}[Definition 2.2.6 p.~47, stability by base change]%\lb{dsbc}
Let $\C$ be a category which admits fiber products and inductive limits indexed by a category $I$.

\nn\emph{(i)} We say that inductive limits in $\C$ indexed by $I$ are \emph{stable by base change}\index{base change}\index{stable by base change} if for any morphism $Y\to Z$ in $\C$, the base change functor $\C_Z\to\C_Y$ given by 
$$
\C_Z\ni(X\to Z)\mt(X\tm_ZY\to Y)\in\C_Y
$$ 
commutes with inductive limits indexed by $I$. 

This is equivalent to saying that for any inductive system $(X_i)_{i\in I}$ in $\C$ and any pair of morphisms $Y\to Z$ and $\col_iX_i\to Z$ in $\C$, we have the isomorphism 
$$
\col_i(X_i\tm_ZY)\xr\sim\left(\col_iX_i\right)\tm_ZY.
$$

\nn\emph{(ii)} If $\C$ admits small inductive limits and \emph{(i)} holds for any small category $I$, we say that \emph{small inductive limits in $\C$ are stable by base change}.
\end{df} 

The following lemma is implicit:

\begin{lem}\lb{bcl} 
Let $I$ and $\C$ be categories, let $Y$ be an object of $\C$, let $U:\C_Y\to\C$ be the forgetful functor, and let $\al:I\to\C_Y$ be a functor such that $\col U\ci\al$ exists in $\C$. Then $\col\al$ exists in $\C_Y$ and is given by the natural morphism $\col U\ci\al\to Y$. More precisely, let $X\to Y$ be a morphism in $\C$, let $p:\al\to\DT(X\to Y)$ be a morphism in $\Fct(I,\C_Y)$, and let $U\star p:U\ci\al\to\DT X$ be the corresponding morphism in $\Fct(I,\C)$. [Recall that $\star$ denotes the horizontal composition defined in Definition~\ref{dil1} p.~\pr{dil1}.] If $U\star p$ is a coprojection (see Definition~\ref{ilim} p.~\pr{ilim}), then so is $p$.
\end{lem}

\begin{proof}
Let $Z\to Y$ be a morphism in $\C$ and $\ld:\al\to\DT(Z\to Y)$ be a morphism in $\Fct(I,\C_Y)$. We must show that there is a unique morphism $f:X\to Z$ in $\C_Y$ such that $\DT f\ci p=\ld$: 
$$
\begin{tikzcd}
\al\ar[r,"p"]\ar[dr,"\ld"']&\DT(X\to Y)\ar[d,"\DT f"]&(X\to Y)\ar[d,dashed,"f"]\\ 
&\DT(Z\to Y)&(Z\to Y).
\end{tikzcd}
$$ 
Let $\mu:U\ci\al\to\DT Z$ be the morphism in $\Fct(I,\C)$ induced by $\ld$. Then there is a unique morphism $f:X\to Z$ such that $\DT f\ci(U\star p)=\mu$: 
$$%\begin{equation}\label{uap}
\begin{tikzcd}
U\ci\al\ar[r,"U\star p"]\ar[dr,"\mu"']&\DT X\ar[d,"\DT f"]&X\ar[d,dashed,"f"]\\ 
&\DT Z&Z.
\end{tikzcd}
$$%\end{equation} 
It remains to check that $f$ is a morphism in $\C_Y$, that is, we must prove 
$$
(X\xr fZ\to Y)=(X\to Y).
$$ 
Let $i$ be in $I$. %By \qr{uap} it suffices to show 
As $U\star p$ is a coprojection, it suffices to show 
$$%\begin{equation}\label{xfzy}
\Big(U(\al(i))\to X\xr fZ\to Y\Big)=\Big(U(\al(i))\to X\to Y\Big).
$$%\end{equation} 
But we have 
$$
\begin{tikzcd}
(U(\al(i))\ar[rr,"(U\star p)_i"]&&X\ar[rr,"f"]&&Z\ar[rr]&&Y)&=\\ 
(U(\al(i))\ar[rrrr,"\ld_i"]&&&&Z\ar[rr]&&Y)&=\\ 
(U(\al(i))\ar[rrrrrr]&&&&&&Y)&=\\ 
(U(\al(i))\ar[rr,"(U\star p)_i"]&&X\ar[rrrr]&&&&Y).
\end{tikzcd}
$$ 
Indeed, the first and third equalities follow from the fact that $U\star p$ is a coprojection, and the second equality follows from the fact that $\ld_i$ is a morphism in $\C_Y$. 
\end{proof} 
\end{s}

\begin{s}%\lb{sbbc}
%P. 47, Definition 2.2.6 (stated below as Definition~\ref{dsbc} p.~\pr{dsbc}), notion of stability by base change. 
We make an easy but useful observation. Let $I,J$ and $\C$ be three categories. 

If $\C$ admits fiber products, and if inductive limits indexed by $I$ exist in $\C$ and are stable by base change, then $\C^J$ admits fiber products, and inductive limits indexed by $I$ exist in $\C^J$ and are stable by base change.
\end{s} 

%%

\sbs{Brief comments}

\begin{s}\lb{fpl}
P.~50, Corollary 2.2.11. We also have:

\emph{A category admits finite projective limits if and only if it admits a terminal object and binary fibered products.}

Indeed, if $f,g:X\parar Y$ is a pair of parallel arrows, and if the square 
$$
\begin{tikzcd}
K\ar{d}\ar{r}&Y\ar{d}{\DT}\\ 
X\ar{r}[swap]{(f,g)}&Y\tm Y
\end{tikzcd}
$$
is cartesian, then $K\iso\Ker(f,g)$. (As usual, $\DT$ is the diagonal morphism.)
\end{s}

%

\begin{s}
P.~50, Definition 2.3.1. The three pieces of notation $\pp_*,\pp^\dg$ and $\pp^\ddg$ are justified by Notation 17.1.5 p.~407 (see also \qr{ttau} p.~\pr{ttau}). 
\end{s} 

% 

\begin{s}\lb{phistar}
P.~50, Definition 2.3.1. Let $\pp:J\to I$ be a functor of small categories, let $\C$ be a category, and consider the functor 
\begin{equation}\lb{pcp}
\pp_*:=\ci\pp:\C^I\to\C^J.
\end{equation}
The following fact results from Proposition 2.1.6 p.~38 of the book (see \S\ref{216} p.~\pr{216}): 

\emph{If $\C$ admits small inductive (resp. projective) limits, then so do $\C^I$ and $\C^J$, and $\pp_*$ commutes with such limits.} 

Recall that we denote horizontal composition of morphisms of functors by $\star$ (see Definition~\ref{dil1} p.~\pr{dil1}).

The effect of the functor \qr{pcp} on morphisms can be described as follows: If $\theta:\al\to\bt$ is a morphism in $\C^I$, then the morphism $\pp_*\theta:\pp_*\al\to\pp_*\bt$ in $\C^J$ is defined by $\pp_*\theta:=\theta\star\pp$, which is in turn defined by $(\theta\star\pp)_j:=\theta_{\pp(j)}$. 
\end{s} 

%

\begin{s}\lb{s232} 
P.~51, Definition 2.3.2. Recall that we have functors $I\xl\pp J\xr\bt\C$. We spell out Definition 2.3.2 using the terminology of Section~\ref{defat} p.~\pr{defat}. %Let $\bt$ be in $\C^J$, and 
Let $\pp^\dg\bt$ and $\pp^\ddg\bt$ be in $\C^I$. 

\nn(a) We say that ``$\pp^\dg\bt$ exists'' if there is a co-unit $\ee_\bt:\bt\to\pp_*\pp^\dg\bt$, that is, for all $\al:I\to\C$ and all $w:\bt\to\pp_*\al$ there is a unique $v:\pp^\dg\bt\to\al$ such that $(v\star\pp)\ci\ee_\bt=w$: 
$$
\begin{tikzcd}
\bt\ar[r,"\ee_\bt"]\ar[dr,"w"']&\pp_*\pp^\dg\bt\ar[d,"v\star\pp"]&\pp^\dg\bt\ar[d,dashed,"v"]\\ 
&\pp_*\al&\al.
\end{tikzcd}
$$

\nn(b) We say that ``$\pp^\ddg\bt$ exists'' if there is a unit $\eta_\bt:\pp_*\pp^\ddg\bt\to\bt$, that is, for all $\al:I\to\C$ and all $w:\pp_*\al\to\bt$ there is a unique $v:\al\to\pp^\ddg\bt$ such that $\eta_\bt\ci(v\star\pp)=w$: 
$$
\begin{tikzcd}
\al\ar[d,dashed,"v"']&\pp_*\al\ar[d,"v\star\pp"']\ar[rd,"w"]\\ 
\pp^\ddg\bt&\pp_*\pp^\ddg\bt\ar[r,"\eta_\bt"']&\bt
\end{tikzcd}
$$

\nn(c) Let $\gamma:I\to\C$ and let $u$ be an endomorphism of the functor $\pp_*\gamma:J\to\C$. The phrase ``$\pp^\ddg\pp_*\gamma$ exists and is isomorphic to $\gamma$ via $u$'' shall mean that for all $\al:I\to\C$ and all $w:\pp_*\al\to\pp_*\gamma$ there is a unique $v:\al\to\gamma$ such that $u\ci(v\star\pp)=w$: 
$$
\begin{tikzcd}
\al\ar[d,dashed,"v"']&\pp_*\al\ar[d,"v\star\pp"']\ar[rd,"w"]\\ 
\gamma&\pp_*\gamma\ar[r,"u"']&\pp_*\gamma.
\end{tikzcd}
$$ 
In particular, the phrase ``$\pp^\ddg\pp_*\gamma$ exists and is isomorphic to $\gamma$ via the identity of $\pp_*\gamma:J\to\C$'' shall mean that, for all $\al:I\to\C$, the map 
$$
\Hom_{\C^I}(\al,\gamma)\to\Hom_{\C^J}(\pp_*\al,\pp_*\gamma),\quad v\mt v\star\pp
$$ 
is bijective. (See \S\ref{713} p.~\pr{713} below.)
\end{s}

%

\begin{s} 
Let $I\xl\pp J\xr\bt\C\xr F\C'$ be functors, let $\bt$ be in $\C^J$ and assume that $\pp^\dg(\bt)$ and $\pp^\dg(F\ci\bt)$ exist:
$$
\begin{tikzcd}
J\ar[ddd,"\bt"']\ar[rrr,"\pp"]&&&I\ar[dddlll,"\pp^\dg(\bt)"']\ar[ddd,"\pp^\dg(F\ci\bt)"]\\ \\ \\ 
\C\ar[rrr,"F"']&&&\C'.
\end{tikzcd}
$$ 

We claim that there is a natural morphism $\pp^\dg(F\ci\bt)\to F\ci\pp^\dg(\bt)$. 

As $F\ci\pp^\dg(\bt)\ci\pp=\pp_*(F\ci\pp^\dg(\bt))$, it suffices to define a natural morphism $F\ci\bt\to F\ci\pp^\dg(\bt)\ci\pp$: 
$$
\begin{tikzcd}
F\ci\bt\ar[r,"\ee_{F\ci\bt}"]\ar[dr]&\pp_*\pp^\dg(F\ci\bt)\ar[d]&\pp^\dg(F\ci\bt)\ar[d,dashed]\\ 
&\pp_*(F\ci\pp^\dg(\bt))&F\ci\pp^\dg(\bt).
\end{tikzcd}
$$ 
As we have $\ee_\bt:\bt\to\pp^\dg(\bt)\ci\pp$, we can take $F\star\ee_\bt:F\ci\bt\to F\ci\pp^\dg(\bt)\ci\pp$. 

\begin{df}\lb{kcw}
If the above morphism $\pp^\dg(F\ci\bt)\to F\ci\pp^\dg(\bt)$ is an isomorphism, we say that $F$ commutes with $\pp^\dg$ at $\bt$.
\end{df}
\end{s}

%

\begin{s} 
P.~51, Definition 2.3.2 (minor variant). We assume that no underlying universe has been given. Let $I\xl\pp J\xr\bt\C$ be functors, let $\bt$ be in $\C^J$, and let $\pp^\dg\bt$ be in $\C^I$. The following conditions are equivalent:

\nn(a) $\pp^\dg\bt$ represents $\Hom_{\C^J}(\bt,\pp_*(\ ))\in(\C^I)^\vee_\U$ for \emph{some} universe $\U$ such that $\C^J$ is a $\U$-category,

\nn(b) $\pp^\dg\bt$ represents $\Hom_{\C^J}(\bt,\pp_*(\ ))\in(\C^I)^\vee_\U$ for \emph{any} universe $\U$ such that $\C^J$ is a $\U$-category. 

\begin{df}[Definition 2.3.2 p. 51]\lb{232} 
If the above equivalent conditions hold, we say that $\pp^\dg\bt$ {\em exists} (this is compatible with \S\ref{s232} (a) p.~\pr{s232}). If $\pp^\dg(F\ci\bt)$ exists and $F$ commutes with $\pp^\dg$ (Definition~\ref{kcw}) at $\bt$ for all functor $F:\C\to\C'$, we say that $\pp^\dg\bt$ exists {\em universally}.
\end{df} 
\end{s}

%%

\sbs{Theorem 2.3.3 (i) p.~52}

Note that projective and inductive limits are particular cases of Kan extensions.

Recall the statement: 

\begin{thm}[Theorem 2.3.3 (i) p.~52]\lb{233i}
Let $I\xl\pp J\xr\bt\C$ be functors. Assume that 
$$
\col_{(\pp(j)\to i)\in J_i}\bt(j)
$$ 
exists in $\C$ for all $i$ in $I$. Then $\pp^\dg(\bt)$ exists and we have 
\begin{equation}\lb{236}
\pp^\dg(\bt)(i)\iso\col_{(\pp(j)\to i)\in J_i}\bt(j)
\end{equation} 
for all $i$ in $I$. In particular, if $\C$ admits small inductive limits and $J$ is small, then $\pp^\dg$ exists. If moreover $\pp$ is fully faithful, then $\pp^\dg$ is fully faithful and the co-unit $\ee_\bt:\id_{\C^J}\to\pp_*\ci\pp^\dg$ (see \S\ref{ldefat} p.~\pr{ldefat}) is an isomorphism.%$\id_{\C^J}\xr\sim\pp_*\ci\pp^\dg$. 
\end{thm}

The proof in the book is divided into three steps, called (a), (b) and (c). 

\subsubsection{Step~(a)}\lb{scji}

% old version:
%https://docs.google.com/document/d/1CSDj0ZCUrvizbqLCHeJn5rkrra1IUVZ57wqqizF2J3U/edit

We define $\pp^\dg(\bt)$ by \qr{236}. The purpose of Step~(a) is to show that $\pp^\dg(\bt)$ is indeed a functor. 

% For another version, see
% https://docs.google.com/document/d/1wI4jmZbaMMD-f_EslRMPowUqU5J9sIUOkwo5NqTBMY8/edit  

For the reader's convenience we reproduce the argument in the book: 

Let $i\to i'$ be a morphism in $I$. It is easily checked that there is a unique morphism $\pp^\dg(\bt)(i)\to\pp^\dg(\bt)(i')$ which make the diagrams 
$$
\begin{tikzcd}
\pp^\dg(\bt)(i)=\ds\col_{(\pp(j)\to i)\in J_i}\bt(j)\ar[rr]&&\ds\col_{(\pp(j)\to i')\in J_{i'}\bt(j)}=\pp^\dg(\bt)(i')\\ 
&\bt(j)\ar[lu,"p(\pp(j)\to i)"]\ar[ru,"q(\pp(j)\to i\to i')"']
\end{tikzcd}
$$ 
commute, where $p(\pp(j)\to i)$ and $q(\pp(j)\to i\to i')$ are the coprojections, and that the assignment $$(i\to i')\mt\Big(\pp^\dg(\bt)(i)\to\pp^\dg(\bt)(i')\Big)$$ is functorial.

%

\subsubsection{Step~(b)}

The purpose of Step~(b) is to prove 
\begin{equation}\lb{stepb}
\Hom_{\C^I}(\pp^\dg(\bt),\al)\iso\Hom_{\C^J}(\bt,\pp_*(\al)) 
\end{equation} 
for all $\al:I\to\C$. As pointed out in the book, this can also be achieved by using Lemma 2.1.15 p.~42. Here is a sketch of the argument. We start with a reminder of Lemma 2.1.15. 

To any category $\A$ we attach the category $\Mor_0(\A)$ defined as follows. The objects of $\Mor_0(\A)$ are the triples $(X,f,Y)$ such that $f$ is a morphism in $\C$ from $X$ to $Y$. The morphisms in $\Mor_0(\A)$ from $(X,f,Y)$ to $(X',f',Y')$ are the pairs $(u,v)$ with $u:X\to X'$, $v:Y'\to Y$, and $f=v\ci f'\ci u$:
$$
\begin{tikzcd}
X\ar{d}[swap]{u}\ar{r}{f}&Y\\ 
X'\ar{r}[swap]{g}&Y'\ar{u}[swap]{v}.
\end{tikzcd}
$$ 
The composition of morphisms is the obvious one. Lemma 2.1.15 can be stated as follows: 

If $I$ and $\A$ are categories, and $a,b:I\parar\A$ are functors, then 
$$
(i,i\to j,j)\mt\Hom_\A(a(i),b(j))
$$ 
is a functor from $\Mor_0(I)^{\op}$ to $\Set$, and there is a natural isomorphism 
\begin{equation}\lb{2115} 
\Hom_{\A^I}(a,b)\xr\sim\lim_{(i\to j)\in\Mor_0(I)}\Hom_\A(a(i),b(j)).
\end{equation}

Returning to \qr{stepb}, we have functors 
$$
\begin{tikzcd}
J\ar{rr}{\pp}\ar{dr}[swap]{\bt}&&I\ar{dl}{\al}\\ 
&\C.
\end{tikzcd}
$$ 
Let us define the categories $M$ and $N$ as follows: an object of $M$ is a pair 
$$
(j,\pp(j)\to i\to i')
$$ 
with $j$ in $J$ and $i,i'$ in $I$. A morphism 
$$
\Big(j_1,\pp(j_1)\to i_1\to i'_1\Big)\to\Big(j_2,\pp(j_2)\to i_2\to i'_2\Big)
$$ 
is given by a triple of morphisms $j_1\to j_2,i_1\to i_2,i'_1\leftarrow i'_2$ such that the obvious diagram commutes. The category $N$ is $\Mor_0(J)$. Consider the functors 
$$
\gamma:M^{\op}\to\Set,\quad\Big(j,\pp(j)\to i\to i'\Big)\mt\Hom_\C\big(\bt(j),\al(i')\big), 
$$ 
$$
\delta:N^{\op}\to\Set,\quad(j\to j')\mt\Hom_\C\big(\bt(j),\al(\pp(j'))\big). 
$$ 
The existence of a natural bijection 
\begin{equation}\lb{nb1}
\Hom_{\C^J}\big(\bt,\pp_*(\al)\big)\xr\sim\lim\delta
\end{equation} 
follows immediately from \qr{2115}. Using \qr{2115} again, it is easy to see that we also have a natural bijection 
\begin{equation}\lb{nb2}
\Hom_{\C^I}\big(\pp^\dg(\bt),\al\big)\xr\sim\lim\gamma.
\end{equation} 
By \qr{nb1} and \qr{nb2}, it suffices to show 

\begin{lem}
There is a natural bijection $\lim\gamma\iso\lim\delta$. 
\end{lem} 

\begin{proof}
To define a map $\lim\gamma\to\lim\delta$, we attach, to a family 
$$
\big(\bt(j)\to\al(i')\big)_{\pp(j)\to i\to i'}\in\lim\gamma
$$ 
and to a morphism $j\to j'$, a morphism $\bt(j)\to\al(\pp(j'))$ by setting 
$$
i:=i':=\pp(j'),\quad(i\to i'):=\id_{\pp(j')},
$$ 
and by taking as $\bt(j)\to\al(\pp(j'))$ the corresponding member of our family. We leave it to the reader to check that this defines indeed a map $\lim\gamma\to\lim\delta$. To define a map $\lim\delta\to\lim\gamma$, we attach, to a family 
$$
\big(\bt(j)\to\al(\pp(j'))\big)_{j\to j'}\in\lim\delta
$$ 
and to a chain of morphisms $\pp(j)\to i\to i'$, a morphism $\bt(j)\to\al(i')$ by setting 
$$
j':=j,\quad(j\to j'):=\id_{j},
$$ 
and by taking as $\bt(j)\to\al(i')$ the composition 
$$
\bt(j)\to\al(\pp(j))\to\al(i)\to\al(i'). 
$$ 
We leave it to the reader to check that this defines indeed a map $\lim\delta\to\lim\gamma$, and that this map is inverse to the map constructed above.
\end{proof}

\subsubsection{Step~(c)}

The purpose of Step~(c) is to prove the last two sentences of the statement of Theorem 2.3.3 (i) p.~52 of the book (stated above as Theorem~\ref{233i} p.~\pr{233i}). We assume that $\C$ admits small inductive limits, that $J$ is small (in particular $\pp^\dg$ exists), and that $\pp$ is fully faithful. We must show that $\pp^\dg$ is fully faithful, and that the co-unit $\ee_\bt:\id_{\C^J}\to\pp_*\ci\pp^\dg$ (see \S\ref{ldefat} p.~\pr{ldefat}) is an isomorphism. %there is an isomorphism $\id_{\C^J}\xr\sim\pp_*\ci\pp^\dg$. 
Recall that we have 
$$
\pp^\dg(\bt)(i):=\col_{(\pp(j)\to i)\in J_i}\bt(j),
$$ 
and let 
$$
p(\pp(j)\to i):\bt(j)\to\pp^\dg(\bt)(i)
$$ 
be the coprojections. By the proof of Step~(b) in the book, the co-unit 
$$
\ee_{\bt,j}:\bt(j)\to\pp^\dg(\bt)(\pp(j))
$$ 
coincides with the coprojection $p(\pp(j)\xr\id\pp(j))$. We shall define a morphism $$\pp^\dg(\bt)(\pp(j))\to\bt(j)$$ and leave it to the reader to check that it is inverse to $\ee_{\bt,j}$. It suffices to define a functorial family of morphisms 
$$
%f(\pp(j')\to\pp(j)):\bt(\pp(j'))\to\bt(\pp(j))
f(\pp(j')\to\pp(j)):\bt(j')\to\bt(j)
$$ 
indexed by $(\pp(j')\to\pp(j))\in J_{\pp(j)}$. Let such a morphism $\pp(j')\to\pp(j)$ be given. As $\pp$ is fully faithful, we get a well-defined morphism $j'\to j$, and we set 
$$
f(\pp(j')\to\pp(j)):=\bt(j'\to j).
$$ 
It is straightforward to verify that the family of morphisms defined this way is functorial.

% removed
% https://docs.google.com/document/d/1MG8V4VHz0r4RCvKSMNpsFtb8frv1A_9RC6CPuHwMXE0/edit

%%

\subsubsection{A Corollary}

Here is a corollary to Theorem~\ref{233i} p.~\pr{233i} (which is Theorem 2.3.3 (i) p.~52 of the book):

\begin{cor}\lb{c233i}
If, in the setting of Theorem \ref{233i}, we have $\C=\Set$ and $J$ is small, then $\pp^\dg(\bt)(i)$ is (in natural bijection with) the quotient of 
$$
\bigsqcup_{j\in J}\ \bt(j)\tm\Hom_I(\pp(j),i) 
$$ 
by the smallest equivalence relation $\sim$ satisfying the following condition: If $j\to j'$ is a morphism in $J$, if $x$ is in $\bt(j)$, and if $\pp(j')\to i$ is a morphism in $I$, then 
$$
p_j(x,\pp(j)\to\pp(j')\to i)\sim p_{j'}(\bt(j\to j')(x),\pp(j')\to i),
$$ 
where $p_j$ is the $j$-coprojection. 
\end{cor}

\begin{proof}
Recall that Theorem~\ref{233i} p.~\pr{233i} states the existence of an isomorphism 
$$
\pp^\dg(\bt)(i)\iso\col_{(\pp(j)\to i)\in J_i}\bt(j).
$$
By Proposition 2.4.1 p.~54 of the book, the right-hand side is, in a natural way, the quotient of 
$$
\bigsqcup_{(\pp(j)\to i)\in J_i}\ \bt(j)
$$ 
by a certain equivalence relation. We have 
$$
\bigsqcup_{(\pp(j)\to i)\in J_i}\bt(j)=\bigsqcup_{j\in J}\ \bigsqcup_{u\in\Hom_I(\pp(j),i)}\ \bt(j)\iso\bigsqcup_{j\in J}\bt(j)\tm\Hom_I(\pp(j),i),
$$ 
and it easy to see that the three data of the above bijection, of the equivalence relation in Proposition 2.4.1 of the book, and of the equivalence relation in Corollary~\ref{c233i} above are compatible.
\end{proof}

Under the same assumptions $\pp^\ddg(\bt)(i)$ is (in natural bijection with) the set of all $x$ in 
$$
\prod_{(i\to\pp(j))\in J^i}\bt(j)
$$ 
such that $x_{i\to\pp(j)\to\pp(j')}=\bt(j\to j')(x_{i\to\pp(j)})$ for all morphism $j\to j'$ in $J$.

%%

\sbs{Brief comments}

\begin{s}
P.~53, Corollary 2.3.4. Recall that we have functors $\C\xl\bt J\xr\pp I$, where $I$ and $J$ are small and $\C$ admits small inductive limits, and that Corollary 2.3.4 says that we have a natural isomorphism $\col\bt\iso\col\pp^\dg\bt$, that is 
\begin{equation}\lb{coco}
\col\bt\iso\col_i\ \col_{(j,u)\in J_i}\bt(j),
\end{equation} 
where $(j,u)$ runs over $J_i$, with $u:\pp(j)\to i$. 

\begin{proof}[Proof of \qr{coco}]
We define morphisms 
$$
\begin{tikzcd} 
\col\bt\ar[yshift=0.7ex]{r}{f}&\ds\col_i\ \col_{(j,u)\in J_i}\bt(j),\ar[yshift=-0.7ex]{l}{g}
\end{tikzcd}
$$ 
and claim that $f$ and $g$ are inverse isomorphisms. We have the coprojections 
$$
\bt(j)\xr{p_j}\col\bt,\qquad\bt(j)\xr{q_{i,j,u}}\col_{(j',u')\in J_i}\bt(j')\xr{r_i}\col_{i'}\ \col_{(j'',u'')\in J_{i'}}\bt(j'').
$$ 

We define $f$ by the condition that we have 
$$
f\ci p_j=r_{\pp(j)}\ci q_{\pp(j),j,\id_{\pp(j)}}
$$ 
for all $j$ in $J$: 
$$
\begin{tikzcd}
\bt(j)\ar{dd}[swap]{p_j}\ar{r}{\id}&\bt(j)\ar{d}{q_{\pp(j),j,\id_{\pp(j)}}}\\ 
&\ds\col_{(j',u)\in J_{\pp(j)}}\bt(j')\ar{d}{r_{\pp(j)}}\\ 
\col\bt\ar{r}[swap]{f}&\ds\col_i\col_{(j',u)\in J_i}\bt(j').
\end{tikzcd}
$$

To define $g$, we form the commutative diagram
\begin{equation}\lb{biju}
\begin{tikzcd}
\bt(j)\ar{d}[swap]{q_{i,j,u}}\ar{r}{\id}&\bt(j)\ar{d}{p_j}\\ 
\ds\col_{(j',u')\in J_i}\bt(j')\ar{d}[swap]{r_i}\ar{r}{g_i}&\col\bt\ar{d}{\id}\\ 
\ds\col_{i'}\col_{(j',u')\in J_{i'}}\bt(j')\ar{r}[swap]{g}&\col\bt.
\end{tikzcd}
\end{equation} 
as follows: We let $i$ be in $I$ and define $g_i$ by the condition that the top square of \qr{biju} commutes for all $(j,u)\in J_i$. Then we define $g$ by the condition that the bottom square of \qr{biju} commutes for all $i$. 

Let us prove that $f\ci g$ is the identity of $\col_i\col_{(j,u)\in J_i}\bt(j)$. We have 
$$
f\ci g\ci r_i\ci q_{i,j,u}=f\ci p_j=r_{\pp(j)}\ci q_{\pp(j),j,\id_{\pp(j)}}
$$ 
for all $i\in J,(j,u)\in J_i$. Let $i\in J,(j,u)\in J_i$. It suffices to show 
\begin{equation}\lb{ri}
r_i\ci q_{i,j,u}=r_{\pp(j)}\ci q_{\pp(j),j,\id_{\pp(j)}},
\end{equation} 
that is, it suffices to show that the diagram
$$ 
\begin{tikzcd}
\bt(j)\ar{d}[swap]{q_{\pp(j),j,\id_{\pp(j)}}}\ar{r}{\id}&\bt(j)\ar{d}{q_{i,j,u}}\\ 
\ds\col_{(j',u')\in J_{\pp(j)}}\bt(j')\ar{d}[swap]{r_{\pp(j)}}\ar{r}{u_*}&\ds\col_{(j',u')\in J_i}\bt(j')\ar{d}{r_i}\\ 
\ds\col_{i'}\col_{(j',u')\in J_i}\bt(j')\ar{r}[swap]{\id}&\ds\col_{i'}\col_{(j',u')\in J_i}\bt(j'),
\end{tikzcd}
$$ 
where $u_*$ denotes the morphism induced by $u$, commutes. The top square commutes by  definition of $u_*$, and the bottom square commutes for obvious reasons. 

We leave the proof of the fact that $g\ci f$ is the identity of $\col\bt$ to the reader. 
\end{proof}
% https://docs.google.com/document/d/1DP4eUKYtcotlHxMne2kaU4Y6in0XlU0P8xHnB303rZU/edit

Display \qr{coco} p.~\pr{coco} above just says that there exists an isomorphism between two given objects of $\C$. But the proof proves much more than that! The proof indeed exhibits a morphism from the first object to the second, a morphism from the second to the first, and a proof that these two morphisms are inverse isomorphisms. When we invoke \qr{coco} in a subsequent argument, we shall often tacitly refer not only to the mere display \qr{coco}, but also to the two morphisms involved in its proof. In many cases it will be clear that the mere existential statement \qr{coco} wouldn't suffice to make the argument in question work, and that the invocation of the \emph{proof} of \qr{coco} is crucial. Such a situation will happen so often that we think it advisable to issue a general warning:

\begin{warning}\lb{warning1}
When we invoke a previously proved statement, we tacitly understand once and for all that the \emph{proof} of the statement in question is also implicitly invoked.
%The proof of a statement often proves much more 
\end{warning}
\end{s} 

% 

%\begin{s} P.~54, end of Section 2.3. Let \begin{equation}\C\xl\bt K\xr\psi J\xr\pp I\end{equation} be a diagram of functors. Assume that $I,J$ and $K$ are small, and that $\C$ admits small projective limits. Then Theorem 2.3.3 (ii) p.~52 of the book implies that the functors $\pp^\ddg(\psi^\ddg(\bt))$ and $(\pp\ci\psi)^\ddg(\bt)$ from $I$ to $\C$ exist and are naturally isomorphic. \end{s}

%

\sbs{Kan extensions of modules}

Let $R$ be a ring, let $\U$ and $\V$ be universes such that $R\in\U\in\V$, put, with self-explanatory notation, 
$$
I:=\Mod^\U(R),\quad\C:=\Mod^\V(R),
$$ 
let $J$ be the full subcategory of $I$ whose single object is $R$, and let $\C\xl\bt J\xr\pp I$ be the inclusion functors. We identify $\Hom_R(R,M)$ to $M$ whenever convenient. 

We claim that the functor $\pp^\dg(\bt):I\to\C$ satisfies 
\begin{equation}\lb{pdb1}
\pp^\dg(\bt)(M)\iso M.
\end{equation} 

To prove \qr{pdb1}, set 
$$
M':=\col_{(x:R\to M)\in J_M}R\in\C, 
$$ 
and let $p_x:R\to M'$ be the coprojections. As Theorem 2.3.3 (i) p.~52 of the book (stated above as Theorem~\ref{233i} p.~\pr{233i}) implies $M'\iso\pp^\dg(\bt)(M)$, it suffices to prove $M'\iso M$. We define a family of linear maps $\Phi_x:R\to M$, indexed by $x:R\to M$, by setting $\Phi_x:=x$, and leave it to the reader to check that the $\Phi_x$ induce a linear map $\Phi:M'\to M$. We define the set theoretic map $\Psi:M\to M'$ by putting $\Psi(x):=p_x(1)$, and leave it to the reader to verify that $\Phi$ and $\Psi$ are mutually inverse bijections. This proves \qr{pdb1}. 

We claim that the functor $\pp^\ddg(\bt):I\to\C$ satisfies 
\begin{equation}\lb{pddb1}
\pp^\ddg(\bt)(M)\iso M^{**}, 
\end{equation} 
where $M^{**}$ is the double dual of $M$. 

To prove \qr{pddb1}, set 
$$
M':=\lim_{(f:M\to R)\in J^M}R\in\C, 
$$ 
and let $p_f:M'\to R$ be the projections. As Theorem 2.3.3 (ii) p.~52 of the book implies $M'\iso\pp^\ddg(\bt)(M)$, it suffices to prove $M'\iso M^{**}$. We define a family of linear maps $\Phi_f:M^{**}\to R$, indexed by $f:M\to R$, by setting $\Phi_f(F):=F(f)$, and leave it to the reader to check that the $\Phi_f$ induce a linear map $\Phi:M^{**}\to M'$. We define the linear map $\Psi:M'\to M^{**}$ by putting $\Psi((\ld_f))(g):=\ld_g$, and leave it to the reader to verify that $\Phi$ and $\Psi$ are mutually inverse linear bijections. This proves \qr{pddb1}. 

Let $R$, $\U$ and $\V$ be as above, put, with self-explanatory notation, 
$$
I:=\Mod^\U(R)^{\op},\quad\C:=\Mod^\V(R^{\op}),
$$ 
let $J$ be the full subcategory of $I$ whose single object is $R$, let $\pp:J\to I$ be the inclusion functor, and let $\bt:J\to\C$ be the obvious functor satisfying $\bt(R)=R^{\op}$. 

We claim that the functor $\pp^\dg(\bt):I\to\C$ satisfies 
\begin{equation}\lb{pdb2}
\pp^\dg(\bt)(M)\iso M^*,
\end{equation} 
where $M^*$ is the dual of $M$. 

To prove \qr{pdb2}, set 
$$
M':=\col_{(R\to M)\in J_M}R^{\op}=\col_{(f:M\to R)\in(J^{\op})^M}R^{\op}\in\C, 
$$ 
and let $p_f:R^{\op}\to M'$ be the coprojections. As Theorem 2.3.3 (i) p.~52 of the book (stated above as Theorem~\ref{233i} p.~\pr{233i}) implies $M'\iso\pp^\dg(\bt)(M)$, it suffices to prove $M'\iso M^*$. We define a family of linear maps $\Phi_f:R^{\op}\to M^*$, indexed by $f:M\to R$, by setting $\Phi_f(1):=f$, and leave it to the reader to check that the $\Phi_f$ induce a linear map $\Phi:M'\to M^*$. We define the set theoretic map $\Psi:M^*\to M'$ by putting $\Psi(f):=p_f(1)$, and leave it to the reader to verify that $\Phi$ and $\Psi$ are mutually inverse bijections. This proves \qr{pdb2}. 

We claim that the functor $\pp^\ddg(\bt):I\to\C$ satisfies 
\begin{equation}\lb{pddb2}
\pp^\ddg(\bt)(M)\iso M^*, 
\end{equation} 
where $M^*$ is the dual of $M$. 

To prove \qr{pddb2}, set 
$$
M':=\lim_{(M\to R)\in J^M}R^{\op}=\lim_{(x:R\to M)\in(J^{\op})_M}R^{\op}\in\C, 
$$ 
and let $p_x:M'\to R^{\op}$ be the projections. As Theorem 2.3.3 (ii) p.~52 of the book implies $M'\iso\pp^\ddg(\bt)(M)$, it suffices to prove $M'\iso M^*$. We define a family of linear maps $\Phi_x:M^*\to R^{\op}$, indexed by $x:R\to M$, by setting $\Phi_x(f):=f(x)$, and leave it to the reader to check that the $\Phi_x$ induce a linear map $\Phi:M^*\to M'$. We define the linear map $\Psi:M'\to M^*$ by putting $\Psi((\ld_x))(y):=\ld_y$, and leave it to the reader to verify that $\Phi$ and $\Psi$ are mutually inverse linear bijections. This proves \qr{pddb2}. 

%% 

\sbs{Brief comments}

\begin{s} 
P.~55, proof of Corollary 2.4.4 (iii) (minor variant).

\begin{prop}
If $I$ is a small category, if $S$ is in $\Set$ and $\DT S:I\to\Set$ is the corresponding constant functor, then there is a canonical bijection
$$
\col\DT S\iso\pi_0(I)\tm S.
$$ 
(See Notation~\ref{diag} p.~\pr{diag}.)
\end{prop} 

\begin{proof}
On the one hand we have 
$$
\pi_0(I):=\Ob(I)/\!\!\sim\ , 
$$
where $\sim$ is the equivalence relation defined on p.~18 of the book. On the other hand we have by Proposition 2.4.1 p.~54 of the book 
$$
\col\DT S\iso(\Ob(I)\tm S)/\!\!\approx\ ,
$$
where $\approx$ is the equivalence relation described in the proposition. In view of the definition of $\approx$ and $\sim$, we get 
$$
(i,s)\approx(j,t)\ \ssi\ [i\sim j\text{ and }s=t].
$$ 
\end{proof}
\end{s}

%%

\sbs{Corollary 2.4.6 p. 56}

Recall the statement: 

\begin{prop}[Corollary 2.4.6 p. 56]\lb{246}
If $X'$ and $X''$ are objects, if $\C$ and $\C'$ are categories, and if $F$ and $G$ are functors satisfying 
\begin{equation}\lb{241s}
X'\in\C'\xl{F}\C\xr{G}\C''\ni X'', 
\end{equation} 
then we have 
\begin{equation}\lb{241} 
\col_{(G(X)\to X'')\in\C_{X''}}\Hom_{\C'}(X',F(X))\iso\col_{(X'\to F(X))\in(\C^{X'})^{\op}}\Hom_{\C''}(G(X),X''). 
\end{equation} 
\end{prop}

\begin{proof} 
Consider the diagram 
$$
\begin{tikzcd}
\ds\col_{(G(X)\to X'')\in\C_{X''}}\Hom_{\C'}(X',F(X))\ar[r,yshift=0.7ex,dashed,"f"]&\ds\col_{(X'\to F(X))\in(\C^{X'})^{\op}}\Hom_{\C''}(G(X),X'')\ar[l,yshift=-0.7ex,dashed,"g"]\\ 
\Hom_{\C'}(X',F(X))\ar{u}{p[G(X)\to X'']}&\Hom_{\C''}(G(X),X'')\ar{u}[swap]{q[X'\to F(X)]},
\end{tikzcd}
$$ 
where the vertical arrows are the coprojections. We leave it to the reader to check firstly that there are maps $f$ and $g$ as in the above diagram satisfying 
$$
f\Big(p\big[G(X)\to X''\big]\big(X'\to F(X)\big)\Big):=q\big[X'\to F(X)\big]\big(G(X)\to X''\big),
$$ 
$$
g\Big(q\big[X'\to F(X)\big]\big(G(X)\to X''\big)\Big):=p\big[G(X)\to X''\big]\big(X'\to F(X)\big)
$$ 
for all morphism $G(X)\to X''$ in $\C''$ and all morphism $X'\to F(X)$ in $\C'$, and secondly that $f$ and $g$ are inverse bijections. 
\end{proof}

% Previous version:
% https://docs.google.com/document/d/1yUAj7vLmCXltxxrSCCzPzjCDDJgoEMOIs7JHmVj07Ig/edit

%%

\sbs{Brief comments}

\begin{s} 
P.~56, proof of Lemma 2.4.7 (minor variant).

\begin{lem} 
If $I$ is a small category, $i_0$ is in $I$, and $\al:I\to\Set$ is the functor $\Hom_I(i_0,\ )$, then $\col\al$ is a terminal object of $\Set$. 
\end{lem}

\begin{proof}
We shall use \qr{cue} p.~\pr{cue}. Let $X=\{x\}$ be a terminal object of $\Set$, let $p:\al\to\DT X$ be the unique morphism from $\al$ to $\DT X$, let $\theta:\al\to\DT Y$ be a morphism in $\Set^I$ (with $Y$ in $\Set$), and let us show that there is a unique map $f:X\to Y$ such that $\DT f\ci p=\theta$: 
$$
\begin{tikzcd}
\al\ar[r,"p"]\ar[dr,"\theta"']&\DT X\ar[d,"\DT f"]&X\ar[d,dashed,"f"]\\ 
&\DT Y&Y.
\end{tikzcd}
$$ 
Unsurprisingly it is better to start with the uniqueness: any such $f$ must satisfy $f(x)=\theta_{i_0}(\id_{i_0})$. For the existence, it is easy to see that the map $f$ defined by the above equality does the job.
\end{proof}

Here is a second proof:

\begin{proof}
Each element of $\col_{i\in I}\Hom_I(i_0,i)$ is represented by some morphism $i_0\to i$ in $I$. Moreover, a composition of the form $i_0\to i\to j$ represents the same element as 
$i_0\to i$. In particular $i_0\to i$ represents the same element as $\id_{i_0}$. 
\end{proof}
\end{s}

%

\begin{s}\lb{252}
P. 57, proof of Proposition 2.5.2 (minor variant). Instead of proving (i)$\then$(v), we prove (i)$\then$(ii), that is, we prove the following statement:

\begin{lem}
If $\pp:J\to I$ and $\bt:I^{\op}\to\Set$ are functors defined on small categories and if the category $J^i$ is connected for all $i$ in $I$, then the natural map 
$$
f:\lim\bt\to\lim\bt\ci\pp^{\op}
$$ 
is bijective. 
\end{lem}

\begin{proof}
We shall define a map 
$$
g:\lim\bt\ci\pp^{\op}\to\lim\bt
$$ 
and leave it to the reader to check that $f$ and $g$ are inverse. Let $y$ be in $\lim\bt\ci\pp^{\op}$. In particular $y$ is of the form $(y_j)_{j\in J}$ with $y_j\in\bt(\pp(j))$. We must define the element $g(y)_i$ in $\bt(i)$, where $i$ is an arbitrary element in $I$. Let us choose a morphism $i\to\pp(j)$ in $I$. It suffices to show that the element $\bt(i\to\pp(j))(y_j)$ in $\bt(i)$ does not depend on the choice of $i\to\pp(j)$, enabling us to set $$g(y)_i:=\bt(i\to\pp(j))(y_j).$$ Given another choice $i\to\pp(j')$, we must prove $$\bt(i\to\pp(j))(y_j)=\bt(i\to\pp(j'))(y_{j'}).$$ As $J^i$ is connected, we may assume that there is a morphism $j\to j'$ in $J$ such that $(i\to\pp(j'))=(i\to\pp(j)\to\pp(j'))$, and the proof is straightforward. %and leave it to the reader to check that $f$ and $g$ are inverse. Let $y$ be in $\lim\bt\ci\pp^{\op}$. In particular $y$ is of the form $(y_j)_{j\in J}$ with $y_j\in\bt(\pp(j))$. We must define the element $g(y)_i$ in $\bt(i)$, where $i$ is an arbitrary element in $I$. Let us choose a morphism $u:i\to\pp(j)$ in $I$. It suffices to show that the element $\bt(u)(y_j)$ in $\bt(i)$ does not depend on the choice of $u$, so that we shall be able to set $g(y)_i:=\bt(u)(y_j)$. Given another choice $u':i\to\pp(j')$, we must prove $\bt(u)(y_j)=\bt(u')(y_{j'})$. As $J^i$ is connected, we may assume that there is a morphism $j\to j'$ in $J^i$, and the proof is straightforward. 
\end{proof}
\end{s}

%

\begin{s} 
P.~58, implication (vi)$\then$(i) of Proposition 2.5.2. Here is a slightly stronger statement:

\begin{prop} 
If $\pp:J\to I$ is a functor, then the obvious map
\begin{equation}\lb{om}
\col\Hom_I(i,\pp)\to\pi_0(J^i)
\end{equation}
is bijective. 
\end{prop}

\begin{proof} 
Let $L_i$ be the left-hand side of \qr{om}, and, for $j$ in $J$, let 
$$
p_j:\Hom_I(i,\pp(j))\to L_i
$$
be the coprojection. It is easy to check that the map 
$$
\Ob(J^i)\to L_i,\quad(j,i\to\pp(j))\mt p_j(i\to\pp(j))
$$
factors through $\pi_0(J^i)$, and that the induced map $\pi_0(J^i)\to L_i$ is inverse to (\ref{om}). %$$\Ob(J^i)\to L_i,\quad\Big(j,s:i\to\pp(j)\Big)\mt p_j(s)$$ factors through $\pi_0(J^i)$, and that the induced map $\pi_0(J^i)\to L_i$ is inverse to (\ref{om}).
\end{proof}
\end{s}

%

\sbs{Proposition 2.6.3 (i) p.~61}

Let $\C$ be a category and let $A$ be in $\C^\wg$. Consider the statements
\begin{equation}\lb{263a}
\ic_{(X\to A)\in\C_A}X\xr\sim A,
\end{equation} 

\begin{equation}\lb{263b}
\col_{(X\to A)\in\C_A}\Hom_\C(Y,X)\xr\sim A(Y)\text{ for all }Y\in\C, 
\end{equation}

\begin{equation}\lb{263c}
\Hom_{\C^\wg}(A,B)\xr\sim\lim_{(X\to A)\in\C_A}B(X)\text{ for all }B\in\C^\wg. 
\end{equation}

We prove \qr{263a}, \qr{263b} and \qr{263c} in \S\ref{263p} p.~\pr{263p} below. [Note that Warning~\ref{warning1} p.~\pr{warning1} applies particularly well to \qr{263a}, \qr{263b} and \qr{263c}.] %[Clearly, \qr{263b} implies \qr{263a} and \qr{263c} --- see \S\ref{216} p.~\pr{216}. See \S\ref{c38} p.~\pr{c38} for the relationship between \qr{263a}, \qr{263b} and \qr{263c}.] %, and the proof of \qr{263b} is straightforward.

Note that \qr{263a} can be stated as follows: If $h:\C\to\C^\wg$ is the Yoneda embedding, then the natural morphism $h^\dg(h)\to\id_{\C^\wg}$ is an isomorphism. This implies in particular that \qr{263a} is functorial in $A$.

% previous version
%https://docs.google.com/document/d/1WNqMtfDDCRbs9WqNffO6yj37ck60FBy6xxHGA8Cf764/edit

\begin{s}\lb{263p}
We shall prove \qr{263a}, \qr{263b} and \qr{263c}. More precisely, we shall spell out these three isomorphisms in terms of Diagram~\qr{cue} p.~\pr{cue} and Diagram~\qr{yfy} p.~\pr{yfy}). 

Warning: In this proof the symbols $X$ and $Y$ will designate either two objects of $\C$ or the image of these objects in $\C^\wg$. The context only will tell which interpretation is the good one. (It seems to me the choice of the correct interpretation will always be obvious.)

\nn$\bu$ Isomorphism \qr{263a} can be decoded as follows: Consider the functor 
$$
\al:\C_A\to\C^\wg,\quad(X\to A)\mt X,
$$ 
and let $p:\al\to\DT A$ be the tautological morphism in $(\C^\wg)^{\C_A}$ defined by 
\begin{equation}\lb{pxa}
p_{X\to A}:=(X\to A)
\end{equation}
for all $X\to A$ in $\C_A$. Let $B$ be in $\C^\wg$ and $\theta:\al\to\DT B$. Diagram \qr{cue} p.~\pr{cue} becomes 
$$
\begin{tikzcd}
\al\ar[r,"p"]\ar[dr,"\theta"']&\DT A\ar[d,"\DT f"]&A\ar[d,dashed,"f"]\\ 
&\DT B&B.
\end{tikzcd}
$$  
The uniqueness of $f$ follows from the fact that the equality 
\begin{equation}\lb{fx1}
\DT f\ci p=\theta
\end{equation} 
implies 
\begin{equation}\lb{fx2} 
(X\to A\xr fB)=\theta_{X\to A}\quad\forall\quad X\to A,
%f\ci(X\to A)=\theta_{X\to A}\quad\forall\quad X\to A, 
\end{equation} 
and the existence of $f$ follows from the fact that \qr{fx2} implies \qr{fx1}. %we can define $f$ by \qr{fx2}. %and we can define $f$ by $f_X(X\to A):=\theta_{X\to A}$.

\nn$\bu$ Isomorphism \qr{263b} can be decoded as follows: Let $Y$ be in $\C$, consider the functor 
$$
\bt:\C_A\to\Set,\quad(X\to A)\mt\Hom_\C(Y,X), 
$$ 
and let $q:\bt\to\DT A(Y)$ be the morphism in $\Set^{\C_A}$ defined by 
$$
q_{X\to A}(Y\to X):=(Y\to X\to A)
$$ 
for all $X\to A$ in $\C_A$. Let $S$ be in $\Set$ and $Y$ in $\C$. Diagram \qr{cue} p.~\pr{cue} becomes 
$$
\begin{tikzcd}
\bt\ar[r,"q"]\ar[dr,"\theta"']&\DT A(Y)\ar[d,"\DT f"]&A(Y)\ar[d,dashed,"f"]\\ 
&\DT S&S.
\end{tikzcd}
$$ 
The equality $\DT f\ci q=\theta$ is equivalent to the condition 
\begin{equation}\lb{fy1}
f(Y\to X\to A)=\theta_{X\to A}(Y\to X)\quad\forall\quad Y\to X\to A. 
\end{equation}
%and we can define $f$ by $f(Y\to A):=\theta_{Y\to A}(\id_Y)$. 
Consider the condition 
\begin{equation}\lb{fy2}
f(Y\to A)=\theta_{Y\to A}(\id_Y)\quad\forall\quad Y\to A. 
\end{equation} 
The uniqueness of $f$ follows from the fact that \qr{fy1} implies \qr{fy2}, and the existence of $f$ follows from the fact that \qr{fy2} implies \qr{fy1}. 

\nn$\bu$ Isomorphism \qr{263c} can be decoded as follows: Let $B$ be in $\C^\wg$, consider the functor 
$$
\gamma:(\C_A)^{\op}\to\Set,\quad(X\to A)\mt B(X),
$$ 
let $r:\DT B(A)\to\gamma$ be the morphism in $\Set^{(\C_A)^{\op}}$ defined by 
$$
r_{X\to A}(A\to B):=(X\to A\to B)
$$ 
for all $X\to A$ in $\C_A$ and all $A\to B$ in $B(A)$, and let $S$ be in $\Set$. Diagram \qr{yfy} p.~\pr{yfy} becomes 
$$
\begin{tikzcd}
S\ar[d,dashed,"f"']&\DT S\ar[d,"\DT f"']\ar[rd,"\theta"]\\ 
B(A)&\DT B(A)\ar[r,"r"']&\gamma.
\end{tikzcd}
$$ 
The uniqueness of $f$ follows from the fact that the equality 
\begin{equation}\lb{fr1}
r\ci\DT f=\theta
\end{equation} 
implies 
\begin{equation}\lb{fr2}
(X\to A\xr{f(s)}B)=\theta_{X\to A}(s)\quad\text{for all $s$ and all $X\to A$}, 
\end{equation} 
and the existence of $f$ follows from the fact that \qr{fr2} implies \qr{fr1}. q.e.d.
%and we can define $f$ by $f(s)_X(X\to A):=\theta_{X\to A}(s)$.
%Isomorphisms \qr{263a}, \qr{263b} and \qr{263c} are functorial. This fact is probably best expressed in terms of the (co)projections. Let us state explicitly the functoriality of \qr{263a}. 

%Let $\al_A$ and $p_A$ be the morphisms denoted above by $\al$ and $p$, let $B$ be in $\C^\wg$, let $\al_B$ and $p_B$ be the obvious morphisms, let $A\to B$ be a morphism from $A$ to $B$, let $(A\to B)\ci:\C_A\to\C_B$ be the obvious morphisms induced by $A\to B$, and recall that $\star$ denotes horizontal composition. Then the diagram $$\begin{tikzcd}\al_A\ar[d,equal]\ar[rrr,"p_A"]&&&\DT A\ar[d,"\DT(A\to B)"]\\ \al_A\ar{rrr}[swap]{p_B\star((A\to B)\ci)}&&&\DT B\end{tikzcd}$$ commutes.
\end{s}

%%

\sbs{Brief comments}

\begin{s} 
% old version: 
% https://docs.google.com/document/d/1RrIqcflYiAihX62SEwaNTTYvF-Pv_LN-U-hB5LWQCOI/edit
P.~61, Proposition 2.6.3 (ii). Here is a Lemma implicitly used in the proof of Proposition 2.6.3 (ii): 

% old version (160509): 
% https://docs.google.com/document/d/167Wrv21BZvIOaVYRo1BDkEFQyjZ8mSHUJaWcHo_zWDU/edit

\begin{lem}\lb{cc}
If $\al:I\to\C$ is a functor admitting an inductive limit and if $\col\al\to Y$ is a morphism in $\C$, then we have 
$$
\col_{i\in I}\ (\al(i)\to\col\al\to Y)\iso(\col\al\to Y).
$$
More precisely, let $X$ be an object of $\C$, let $p:\al\to\DT X$ be a coprojection, let $f:X\to Y$ be a morphism in $\C$, and define $\bt:I\to\C_Y$ by $\bt(i):=(\al(i)\xr{p_i}X\xr fY)$ and $q:\bt\to\DT(X\xr fY)$ by $q_i:=p_i$: 
$$
\begin{tikzcd}
\al(i)\ar[rd,"f\ci p_i"']\ar[rr,"p_i"]&&X\ar[dl,"f"]\\ 
&Y.
\end{tikzcd}
$$ 
Then $q$ is a coprojection. 
\end{lem}

\begin{proof}
For all $Z$ in $\C$ and all morphism $\pp:\al\to\DT Z$ in $\C^I$ there is a unique morphism $g:X\to Z$ in $\C$ such that $\DT g\ci p=\pp$: 
$$
\begin{tikzcd}
\al\ar[r,"p"]\ar[dr,"\pp"']&\DT X\ar[d,"\DT g"]&X\ar[d,dashed,"g"]\\ 
&\DT Z&Z.
\end{tikzcd}
$$ 
Let $(Z\xr hY)$ be an object of $\C_Y$ and $\psi:\bt\to\DT(Z\xr hY)$ a morphism in $(\C_Y)^I$. We must show that there is a unique morphism $k:(X\xr fY)\to(Z\xr hY)$, which it is more suggestive to display as 
$$
\begin{tikzcd}
X\ar[rd,"f"']\ar[rr,"k"]&&Z\ar[dl,"h"]\\ 
&Y,
\end{tikzcd}
$$ 
such that $\DT k\ci q=\psi$: 
$$
\begin{tikzcd}
\bt\ar[r,"q"]\ar[dr,"\psi"']&\DT(X\xr fY)\ar[d,"\DT k"]&(X\xr fY)\ar[d,dashed,"k"]\\ 
&\DT(Z\xr hY)&(Z\xr hY).
\end{tikzcd}
$$ 
It is straightforward to check that the formulas $\pp_i:=\psi_i$ and $k:=g$ do the job. 
\end{proof}
\end{s}

% the \S with the label \lb{b} was here

\begin{s} 
Proposition~\ref{copr} yields the following minor variant of Theorem 2.3.3 (i) p.~52 of the book (stated above as Theorem~\ref{233i} p.~\pr{233i}). Let $I\xl\pp J\xr\bt\C$ be functors and let $\bt$ be in $\C^J$. 

\begin{thm}\lb{233}%[Theorem 2.3.3 (i) p. 52]
If 
\begin{equation}\lb{e233}
\col_{(\pp(j)\to i))\in J_i}\bt(j)
\end{equation}
exists in $\C$ for all $i$ in $I$, then $\pp^\dg\bt$ exists, and $(\pp^\dg\bt)(i)$ is isomorphic to \qr{e233}. If, in addition, \qr{e233} exists universally in the sense of Definition~\ref{uil} p.~\pr{uil}, then $\pp^\dg\bt$ exists universally in the sense of Definition~\ref{232} p.~\pr{232}. 
\end{thm}
\end{s}

%

\begin{s}\lb{c271b}
P.~62, Proposition 2.7.1. Consider the commutative diagram 
$$
\begin{tikzcd}
\C\ar{r}{\hy_\C}\ar{dr}[swap]{F}&\C^\wg\ar{d}{\widetilde F}&I\ar{l}[swap]{\al}\\
&\A,
\end{tikzcd}
$$
where $I$ is a small category and $\widetilde F$ satisfies 
$$
\widetilde F(A)\iso\col_{(X\to A)\in\C_A}F(X)
$$ 
for all $A$ in $\C^\wg$. Let us rewrite the proof of the fact that the natural morphism $\col\widetilde F\ci\al\to\widetilde F\left(\col\al\right)$ is an isomorphism. 

By Proposition 2.1.10 p. 40 of the book (stated on p.~\pr{2.1.10c} above as Corollary~\ref{2.1.10c}), it suffices to check that the functor $G:\A\to\C^\wg$ defined by 
$$
G(X')(X):=\Hom_{\A}(F(X),X').
$$ 
is right adjoint to $\widetilde F$. This results from the following computation: 
$$
\Hom_{\A}\left(\widetilde F(A),X'\right)\iso
\Hom_{\A}\left(\col_{(X\to A)\in\C_A}F(X),X'\right)\iso 
\lim_{(X\to A)\in\C_A}\Hom_{\A}(F(X),X')
$$
$$
=\lim_{(X\to A)\in\C_A}G(X')(X)\iso\Hom_{\C^\wg}(A,G(X')), 
$$ 
the last isomorphism following from \qr{263c} p.~\pr{263c}. q.e.d.
\end{s}

%

\begin{s}\lb{bil}
P.~62. In the setting of Proposition 2.7.1, the functors
$$
\A^\C\to\A,\quad F\mt(\oo h_\C^\dg F)(A)\quad\text{and}\quad
\C^\wg\to\A,\quad A\mt(\oo h_\C^\dg F)(A)
$$ 
commute with small inductive limits. 

Indeed, for the first functor the conclusion follows from the isomorphism 
\begin{equation}\lb{hfa}
(\oo h_\C^\dg F)(A)\iso\col_{(U\to A)\in\C_A}F(U),
\end{equation}
and, for the second functor it follows from Proposition 2.7.1 p.~62 of the book.
\end{s} 

%%

\sbs{Three formulas}

Here is a complement to Section 2.3 pp~52-54 of the book, complement which will be used in \S\ref{prepa5} p.~\pr{prepa5} to prove Proposition 17.1.9 p.~409 of the book. 

In this section we shall use the following notation: The Yoneda embedding $\C\to\C^\wg$ will be denoted by $h(\C)$, and the forgetful functor $\C_A\to\C$ by $j(\C_A)$: 
$$
h(\C):\C\to\C^\wg,\qquad j(\C_A):\C_A\to\C.
$$

\subsubsection{Preliminaries}

Let $\C$ be a category and $A$ an object of $\C^\wg$. Recall that there is a unique functor 
$$
\ld:(\C^\wg)_A\to(\C_A)^\wg
$$ 
such that 
\begin{equation}\lb{lambda}
\ld(B\to A)(U\to A)=\Hom_{(\C^\wg)_A}(U\to A,B\to A)
\end{equation} 
for all $(B\to A)$ in $(\C^\wg)_A$ and all $(U\to A)$ in $\C_A$. Moreover 
\begin{equation}\lb{leq}
\text{$\ld$ is an equivalence,}
\end{equation} 
and we have 
\begin{equation}\lb{lcom}
\ld\ci h(\C)_A\iso h(\C_A),
\end{equation}
that is, the diagram
$$
\begin{tikzcd}
\C_A\ar{r}{h(\C)_A}\ar{rd}[swap]{h(\C_A)}&(\C^\wg)_A\ar{d}{\ld}\\ 
&(\C_A)^\wg
\end{tikzcd}
$$ 
quasi-commutes. (See Lemma 1.4.12 p.~26 of the book.) 

The statement below follows from Proposition 2.7.1 p.~62 of the book:

\begin{prop}\lb{271}
Let $F:\C\to\A$ be a functor, assume that $\C$ is small and that $\A$ admits small inductive limits. Then the functor $h(\C)^\dg(F):\C^\wg\to\A$ exists, commutes with small inductive limits and satisfies $h(\C)^\dg(F)\ci h(\C)\iso F$.
\end{prop}

%For the reader's convenience we paste Lemma 2.1.13 (i) p.~41 of the book:

%\begin{lem}[Lemma 2.1.13 (i)]\lb{2113}
%Let $I$ be a category. Assume that $\C$ admits inductive limits indexed by $I$, and that $A:\C^{\op}\to\Set$ commutes with projective limits indexed by $I$ (i.e. $A(\col_{i\in I}X_i)\iso\lim_{i\in I} A(X_i)$ for any inductive system $(X_i)_{i\in I}$ in $\C$). Then $\C_A$ admits inductive limits indexed by $I$ and $j(\C_A):\C_A\to\C$ commutes with such limits.
%\end{lem}

%The following lemma will be useful:

% previous version
% https://docs.google.com/document/d/1h1cHxZ_Bdwh5o2DuO1et13-OkpRBe2bhjqtLgkhMbQY/edit

Let $F:\C\to\A$ be a functor and $A$ an object of $\C^\wg$, and assume that $\C$ is small and that $\A$ admits small inductive and projective limits.

\subsubsection{First Formula}

% previous version
% https://docs.google.com/document/d/1viwkdjeF7abC7eEjaRiufDMDTcof8h3Qc2pUzmVQ6zU/edit

We claim 

\begin{equation}\lb{prepa1}
h(\C_A)^\dg(F\ci j(\C_A))\ci\ld\iso h(\C)^\dg(F)\ci j((\C^\wg)_A)
\end{equation}

\nn(see the commutative diagram \qr{qcd} below).

\begin{proof} 
Consider the commutative diagram 
\begin{equation}\lb{qcd}
\begin{tikzcd}
\C_A\ar[bend left]{rrrr}{j(\C_A)}\ar{rr}{F\ci j(\C_A)}\ar{dd}[swap]{h(\C_A)}&&\A&&\C\ar{ll}[swap]{F}\ar{dd}{h(\C)}\\ 
{}\\ 
(\C_A)^\wg\ar{uurr}[swap]{h(\C_A)^\dg(F\ci j(\C_A))}&&(\C^\wg)_A\ar{ll}{\ld}\ar{rr}[swap]{j((\C^\wg)_A)}&&\C^\wg.\ar{uull}[swap]{h(\C)^\dg(F)}
\end{tikzcd}
\end{equation} 

For $B$ in $\C^\wg$ and $B\to A$ in $(\C^\wg)_A$, setting 

$$X:=h(\C_A)^\dg\Big(F\ci j(\C_A)\Big)\Big(\ld(B\to A)\Big),$$

\nn we get 

\begin{align*}
X&\iso h(\C_A)^\dg(F\ci j(\C_A))\left(\ld\left(\left(\col_{(U\to B)\in\C_B}h(\C)(U)\right)\to A\right)\right)&\text{by \qr{263a}}\\ 
&\iso h(\C_A)^\dg(F\ci j(\C_A))\left(\ld\left(\col_{(U\to B)\in\C_B}\Big(h(\C)(U)\to A\Big)\right)\right)&\text{by Lemma~\ref{cc}}\\ 
&\iso\col_{(U\to B)\in\C_B}h(\C_A)^\dg\Big(F\ci j(\C_A)\Big)\bigg(\ld\Big(h(\C)(U)\to A\Big)\bigg)&\text{by \qr{leq} \& Prop.~\ref{271}}\\ 
&\iso\col_{(U\to B)\in\C_B}h(\C_A)^\dg\Big(F\ci j(\C_A)\Big)\bigg(\ld\Big(h(\C)_A(U\to A)\Big)\bigg)\\ 
&\iso\col_{(U\to B)\in\C_B}h(\C_A)^\dg\Big(F\ci j(\C_A)\Big)\bigg(h(\C_A)(U\to A)\bigg)&\text{by \qr{lcom}}\\ 
&\iso\col_{(U\to B)\in\C_B}\big(F\ci j(\C_A)\big)(U\to A)&\text{by Prop.~\ref{271}}\\ 
&\iso\col_{(U\to B)\in\C_B}F(U)\\ 
&\iso\bigg(h(\C)^\dg(F)\ci j\Big((\C^\wg)_A\Big)\bigg)\left(B\to A\right)&\text{by \qr{hfa}.} 
\end{align*}
\end{proof}

%

\subsubsection{Second Formula}

\begin{prop}\lb{prepa2a}
Consider the quasi-commutative diagram 
$$
\begin{tikzcd}
\C_A\ar{rr}{j(\C_A)}\ar{dr}[swap]{G}&&\C\ar{dl}{j(\C_A)^\ddg(G)}\\ 
{}&\A,
\end{tikzcd}
$$ 
and let $U$ be in $\C$. Then we have 
\begin{equation}\lb{prepa2}
j(\C_A)^\ddg(G)(U)\iso\prod_{U\to A}G(U\to A).
\end{equation} 
\end{prop}

\begin{lem}\lb{cicau}
The discrete category $A(U)$ is cocofinal in $(\C_A)^U$.
\end{lem}

\begin{proof}[Proof of Lemma \ref{cicau}]
We probably give too many details, and the reader may want to skip this proof. An object $a$ of $(\C_A)^U$ is given by a triple 
$$
a=(U_a,U_a\xr{u_a}A;U\xr{s_a}U_a),
$$ 
and a morphism from $a$ to 
$$
b=(U_b,U_b\xr{u_b}A;U\xr{s_b}U_b)\in(\C_A)^U
$$ 
is given by a commutative diagram 
$$
\begin{tikzcd}
{}&U\ar{dl}[swap]{s_a}\ar{dr}{s_b}\\ 
U_a\ar{dr}[swap]{u_a}\ar{rr}{c}&&U_b\ar{dl}{u_b}\\ 
{}&A.
\end{tikzcd}
$$ 
The embedding $\pp:A(U)\to(\C_A)^U$ implicit in the statement of Lemma~\ref{cicau} is given by 
$$
\pp(u)=(U,U\xr uA;U\xr{\id_U}U). 
$$ 
It is easy to see that, for any $b$ in $(\C_A)^U$, there is precisely one pair $(u,c)$ such that $u$ is in $A(U)$ and $c$ is a morphism from $U$ to $U_b$ making the diagram 
$$
\begin{tikzcd}
{}&U\ar{dl}[swap]{\id_U}\ar{dr}{s_b}\\ 
U\ar{dr}[swap]{u}\ar{rr}{c}&&U_b\ar{dl}{u_b}\\ 
{}&A
\end{tikzcd}
$$ 
commute. This implies the lemma. 
\end{proof}

\begin{proof}[Proof of Proposition \ref{prepa2a}]
We have 
$$
j(\C_A)^\ddg(G)(U)\iso\lim_{(U\to j(\C_A)(V\to A))\in(\C_A)^U}G(V\to A)
$$
$$
\iso\lim_{U\to A}G(U\to A)\iso\prod_{U\to A}G(U\to A),
$$
the penultimate isomorphism following from Lemma~\ref{cicau}. 
\end{proof} 

% 

\subsubsection{Third Formula}

Put $j:=j(\C_A),\ h:=h(\C),\ h_A:=h(\C_A)$, and consider the quasi-commutative diagram 
$$
\begin{tikzcd}
\C^\wg\ar{d}[swap]{h^\dg(j^\dg(G))}&
\C\ar{l}[swap]{h}\ar{d}[swap]{j^\dg(G)}&
\C_A\ar{l}[swap]{j}\ar{r}{h_A}\ar{d}{G}&
(\C_A)^\wg\ar{d}{(h_A)^\dg(G)}&
(\C^\wg)_A\ar{l}[swap]{\ld}\\ 
\A\ar[equal]{r}&\A\ar[equal]{r}&\A\ar[equal]{r}&\A.
\end{tikzcd}
$$ 
(See \qr{lambda} p.~\pr{lambda} for the definition of $\ld$.) Let $B$ be in $\C^\wg$. We claim 
\begin{equation}\lb{prepa3}
h^\dg(j^\dg(G))(B)\iso(h_A)^\dg(G)(\ld(B\tm A\to A)).
\end{equation} 

\begin{proof}
We have, for $U$ in $\C$, 
$$
j^\dg(G)(U)\iso\col_{(j(V\to A)\to U)\in(\C_A)_U}G(V\to A)\iso\col_{(A\leftarrow V\to U)\in(\C_A)_U}G(V\to A)
$$
$$
\iso\col_{((V\to A)\to(U\tm A\to A))\in(\C_A)_{U\tm A\to A}}G(V\to A)\iso(h_A)^\dg(G)(\ld(U\tm A\to A)),
$$
that is: 
\begin{equation}\lb{prepa4}
j^\dg(G)(U)\iso(h_A)^\dg(G)(\ld(U\tm A\to A)).
\end{equation} 
For $B$ in $\C^\wg$ we get 

$$h^\dg(j^\dg(G))(B)\iso\col_{(U\to B)\in\C_B}j^\dg(G)(U)\os{(\text a)}\iso\col_{(U\to B)\in\C_B}(h_A)^\dg(G)(\ld(U\tm A\to A))$$

$$\os{(\text b)}\iso(h_A)^\dg(G)\left(\ld\left(\col_{(U\to B)\in\C_B}(U\tm A\to A)\right)\right)$$

$$\os{(\text c)}\iso(h_A)^\dg(G)\left(\ld\left(\left(\col_{(U\to B)\in\C_B}(U\tm A)\right)\to A\right)\right)$$

$$\os{(\text d)}\iso(h_A)^\dg(G)\left(\ld\left(\left(\icolim_{(U\to B)\in\C_B}U\right)\tm A\to A\right)\right)\os{(\text e)}\iso(h_A)^\dg(G)(\ld(B\tm A\to A)),$$ 

\nn where 
(a) follows from \qr{prepa4}; 
(b) follows from Proposition~\ref{271} p.~\pr{271} and \qr{leq} p.~\pr{leq};  
(c) follows from Lemma~\ref{cc} p.~\pr{cc}; 
(d) follows from the fact that small inductive limits in $\Set$ are stable by base change (Definition 2.2.6 p.~47 of the book; see \S\ref{parsbc} p.~\pr{parsbc}); and 
(e) follows from \qr{263a} p.~\pr{263a}.
\end{proof}

%%

\sbs{Notation 2.7.2 p. 63}

Recall that $F:\C\to\C'$ is a functor of small categories. The formula 
$$
\widehat F(A)(X')=\col_{(X\to A)\in\C_A}\Hom_{\C'}(X',F(X))
$$ 
may also be written as 
\begin{equation}\lb{c}
\widehat F(A)=\icolim_{(X\to A)\in\C_A}F(X).
\end{equation}
It might be worth stating explicitly the isomorphism 
$$
\widehat F\ci\hy_\C\xr\sim\hy_{\C'}\ci F, 
$$ 
which says that the diagram 
$$
\begin{tikzcd}
\C\ar{r}\ar{d}[swap]{\hy_\C}\ar{r}{F}&\C'\ar{d}{\hy_{\C'}}\\ 
\C^\wg\ar{r}[swap]{\widehat F}&\C'^\wg.
\end{tikzcd}
$$ 
quasi-commutes. 

\begin{rk}\lb{cof}
Recall that $F:\C\to\C'$ is a functor of small categories. Let $A'$ be in $\C'^\wg$, and let $\C_{A'\ci F}\xr\pp\C'_{A'}\xr\psi\C'^\wg$ be the natural functors. The natural morphism $\col\psi\ci\pp\to\col\psi$ induces a morphism $f:\widehat F(A'\ci F)\to A'$ functorial in $A'$: 
$$
\widehat F(A'\ci F)=\col\psi\ci\pp\to\col\psi\iso A',
$$ 
the equality $F(A'\ci F)=\col\psi\ci\pp$ and the isomorphism $\col\psi\iso A'$ following respectively from \qr{c} and \qr{263a} p.~\pr{263a}. Moreover $f$ is an isomorphism whenever $\pp$ is cofinal. Note that the condition that $f$ is an isomorphism means that, for each $X'$ in $\C'$, the natural map 
$$
\col_{(X\to A'\ci F)\in\C_{A'\ci F}}\Hom_{\C'}(X',F(X))\to A'(X')
$$ 
is bijective. This condition depends only on the functor $F:\C\to\C'$ and the projective system of sets $(A'(X'))_{X'\in\C'}$. (This remark will be used to prove Proposition~\ref{myprop1} p.~\pr{myprop1}.)
\end{rk}

The proof is obvious.

\begin{rk}
If $F$ is fully faithful, then there is an isomorphism $\widehat F(A)\ci F\xr\sim A$ functorial in $A\in\C^\wg$.
\end{rk} 

\begin{proof}
We have 
$$
\widehat F(A)(F(X))=\col_{(Y\to A)\in\C_A}\Hom_{\C'}(F(X),F(Y))
$$
$$
\iso\col_{(Y\to A)\in\C_A}\Hom_\C(X,Y)\xr\sim A(X),
$$
the last isomorphism following from \qr{263b} p.~\pr{263b}.
\end{proof}

As observed in the book (see also \S\ref{c271b} p.~\pr{c271b}):

\begin{rk}\lb{272}
The functor $\widehat F$ commutes with small inductive limits.
\end{rk} 

Let $X$ be in $\C$ and $A$ a terminal object of $\C^\wg$. We have 
$$
\widehat F(A)(F(X))\iso\bigsqcup_{Y\in\C}\Hom_{\C'}(F(X),F(Y)).
$$ 
Let us identify these two sets. 

\begin{rk}\lb{272b}
Assume $A$ is a terminal object of $\C^\wg$, and define, using the above identification, $G:\C\to\C'_{\widehat F(A)}$ by 
$$
G(X):=(F(X),p_X(\id_{F(X)})),
$$ 
where $p_X:\Hom_{\C'}(F(X),F(X))\to\widehat F(A)(F(X))$ is the coprojection. Then the composition of $G$ with the forgetful functor $\C'_{\widehat F(A)}\to\C'$ is $F$.
\end{rk} 

The proof is obvious. 

%%

\sbs{Brief comments} 

\begin{s}\lb{opddagg} 
P.~63, Corollary 2.7.4. Here is a variant: 

Let $\C$ be a category and $\A$ a category admitting small projective limits, let $h:\C\to\C^\wg$ the Yoneda embedding, and let $\Fct^{p\ell}((\C^\wg)^{\op},\A)$ be the category of functors from $(\C^\wg)^{\op}$ to $\A$ commuting with small projective limits. Then the functors 
$$
\begin{tikzcd}
\Fct^{p\ell}((\C^\wg)^{\op},\A)\ar[yshift=0.7ex]{r}{(h^{\op})_*}&\Fct(\C^{\op},\A)\ar[yshift=-0.7ex]{l}{(h^{\op})^\ddg}
\end{tikzcd}
$$
are mutually quasi-inverse equivalences. 

Let $F$ be in $\Fct((\C^\wg)^{\op},\A)$. Assume $(A_i)$ is a projective system in $(\C^\wg)^{\op}$, or, equivalently, $(A_i)$ is an inductive system in $\C^\wg$. In particular $(F(A_i))$ is a projective system in $\A$. 

Then $F$ is in $\Fct^{p\ell}((\C^\wg)^{\op},\A)$ if and only if the following condition holds: 

For any system $(A_i)$ as above, the natural morphism 
$$
F\left(\col_iA_i\right)\to\lim_iF(A_i)
$$ 
is an isomorphism. 

The functor $(h^{\op})^\ddg$ is given by 
$$ 
(h^{\op})^\ddg(F)(A)=\lim_{(U\to A)\in\C_A}F(U). 
$$ 

The functors 
$$
\A^{\C^{\op}}\to\A,\quad F\mt(h^{\op})^\ddg(F)(A)\quad\text{and}\quad
\C^\wg\to\A,\quad A\mt(h^{\op})^\ddg(F)(A)
$$ 
commute with small projective limits. (For a justification, see \S\ref{bil} p.~\pr{bil}.)
\end{s}

%

\begin{s} 
P.~64. It might be worth displaying the formula 
\begin{equation}\lb{275}
\widehat F(A)(X')\iso\col_{(X\to A)\in\C_A}\Hom_{\C'}(X',F(X))\iso
\col_{(X'\to F(X))\in\C^{X'}}A(X),
\end{equation} 
which is contained in the proof of Proposition 2.7.5 p.~64 of the book, and which follows from Corollary 2.4.6 p.~56 of the book (see Proposition~\ref{246} p.~\pr{246}). Recall that $F:\C\to\C'$ is a functor of small categories, that $A$ is in $\C^\wg$, and that $X'$ is in $\C'$. 

For the reader's convenience we reproduce the statement of Proposition 2.7.5: 

\begin{prop}[Proposition 2.7.5 p.~64] 
If $F:\C\to\C'$ is a functor of small categories, then the functors $\widehat F$ and $(F^{\op})^\dg$ from $\C^\wg$ to $\C'^\wg$ are isomorphic. 
\end{prop} 

This follows from \qr{275}.
\end{s}

%

\begin{s}
P.~64, end of Chapter 2. One could add the following observation: 

\emph{If $\C$ is a small category, if $A$ is in $\C^\wg$, if $B$ is a terminal object of $(\C_A)^\wg$, and if $F:\C_A\to\C$ is the forgetful functor, then we have} 
$$
\widehat F(B)\iso A.
$$ 

Indeed, we have 
$$
\widehat F(B)(X)\iso\col_{((Y\to A)\to B)\in(\C_A)_B}\Hom_\C(X,F(Y\to A))
$$
$$
\iso\col_{(Y\to A)\in\C_A}\Hom_\C(X,Y)\iso A(X),
$$ 
the last isomorphism following from \qr{263b} p.~\pr{263b}. 
\end{s}

%

\begin{s}
P.~64, Exercise 2.4. Here is (with some minor changes) the statement of Exercise~2.4. 

Let $f:X\to Y$ be a morphism in a category admitting fiber products. Set $P:=X\tm_YX$; let $p_1,p_2:P\to X$ be the projections; and let $\delta:X\to P$ be the diagonal morphism.

\nn(i) We have $p_1\ci\delta=\id_X=p_2\ci\delta$. In particular $p_1$ and $p_2$ are epimorphisms and $\delta$ is a monomorphism.

\nn(ii) We have: $f$ monomorphism $\ssi p_1=p_2\ssi\delta$ isomorphism $\ssi\delta$ epimorphism. %\nn $f$ monomorphism $\ssi p_1=p_2\ssi\delta$ isomorphism $\ssi\delta$ epimorphism.

Solution: Claim (i) is obvious. Let us prove (ii):

\nn $f$ monomorphism $\then$ $p_1=p_2$: we have $f\ci p_1=f\ci p_2$;

\nn$p_1=p_2\ \then$ $\delta$ isomorphism: $p_i\ci\delta\ci p_j=p_i\ci\id_P$ for all $i,j$, and thus $\delta\ci p_j=\id_P$ for all $j$;

\nn$\delta$ isomorphism $\then\delta$ epimorphism: obvious;

\nn$\delta$ epimorphism $\then\ f$ monomorphism: let $g,h:Z\parar X$ satisfy $f\ci g=f\ci h$; let $k:Z\to P$ satisfy $p_1\ci k=g,\ p_2\ci k=h$; % and note that (i) implies $p_1=p_2$, and thus $g=h$. 
then the assumption that $\delta$ is an epimorphism and the equality $p_1\ci\delta=p_2\ci\delta$ observed in (i) imply $p_1=p_2$, and thus $g=h$.
\end{s}

%

\begin{s}\lb{s3cat}
P.~65, Exercise 2.7. (See \S\ref{27i} p.~\pr{27i}.) Clearly, the following proposition holds and implies Exercise~2.7. 
\begin{prop}\lb{p3cat} 
Let $\U$ be a universe, let $\Set$ be the category of $\U$-sets, let $Z$ be in $\Set$, and let $Z'$ be the discrete category whose set of object is $Z$. Then there are canonical isomorphisms  
$$
\Set_Z\iso\prod_{z\in Z}\Set\iso\Fct(Z',\Set).
$$ 
\end{prop}
\end{s}

%%

\section{About Chapter 3}

\sbs{Brief comments}

\begin{s} 
P.~72, proof of Lemma 3.1.2. Here is a minor variant of the proof of the following statement: 

\emph{If $\pp:J\to I$ is a functor with $I$ filtrant and $J$ finite, then $\lim\Hom_I(\pp,i)\neq\vi$ for some $i$ in $I$.} 

Indeed, let $S$ be a set of morphisms in $J$. It is easy to prove 
$$
(\exists\ i\in I)\left(\exists\ a\in\prod_{j\in J}\Hom_I(\pp(j),i)\right)\ (\forall\ (s:j\to j')\in S)\ (a_{j'}\ci\pp(s)=a_j) 
$$ 
by induction on the cardinal of $S$, and to see that this implies the claim. q.e.d.
\end{s}

% removed
% https://docs.google.com/document/d/1BYGPoHg2m2Y7p9DfLzJ6Qraz0PfXqpy-r8PiNdfH2XQ/edit

\begin{s} 
P.~74, Theorem 3.1.6. The proof of Theorem 3.1.6 implies: 
\begin{prop}\lb{316}
Let $I$ be a (not necessarily small) filtrant $\U$-category, $J$ a finite category, and $\al:I\tm J^{\op}\to\Set$ a functor such that $\col_i\al(i,j)$ exists in $\Set$ for all $j$. Then $\col_i\lim_j\al(i,j)$ exists in $\Set$, and the natural map 
$$
\col_i\lim_j\al(i,j)\to
\lim_j\col_i\al(i,j)
$$ 
is bijective. 
\end{prop} 
This corollary is implicitly used in the proof of Proposition 3.3.13 p.~84 (see Proposition~\ref{3313} p.~\pr{3313} below).
\end{s}

%

\begin{s} 
P.~75, Proposition 3.1.8 (i). In the proof of Proposition 3.3.15 p.~85 of the book, a slightly stronger result is needed (see \S\ref{3315} p.~\pr{3315}). We state and prove this stronger result. 
\begin{prop}\lb{318i} 
Let 
$$
\begin{tikzcd}
J\ar{r}{\pp}&I\ar{r}{\theta}&L&K\ar{l}[swap]{\psi}
\end{tikzcd}
$$
be a diagram of categories. Assume that $\psi$ is cofinal, and that the obvious functor $\pp_k:J_{\psi(k)}\to I_{\psi(k)}$ is cofinal for all $k$ in $K$. Then $\pp$ is cofinal. 
\end{prop} 
%
\begin{proof}
Pick a universe making $I,J,K$ and $L$ small, let $\al:I\to\Set$ be a functor, and consider the commutative diagrams 
$$
\begin{tikzcd}
\al(\pp(j))\ar{dd}[swap]{p_j}\ar{r}{\id}&\al(\pp(j))\ar{d}\ar{r}{\id}&\al(\pp(j))\ar{d}\\ 
&\ds\col_{\theta(\pp(j))\to\ell}\al(\pp(j))\ar{d}\ar{r}&\ds\col_{\theta(\pp(j))\to\psi(k)}\al(\pp(j))\ar{d}\\ 
\col\al\ci\pp\ar{r}[swap]{a}&\ds\col_{\ell}\col_{\theta(\pp(j))\to\ell}\al(\pp(j))\ar{r}[swap]{b}&\ds\col_{k}\col_{\theta(\pp(j))\to\psi(k)}\al(\pp(j)),
\end{tikzcd}
$$ 

$$
\begin{tikzcd}
\al(\pp(j))\ar{d}\ar{r}{\id}\ar{d}&\al(\pp(j))\ar{d}\ar{r}{\id}&\al(\pp(j))\ar{d}\\ 
\ds\col_{\theta(\pp(j))\to\psi(k)}\al(\pp(j))\ar{d}\ar{r}&\ds\col_{\theta(i)\to\psi(k)}\al(i)\ar{d}\ar{r}{\id}&\ds\col_{\theta(i)\to\psi(k)}\al(i)\ar{d}\\ 
\ds\col_{k}\col_{\theta(\pp(j))\to\psi(k)}\al(\pp(j))\ar{r}[swap]{c}&\ds\col_{k}\col_{\theta(i)\to\psi(k)}\al(i)\ar{r}[swap]{d}&\ds\col_{\ell}\col_{\theta(i)\to\ell}\al(i),
\end{tikzcd}
$$ 

$$
\begin{tikzcd}
\al(\pp(j))\ar{d}&\al(\pp(j))\ar{l}[swap]{\id}\ar{dd}{q_j}\\ 
\ds\col_{\theta(i)\to\psi(k)}\al(i)\ar{d}&\ar{d}\\ 
\ds\col_{\ell}\col_{\theta(i)\to\ell}\al(i)&\col\al.\ar{l}{e}
\end{tikzcd}
$$ 
Note that the last row of the first (resp. second) diagram coincides with the first row of the second (resp. third) diagram. Moreover the vertical arrows are coprojections, the squares above $a$ and $e$ result from the proof of \qr{coco} p.~\pr{coco}, the squares above $b$ and $d$ result from the cofinality of $\psi$, and the squares above $c$ result from the cofinality of $\pp_k$. In particular, the maps $a,b,c,d$ and $e$ are bijective. As the bijection $f:=e^{-1}\ci d\ci c\ci b\ci a$ satisfies $f\ci p_j=q_j$, it is the natural map from $\col\al\ci\pp$ to $\col\al$.
\end{proof}
\end{s}

%

\begin{s}\lb{cipc}
P.~75. Throughout the section about the IPC Property, one can assume that $\A$ is a big category. This applies in particular to Corollary 3.1.12 p.~77, corollary used in this generalized form at the end of the proof of Proposition 6.1.16 p.~136 of the book.
\end{s}

%

\begin{s}\lb{77}
P.~77, Proposition 3.1.11 (ii). This proposition says that $\Set$ has the IPC property. Recall the setting:

Let $S$ be a small set, for each $s$ in $S$ let $I_s$ be a small set and $\al_s:I_s\to\Set$ a functor, put $I:=\prod_{s\in S}I_s$, let 
$$
p_j:\prod_{s\in S}\ \al_s(j_s)\ \to\ \col_{i\in I}\ \prod_{s\in S}\ \al_s(i_s),\quad q_{j_s}:\al_s(j_s)\ \to\ \col_{i_s\in I_s}\ \al_s(i_s)
$$ 
be the coprojections, and define  
$$
f:\col_{i\in I}\ \prod_{s\in S}\ \al_s(i_s)\ \to\ \prod_{s\in S}\ \col_{i_s\in I_s}\ \al_s(i_s)
$$ 
by $(f(p_j(x)))_s:=q_{j_s}(x_s)$. Let 
\begin{equation}\lb{eg}
g:\prod_{s\in S}\ \col_{i_s\in I_s}\ \al_s(i_s)\ \to\ \col_{i\in I}\ \prod_{s\in S}\ \al_s(i_s)
\end{equation} 
and consider the following condition on $g$:
%\begin{equation}\lb{g}  y\in\prod_{s\in S}\ \al_s(j_s)\implies g((q_{j_s}(y_s))_{s\in S})=p_j(y).\end{equation} 
\begin{cond}\lb{cg} 
We have 
$$
g((q_{j_s}(y_s))_{s\in S})=p_j(y)
$$ 
for all $j$ in $I$ and all $y$ in $\prod_{s\in S}\ \al_s(j_s)$.
\end{cond} 
%\begin{prop}  If $g$ satisfies \qr{g}, then $f$ and $g$ are inverse bijections. If $I_s$ is filtrant for all $s$ in $S$, then $g$ satisfies \qr{g}.\end{prop} 

Clearly the proposition below implies Proposition 3.1.11 (ii) in the book.

\begin{prop}  
If $g$ satisfies Condition~\ref{cg}, then $f$ and $g$ are inverse bijections. If $I_s$ is filtrant for all $s$ in $S$, then there is a $g$ as in \qr{eg} satisfying Condition~\ref{cg}.
\end{prop} 
\begin{proof}
The proof of the first sentence is straightforward. To prove the second sentence, let $j$ and $k$ be in $I$, let $y$ in $\prod_{s\in S}\al_s(j_s)$ and $z$ in $\prod_{s\in S}\al_s(k_s)$ satisfy $q_{j_s}(y_s)=q_{k_s}(z_s)$ for all $s$ in $S$. It suffices to show $p_j(y)=p_k(z)$. By Corollary 3.1.4 (ii) p.~73 in the book, for each $s$ in $S$ there is a diagram 
$$
j_s\xr{u_s}\ell_s\xl{v_s}k_s
$$ 
in $I_s$ and an element $w_s$ in $\al_s(\ell_s)$ such that $\al_s(u_s)(y_s)=w_s=\al_s(v_s)(z_s)$. This implies $p_j(y)=p_\ell(w)=p_k(z)$, and thus $p_j(y)=p_k(z)$ as requested.
\end{proof}

In view of Proposition~\ref{p3cat} p.~\pr{p3cat} the above proposition implies 
\begin{prop}  
If $Z\in\Set$, then $\Set_Z$ has the IPC property.
\end{prop} 
\end{s} 

%

\begin{s} 
P.~78, Proposition 3.2.2. It is easy to see that Condition (iii) is equivalent to
\begin{equation}\lb{78} 
\col\ \Hom_I(i,\pp)\iso\pt\quad\text{for all }i\in I, 
\end{equation} 
which is Condition (vi) in Proposition 2.5.2 p.~57 of the book. (Proposition 2.5.2 states, among other things, that \qr{78} is equivalent to the cofinality of $\pp$.)
\end{s}

%

\begin{s} 
P.~79, proof of Corollary 3.2.3 (ii). Here are more details: For $(i,j)\in I\tm J$ we have $I^{(i,j)}\iso(I^i)^j$. Part (i) implies that $I^i$ is filtrant and the forgetful functor $I^i\to I$ is cofinal. Then Proposition 3.2.2, (i) $\then$ (ii), p.~78 of the book implies that $(I^i)^j$ is filtrant. Finally, Proposition 3.2.2, (ii) $\then$ (i) implies that the diagonal functor $I\to I\tm I$ is cofinal.
\end{s}

%

\begin{s}
P. 79, Proposition 3.2.5. It is claimed that (ii) is a particular case of (iv). More precisely, (ii) is obtained from (iv) by replacing the setting 
$$
I\xr\pp J\xr\psi K,\quad u:k\to\psi(j)
$$ 
with 
$$
I\xr{\id_I}I\xr\pp J,\quad\id_{\pp(i)}:\pp(i)\to\pp(i).
$$ 
\end{s}

%

\begin{s} 
P.~80. Propositions 3.2.4 and 3.2.6 can be combined as follows. 

\begin{prop}\lb{comb}
Let $\pp:J\to I$ be fully faithful. Assume that $I$ is filtrant and cofinally small, and that for each $i$ in $I$ there is a morphism $i\to\pp(j)$ for some $j$ in $J$. Then $\pp$ is cofinal and $J$ is filtrant and cofinally small. 
\end{prop} 

\begin{proof}
In view of Proposition 3.2.4 it suffices to show that $J$ is cofinally small. By Proposition 3.2.6, there is a small full subcategory $S\subset I$ cofinal to $I$. For each $s$ in $S$ pick a morphism $s\to\pp(j_s)$ with $j_s$ in $J$. Then, for each $j$ in $J$ there are morphisms $\pp(j)\to s\to\pp(j_s)$ with $s$ in $S$. As $\pp$ is full there is a morphism $j\to j_s$, and we conclude by using again Proposition 3.2.6.
\end{proof}
\end{s}

%

\begin{s} 
P.~80, proof of Lemma 3.2.8 (minor variant). As already pointed out, a $\ds\ilim$ is missing in the last display. Recall the statement:
\begin{lem}
Let $I$ be a small ordered set, let $\al:I\to\C$ be a functor, let $\J$ be the set of finite subsets of $I$ ordered by inclusion, and for each $J$ in $\J$ let $\al_J:\J\to\C$ be the restriction of $\al$ to $\J$. Then $\J$ is small and filtrant, and we have
$$
\col\al\iso\col_{J\in\J}\col\al_J.
$$ 
in $\C^\vee$.
\end{lem}
\begin{proof}
Set
$$
A:=\col\al,\quad
\bt(J):=\col\al_J,\quad
B:=\col\bt.
$$
Let 
$$
p_i:\al(i)\to A,\quad 
p_{i,J}:\al(i)\to\bt(J),\quad 
p_J:\bt(J)\to B
$$
be the coprojections. Note that $p_{i,J}$ is defined only for $i$ in $J$. We easily check that 

\nn$\bu$ the morphisms$f_i:=p_{\{i\}}\ci p_{i,\{i\}}:\al(i)\to B$ induce a morphism $f:A\to B$, 

\nn$\bu$ the morphisms $g_{i,J}:=p_i:\al(i)\to A$ (with $i$ in $J$) induce a morphism $g_J:\bt(J)\to A$, 

\nn$\bu$ the morphisms $g_J$ induce a morphism $g:B\to A$, 

\nn$\bu$ $f$ and $g$ are mutually inverse isomorphisms.
\end{proof}
\end{s}

% 

For the reader's convenience we reproduce Definition 3.3.1 p.~81. 

\begin{df}[Definition 3.3.1, exactness] 
Let $F:\C\to\C'$ be a functor.

\nn\emph{(i)} We say that $F$ is \emph{right exact}\index{right exact} if the category $\C_{X'}$ is filtrant for all $X'$ in $\C'$. 

\nn\emph{(ii)} We say that $F$ is \emph{left exact}\index{left exact} if $F^{\op}:\C^{\op}\to\C'^{\op}$ is right exact, or equivalently if the category $\C^{X'}$ is cofiltrant for all $X'$ in $\C'$.

\nn\emph{(iii)} We say that $F$ is \emph{exact}\index{exact functor} if it is both right and left exact.
\end{df}

\begin{s}% 4:35 PM Saturday, May 28, 2016
% https://docs.google.com/document/d/1cvapjvUY-qwpIWY90ncOCdZt_1pYSS07lZrnvBDp5rM/edit
P.~81, proof of Proposition 3.3.2 (minor variant). Recall the statement:

\begin{prop}[Proposition 3.3.2 p.~81]\lb{p332} 
Consider functors $I\xr\al\C\xr F\C'$, and assume that $I$ is finite, that $F$ is right exact, and that $\col\al$ exists in $\C$. Then $\col F\ci\al$ exists in $\C'$, and the natural morphism $\col F\ci\al\to F(\col\al)$ is an isomorphism. 
\end{prop} 

\begin{proof}
Let $X'$ be in $\C'$. It suffices to show that the natural map 
$$
\Hom_{\C'}(F(\col\al),X')\to\lim\Hom_{\C'}(F\ci\al,X')
$$ 
is bijective. We claim 
\begin{subequations}\lb{fyx'}% other version: 
% https://docs.google.com/document/d/1cvapjvUY-qwpIWY90ncOCdZt_1pYSS07lZrnvBDp5rM/edit
\begin{align}
\col_{(F(Y)\to X')\in\C_{X'}}\Hom_\C(X,Y)&\iso\col_{(X\to Y)\in(\C^X)^{\op}}\Hom_{\C'}(F(Y),X')\lb{fyx'1}\\
&\iso\Hom_{\C'}(F(X),X').\lb{fyx'2}
\end{align}
\end{subequations}
Indeed, we obtain \qr{fyx'1} by replacing the setting \qr{241s} p.~\pr{241s} with 
$$
X\in\C\xl{\id_\C}\C\xr{F}\C'\ni X' 
$$ 
in the isomorphism \qr{241} p.~\pr{241}, and we prove \qr{fyx'2} by noting that the identity of $X$ is an initial object of $\C^X$. We have five sets and four bijections: 
$$ 
\Hom_{\C'}(F(\col\al),X')\xr\sim\col_{(F(Y)\to X')\in\C_{X'}}\Hom_\C(\col\al,Y)
$$
$$
\xr\sim\col_{(F(Y)\to X')\in\C_{X'}}\lim\Hom_\C(\al,Y)\xr\sim\lim\col_{(F(Y)\to X')\in\C_{X'}}\Hom_\C(\al,Y)
$$
$$
\xr\sim\lim\Hom_{\C'}(F\ci\al,X'). 
$$ 
The first and last bijections follow from \qr{fyx'}, the second one is clear, and the third one can be justified as follows: Inductive limits over the category $\C_{X'}$, which is filtrant because $F$ is right exact, commute with projective limits over the finite category $I$. 

Let us denote these five sets and four bijections by 
$$
S_1\xr{f_1}S_2\xr{f_2}S_3\xr{f_3}S_4\xr{f_4}S_5,
$$ 
and let 
$$
f:\Hom_{\C'}(F(\col\al),X')\to\lim\Hom_{\C'}(F\ci\al,X').
$$ 
be the natural map. It remains to show 
\begin{equation}\lb{4321}
f_4\ci f_3\ci f_2\ci f_1=f.
\end{equation} 

%To prove this we shall use the following abuse of notation: For any functor $\bt:J\to\A$ admitting an inductive limit, and for any $j$ in $J$, we denote by $$q[j]:\bt(j)\to\col\bt$$ the coprojection. 

Let $Y$ be in $\C$, let $F(Y)\to X'$ be a morphism in $\C'$, and let 
$$
p[F(Y)\to X']:\Hom_\C(\col\al,Y)\to\col_{(F(Y)\to X')\in\C_{X'}}\Hom_\C(\col\al,Y),
$$ 
$$
q[F(Y)\to X']:\lim\Hom_\C(\al,Y)\to\col_{(F(Y)\to X')\in\C_{X'}}\lim\Hom_\C(\al,Y),
$$ 
$$
r[F(Y)\to X']:\Hom_\C(\al,Y)\to\col_{(F(Y)\to X')\in\C_{X'}}\Hom_\C(\al,Y)
$$ 
be the coprojections.

We shall use implicitly, not only the statements of the bijections \qr{fyx'1} and \qr{fyx'2}, but also their proofs (see Warning~\ref{warning1} p.~\pr{warning1}). 

For $F(\col\al)\to X'$ in $\Hom_{\C'}(F(\col\al),X')$, we have (omitting most of the parenthesis) 
$$
f_4f_3f_2f_1(F(\col\al)\to X')
$$ 
$$
=f_4f_3f_2\left(p[F(\col\al)\to X']\left(\col\al\xr{\id}\col\al\right)\right)
$$ 
$$
=f_4f_3\bigg(q[F(\col\al)\to X']\Big(\big(\al(i)\to\col\al\big)_i\Big)\bigg)
$$ 
$$
=f_4\bigg(\Big(r[F(\col\al)\to X']\big(\al(i)\to\col\al\big)\Big)_i\bigg)
$$ 
$$
=\Big(\big(F(\al(i))\to F(\col\al)\to X'\big)_i\Big).
$$ 
This proves \qr{4321}.
\end{proof}  
\end{s}

%

\begin{s} Some more details in the proof of Proposition 3.3.12 p. 84:
\begin{prop}[Proposition 3.3.12 p. 84] 
Let $F:\C\to\C'$ and $G:\C'\to\C''$ be two functors. If $F$ and $G$ are right exact, then $G\ci F$ is right exact.
\end{prop}
\begin{proof}
Since $G$ is right exact, $\C'_{X''}$ is filtrant for any $X''$ in $\C''$. The obvious functor $\C_{X''}\to\C'_{X''}$ is again right exact. Indeed, for any $G(X')\to X''$ in $\C'_{X''}$, the category $(\C_{X''})_{G(X')\to X''}\iso\C_{X'}$ is filtrant because $F$ is right exact. Hence, Proposition 3.3.11 implies that $\C_{X''}$ is filtrant.
%Indeed, for any $(X',u)$ in $\C'_{X''}$, with $X'$ in $\C'$ and $u:G(X')\to X''$, the category $(\C_{X''})_{(X',u)}\iso\C_{X'}$ is filtrant because $F$ is right exact. Hence, Proposition 3.3.11 implies that $\C_{X''}$ is filtrant.
\end{proof}
\end{s}

%

\begin{s}
P.~84, Proposition 3.3.13. Recall the statement:

\begin{prop}[Proposition 3.3.13 p. 84]\lb{3313} 
Let $\C$ be a category admitting finite inductive limits, and let $A$ be in $\C^\wg$. Then $A$ is left exact if and only if $\C_A$ is filtrant.
\end{prop}

We spell out the details of the proof of the implication $\C_A$ is filtrant $\then$ $A$ left exact.

By Proposition 3.3.3 of the book, stated as Proposition~\ref{333} p.~\pr{333} below, it suffices to show that $A$ commutes with finite projective limits. Let $(X_i)$ be a finite inductive system in $\C$. We must check that the natural map 
$$
e:A\left(\col_i X_i\right)\to\lim_iA(X_i)
$$ 
is bijective. Let us abbreviate $(Y\to A)\in\C_A$ by $Y$, and consider the commutative diagram 
$$
\begin{tikzcd}
\col_Y\Hom_\C(\col_iX_i,Y)\ar{r}{a}\ar{dd}[swap]{d}&\col_Y\lim_i\Hom_\C(X_i,Y)\ar{d}{b}\\ 
{}&\lim_i\col_Y\Hom_\C(X_i,Y)\ar{d}{c}\\ 
A(\col_iX_i)\ar{r}[swap]{e}&\lim_iA(X_i),
\end{tikzcd}
$$ 
where $a$ is defined by \S\ref{sv} p.~\pr{sv}, $c$ and $d$ are defined by \qr{263b} p.~\pr{263b} and $b$ is defined by %Theorem 3.1.6 p.~74 of the book
Proposition~\ref{316} p.~\pr{316} (see Warning~\ref{warning1} p.~\pr{warning1}). %It follows from their definitions that t
These four maps are clearly bijective. We leave it to the reader to check that this diagram commutes. %The maps $a,b,c$ and $d$ are bijective for the following reasons: clearly for $a$, because of \qr{263b} p.~\pr{263b} for $c$ and $d$, and by Theorem 3.1.6 p.~74 of the book for $b$. Thus 
This implies that $e$ is bijective. 
\end{s}

%

\begin{s}\lb{3315}
P.~85, proof of Proposition 3.3.15. To prove that $\A\to\C$ is cofinal, one can apply Proposition~\ref{318i} p.~\pr{318i} with $J=\A,I=\C,L=\C',K=\SSS$. 
\end{s}

%

\begin{s}
P. 86, proof of Theorem 3.3.18 (b). The proof uses the following fact, whose proof is straightforward:

Let $\al:I\tm J^{\op}\to\C$ be a functor. %and set $X_{ij}:=\al(i,j)$. 
Assume that $\C$ admits inductive limits indexed by $I$ and projective limits indexed by $J^{\op}$. Then the morphism obtained by composing the canonical morphism 
$$
\col_{i\in I}\lim_{j\in J^{\op}}\al(i,j)\to\lim_{j\in J^{\op}}\col_{i\in I}\al(i,j)
$$ 
with the projection 
$$
\lim_{j\in J^{\op}}\col_{i\in I}\al(i,j)\to\col_{i\in I}\al(i,j)
$$ 
coincides with the morphism obtained by applying the functor $\col_{i\in I}$ to the projection 
$$
\lim_{j\in J^{\op}}\al(i,j)\to \al(i,j).
$$ 
\end{s} 

%%%

\sbs{Proposition 3.4.3 (i) p.~88} 

% old version:
%https://docs.google.com/document/d/1ODl6Kwqz4LekRjoydQLeDXX23NiBk034UbtlK4fLzrI/edit

\begin{lem}
If $I\xr\pp K\xl\psi J$ are functors between small categories, if 
$$
M:=M[I\xr\pp K\xl\psi J] 
$$ 
is the category defined in Definition 3.4.1 p.~87 of the book, if $\al:M\to\C$ is a functor, and if $\C$ admits small inductive limits, then there is a natural functor (described in the proof) from $J$ to $\C$ mapping $j\in J$ to 
$$
\col_{(i,u)\in I_{\psi(j)}}\al(i,j,u) 
$$ 
($u$ being a morphism in $K$ from $\pp(i)$ to $\psi(j)$).%is a functor from $J$ to $\C$.
\end{lem}

\begin{proof}
Let $j\to j'$ be a morphism in $J$. It is easily checked that there is a unique dashed arrow which make all diagrams 
$$
\begin{tikzcd}
\ds\col_{(i,u)\in I_{\psi(j)}}\al(i,j,u)\ar[rr,dashed]&&\ds\col_{(i,u)\in I_{\psi(j')}}\al(i,j',u)\\ 
\al(i,j,u)\ar{u}{p_{iu}}\ar{rr}[swap]{\al(\id_i,j\mt j')}&&\al(i,j',u')\ar{u}[swap]{q_{iu'}}
\end{tikzcd}
$$
commute, where $p_{iu}$ and $q_{iu'}$ are the coprojections and $u'$ is the obvious composition 
$$
\pp(i)\xr u\psi(j)\to\psi(j'),
$$ 
and that the assignment 
$$
(j\to j')\mt\left(\col_{(i,u)\in I_{\psi(j)}}\al(i,j,u)\to\col_{(i,u)\in I_{\psi(j')}}\al(i,j',u)\right)
$$ 
is functorial.
\end{proof} 

\begin{prop}[Proposition 3.4.3 (i) p. 88]\lb{coco2}
We have an isomorphism 
$$
\col\al\iso\col_j\ \col_{i,u}\al(i,j,u),
$$  
where $(i,u)$ runs over $I_{\psi(j)}$ (with $u:\pp(i)\to\psi(j)$). This isomorphism is explicitly described in  the proof.
\end{prop} 

\begin{proof}
Let  
$$
\al(i,j,u)\xr{p_{iju}}\col\al,\qquad\al(i,j,u)\xr{q_{iju}}\col_{i,u}\al(i,j,u)\xr{r_j}\col_j\ \col_{i,u}\al(i,j,u)
$$ 
be the coprojections. There is a unique morphism 
$$
f:\col\al\to\col_j\ \col_{i,u}\al(i,j,u)
$$ 
such that $f\ci p_{iju}=r_j\ci q_{iju}$ for all $i,j,u$: 
$$
\begin{tikzcd}
\al(i,j,u)\ar{dd}[swap]{p_{iju}}\ar{r}{\id}&\al(i,j,u)\ar{d}{q_{iju}}\\ 
&\ds\col_{i,u}\al(i,j,u)\ar{d}{r_j}\\ 
\col\al\ar{r}[swap]{f}&\ds\col_j\col_{i,u}\al(i,j,u).
\end{tikzcd}
$$
We construct the commutative diagram 
\begin{equation}\lb{iju}
\begin{tikzcd}
\al(i,j,u)\ar{d}[swap]{q_{iju}}\ar{r}{\id}&\al(i,j,u)\ar{d}{p_{iju}}\\ 
\ds\col_{i,u}\al(i,j,u)\ar{d}[swap]{r_j}\ar{r}{g_j}&\col\al\ar{d}{\id}\\ 
\ds\col_j\col_{i,u}\al(i,j,u)\ar{r}[swap]{g}&\col\al.
\end{tikzcd}
\end{equation} 
as follows: We fix $j$ and define $g_j$ by the condition that the top square of \qr{iju} commutes for all $(i,u)$. Then we define $g$ by the condition that the bottom square of \qr{iju} commutes for all $j$. We leave it to the reader to check that $f$ and $g$ are inverse isomorphisms. 
\end{proof}

In view of Proposition~\ref{316} p.~\pr{316}, %Theorem 3.1.6 p.~74 of the book, 
Proposition~\ref{coco2} implies 
\begin{prop}\lb{cocop} 
If $J$ and $I_{\psi(j)}$ are filtrant for all $j$ in $J$, then $M$ is filtrant.
\end{prop}

%

% previous version 
% https://docs.google.com/document/d/1F9EeXTF75Ruv4EF1BriNu80PrDzzAmnJKNIX63BkRZs/edit

%%

\sbs{Brief comments}

\begin{s} 
We prove Proposition 3.4.3 (ii) p.~88. Recall the statement:

\emph{If $\psi$ is cofinal, then $M[I\to K\leftarrow J]\to I$ is cofinal.} 

To prove this, we let $\al:I\to\Set$ be a functor, we denote by $\bt$ the composition $M[I\to K\leftarrow J]\to I\to\Set$, and we verify that the natural map $\col\bt\to\col\al$ is bijective as follows. 

In the commutative diagram below we write $u$ for a generic morphism $\pp(i)\to\psi(j)$ and $v$ for a generic morphism $\pp(i)\to k$, with $i\in I, j\in J, k\in K$, and we abbreviate $\col_{(i,u)\in I_{\psi(j)}}$ by $\col_{i,u}$ and $\col_{(i,v)\in I_k}$ by $\col_{i,v}$: 
$$
\begin{tikzcd}
\al(i)\ar{dd}{a_{iju}}\ar{r}{\id}&\al(i)\ar{d}{b_{iju}}\ar{r}{\id}&\al(i)\ar{d}{b_{i\psi(j)u}}\ar{r}{\id}&\al(i)\ar{dd}{e_i}\\ 
&\ds\col_{i,u}\al(i)\ar{d}{c_j}\ar{r}{\id}&\ds\col_{i,u}\al(i)\ar{d}{d_{\psi(j)}}\\ 
\col\bt\ar{r}{f}&\ds\col_j\col_{i,u}\al(i)\ar{r}\ar{r}{g}&\ds\col_k\col_{i,v}\al(i)\ar{r}\ar{r}{h}&\col\al.
\end{tikzcd}
$$ 
%The third column involves the coprojections $$\al(i)\xr{d_{ikv}}\col_{(i,v)\in I_k}\al(i)\xr{e_k}\col_{k\in K}\col_{(i,v)\in I_k}\al(i).$$ 
(The vertical arrows are the various coprojections.) The diagram being commutative, $h\ci g\ci f$ is the natural map $\col\bt\to\col\al$. Moreover $f$ is bijective by the proof of Proposition~\ref{coco2}, $g$ is bijective because $\psi$ is cofinal and $h$ is bijective by the proof of \qr{coco} p.~\pr{coco}.

% previous version
% https://docs.google.com/document/d/14DUaE9PkHOqyvFdvsYAAmet7K2fkvWOj-fsropw3tMk/edit
 
\end{s}

%

\begin{s} 
P.~89, Proposition 3.4.5 (iii). The proof uses implicitly the following fact: 

\begin{prop}\lb{355}
If $F$ is a cofinally small filtrant category, then there is a small {\em filtrant} full subcategory of $F$ cofinal to $F$. 
\end{prop}

This results immediately from Corollary 2.5.6 p.~59 and Proposition 3.2.4 p.~79 (see Proposition~\ref{comb} p.~\pr{comb}). This fact also justifies the sentence ``We may replace `filtrant and small' by `filtrant and cofinally small' in the above definition'' p.~132, Lines 4 and 5 of the book.
\end{s}

%%

\sbs{Five closely related statements}

For the reader's convenience we collect five statements closely related to Exercise 3.4 (i) p.~90 of the book. 

\subsubsection{Proposition 2.1.10 p.~40}

\begin{prop}[Proposition 2.1.10 p.~40]\lb{2.1.10}
If $F:\C\to\C'$ is a functor admitting a left adjoint, if $I$ is a category, and if $\C$ admits projective limits indexed by $I$, then $F$ commutes with such limits.
\end{prop}

(This fact has already been stated as Corollary~\ref{2.1.10c} p.~\pr{2.1.10c}.)

\subsubsection{Exercise 2.7 (ii) p.~65}

\begin{prop}[Exercise 2.7 (ii) p.~65]\lb{sbcs}
The base change functors (see \S\ref{parsbc} p.~\pr{parsbc}) in $\Set$ commute with small inductive and projective limits limits. In particular, small inductive limits in $\Set$ are stable by base change.
\end{prop}

(See \S\ref{27i} p.~\pr{27i}.) Note that Proposition~\ref{sbcs} generalizes the distributivity of multiplication over addition in $\bb N$. 

\subsubsection{Proposition 3.3.3 p.~82}

\begin{prop}[Proposition 3.3.3 p.~82]\lb{333}
Let $F:\C\to\C'$ be a functor and assume that $\C$ admits finite projective limits. Then $F$ is left exact if and only if it commutes with such limits.
\end{prop}

\begin{cor}\lb{bre}
In the setting of Proposition 2.7.1 p.~62 of the book, the functors
$$
\A^\C\to\A,\quad F\mt(\oo h_\C^\dg F)(A)\quad\text{and}\quad\C^\wg\to\A,\quad A\mt(\oo h_\C^\dg F)(A)
$$ 
are right exact. 
\end{cor}

\begin{proof}
This follows from Proposition~\ref{333} and \S\ref{bil} p.~\pr{bil}.
\end{proof} 

\begin{cor}\lb{bre2}
In the setting of \S\ref{opddagg} p.~\pr{opddagg}, the functors
$$
\A^{\C^{\op}}\to\A,\quad F\mt(h^{\op})^\ddg(F)(A)\quad\text{and}\quad\C^\wg\to\A,\quad A\mt(h^{\op})^\ddg(F)(A)
$$ 
are left exact. 
\end{cor} 

\subsubsection{Proposition 3.3.6 p.~83}

\begin{prop}[Proposition 3.3.6 p.~83]\lb{336}
A functor admitting a left adjoint is left exact.
\end{prop}

\subsubsection{Exercise 3.4 (i) p.~90}

\begin{prop}[Exercise 3.4 (i) p.~90]\lb{34i}
If $F:\C\to\C'$ is a right exact functor and $f:X\epi Y$ is an epimorphism in $\C$, then $F(f):F(X)\to F(Y)$ is an epimorphism in $\C'$.
\end{prop}

(This exercise is used in the second sentence of p.~227 of the book.)

\begin{proof}
Let $f'_1,f'_2:F(Y)\rightrightarrows X'$ be morphisms in $\C'$ satisfying 
$$
f'_1\ci F(f)=f'_2\ci F(f)=:f'.
$$
This is visualized by the diagram
$$
\begin{tikzcd}
\Big(F(X)\ar{r}{f}&F(Y)\ar[yshift=.7ex]{r}{f'_1}\ar[yshift=-.7ex]{r}[swap]{f'_2}&X'\Big)=\Big(F(X)\ar{r}{f'}&X'\Big).
\end{tikzcd}
$$ 
It suffices to prove $f'_1=f'_2$. For $i=1,2$ let $f_i$ be the morphism $f$ viewed as a morphism from $(X,f')$ to $(Y,f'_i)$ in $\C_{X'}$: 
$$
\begin{tikzcd}
F(X)\ar{dr}[swap]{f'}\ar{rr}{F(f)}&&F(Y)\ar{dl}{f'_i}\\ 
{}&X'.
\end{tikzcd}
$$
As $\C_{X'}$ is filtrant, there are morphisms $\gamma_i:(Y,f'_i)\to(Z,g')$, defined by morphisms $g_i:Y\to Z$, such that $\gamma_1\ci f_1=\gamma_2\ci f_2$:
$$
\begin{tikzcd}
F(X)\ar{d}[swap]{f'}\ar{r}{F(f)}&F(Y)\ar{d}[swap]{f'_i}\ar{r}{F(g_i)}&F(Z)\ar{d}{g'}\\ 
X'\ar[equal]{r}&X'\ar[equal]{r}&X'.
\end{tikzcd}
$$
As $f$ is an epimorphism, the equality $g_1\ci f=g_2\ci f$ implies $g_1=g_2=:g$, and thus $f'_1=g'\ci F(g)=f'_2$.
\end{proof}

\begin{cor}\lb{ccprime}
Let $\C$ be a category, let $\C'$ be a category admitting finite inductive limits, and let $\theta:F\to G$ be a morphism in $\C'^\C$. Then $\theta$ is an epimorphism if and only if $\theta_X:F(X)\to G(X)$ is an epimorphism for all $X$ in $\C$. 
\end{cor} 

\begin{proof}
This follows from Proposition~\ref{333} p.~\pr{333} and Proposition~\ref{34i} just above.
\end{proof}

%%%

\section{About Chapter 4}

\begin{s}
P.~93, Lemma 4.1.2. Here is a slightly more general statement:

\begin{lem}\lb{proj3}
Let $\C$ be a category, let $P:\C\to\C$ be a functor, let $\ee:\id_\C\to P$ be a morphism of functors, and let $X$ be an object of $\C$. Then the following conditions are equivalent:

\nn{\em(a)} $\ee_{P(X)}$ is an isomorphism and $P(\ee_X)$ is an epimorphism,

\nn{\em(b)} $P(\ee_X)$ is an isomorphism and $\ee_{P(X)}$ is a monomorphism,

\nn{\em(c)} $\ee_{P(X)}$ and $P(\ee_X)$ are equal isomorphisms.
\end{lem}

\begin{proof} It is enough to prove (a)$\then$(c)$\Leftarrow$(b). 

\nn(a)$\then$(c): Put $u:=(\ee_{P(X)})^{-1}\ci P(\ee_X)$. It suffices to show 
\begin{equation}\lb{uidpx}
u=\id_{P(X)}.
\end{equation} 
We have 
$$
u\ci\ee_X=(\ee_{P(X)})^{-1}\ci P(\ee_X)\ci\ee_X=(\ee_{P(X)})^{-1}\ci\ee_{P(X)}\ci\ee_X=\ee_X,
$$ 
and thus 
$$
P(u)\ci P(\ee_X)=P(\ee_X)=\id_{P^2(X)}\ci P(\ee_X).
$$
As $P(\ee_X)$ is an epimorphism, this implies $P(u)=\id_{P^2(X)}$, and thus 
$$
\ee_{P(X)}\ci u=P(u)\ci\ee_{P(X)}=\ee_{P(X)}.
$$ 
As $\ee_{P(X)}$ is an isomorphism, this implies \qr{uidpx}, as required.

\nn(b)$\then$(c): We shall use several times the assumption that $P(\ee_X)$ is an isomorphism. Put $v:=P(\ee_X)^{-1}\ci\ee_{P(X)}$. It suffices to show 
%
\begin{equation}\lb{vidpx}
v=\id_{P(X)}.
\end{equation}
%
We have 
$$
v\ci\ee_X=P(\ee_X)^{-1}\ci\ee_{P(X)}\ci\ee_X=P(\ee_X)^{-1}\ci P(\ee_X)\ci\ee_X=\ee_X,
$$ 
$$
P(v)\ci P(\ee_X)=P(\ee_X),
$$
$$
P(v)=\id_{P^2(X)},
$$
$$
\ee_{P(X)}\ci v=P(v)\ci\ee_{P(X)}=\ee_{P(X)}=\ee_{P(X)}\ci\id_{P(X)}.
$$ 
As $\ee_{P(X)}$ is a monomorphism, this implies \qr{vidpx}, as required. 
\end{proof}

Definition 4.1.1 p.~93 of the book can be stated as follows:

\begin{df}[Definition 4.1.1 p.~93, projector] 
Let $\C$ be a category. A {\em projector}\index{projector} on $\C$ is the data of a functor $P:\C\to\C$ and a morphism $\ee:\id_\C\to P$ such that each object $X$ of $\C$ satisfies the equivalent conditions of Lemma~\ref{proj3}. 
\end{df}
\end{s}

%

\begin{s} 
P.~94, proof of (a)$\then$(b) in Proposition 4.1.3 (ii) (additional details): In the commutative diagram 
$$
\begin{tikzcd}
\Hom_\C(P(Y),X)\ar{d}{\sim}[swap]{\ee_X\ci}\ar{r}{\ci\ee_Y}&\Hom_\C(Y,X)\ar{d}{\ee_X\ci}[swap]{\sim}\\ 
\Hom_\C(P(Y),P(X))\ar{r}[swap]{\ci\ee_Y}{\sim}&\Hom_\C(Y,P(X)),
\end{tikzcd}
$$ 
the vertical arrows are bijective by (a), and the bottom arrow is bijective by (i).
\end{s}

%%

\begin{s}
P. 95, end of the proof of Proposition 4.1.3. Recall that we have functors 
$$
\begin{tikzcd} 
\C_0\ar[yshift=0.7ex,hook]{r}{\iota}&\C.\ar[yshift=-0.7ex]{l}{P}
\end{tikzcd}
$$
The last sentence of the proof says that $P$ ``is a left adjoint to $\iota$ by (i)''. One could also write that $P$ ``is a left adjoint to $\iota$ by Condition (b) in Part (ii)''. Indeed, Condition (b) in Part (ii) asserts that the map
$$
\Hom_\C(P(Y),X)\xr{\ci\ee_Y}\Hom_\C(Y,X),
$$ 
that is 
$$
\Hom_{\C_0}(P(Y),X)\xr{\ci\ee_Y}\Hom_\C(Y,\iota(X)),
$$
is bijective for all $Y$ in $\C$. 
\end{s}

%%

\begin{s} 
P.~95, proof of Proposition 4.1.4 (i) (additional details). The authors write: ``The two compositions 
$$
\begin{tikzcd}
P\ar[yshift=0.7ex]{r}{\ee\ci P}\ar[yshift=-0.7ex]{r}[swap]{P\ci\ee}&P^2\ar{r}{R\eta L}&P
\end{tikzcd}
$$ 

\nn are equal to $\id_P$''. If we translate this statement into the language of Notation~\ref{nhove} p.~\pr{nhove} and Notation~\ref{nmat} p.~\pr{nmat}, we get 

\begin{equation}\lb{tra}
\begin{pmatrix}R\star\eta\star L\\ \ee\star R\star L\end{pmatrix}
=RL=\begin{pmatrix}R\star\eta\star L\\ R\star L\star\ee\end{pmatrix}.
\end{equation}

\nn To prove \qr{tra}, write 

$$
\begin{pmatrix}R\star\eta\star L\\ \ee\star R\star L\end{pmatrix}
=\begin{pmatrix}R\star\eta&L\\ \ee\star R&L\end{pmatrix}
=\begin{pmatrix}R\star\eta\\ \ee\star R\end{pmatrix}\star\begin{pmatrix}L\\ L\end{pmatrix}
\os{\text{(a)}}{=}RL
$$

$$
\os{\text{(b)}}{=}\begin{pmatrix}R\\ R\end{pmatrix}\star\begin{pmatrix}\eta\star L\\ L\star\ee\end{pmatrix}
=\begin{pmatrix}R&\eta\star L\\ R&L\star\ee\end{pmatrix}
=\begin{pmatrix}R\star\eta\star L\\ R\star L\star\ee\end{pmatrix},
$$ 

\nn Equalities (a) and (b) resulting respectively from \qr{159} p.~\pr{159} and \qr{158} p.~\pr{158}, and the other equalities following from Proposition~\ref{pil1} p.~\pr{pil1}. 
\end{s}

%%

\section{About Chapter 5}

\sbs{Beginning of Section 5.1 p.~113}

We want to define the notions of coimage (denoted by $\Coim$) and image (denoted by $\Ima$) in a slightly more general way than in the book. To this end we start by defining these notions in a particular context in which they coincide. To avoid confusions we (temporarily) use the notation $\IM$ for these particular cases. The proof of the following lemma is obvious. 

\begin{lem}\lb{imset} 
For any set theoretical map $g:U\to V$ we have natural bijections 
$$ 
\Coker(U\tm_VU\parar U)\iso\IM g\iso\Ker(V\parar V\sqcup_UV),
$$ 
where $\IM g$ denotes the image of $g$. 
\end{lem} 

Let $\C$ be a $\U$-small category, and let us denote by $\hy:\C\to\C^\wg$ and $\ky:\C\to\C^\vee$ the Yoneda embeddings. For any morphism $f:X\to Y$ in $\C$ define $\IM\hy(f)$ in $\C^\wg$ and $\IM\ky(f)$ in $\C^\vee$ by
$$
(\IM\hy(f))(Z):=\IM\,\hy(f)_Z,\quad(\IM\ky(f))(Z):=\IM\,\ky(f)_Z 
$$
for any $Z$ in $\C$. Note the equalities 
$$
\IM\,\hy(f)_Z=f\ci\Hom_\C(Z,X)=\{f\ci x\ |\ x\in\Hom_\C(Z,X)\},
$$ 
$$
\IM\,\ky(f)_Z=\Hom_\C(Y,Z)\ci f=\{y\ci f\ |\ y\in\Hom_\C(Y,Z)\}.
$$ 
Lemma \ref{imset} implies

\begin{equation}\lb{IM2}
\begin{split}
\IM\hy(f)\iso\Coker(\hy(X)\tm_{\hy(Y)}\hy(X)\parar\hy(X)),\\ \\ 
\IM\ky(f)\iso\Ker(\ky(Y)\parar\ky(Y)\sqcup_{\ky(X)}\ky(Y)).
\end{split}
\end{equation}

\begin{df}[coimage, image]%\lb{dci}
In the above setting, the {\em coimage}\index{coimage} of $f$ is the object $\Coim f$ of $\C^\vee$ defined by 
$$ 
(\Coim f)(Z):=\Hom_{\C^\wg}(\IM\hy(f),\hy(Z))
$$ 
for all $Z$ in $\C$, and the {\em image}\index{image} of $f$ is the object $\Ima f$ of $\C^\wg$ defined by 
$$ 
(\Ima f)(Z):=\Hom_{\C^\vee}(\ky(Z),\IM\ky(f)) 
$$ 
for all $Z$ in $\C$. 
\end{df} 

\begin{prop}\lb{reg}
We may regard $(\Coim f)(Z)$ as a subset of $\Hom_\C(X,Z)$, and $(\Ima f)(Z)$ as a subset of $\Hom_\C(Z,Y)$. (These subsets will be spelled out by Proposition~\ref{epimono} below.)
\end{prop}

\begin{proof}
We prove that $(\Coim f)(Z)$ is naturally embedded in $\Hom_\C(X,Z)$. The morphisms 
$$
\hy(X)\to\IM \hy(f)\to\hy(Y)
$$ 
are given by the maps 
$$
\Hom_\C(Z,X)=\hy(X)(Z)\to\IM\hy(f)_Z\to\hy(Y)(Z)=\Hom_\C(Z,Y).
$$ 
In view of the definition of $\Coim(f)$, it suffices to check that $\hy(X)\to\IM\hy(f)$ is an epimorphism in $\C^\wg$, that is, it suffices, by Corollary~\ref{ccprime} p.~\pr{ccprime}, to check that the map $\hy(X)(Z)\to\IM\hy(f)_Z$ is surjective for all $Z$ in $\C$. But this is clear. 

We prove that $(\Ima f)(Z)$ is naturally embedded in $\Hom_\C(Z,Y)$. The morphisms 
$$
\ky(X)\to\IM\ky(f)\to\ky(Y)
$$ 
in $\C^\vee$ are given by the morphisms 
$$
\ky(Y)\to\IM\ky(f)\to\ky(X)
$$ 
in $\Set^\C$, which are, in turn, given by the maps 
$$
\Hom_\C(Y,Z)=\ky(Y)(Z)\to\IM\ky(f)_Z\to\ky(X)(Z)=\Hom_\C(X,Z).
$$ 
In view of the definition of $\Ima(f)$, it suffices to check that $\ky(Y)\to\IM\ky(f)$ is an epimorphism in $\Set^\C$, that is, it suffices, by Corollary~\ref{ccprime} p.~\pr{ccprime}, to check that the map $\ky(Y)(Z)\to\IM\ky(f)_Z$ is surjective for all $Z$ in $\C$. But this is clear. 
\end{proof} 

According to Proposition \ref{reg} we regard from now on $(\Coim f)(Z)$ as a subset of $\Hom_\C(X,Z)$ and $(\Ima f)(Z)$ as a subset of $\Hom_\C(Z,Y)$.\bigskip

\begin{conv}\lb{bra}
If $A\parar B\to C$ is a diagram in a given category, then the notation $[A\parar B\to C]$ shall mean that the two compositions coincide.
\end{conv}\bigskip

\begin{prop}\lb{epimono}
If $f:X\to Y$ is a morphism in a category $\C$, and if $Z$ is an object of $\C$, then we have 
$$
(\Coim f)(Z)=\left\{x:X\to Z\ \bigg|\ \left[W\parar X\xr fY\right]\then\left[W\parar X\xr xZ\right]\ \forall\ W\in\C\right\},
$$
$$
(\Ima f)(Z)=\left\{y:Z\to Y\ \bigg|\ \left[X\xr fY\parar W\right]\then\left[Z\xr yY\parar W\right]\ \forall\ W\in\C\right\}.
$$ 
In particular, these two sets do not depend on the universe $\U$ making $\C$ a $\U$\--category. There are natural morphisms 
$$
\ky(X)\to\Coim f\to\ky(Y),\quad\hy(X)\to\Ima f\to\hy(Y)
$$ 
in $\C^\vee$ and $\C^\wg$ respectively. Moreover, $\ky(X)\to\Coim f$ is an epimorphism, and $\Ima f\to\hy(Y)$ is a monomorphism. 
\end{prop} 

For the sake of emphasis we write
$$
\ky(X)\epi\Coim f\to\ky(Y),\quad\hy(X)\to\Ima f\mono\hy(Y).
$$ 

\begin{proof}
%Let $x:X\to Z$ and assume \begin{equation}\lb{lw}\left[W\parar X\xr fY\right]\then\left[W\parar X\xr xZ\right]\ \forall\ W\in\C.\end{equation} Let us prove that $x$ is in $\Coim f$, that is, let us prove that there is a morphism $\IM\hy(f)\to\hy(Z)$ in $\C^\wg$ such that the composition $$\hy(X)\to\IM\hy(f)\to\hy(Z)$$ is $\hy(x)$. Let $W$ be in $\C$. Recalling that $(\IM\hy(f))(W):=\IM\hy(f)_W$, we see that it suffices to show that there is a map $$\IM\hy(f)_W\to\Hom_\C(W,Z)$$ such that the composition $$\Hom_\C(W,X)\to\IM\hy(f)_W\to\Hom_\C(W,Z)$$ is $\hy(f)_W:\hy(X)(W)\to\hy(Z)(W)$. But this follows from \qr{lw}. 
To prove the first equality, let $x:W\to X$ be a morphism in $\C$ and consider the condition

\nn(a) there is a map $u:f\ci\Hom_\C(W,X)\to\Hom_\C(W,Z)$ such that $u(g)=x\ci g$ for all $g$ in $\Hom_\C(W,X)$: 
$$
\begin{tikzcd}
\Hom_\C(W,X)\ar[rd]\ar[rr,"x\ci"]&&\Hom_\C(W,Z)\\ 
&f\ci\Hom_\C(W,X).\ar[ru,dashed,"u"']
\end{tikzcd} 
$$ 
It suffices to show that (a) is equivalent to 

\nn(b) $\ds\left[W\parar X\xr fY\right]\then\left[W\parar X\xr xZ\right]$.

To show (a)$\then$(b), let $g_1$ and $g_2$ in $\Hom_\C(W,X)$ satisfy $f\ci g_1=f\ci g_2$. This yields $x\ci g_1=u(f\ci g_1)=u(f\ci g_2)=x\ci g_2$. 

To show (b)$\then$(a), given $g$ in $\Hom_\C(W,X)$ we must prove that the morphism $x\ci g$ does depends only on $f\ci g$, and not on $g$ itself. But this is precisely what (b) says. This proves the first equality in the statement of the proposition.

Let us show that the natural morphism $\ky(X)\to\Coim f$ is an epimorphism. As $\ky(X)\to\Coim f$ is a morphism in $\C^\vee$, it is given by a morphism $\Coim f\to\ky(X)$ in $\Set^\C$, and we must check that $\Coim f\to\ky(X)$ is a monomorphism in $\Set^\C$. But in Proposition~\ref{reg}, we noticed that $(\Coim f)(Z)$ could be viewed as a subset of $\Hom_\C(X,Z)=\ky(X)(Z)$ for any $Z$ in $\C$. 

Let us show that the natural morphism $\Ima f\to\hy(Y)$ is an monomorphism. But in Proposition~\ref{reg}, we noticed that $(\Ima f)(Z)$ could be viewed as a subset of $\Hom_\C(Z,Y)=\hy(Y)(Z)$ for any $Z$ in $\C$. 

The rest of the proof is left to the reader.
\end{proof}

By \qr{IM2} we have 
$$ 
(\Coim f)(Z)\iso\Ker\Big(\Hom_\C(X,Z)\parar\Hom_{\C^\wg}\big(\hy(X)\tm_{\hy(Y)}\hy(X),\hy(Z)\big)\Big), 
$$ 
$$ 
(\Ima f)(Z)\iso\Ker\Big(\Hom_\C(Z,Y)\parar\Hom_{\C^\vee}\big(\ky(Z),\ky(Y)\sqcup_{\ky(X)}\ky(Y)\big)\Big). 
$$
 
This implies 

\begin{prop}\lb{coimim}
If $P:=X\tm_YX$ exists in $\C$, then $\Coim f$ is naturally isomorphic to $\Coker(P\parar X)\in\C^\vee$. If $S:=Y\sqcup_XY$ exists in $\C$, then $\Ima f$ is naturally isomorphic to $\Ker(Y\parar S)\in\C^\wg$. 
\end{prop} 

In view of Lemma~\ref{imset} and Proposition~\ref{coimim} we can replace the notation $\IM$ with $\Ima$ (or $\Coim$). The following proposition is obvious: 

\begin{prop}\lb{fun}
We have: 

$f\mt\Ima\hy(f)$ and $\Ima$ are functors from $\Mor(\C)$ to $\C^\wg$, 

$f\mt\Ima\ky(f)$ and $\Coim$ are functors from $\Mor(\C)$ to $\C^\vee$. 
\end{prop}

\begin{df}[strict epimorphism] 
A morphism $f:X\to Y$ in a category $\C$ is a {\em strict epimorphism}\index{strict epimorphism} if the morphism $\Coim f\to\ky(Y)$ in $\C^\vee$ is an isomorphism.
\end{df} 

The proposition below is obvious:

\begin{prop}\lb{strepi}
A morphism $f:X\to Y$ in a category $\C$ is a strict epimorphism if and only if, for all $Z$ in $\C$, the map 
$$
\ci f:\Hom_\C(Y,Z)\to\Hom_\C(X,Z)
$$ 
induces a bijection 
$$
\Hom_\C(Y,Z)\xr\sim(\Coim f)(Z).
$$ 
By Proposition~\ref{epimono} p.~\pr{epimono}, this condition does not depend on the universe $\U$ making $\C$ a $\U$-category. Moreover, a strict epimorphism is an epimorphism. 
\end{prop}

%%

\sbs{Brief comments}

\begin{s}%\lb{515i}
P.~115, Proposition 5.1.5 (i). For the sake of completeness we spell out some details, and, for the reader's convenience we reproduce Proposition 5.1.5 (i) p.~115 of the book. 

\begin{prop}[Proposition 5.1.5 (i) p.~115]\lb{515i} 
If $\C$ is a category admitting finite inductive and projective limits, then the following five conditions on a morphism $f:X\to Y$ are equivalent:

\nn\emph{(a)} $f$ is an epimorphism and $\Coim f\to\Ima f$ is an isomorphism,

\nn\emph{(b)} $\Coim f\xr\sim Y$,

\nn\emph{(c)} the sequence $X\tm_YX\parar X\to Y$ is exact,

\nn\emph{(d)} there exists a pair of parallel arrows $g,h:Z\parar X$ such that $f\ci g=f\ci h$ and $\Coker(g,h)\to Y$ is an isomorphism,

\nn\emph{(e)} for any $Z$ in $\C$, the set $\Hom_\C(Y,Z)$ is isomorphic to the set of morphisms $u:X\to Z$ satisfying $u\ci v_1=u\ci v_2$ for any pair of parallel morphisms $v_1,v_2:W\parar X$ such that $f\ci v_1=f\ci v_2$. 
\end{prop} 

Here are the additional details: 

\nn(b)$\then$(a): The composition $\Coim f\to\Ima f\to Y$ being an isomorphism by assumption, $\Ima f\to Y$ is an epimorphism. Then Proposition 5.1.2 (iv) of the book implies that $f$ is an epimorphism and that $\Ima f\to Y$ is an isomorphism, from which we conclude that $\Coim f\to\Ima f$ is an isomorphism.

%\nn(c)$\ssi$(e): Write $P$ for $X\tm_YX$. Recall that (c) says that $P\parar X\to Y$ is exact. Proposition~\ref{epimono} p.~\pr{epimono} implies that Condition~(e) is equivalent to the condition in Proposition~\ref{strepi} p.~\pr{strepi}. Thus, it suffices to show that, letting $Z$ be an object of $\C$, we have in the notation of Convention~\ref{bra} p.~\pr{bra} $$[P\parar X\xr xZ]\iff\Big((\forall\ W\in\C)\ [W\parar X\xr fY]\then[W\parar X\xr xZ]\Big).$$ Implication $\si$ is clear. Let us prove $\then$. Assuming $$[P\parar X\xr xZ]\quad\text{and}\quad[W\parar X\xr fY],$$ we must check $[W\parar X\xr xZ]$. As the morphisms $W\parar X$ factor through $P\parar X$ by definition of $P$, the statement is obvious. 

%(Note that the equivalence (c)$\ssi$(e) also follows from Proposition~\ref{epimono} p.~\pr{epimono}, Proposition~\ref{coimim} p.~\pr{coimim} and Proposition~\ref{strepi} p.~\pr{strepi}.)Exercise 3.4 (i) p.~90]\lb{34i}
\end{s}

%

\begin{s}
Proposition 5.1.5 p. 115. Here is a corollary to Proposition 5.1.5 and to Proposition~\ref{34i} p.~\pr{34i}:

\begin{cor}\lb{fg}
Let $F$ and $G$ be functors from a category $\C$ to a category $\C'$, let $\theta:F\to G$ be a morphism of functors, and consider the following conditions: 

\nn\emph{(a)} $\C'$ admits finite inductive and projective limits,

\nn\emph{(b)} $\theta$ is an epimorphism,

\nn\emph{(c)} $\theta$ is a strict epimorphism,

\nn\emph{(d)} $\theta_X:F(X)\to G(X)$ is an epimorphism for all $X$ in $\C$,

\nn\emph{(e)} $\theta_X:F(X)\to G(X)$ is a strict epimorphism for all $X$ in $\C$,

\nn\emph{(f)} $\theta$ is a monomorphism,

\nn\emph{(g)} $\theta_X:F(X)\to G(X)$ is a monomorphism for all $X$ in $\C$.

\nn Then \emph{(d)} $\then$ \emph{(b)}, \emph{(g)} $\then$ \emph{(f)}, \emph{(a)} and \emph{(b)} imply \emph{(d)}, \emph{(a)} and \emph{(f)} imply \emph{(g)}, \emph{(a)} implies that \emph{(c)} and \emph{(e)} are equivalent. 
\end{cor}
\end{s}

%

\begin{s} 
P.~116, proof of Proposition 5.1.7 (i) (minor variant). Recall the statement: 

\begin{prop}[Proposition 5.1.7 (i) p. 116]
Let $\C$ be a category admitting finite inductive and projective limits in which epimorphisms are strict. Let us denote by $I'_g$ the coimage of any morphism $g$ in $\C$. Let $f:X\to Y$ be a morphism in $\C$ and $X\xr u I'_f\xr v Y$ its factorization through $I'_f$. Then $v$ is a monomorphism. 
\end{prop}

\begin{proof}
Consider the commutative diagram
$$
\begin{tikzcd}
X\ar[two heads]{d}[swap]{b}\ar[two heads]{r}{u}&I'_f\ar[two heads]{d}{a}\ar{r}{v}&Y\\
I'_{a\ci u}\ar{ur}{d}\ar[two heads]{r}[swap]{c}&I'_v.\ar{ru}
\end{tikzcd}
$$ 
(We first form $a$, then $b$ and $c$, and finally $d$; the existence of $d$ is a very particular case of Proposition~\ref{fun} p.~\pr{fun}.) By (the dual of) Proposition 5.1.2 (iv) p.~114 of the book, it suffices to show that $a$ is an isomorphism. As $a\ci u$ is a strict epimorphism, Proposition~\ref{515i}, (a)$\then$(b), p.~\pr{515i}, implies that $c$ is an isomorphism. We claim that $d\ci c^{-1}$ is inverse to $a$. We have 
$$
a\ci d\ci c^{-1}=c\ci c^{-1}=\id_{I'_v}
$$ 
and 
$$
d\ci c^{-1}\ci a\ci u=d\ci c^{-1}\ci c\ci b=d\ci b=u=\id_{I'_f}\ci u,
$$ 
and the conclusion follows from the fact that $u$ is an epimorphism.
\end{proof}
\end{s}

%

\begin{s}
P.~117, Definition 5.2.1 (definition of a system of generators). There is an important comment about this in Pierre Schapira's Errata 

\href{https://webusers.imj-prg.fr/~pierre.schapira/books/Errata.pdf}{https://webusers.imj-prg.fr/$\sim$pierre.schapira/books/Errata.pdf}.

As observed at the bottom of p.~121 of the book, the definition can be stated as follows:

\begin{df}[generator, system of generators]\lb{dg1} 
Let $S$ be a set of objects of a category $\C$ and $\SSS$ the corresponding full subcategory. We say that $S$ is a {\em system of generators}\index{generator}\index{system of generators} if the functor $\pp:\C\to\SSS^\wg$, $X\mt\Hom_\C(\ ,X)$ is conservative. The notions of co-generator and system of co-generators are defined in the obvious way. 
\end{df}
\end{s}
	
%

\begin{s}
P.~118, second display: the isomorphism 
$$
\Hom_{\Set}\Big(\Hom_\C(G,X),\Hom_\C(G,X)\Big)\iso\Hom_{\C^\vee}(G^{\sqcup\Hom_\C(G,X)},X)
$$ 
is a particular case of the following isomorphism, which holds for any $\U$-set $S$ and any objects $G$ and $X$ of $\C$: 
$$
\Hom_{\Set}(S,\Hom_\C(G,X))\iso\Hom_{\C^\vee}(G^{\sqcup S},X).
$$ 
\end{s}

%

\begin{s}
P.~118, proof of Proposition 5.2.3: the proof of (ii) uses Proposition~\ref{34i} p.~\pr{34i} and Proposition 3.3.7 (i) p.~83 of the book. 
\end{s} 

%

\begin{s}
P.~119, Theorem 5.2.5: see Corollary~\ref{c2111} p.~\pr{c2111}.
\end{s}

%

\begin{s}
P.~121. Corollary 5.2.10 follows from Theorem 5.2.6 p.~119 and Proposition 5.2.9 p.~121 of the book.
\end{s}

%

\begin{s}
P.~122, sentence following Definition 5.3.1. This sentence is ``Note that if $\F$ is strictly generating,\index{strictly generating subcategory} then $\Ob(\F)$ is a system of generators''. See \S\ref{ffc} p.~\pr{ffc}. 
\end{s}  

%%

\sbs{Lemma 5.3.2 p.~122} 

Here is a minor variant of the proof of Lemma 5.3.2. 

\begin{lem}[Lemma 5.3.2 p.~122]
If $\F\subset\G$ are full subcategories of a category $\C$, and if $\F$ is strictly generating, then $\G$ is strictly generating. 
\end{lem} 

\begin{proof}
Let 
$$
\begin{tikzcd}
\C\ar{r}{\gamma}\ar{dr}[swap]{\pp}&\G^\wg\ar{d}{\rho}\\
&\F^\wg
\end{tikzcd}
$$ 
be the natural functors ($\rho$ being the restriction), and let $X$ and $Y$ be in $\C$. We have 
$$
\begin{tikzcd}
\Hom_\C(X,Y)\ar{r}{\gamma'}\ar{dr}{\sim}[swap]{\pp'}&\Hom_{\G^\wg}(\gamma(X),\gamma(Y))\ar{d}{\rho'}\\ 
{}&\Hom_{\F^\wg}(\pp(X),\pp(Y)). 
\end{tikzcd}
$$ 
We want to prove that $\gamma'$ is bijective. As $\pp'$ is bijective, it suffices to show that $\gamma'$ is surjective. Let $\xi$ be in $\Hom_{\G^\wg}(\gamma(X),\gamma(Y))$. There is a (unique) $f$ in $\Hom_\C(X,Y)$ such that  
\begin{equation}\lb{rhoxi}
\rho(\xi)=\pp(f),
\end{equation}
and it suffices to prove $\xi=\gamma(f)$. Let $Z$ be in $\G$ and $z$ be in $\Hom_\C(Z,X)$. It suffices to show that the morphisms 
$$
\begin{tikzcd}
Z\ar[yshift=0.7ex]{r}{\xi_Z(z)}\ar[yshift=-0.7ex]{r}[swap]{f\ci z}&Y
\end{tikzcd}
$$ 
coincide. As $\F$ is strictly generating, it suffices to show that the morphisms 
$$
\begin{tikzcd}
\pp(Z)\ar[yshift=0.7ex]{rr}{\pp(\xi_Z(z))}\ar[yshift=-0.7ex]{rr}[swap]{\pp(f\ci z)}&&\pp(Y)
\end{tikzcd}
$$ 
coincide. Let $W$ be in $\F$. It suffices to show that the maps 
$$
\begin{tikzcd}
\pp(Z)(W)\ar[yshift=0.7ex]{rr}{\pp(\xi_Z(z))_W}\ar[yshift=-0.7ex]{rr}[swap]{\pp(f\ci z)_W}&&\pp(Y)(W)
\end{tikzcd}
$$ 
coincide, that is, it suffices to show that the maps 
$$
\begin{tikzcd}
\Hom_\C(W,Z)\ar[yshift=0.7ex]{rr}{\xi_Z(z)\ci}\ar[yshift=-0.7ex]{rr}[swap]{f\ci z\ci}&&\Hom_\C(W,Y)
\end{tikzcd}
$$ 
coincide. We have, for $w$ in $\Hom_\C(W,Z)$,
$$
\xi_Z(z)\ci w
\os{\text{(a)}}{=}\xi_W(z\ci w)
\os{\text{(b)}}{=}\rho(\xi)_W(z\ci w)
\os{\text{(c)}}{=}\pp(f)_W(z\ci w)
\os{\text{(d)}}{=}f\ci z\ci w, 
$$ 
Equality~(a) following from the functoriality of $\xi$ (see diagram below), Equality~(b) following from the definition of $\rho$, Equality~(c) following from \qr{rhoxi}, and Equality~(d) following from the definition of $\pp$.
\end{proof}

For the reader's convenience, we add the commutative diagram 
$$
\begin{tikzcd}
Z&z\in\Hom_\C(Z,X)\ar{r}{\xi_Z}\ar{d}[swap]{\ci w}&\Hom_\C(Z,Y)\ar{d}{\ci w}\\ 
W\ar{u}{w}&\Hom_\C(W,X)\ar{r}[swap]{\xi_W}&\Hom_\C(W,X).
\end{tikzcd}
$$

%%

\sbs{Brief comments}

\begin{s}
P.~122. The proof of Lemma 5.3.3 proves a statement that is much stronger than Lemma 5.3.3. This stronger statement can be phrased as follows:
\begin{lem}\lb{533}
Let $\C$ be a category which admits small inductive limits, let $\F$ be a small full subcategory of $\C$, let $F$ be in $\F^\wg$, set 
$$
\psi(F):=\col_{(Y\to F)\in\F_F}Y,
$$ 
let $X$ be in $\C$, let $f:\psi(F)\to X$ be a morphism in $\C$ and, for each $(Y\to F)\in\F_F$, let $f_{Y\to F}:Y\to X$ be the composition of $f$ with the coprojection $Y\to\psi(F)$. Then there is a unique morphism $\theta:F\to\pp(X)$ in $\F^\wg$ such that 
$$
\theta_Y(Y\to F)=f_{Y\to F}
$$ 
for all $Y$ in $\F$. Moreover, the map 
$$
\Hom_\C(\psi(F),X)\to\Hom_{\F^\wg}(F,\pp(X)),\quad f\mt\theta
$$ 
is bijective and functorial in $F$ and $X$. In particular $\psi:\F^\wg\to\C$ is left adjoint to $\pp:\C\to\F^\wg$. 
\end{lem} 

\begin{proof}
We have, for $X$ in $\C$ and $A$ in $\F^\wg$, 
$$
\Hom_\C\left(\col_{(Y\to A)\in\F_A}Y,X\right)\xr\sim\lim_{(Y\to A)\in\F_A}\Hom_\C(Y,X)
$$ 
$$
\xr\sim\lim_{(Y\to A)\in\F_A}\pp(X)(Y)\xr\sim\Hom_{\F^\wg}(A,\pp(X)),
$$  
the last isomorphism following from \qr{263c} p.~\pr{263c}, and it is straightforward to check that the composition of these bijections coincides with the map $f\mt\theta$ in the statement of our lemma. 
\end{proof}
\end{s} 

%

\begin{s}
P. 123, proof of Theorem 5.3.4. The following fact is implicit in the proof: The map 
$$
f:\Hom_{\F^\wg}(\pp\psi(G),F)\to\Hom_{\F^\wg}(G,F),
$$ 
obtained by composing the chain of isomorphisms in the proof, is equal to $\ci\ee_G$. This equality is easily checked using Lemma~\ref{533}. 
\end{s} 

%%

\sbs{Theorem 5.3.6 p.~124} 

% previous version
% https://docs.google.com/document/d/1PDpFEz1ajTYw07OCkt00ZlY8ZX3xSNHy8nCedJpGQT4/edit

Recall that $\C$ is a category, that $\F$ is an essentially small full subcategory of $\C$, that the functor $\pp:\C\to\F^\wg$ is defined by $\pp(X)(Y):=\Hom_\C(Y,X)$, and that, by definition, $\F$ is strictly generating if and only if $\pp$ is fully faithful.

\begin{thm}[Theorem 5.3.6 p.~124]\lb{536} 
Let $\C$ be a category satisfying the conditions \emph{(i)-(iii)} below:

\nn\emph{(i)} $\C$ admits small inductive limits and finite projective limits, 

\nn\emph{(ii)} small filtrant inductive limits are stable by base change (Definition 2.2.6 p.~47 of the book; see \S\ref{parsbc} p.~\pr{parsbc}), 

\nn\emph{(iii)} any epimorphism is strict. 

\nn Let $\F$ be an essentially small full subcategory of $\C$ such that 

\nn\emph{(a)} $\Ob(\F)$ is a system of generators,

\nn\emph{(b)} $\F$ is closed by finite coproducts in $\C$. 

\nn Then $\F$ is strictly generating.
\end{thm}

\begin{proof} % previous version
% https://docs.google.com/document/d/1w-6neN0st96QuZpMjrMb-wHMEcPjQdrHK5MsjeXpSxc/edit
We may assume from the beginning that $\F$ is small.

\nn\textbf{Step 1.} By Proposition 5.2.3 (i) p.~118 of the book, the functor $\pp$ is conservative and faithful.

\nn\textbf{Step 2.} By Proposition 1.2.12 p.~16 of the book, a morphism $f$ in $\C$ is an epimorphism as soon as $\pp(f)$ is an epimorphism.

\nn\textbf{Step 3.} Let $C$ be in $\F^\wg$, and let $(B_i\to C)_{i\in I}$ be a small filtrant inductive system in $(\F^\wg)_C$. We claim that the natural morphism
\begin{equation}\lb{colcoim}
\col_i\Coim(B_i\to C)\to\Coim\left(\col_iB_i\to C\right)
\end{equation}
is an isomorphism.

Let $X$ and $Y$ be in $\C$.

\nn\textbf{Step 4.} If $z:Z\to X$ is in $\F_X$, then the natural map 
\begin{equation}\lb{czy}
\Hom_\C(Z,Y)\to\Hom_{\F^\wg}\big(\pp(Z),\pp(Y)\big),
\end{equation} 
which is bijective by the Yoneda Lemma, induces a bijection 
\begin{equation}\lb{cczy}
\Hom_\C(\Coim z,Y)\xr\sim\Hom_{\F^\wg}(\Coim\pp(z),\pp(Y))
\end{equation} 
in the following sense:

There are natural isomorphisms 
$$
\Hom_\C(\Coim z,Y)\iso\Ker\big(\Hom_\C(Z,Y)\rightrightarrows\Hom_\C(Z\tm_XZ,Y)\big),
$$
$$
\Hom_{\F^\wg}(\Coim\pp z,\pp Y)\iso\Ker\Big(\Hom_{\F^\wg}\big(\pp Z,\pp Y\big)\rightrightarrows\Hom_{\F^\wg}\big(\pp Z\tm_{\pp X}\pp Z,\pp Y\big)\Big).
$$ 
(We have omitted some parenthesis to save space.) Let $Z\to Y$ be a morphism in $\C$. Then $Z\to Y$ is in %$\Hom_\C(Z,Y)$, the morphism $Z\to Y$ is in 
$$
\Ker\big(\Hom_\C(Z,Y)\rightrightarrows\Hom_\C(Z\tm_XZ,Y)\big)
$$ 
if and only if its image $\pp(Z)\to\pp(Y)$ is in 
$$
\Ker\Big(\Hom_{\F^\wg}\big(\pp(Z),\pp(Y)\big)\rightrightarrows\Hom_{\F^\wg}\big(\pp(Z)\tm_{\pp(X)}\pp(Z),\pp(Y)\big)\Big).
$$ 

[To make our argument work, it is not enough that the natural bijection \qr{cczy} exists; the fact that it is induced by \qr{czy} will be crucial.] 

Let us denote by $I$ the set of finite subsets of $\Ob(\F_X)$, ordered by inclusion. Regarding $I$ as a category, it is small and filtrant. For any $A$ in $I$ set $Z_A:=\bigsqcup_{a\in A}\zeta(a)$, where $\zeta:\C_X\to\C$ is the forgetful functor.

\nn\textbf{Step 5.} We claim that the natural morphism 
$$
\col_{A\in I}\pp(Z_A)\to\pp(X) 
$$ 
is an epimorphism.

\nn\textbf{Step 6.} We claim that the natural morphism 
$$
\col_{A\in I}\Coim(Z_A\to X)\to X
$$ 
is an isomorphism.

These steps imply the theorem: Indeed, we have, in the above setting, 
\begin{align*} 
\Hom_\C(X,Y)&\xr\sim\Hom_\C\left(\col_{A\in I}\Coim(Z_A\to X),Y\right)&\text{by Step 6}\\ \\ 
&\xr\sim\lim_{A\in I}\Hom_\C(\Coim(Z_A\to X),Y)\\ \\ 
&\xr\sim\lim_{A\in I}\Hom_{\F^\wg}\Big(\Coim(\pp(Z_A)\to\pp(X)),\pp(Y)\Big)&\text{by Step 4}\\ \\ 
&\xr\sim\Hom_{\F^\wg}\left(\col_{A\in I}\Coim(\pp(Z_A)\to\pp(X)),\pp(Y)\right)\\ \\ 
&\xr\sim\Hom_{\F^\wg}\left(\Coim\left(\col_{A\in I}\pp(Z_A)\to\pp(X)\right),\pp(Y)\right)&\text{by Step 3}\\ \\ 
&\xr\sim\Hom_{\F^\wg}(\pp(X),\pp(Y))&\text{by Step 5.}
\end{align*} 

\begin{lem}\lb{sixbij} 
Taking Steps 1 to 6 for granted, the composition of the six above bijections coincides with the natural map $\Hom_\C(X,Y)\to\Hom_{\F^\wg}(\pp(X),\pp(Y))$.  
\end{lem}

\begin{proof}
Let us denote these six bijections by $f_1,\dots,f_6$; let $u:X\to Y$ be in $\Hom_\C(X,Y)$; and let us compute successively $f_1u,f_2f_1u,\dots,f_6\dots f_1u$. We have 
$$
f_1u=\left(\col_{A\in I}\Coim(Z_A\to X)\xr\sim X\xr uY\right)\in\Hom_\C\left(\col_{A\in I}\Coim(Z_A\to X),Y\right).
$$ 
Then $f_2f_1u$ is the obvious family 
\begin{equation}\lb{za1}
(\Coim(Z_A\to X)\to X\xr uY)_A\in\lim_{A\in I}\Hom_\C(\Coim(Z_A\to X),Y).
\end{equation}

Let us compute $f_3f_2f_1u$ thanks to Step~4. Firstly, to the family \qr{za1} we attach a certain family $(Z_A\to Y)_A$, each of whose member $Z_A\to Y$ is in 
$$
\Ker\big(\Hom_\C(Z_A,Y)\rightrightarrows\Hom_\C(Z_A\tm_XZ_A,Y)\big).
$$ 
Secondly, applying the functor $\pp$ to the family $(Z_A\to Y)_A$ we get a certain family $(\pp(Z_A)\to\pp(Y))_A$, each of whose member $\pp(Z_A)\to\pp(Y)$ is in 
$$
\Ker\Big(\Hom_{\F^\wg}\big(\pp(Z_A),\pp(Y)\big)\rightrightarrows\Hom_{\F^\wg}\big(\pp(Z_A)\tm_{\pp(X)}\pp(Z_A),\pp(Y)\big)\Big).
$$ 
Thirdly, to the family $(\pp(Z_A)\to\pp(Y))_A$ we attach a family of morphisms 
\begin{equation}\lb{za2}
\Coim(\pp(Z_A)\to\pp(X))\to\pp(X)\xr{\pp(u)}\pp(Y),
\end{equation} 
family which makes up the sought-for morphism $f_3f_2f_1u$. The morphisms \qr{za2} give rise to a morphism 
\begin{equation}\lb{za3}
\col_{A\in I}\Coim(\pp(Z_A)\to\pp(X))\to\pp(X)\xr{\pp(u)}\pp(Y),
\end{equation} 
morphism equals to $f_4f_3f_2f_1u$. The morphism \qr{za3} induces a morphism 
$$
\Coim\left(\col_{A\in I}\pp(Z_A)\to\pp(X)\right)\to\pp(X)\xr{\pp(u)}\pp(Y),
$$ 
morphism equals to $f_5f_4f_3f_2f_1u$. This shows that $f_6f_5f_4f_3f_2f_1u$ is indeed equal to $\pp(X)\xr{\pp(u)}\pp(Y)$. 
\end{proof}

It remains to prove Steps 3, 4, 5 and 6. 

\begin{proof}[Proof of Step 3] 
Set $B:=\col_iB_i$. 

\begin{lem}\lb{bibj}
The natural morphism $\col_iB_i\tm_CB_i\to B\tm_CB$ is an isomorphism. 
\end{lem}

In the diagrams used to prove Step~3, the undefined arrows are the obvious ones.

\begin{proof}[Proof of Lemma \ref{bibj}]
Consider the commutative diagrams 

$$\begin{tikzcd}
\ds\col_iB_i\tm_CB_i\ar[r,"a"]&\ds\col_{i,j}B_i\tm_CB_j\\ 
B_i\tm_CB_i\ar[u]\ar[r,"\id"']&B_i\tm_CB_i,\ar[u]
\end{tikzcd}$$ 

$$\begin{tikzcd}
\ds\col_{i,j}B_i\tm_CB_j\ar[r,"b"]&\ds\col_i\col_jB_i\tm_CB_j\ar[r,"c"]&\ds\col_iB_i\tm_CB\\ 
&\ds\col_jB_i\tm_CB_j\ar[u]\ar[ur]\\ 
B_i\tm_CB_j\ar[uu]\ar[r,"\id"']&B_i\tm_CB_j\ar[u]\ar[r]&B_i\tm_CB,\ar[uu]
\end{tikzcd}$$ 

$$\begin{tikzcd}
\ds\col_iB_i\tm_CB\ar[r,"d"]&B\tm_CB\\ 
B_i\tm_CB.\ar[u]\ar[ur]
\end{tikzcd}$$ 

%(Note that the last row of the first diagram coincides with the first row of the second diagram, and the last row of the second diagram coincides with the first row of the third diagram.) 
\nn The composition $d\ci c\ci b\ci a$ equals \qr{colcoim}, and $a$ is an isomorphism by Corollary 3.2.3 (ii) p.~79 of the book, $b$ is an isomorphism by \S\ref{217} p.~\pr{217}, % for obvious reasons, 
$c$ and $d$ are isomorphisms because inductive limits in $\F^\wg$ indexed by $I$ are stable by base change (see Section~\ref{sbsarsbc} p.~\pr{sbsarsbc}). This proves Lemma~\ref{bibj}.
\end{proof}

Taking the definition of $\Coim$ into account, we have 
$$
\Coim(B_i\to C)=\Coker(B_i\tm_CB_i\parar B_i),
$$ 
$$
\Coker(B\tm_CB\parar B)=\Coim(B\to C).
$$ 
Moreover, there is an obvious commutative diagram
$$
\begin{tikzcd}
\ds\col_i\Coker(B_i\tm_CB_i\parar B_i)\ar[r,"e"]&\ds\Coker(\col_i(B_i\tm_CB_i)\parar B)\\ 
\Coker(B_i\tm_CB_i\parar B_i),\ar[u]\ar[r,"\id"']&\Coker(B_i\tm_CB_i\parar B_i),\ar[u]
\end{tikzcd}
$$ 
where $e$ is an isomorphism, and Lemma \ref{bibj} yields a commutative diagram 
$$
\begin{tikzcd}
\ds\Coker(\col_i(B_i\tm_CB_i)\parar B)\ar[r,"f"]&\ds\Coker(B\tm_CB\parar B)\\ 
\Coker(B_i\tm_CB_i\parar B_i)\ar[u]\ar[r,"\id"']&\Coker(B_i\tm_CB_i\parar B_i),\ar[u]
\end{tikzcd}
$$ 
where $f$ is an isomorphism. This implies that \qr{colcoim} is an isomorphism, completing the proof of Step~3. 
\end{proof} 

\begin{proof}[Proof of Step 4] 
We have 
$$
\Hom_\C(\Coim z,Y)=\Hom_\C\big(\Coker(Z\tm_XZ\rightrightarrows Z),Y\big)
$$
$$
\iso\Ker\big(\Hom_\C(Z,Y)\rightrightarrows\Hom_\C(Z\tm_XZ,Y)\big),
$$ 
and similarly 
$$
\Hom_{\F^\wg}(\Coim\pp z,\pp Y)\iso\Ker\Big(\Hom_{\F^\wg}\big(\pp Z,\pp Y\big)\rightrightarrows\Hom_{\F^\wg}\big(\pp Z\tm_{\pp X}\pp Z,\pp Y\big)\Big).
$$ 
The natural map 
$$
\Hom_\C(Z,Y)\to\Hom_{\F^\wg}(\pp(Z),\pp(Y))
$$ 
is bijective by the Yoneda Lemma. As $\pp$ is faithful, the natural map 
$$
\Hom_\C(Z\tm_XZ,Y)\to\Hom_{\F^\wg}\big(\pp(Z\tm_XZ),\pp(Y)\big)
$$
$$
\iso\Hom_{\F^\wg}\big(\pp(Z)\tm_{\pp(X)}\pp(Z),\pp(Y)\big).
$$ 
is injective. This implies our claims. 
\end{proof}

\begin{proof}[Proof of Step 5]
Let $Z$ be in $\F$. We must show that the natural map 
$$
\col_{A\in I}\ \pp(Z_A)(Z)\to\pp(X)(Z):=\Hom_\C(Z,X) 
$$
is surjective. Let $z$ be in $\Hom_\C(Z,X)$. It suffices to check that $z$ is in the image of the natural map 
$$
\pp(Z_{\{z\}})(Z)=\Hom_\C(Z,Z)\xr{z\ci}\Hom_\C(Z,X),
$$
which is obvious. 
\end{proof}

\begin{proof}[Proof of Step 6] 
As Step 3 implies 
$$
\col_{A\in I}\Coim(Z_A\to X)\iso\Coim\left(\col_{A\in I}Z_A\to X\right),
$$ 
it suffices to prove 
\begin{equation}\lb{step6a}
\Coim\left(\col_{A\in I}Z_A\to X\right)\iso X.
\end{equation} 
Epimorphisms being strict, it is enough, in view of Proposition~\ref{515i}, (a)$\then$(b), p.~\pr{515i} to check that 
\begin{equation}\lb{step6b}
\col_{A\in I}Z_A\to X
\end{equation} 
is an epimorphism. Let 
$$
\col_{A\in I}\pp(Z_A)\xr{b}\pp\left(\col_{A\in I}Z_A\right)\xr{a}\pp(X)
$$
be the natural morphisms. As $a\ci b$ is an epimorphism by Step~5, $a$ is an epimorphism, and Step~2 implies that \qr{step6b} is also an epimorphism. 
\end{proof}
\end{proof}

%%

\sbs{Brief comments}

\begin{s}
P.~127, proof of Theorem 5.3.8. 

The sentence ``Since $\pp$ is conservative by (a), it remains to show that $\pp(u)$ is a monomorphism'' is justified by Proposition 5.1.5 (ii) p.~115 of the book and Corollary~\ref{fg} p.~\pr{fg} above.

The phrase ``the two arrows $\pp(X_{i_1}\tm_XX_{i_1})\parar\col_i\pp(X_i)$ coincide'' can be justified as follows: The two compositions  
$$
\begin{tikzcd}
X_{i_1}\tm_XX_{i_1}\ar[yshift=0.7ex]{r}{\xi_1}\ar[yshift=-0.7ex]{r}[swap]{\xi_2}&X_0\ar{r}{u}&X
\end{tikzcd}
$$ 
coincide by definition. Thus the two compositions  
$$
\pp(X_{i_1}\tm_XX_{i_1})\parar\pp(X_0)\to\pp(X),
$$ 
which can be written as 
$$
\pp(X_{i_1}\tm_XX_{i_1})\parar\col_i\pp(X_i)\to\pp(X_0)\to\pp(X),
$$
coincide. The composition $\col_i\pp(X_i)\to\pp(X_0)\to\pp(X)$ being an isomorphism, the two morphisms $\pp(X_{i_1}\tm_XX_{i_1})\parar\col_i\pp(X_i)$ coincide.
\end{s}

%

\begin{s}
P.~128, Theorem 5.3.9. To prove the existence of $\F$, one can also argue as follows. 

\begin{lem} 
Let $\C$ be a category admitting finite inductive limits, and let $\A$ be a small full subcategory of $\C$. Then:

\nn{\em(a)} There is a small full subcategory $\B$ of $\C$ such that $\A\subset\B\subset \C$ and that $\B$ is closed by finite inductive limits in the following sense: if $(X_i)$ is a finite inductive system in $\B$ and $X$ is an inductive limit of $(X_i)$ in $\C$, then $X$ is isomorphic to some object of $\B$.

\nn{\em(b)} There is a small full subcategory $\A'$ of $\C$ such that $\A\subset\A'\subset \C$ and that each finite inductive system in $\A$ has a limit in $\A'$. 
\end{lem} 

\begin{proof}
Since there are only countably many finite categories up to isomorphism, (b) is clear. To prove (a), let $\A\subset\A'\subset\A''\subset\cdots$ be a tower of full subcategories obtained by iterating the argument used to prove (b), and let $\B$ be the union of the $\A^{(n)}$.
\end{proof}
\end{s}

%%%

\section{About Chapter 6}

\sbs{Definition 6.1.1 p. 131}\lb{131}

Here is an example of an object of $\Ind(\Set)$ which is isomorphic (in $\Ind(\Set)$) to no object of $\Set$. 

For each $n$ in $\bb N$ set 
$$
\mbf n:=\{0,\dots,n-1\}\subset\bb N,
$$ 
let $\bb N\to\Set,n\mt\mbf n$ be the obvious functor, and set 
$$
\bb N':=\ic_n\mbf n. 
$$ 
(Note that $A(X)$ can be identified to the set of bounded maps from $X$ to $\mathbb N$.) We clearly have $\bb N'\in\Ind(\Set)$. Let $X$ be in $\Set$ and let $u:X\to\bb N'$ be a morphism. To prove that $\bb N'$ is isomorphic to no object of $\Set$, it suffices to show that $u$ is \emph{not} an epimorphism. As $u$ factors through some coprojection $p_n:\mbf n\to\bb N'$, if $u$ were an epimorphism, so would be $p_n$.  

Claim: $p_n$ is \emph{not} an epimorphism. 

Proof: Let $f:\bb N'\to\bb N$ be the natural morphism. There is a morphism $g:\bb N'\to\bb N$ such that 
$$
g(p_i(j))=\begin{cases}j&\text{ if }j\le n\\ n&\text{ if }j\ge n\end{cases}
$$ 
whenever $0\le j<i$, and we have $g\ci p_n=f\ci p_n$ but $g\neq f$. This proves the claim, and, thus, the fact that $\bb N'$ is isomorphic to no object of $\Set$. 

%We have just seen that the natural morphism $f:\bb N'\to\bb N$ is \emph{not} an isomorphism. 

Claim: The natural morphism $f:\bb N'\to\bb N$ is a monomorphism. 

Proof: Let 
$$
\begin{tikzcd}
A\ar[r,yshift=4pt,"g"]\ar[r,yshift=-4pt,"h"']&\bb N'\ar[r,"f"]&\bb N
\end{tikzcd}
$$ 
be a diagram in $\Ind(\Set)$ with $g\neq h$. It suffices to prove $f\ci g\neq f\ci h$. We can assume that $A$ is in $\Set$. There is an $a$ in $A$ satisfying $g(a)\neq h(a)$. Recall that $p_n:\mbf n\to\bb N'$ is the $n$-th coprojection. There is an $n$ in $\bb N$ and there are $g_n,h_n:A\parar\mbf n$ such that $g=p_n\ci g_n,\ h=p_n\ci h_n$: 
$$
\begin{tikzcd}
A\ar[d,xshift=4pt,"h_n"]\ar[d,xshift=-4pt,"g_n"']\ar[r,yshift=4pt,"g"]\ar[r,yshift=-4pt,"h"']&\bb N'\ar[r,"f"]&\bb N\\ 
\mbf{n}.\ar[ru,bend right,"p_n"']
\end{tikzcd}
$$ 
The map $f\ci p_n$ being injective, this yields
$$
(f\ci g)(a)=(f\ci p_n)(g_n(a))\neq(f\ci p_n)(h_n(a))=(f\ci h)(a),
$$ 
and thus $f\ci g\neq f\ci h$. Hence, $f$ is a monomorphism. 

Claim: $f$ is \emph{not} an epimorphism. 

Proof 1: Use Proposition~\ref{si} p.~\pr{si} below. 

Proof 2: (Proof 2 is more direct.) Define $u:\bb N\to\bb N$ by 
$$
u(i):=\begin{cases}i-1&\text{if }i\neq0\\ 0&\text{if }i=0,\end{cases}
$$ 
and define the functor $\al:\bb N\to\Set$ as follows. To the object $i$ of $\bb N$ we attach the object $\bb N$ of $\Set$, and to the inequality $i\le j$ in $\bb N$ we attach the endomap $u^{j-i}$ of $\bb N$. Set $A:=\ic\al$, let $q_i:\bb N\to A$ be the $i$-th coprojection and define $g:\bb N\to\bb N$ by $g(i)=0$ for all $i\in\bb N$. It is easy to check that we have $q_0\neq q_0\ci g$ and $q_0\ci f=q_0\ci g\ci f$. This proves the claim. 

%%

\sbs{Theorem 6.1.8 p. 132} 

Recall the statement:

\begin{thm}[Theorem 6.1.8 p.~132]\lb{618}
If $\C$ is a category, then the category $\Ind(\C)$ admits small filtrant inductive limits and the natural functor $\Ind(\C)\to\C^\wg$ commutes with such limits.
\end{thm} 

%The proposition below follows from Theorem~\ref{618} above and Proposition~\ref{333} p.~\pr{333}. 

%\begin{prop} If $\C$ is a category, then small filtrant inductive limits exist in $\Ind(\C)$ and are right exact. \end{prop}

Here is a minor variant of Step~(i) of the proof of Theorem 6.1.8. We must show:  

\begin{lem} 
If $\al:I\to\Ind(\C)$ is a functor, if $I$ is small and filtrant, and if we define $A\in\C^\wg$ by $A=\ic\al$, then $\C_A$ is filtrant. 
\end{lem} 

\begin{proof}
Let $M$ be the category attached by Definition 3.4.1 p. 87 of the book to the functors 
$$
\C\xr h\C^\wg\xl{\iota\ci\al}I,
$$ 
where $h:\C\to\C^\wg$ and $\iota:\Ind(\C)\to\C^\wg$ are the natural embeddings. Proposition~\ref{cocop} p.~\pr{cocop} implies that $M$ is filtrant, and that it suffices to check that Conditions (iii) (a) and (iii) (b) of Proposition 3.2.2 p.~78 of the book hold for the obvious functor $\pp:M\to\C_A$. Let us do it for Condition (iii) (b), the case of Condition (iii) (a) being similar and simpler. 

For all $i$ in $I$ and all $X$ in $\C$ let 
$$
p_i:\al(i)\to A\quad\text{and}\quad p_i(X):\Hom_\C(X,\al(i))\to A(X)
$$
be the coprojections. Note that $p_i(X)=p_i\ci$. 

Given an object $c$ of $\C_A$, and object $m$ of $M$, and a pair of parallel morphisms $\sigma,\sigma':c\parar\pp(m)$ in $\C_A$, we must find a morphism $\tau:m\to n$ in $M$ satisfying $\pp(\tau)\ci\sigma=\pp(\tau)\ci\sigma'$. 

Let $c$ be given by the morphism $X\to A$ in $\C^\wg$, let $m$ be given by the morphism $Y\to\al(i)$ in $\Ind(\C)$, and let $\sigma$ and $\sigma'$ be given by the morphisms $s,s':X\parar Y$ making the diagram below commute:
$$
\begin{tikzcd}
X\ar{dd}\ar[yshift=0.7ex]{r}{s}\ar[yshift=-0.7ex]{r}[swap]{s'}&Y\ar{d}{y}\\ 
{}&\al(i)\ar{d}{p_i}\\ 
A\ar[equal]{r}&A.
\end{tikzcd}
$$ 

Then we are looking for and object $n$ of $M$ given by a morphism $z:Z\to\al(j)$, and for a morphism $t:Y\to Z$ defining the sought-for morphism $\tau$. 

As $p_i(X)(y\ci s)$ equals $p_i(X)(y\ci s')$ in $A(X)\iso\col\Hom_\C(X,\al)$ and $I$ is filtrant, there is a morphism $t:i\to j$ in $I$ such that $\al(t)\ci y\ci s=\al(t)\ci y\ci s'$, and we can set $Z:=\al(j)$ and $z:=\id_{\al(j)}$. The situation is depicted by the commutative diagram
$$
\begin{tikzcd}
X\ar{dd}\ar[yshift=0.7ex]{r}{s}\ar[yshift=-0.7ex]{r}[swap]{s'}&Y\ar{d}{y}\ar{r}&\al(j)\ar[equal]{d}\\ 
{}&\al(i)\ar{d}{p_i}\ar{r}{\al(t)}&\al(j)\ar{d}{p_j}\\ 
A\ar[equal]{r}&A\ar[equal]{r}&A.
\end{tikzcd}
$$
\end{proof}

%%

\sbs{Proposition 6.1.9 p.~133} 

\subsubsection{Proof of Proposition 6.1.9}

The following point is implicit in the book, and we give additional details for the reader's convenience. Proposition 6.1.9 results immediately from the statement below:

\begin{prop} 
Let $\A$ be a category which admits small filtrant inductive limits, let $F:\C\to\A$ be a functor, and let $\C\xr i\Ind(\C)\xr j\C^\wg$ be the natural embeddings. Then the functor $i^\dg(F):\Ind(\C)\to\A$ exists, commutes with small filtrant inductive limits, and satisfies $i^\dg(F)\ci i\iso F$. Conversely, any functor $\widetilde F:\Ind(\C)\to\A$ commuting with small filtrant inductive limits with values in $\C$, and satisfying $\widetilde F\ci i\iso F$, is isomorphic to $i^\dg(F)$. 
\end{prop} 

\begin{proof}
The proof is essentially the same as that of Proposition 2.7.1 on p.~62 of the book. (See also \S\ref{c271b} p.~\pr{c271b}.) Again, we give some more details about the proof of the fact that $i^\dg(F)$ commutes with small filtrant inductive limits. Put $\widetilde F:=i^\dg(F)$. 

Let us attach the functor $B:=\Hom_\A(F(\ ),Y)\in\C^\wg$ to the object $Y$ of $\A$. To apply Proposition~\ref{2.1.10b} p.~\pr{2.1.10b} to the diagram 
$$
\begin{tikzcd}
I\ar{r}{\al}&\Ind(\C)\ar{d}[swap]{j}\ar{r}{\widetilde F}&\A\\
&\C^\wg
\end{tikzcd}
$$
(where $I$ is a small filtrant category), it suffices to check that there is an isomorphism 
$$
\Hom_\A\left(\widetilde F(\ ),Y\right)\iso\Hom_{\C^\wg}(\ \ ,B)
$$ 
in $\Ind(\C)^\wg_\V$, where $\V$ is a universe containing $\U$ such that $\C^\wg$ is a $\V$-category. We have 
$$
\widetilde F(A):=\col_{(X\to A)\in\C_A}F(X),
$$ 
as well as the following bijections functorial in $A\in\Ind(\C)$:
$$
\Hom_\A\left(\widetilde F(A),Y\right)=\Hom_\A\left(\col_{(X\to A)\in\C_A}F(X),Y\right)\iso\lim_{(X\to A)\in\C_A}B(X)
$$
$$
\iso\lim_{(X\to A)\in\C_A}\Hom_{\C^\wg}((j\ci i)(X),B)\iso\Hom_{\C^\wg}\left(\icolim_{(X\to A)\in\C_A}X,B\right)\iso\Hom_{\C^\wg}(j(A),B).
$$
\end{proof}

\subsubsection{Comments about Proposition 6.1.9} 

Let us record Part (i) of the proposition as 
\begin{equation}\lb{133i}
IF\ci\iota_\C\iso\iota_{\C'}\ci F, 
\end{equation} 
and note that we have, in the setting of Corollary 6.3.2 p.~140, 
\begin{equation}\lb{140}
\col F\ci\al\xr\sim(JF)(\ic\al).
\end{equation} 
Let us also record Part (ii) of the proposition as 
\begin{equation}\lb{133ii}
\ic(IF\ci\al)\xr\sim IF(\ic\al).
\end{equation} 
(See \S\ref{s133ii} p.~\pr{s133ii}.)

Also note that the proof of Proposition 6.1.9 shows

\begin{prop}\lb{f}
If $F:\C\to\C'$ is a functor of small categories, then the functor $\widehat F:\C^\wg\to\C'^\wg$ defined in Notation 2.7.2 p.~63 of the book induces the functor $IF:\Ind(\C)\to\Ind(\C')$.
\end{prop}

%%
% old version:
% https://docs.google.com/document/d/1CDoSFxs8pfmtM0ibrH0OiKADadx50sqoLAbaWn-u00U/edit

\sbs{Proposition 6.1.12 p.~134}\lb{6112}

We give some more details about the proof. Recall the setting: We have two categories $\C_1$ and $\C_2$, and we shall define functors
$$
\begin{tikzcd}
\Ind(\C_1\tm\C_2)\ar[yshift=0.7ex]{r}{\theta}&\Ind(\C_1)\tm\Ind(\C_2),\ar[yshift=-0.7ex]{l}{\mu}
\end{tikzcd}
$$ 
and prove that they are mutually quasi-inverse equivalences. (In fact, we shall only define the effect of $\theta$ and $\mu$ on objects, leaving also to the reader the definition of the effect of these functors on morphisms.) But first let us introduce some notation. We shall consider functors 
$$
A\in\Ind(\C_1\tm\C_2);\quad A_i,B_i\in\Ind(\C_i);
$$ 
objects $X_i,Y_i,\dots$ in $\C_i$; and elements 
$$
x\in A(X_1,X_2),\ y\in A(Y_1,Y_2),\dots,\quad x_i\in A_i(X_i),\ y_i\in A_i(Y_i),\dots
$$ 
When we write 
$$
\col_x\ \cdots,\quad\col_{x_i}\ \cdots,\quad\col_{x_1,x_2}\ \cdots,
$$ 
we mean, in the first case, not only that $x$ runs over the elements of $A(X_1,X_2)$, but also that $X_1$ and $X_2$ themselves run over the objects of $\C_1$ and $\C_2$, so that we are taking the inductive limit of some functor defined over $(\C_1\tm\C_2)_A$. In the other cases, the interpretation is similar. 

Let us define $\theta$ and $\mu$: We define $\theta$ by setting $\theta(A)=(A_1,A_2)$ with
\begin{equation}\lb{ai}
A_i:=\ic_x\ X_i, 
\end{equation} 
and we define $\mu$ by putting $\mu(A_1,A_2):=A_1\tm A_2$ with 
$$
(A_1\tm A_2)(X_1,X_2):=A_1(X_1)\tm A_2(X_2)
$$ 
for all $X_i$ in $\C_i$ ($i=1,2$).%$$\mu(A_1,A_2):=\ic_{x_1,x_2}\ (X_1,X_2). $$ 

\begin{prop}[Proposition 6.1.12 p.~134]\lb{p6112}
The functors $\theta$ and $\mu$ are mutually quasi-inverse.
\end{prop}

\begin{proof}
Let us prove
\begin{equation}\lb{6112a}
\theta\ci\mu\iso\id_{\Ind(\C_1)\tm\Ind(\C_2)}.
\end{equation}
If $A_i$ is in $\Ind(\C_i)$ for $i=1,2$; if $A$ is $A_1\tm A_2$; and if $(B_1,B_2)$ is $\theta(A)$, then we have 
$$ 
B_1
\os{\text{(a)}}{\iso}\ic_x\ X_1
\os{\text{(b)}}{\iso}\ic_{x_1,x_2}\ X_1
\os{\text{(c)}}{\iso}\ic_{x_1}\ X_1
\os{\text{(d)}}{\iso}A_1.
$$ 
Indeed, Isomorphism~(a) follows from \qr{ai}, Isomorphism~(b) from the definition of $A$, Isomorphism~(c) from the fact that the projection 
$$
(\C_1)_{A_1}\tm(\C_2)_{A_2}\to(\C_1)_{A_1}
$$ 
is cofinal by Lemma~\ref{proj} below coupled with the fact that $(\C_2)_{A_2}$ is connected, and Isomorphism~(d) from our old friend \qr{263a} p.~\pr{263a}. (By the way, in this proof we are using \qr{263a} a lot without explicit reference.)

\begin{lem}\lb{proj}
If $I$ and $J$ are categories and if $J$ is connected, then the projection $I\tm J\to I$ is cofinal.
\end{lem}

\begin{proof}
Let $i_0$ be in $I$. We must check that $(I\tm J)^{i_0}$ is connected. We have $(I\tm J)^{i_0}\iso I^{i_0}\tm J$, and it is easy to see that a product of two connected categories is connected. 
\end{proof}

This ends the proof of \qr{6112a}.

Let us prove
\begin{equation}\lb{6112b}
\mu\ci\theta\iso\id_{\Ind(\C_1\tm\C_2)}.
\end{equation}

Let $A$ be in $\Ind(\C_1\tm\C_2)$ and set $(A_1,A_2):=\theta(A)$. We shall define morphisms $A\to A_1\tm A_2$ and $A_1\tm A_2\to A$, and leave it to the reader to check that these morphisms are mutually inverse isomorphisms of functors. 

\nn$\bu$ Definition of the morphism $A\to A_1\tm A_2$: Let $X_i$ be in $\C_i$ ($i=1,2$). We must define a map $A(X_1,X_2)\to A_1(X_1)\tm A_2(X_2)$. We shall define firstly a map 
\begin{equation}\lb{ax1}
A(X_1,X_2)\to A_1(X_1). 
\end{equation}  
This will enable us to define a map $A(X_1,X_2)\to A_2(X_2)$ similarly, yielding our map $A(X_1,X_2)\to A_1(X_1)\tm A_2(X_2)$. As we have an isomorphism 
$$
A_1(X_1)\iso\col_y\Hom_{\C_1}(X_1,Y_1)
$$ 
and coprojections 
$$
p_{1y}:\Hom_{\C_1}(X_1,Y_1)\to A_1(X_1),
$$ 
we can define \qr{ax1} by $x\mt p_{1x}(\id_{X_1})$. We leave it to the reader to check that this does define a morphism $A\to A_1\tm A_2$.

\nn$\bu$ Definition of the morphism $A_1\tm A_2\to A$. Letting $X_i$ be in $\C_i$ as above, we must define a map $A_1(X_1)\tm A_2(X_2)\to A(X_1,X_2)$. Letting $x_i$ be in $A_i(X_i)$, we must define an element $x$ in $A(X_1,X_2)$. We have $x_1=p_{1y}(f_1)$ and $x_2=p_{2z}(f_2)$ for some $y$ and $z$ in $(\C_1\tm\C_2)_A$, some $f_1$ in $\Hom_{\C_1}(X_1,Y_1)$ and some $f_2$ in $\Hom_{\C_2}(X_2,Z_2)$. The category $(\C_1\tm\C_2)_A$ being filtrant, we can assume $z=y$. We have an isomorphism 
$$
A(X_1,X_2)\iso\col_w\Hom_{\C_1}(X_1,W_1)\tm\Hom_{\C_2}(X_2,W_2)
$$ 
and coprojections $q_w:\Hom_{\C_1}(X_1,W_1)\tm\Hom_{\C_2}(X_2,W_2)\to A(X_1,X_2)$, we can define $x$ by $x:=q_y(f_1,f_2)$. We leave it to the reader to check that this does define a morphism $A_1\tm A_2\to A$, and that this morphism is an inverse to the morphism $A\to A_1\tm A_2$ defined above.

This ends the proofs of Isomorphism~\qr{6112b} p.~\pr{6112b} and Proposition~\ref{p6112} p.~\pr{p6112}. 
\end{proof} 

%%

\sbs{Corollary 6.1.15 p.~135}\lb{s135b} 

%\begin{s}\lb{s135b} P.~135, Corollary 6.1.15. 
Recall the statement (see \S\ref{s135a} p.~\pr{s135a} above):

\begin{cor}[Corollary 6.1.15 p.~135]\lb{c135}
Let $f,g:A\parar B$ be two morphisms in $\Ind(\C)$. Then there exist a small and filtrant category $I$ and morphisms $\pp,\psi:\al\parar\bt$ of functors from $I$ to $\C$ such that $A\iso\ic\al$, $B\iso\ic\bt$, $f\iso\ic\pp$, $g\iso\ic\psi$. (The last two isomorphisms take place in $\Mor(\Ind(\C))$.)
\end{cor}

\begin{lem}\lb{l135}
Let $\al_1:I\to\C_1$ and $\al_2:I\to\C_2$ be functors defined on a small filtrant category $I$. Define $\al:I\to\C_1\tm\C_2$ by $\al(i):=(\al_1(i),\al_2(i))$ and let $X_k$ be in $\C_k\ (k=1,2)$. Then the natural map 
\begin{equation}\lb{e135}
(\ic\al)(X_1,X_2)\to(\ic\al_1)(X_1)\tm(\ic\al_2)(X_2)
\end{equation}
is bijective.
\end{lem}

\begin{proof}[Proof of Lemma \ref{l135}] 
This follows from Corollary 3.2.3 (ii) p.~79 and Proposition 3.1.11 (ii) of the book. Let us just add that \qr{e135} is the natural composition 
$$
\col_{i\in I}\Big(\Hom_{\C_1}(X_1,\al_1(i))\tm\Hom_{\C_2}(X_2,\al_2(i))\Big)\to
$$ 
$$
\col_{(i,j)\in I\tm I}\Big(\Hom_{\C_1}(X_1,\al_1(i))\tm\Hom_{\C_2}(X_2,\al_2(j))\Big)\to
$$
$$
\Big(\col_{i\in I}\Hom_{\C_1}(X_1,\al_1(i))\Big)\tm\Big(\col_{j\in I}\Hom_{\C_2}(X_2,\al_2(j))\Big).
$$ 
\end{proof}

\begin{proof}[Proof of Corollary \ref{c135}] 
Let $I$ and $J$ be small filtrant categories and let $\al:I\to\C$ and $\bt:J\to\C$ be two functors such that $A\iso\ic\al$ and $B\iso\ic\bt$. Denote by $\widetilde\al:I\to\C\tm\C$ the functor $i\mt(\al(i),\al(i))$, and similarly with $\bt$. 

Recall that there are quasi-inverse equivalences
$$
\begin{tikzcd}
\Ind(\C\tm\C)\ar[yshift=0.7ex]{r}{\theta}&\Ind(\C)\tm\Ind(\C),\ar[yshift=-0.7ex]{l}{\mu}
\end{tikzcd}
$$ 
that we sometimes write $A_1\tm A_2$ for $\mu(A_1,A_2)$, and that we have  
$$
(A_1\tm A_2)(X_1,X_2):=A_1(X_1)\tm A_2(X_2)
$$ 
for all $X_i$ in $\C$ ($i=1,2$). (See Section~\ref{6112} p.~\pr{6112}.)

Then $A\tm A\iso\ic\widetilde\al$ and $B\tm B\iso\ic\widetilde\bt$. By Lemma~\ref{l135} and the above reminder, the morphism $(f,g):(A,A)\to(B,B)$ in $\Ind(\C)\tm\Ind(\C)$ defines a morphism $f\tm g:A\tm A\to B\tm B$ in $\Ind(\C\tm\C)$. Applying Proposition 6.1.13 p.~134 in the book, we find a small and filtrant category $K$, functors $p_I:K\to I,p_J:K\to J$ and a morphism of functors $(\pp,\psi)$ from $\al\ci p_I$ to $\bt\ci p_J$ such that $f\tm g\iso\ic(\pp,\psi)$. It follows that $f\iso\ic\pp$ and $g\iso\ic\psi$.
\end{proof}
%\end{s}

%% 

\sbs{Brief comments} 

\begin{s} 
The proofs of Propositions 6.1.16 and 6.1.18 p.~136 in the book use the following lemma

\begin{lem}\lb{ci} 
Let $I$ be a small filtrant category, let $\C$ be category, and let $F:\C^I\to\Ind(\C)$ be the functor $\al\mt\ic\al$. 

\nn\emph{(a)} If $\C$ admits finite projective limits, $F$ is left exact.  

\nn\emph{(b)} If $\C$ admits finite inductive limits, $F$ is right exact. 
\end{lem} 

\begin{proof} 
Let $h:\C\to\C^\wg$ be the Yoneda embedding and $\iota:\C\to\Ind(\C)$ the natural embedding. If $\al$ is in $\C^I$, then $\ic\al$ can be defined as $\col h\ci\al$ or as $\col\iota\ci\al$ (Theorem 6.1.8 p.~132 in the book). Let $J$ be a finite category. 

\nn(a) Let $\al:I\tm J^{\op}\to\C$ be a functor. We claim 
$$
\ic_i\lim_j\al(i,j)\xr\sim\col_i h\left(\lim_j\al(i,j)\right). 
$$ 
Clearly $\ds\lim_j$ commutes with $h$. As $\ds\lim_j$ commutes also with $\col_i$ as far as $\C^\wg$-valued functors are concerned, $\ds\ic_i$ commutes with $\ds\lim_j$, and the claim is proved. 

\nn(b) Let $\al:I\tm J\to\C$ be a functor. We claim 
$$
\ic_i\col_j\al(i,j)\xr\sim\col_i\iota\left(\col_j\al(i,j)\right). 
$$ 
The functor $\iota$, being right exact (Corollary 6.1.6 p.~132 in the book), commutes with $\ds\col_j$. As $\ds\col_j$ commutes with $\ds\col_i$ for obvious reasons, it commutes with $\ds\ic_i$, and the claim is proved.
\end{proof} 

\end{s} 

%

\begin{s}

\begin{prop}\lb{si} 
If $\C$ is a category admitting finite inductive and projective limits, and if $\C$ is strict, then $\Ind(\C)$ is strict. 
\end{prop} 

\begin{lem}\lb{fcc'} 
If $F:\C\to\C'$ is an exact functor between categories admitting finite inductive and projective limits, and if $f$ is a strict morphism in $\C$, then $F(f)$ is a strict morphism in $\C'$. 
\end{lem}

\begin{proof} 
This is obvious. 
\end{proof} 

\begin{proof}[Proof of Proposition~\ref{si}] 
Let $f:A\to B$ be a morphism in $\Ind(\C)$. By Corollary 6.1.14 p.~135 in the book, there is a small filtrant category $I$ and a morphism $\pp$ in $\C^I$ such that $\ic\pp\iso f$. Clearly $\C^I$ is strict, and the theorem follows now from Lemmas \ref{ci} and \ref{fcc'}. 
\end{proof}

\end{s}

% 

\begin{s} 
P.~137, proof of Proposition 6.1.19. We just add a few references in the proof. Recall the statement:
\begin{prop}
If a category $\C$ admits finite inductive limits and finite projective limits, then small filtrant inductive limits are exact in $\Ind(\C)$.
\end{prop}
\begin{proof}
By Proposition~\ref{333} p.~\pr{333} it suffices to check that small filtrant inductive limits commute with finite projective limits in $\Ind(\C)$. Since the embedding $\Ind(\C)\to\C^\wg$ commutes with small filtrant inductive limits by Theorem 6.1.8 p.~132, and with finite projective limits by Corollary 6.1.17 (i) p.~136, this follows from the fact that small filtrant inductive limits are exact in $\C^\wg$ (see Exercise 3.2 p.~90).
\end{proof}
\end{s}

\begin{s} 
P.~137, table. In view of Corollary 6.1.17 p.~136, one can add two lines to the table:\bigskip 

\begin{center}
\begin{tabular}{|c|c|c|c|}\hline
&&$\C\to\Ind(\C)$&$\Ind(\C)\to\C^\wg$\\ \hline
1&finite inductive limits&$\ci$&$\tm$\\ \hline
2&finite coproducts&$\ci$&$\tm$\\ \hline
3&small filtrant inductive limits&$\tm$&$\ci$\\ \hline
4&small coproducts&$\tm$&$\tm$\\ \hline
5&small inductive limits&$\tm$&$\tm$\\ \hline
6&finite projective limits&$\ci$&$\ci$\\ \hline
7&small projective limits&$\ci$&$\ci$\\ \hline
\end{tabular}
\end{center}%\bigskip 
\nn(In Line 6 we assume that $\C$ admits finite projective limits, whereas in Line 7 we assume that $\C$ admits small projective limits.)%\bigskip 
\end{s}

%

\begin{s} 
P.~138, Corollary 6.1.17. If $\C$ admits finite projective limits, then $\C$ is exact in $\Ind(\C)$. This follows from Corollary 6.1.17, Corollary 6.1.6 p.~132 and Proposition~\ref{333} p.~\pr{333}.%Proposition 3.3.3 p.~82 of the book.
\end{s}

%

\begin{s} 
P.~138, proof of Proposition 6.1.21. One can also argue as follows. Assume $\C$ admits finite projective limits. By Remark 2.6.5 p.~62 and Corollary 6.1.17 p.~136, all inclusions represented in the diagram 
\[
\begin{tikzcd}
{}&\C^\wg_\V\ar[-]{ld}\ar[-]{rd}\\
\C^\wg_\U\ar[-]{rd}&&\Ind^\V(\C)\ar[-]{ld}{i}\\
&\Ind^\U(\C)\ar[-]{d}\\
&\C,
\end{tikzcd}
\]
except perhaps inclusion $i$, commute with finite projective limits. Thus inclusion $i$ commutes with finite projective limits. The argument for $\U$-small projective limits is the same. q.e.d.
\end{s}

%%

\sbs{Proposition 6.2.1 p.~138}\lb{138}

We add a few details to the proof. Recall the statement: 
\begin{prop}[Proposition 6.2.1 p.~138] 
Let $\al:I\to \C$ be a functor with $I$ filtrant and let $Z\in\C$. The conditions below are equivalent:

\nn\emph{(i)} $Z$ is a universal inductive limit of $\al$ in the sense of Definition~\ref{uil} p.~\pr{uil}. % $\ic\al$ is representable by $Z$,

\nn\emph{(ii)} there exist an object $i_0\in I$ and a morphism $\tau:Z\to\al(i_0)$ satisfying the property:

\nn for any morphism $s:i_0\to i$, there exist a morphism $p_i:\al(i)\to Z$ and a morphism $t:i\to j$ satisfying 

\nn\emph{(a)} $p_i\ci\al(s)\ci\tau=\id_Z$,

\nn\emph{(b)} $\al(t)\ci\al(s)\ci\tau\ci p_i=\al(t)$.
\end{prop}

\begin{proof} 
Choose a universe making $I$ and $\C$ small, set $A:=\ic\al$ and let $q_i:\al(i)\to A$ be the coprojections. 

\nn$\bu$ (i) implies (ii). Let $\pp:Z\to A$ be an isomorphism. By definition of $A$ the isomorphism $\pp$ factors as $Z\xr\tau\al(i_0)\xr{q_{i_0}}A$. Let $s:i_0\to i$. We define $p_i:\al(i)\to Z$ as being the composition $\al(i)\xr{q_i}A\begin{smallmatrix}\pp\\ \longleftarrow\\ \sim\end{smallmatrix}Z$. As the three small triangles in the diagram 
$$
\begin{tikzcd}
\al(i)\ar[rr,"p_i"]\ar[rd,"q_i"]&&Z\ar[ld,"\pp","\sim"']\ar[ldd,bend left,"\tau"]\\ 
&C\\ 
&D\ar[uul,bend left,"\al(s)"]\ar[u,"q_{i_0}"']
\end{tikzcd}
$$ 
commute, we get 
$$
\pp\ci p_i\ci\al(s)\ci\tau=q_i\ci\al(s)\ci\tau=\pp,
$$ 
which implies (a). %Let us prove (b). 
The coprojection 
$$
\Hom_\C(\al(i),\al(i))\to\col\Hom_\C(\al(i),\al)=\Hom_\C(\al(i),A)
$$ 
being the map $q_i\ci$, and $I$ being filtrant, the equalities 
$$
q_i\ci\al(s)\ci\tau\ci p_i=q_{i_0}\ci\tau\ci p_i=\pp\ci p_i=q_i=q_i\ci\id_{\al(i)}
$$ 
imply the existence of a morphism $t:i\to j$ satisfying (b). 

\nn$\bu$ (ii) implies (i). Let $\pp:Z\to A$ be the composition $Z\xr\tau\al(i_0)\xr{q_{i_0}}A$. It suffices to show that $\pp$ is an isomorphism. Let $X$ be an object of $\C$. It suffices to show that the map 
$$ 
\pp_X:\Hom_\C(X,Z)\to\Hom_\C(X,A)\iso\col\Hom_\C(X,\al),\quad u\mt\pp\ci u
$$ 
is bijective. 

\nn$\star\ \pp_X$ is injective. Let $u,v\in\Hom_\C(X,Z)$ satisfy $\pp\ci u=\pp\ci v$, that is 
$$
q_{i_0}\ci\tau\ci u=q_{i_0}\ci\tau\ci v.
$$ 
As $I$ is filtrant and as $\Hom_\C(X,A)\iso\col\Hom_\C(X,\al)$, there is a morphism $s:i_0\to i$ such that $\al(s)\ci\tau\ci u=\al(s)\ci\tau\ci v$. We have $q_i\ci\al(s)\ci\tau\ci u=q_{i_0}\ci\tau\ci u=\pp\ci u$, and, similarly, $q_i\ci\al(s)\ci\tau\ci v=\pp\ci v$, yielding 
$$
\pp\ci u=q_i\ci\al(s)\ci\tau\ci u=q_i\ci\al(s)\ci\tau\ci v=\pp\ci v,
$$ 
and thus $u=v$. 

\nn$\star\ \pp_X$ is surjective. Let $w:X\to\al(i)$ be a morphism. It suffices to show that there is a morphism $u:X\to Z$ such that $\pp\ci u=w$. We may assume that there is a morphism $i_0\to i$. If $p_i:\al(i)\to Z$ and $t:i\to j$ are as in (ii), we get
$$
q_i\ci w=q_j\ci\al(t)\ci w=q_j\ci\al(t)\ci\al(s)\ci\tau\ci p_i\ci w=q_{i_0}\ci\tau\ci p_i\ci w=\pp\ci p_i\ci w.
$$ 
\end{proof} 

\begin{cor}
Let $\al:I\to X$ be a functor from a filtrant category $I$ to an ordered set $X$, let $f:\Ob(I)\to X$ be the obvious map, and let $x_0$ be in $X$. Then $x_0=\col\al$ if and only if $x_0=\sup\ \Ima f$. Moreover this inductive limit is universal in the sense of Definition~\ref{uil} p.~\pr{uil} if and only the supremum $x_0$ is reached by $f$.
\end{cor} 

%% 

\sbs{Brief comments}

\begin{s} 
P.~140, proof of Corollary 6.3.2. For $X$ in $\J$ we have 
$$
F(X)\iso\sigma_\C(\iota_\C(F(X)))\iso\sigma_\C(IF(\iota_\J(X))).
$$
\end{s}

%

\begin{s} 
P.~140, Definition 6.3.3. Recall this definition:

\begin{df}[Definition 6.3.3. p.~140]\lb{633}
Assume that $\C$ admits small filtrant inductive limits. We say that an object $X$ of $\C$ is \emph{of finite presentation} if for any $\al:I\to\C$ with $I$ small and filtrant, the natural morphism 
$$
\col\Hom_\C(X,\al)\to\Hom_\C(X,\col\al)
$$ 
is an isomorphism, that is, if 
$$
\Hom_{\Ind(\C)}(X,A)\to\Hom_\C(X,\sigma_\C(A))
$$ 
is an isomorphism for any $A\in\Ind(\C)$.
\end{df}

We spell out some details. Recall that the embedding functor $\iota:\C\to\Ind(\C)$ has a left adjoint functor $\sigma$:
$$
\begin{tikzcd}
\C\ar[xshift=.7ex]{d}{\iota}\\ 
\Ind(\C).\ar[xshift=-.7ex]{u}{\sigma}
\end{tikzcd}
$$ 
In particular, for each $A$ in $\Ind(\C)$ we have a morphism $\ee_A:A\to\iota(\sigma(A))$. Recall also that $\C$ is a category admitting small filtrant limits. Consider the following conditions on an object $X$ in $\C$:

\nn(a) The natural map $\col\Hom_\C(X,\al)\to\Hom_\C(X,\col\al)$ is bijective for all functor $\al:I\to\C$ with $I$ small and filtrant.

\nn(b) The map $\ee_A\ci:\Hom_{\Ind(\C)}(\iota(X),A)\to\Hom_{\Ind(\C)}(\iota(X),\iota(\sigma(A)))$ is bijective for all $A$ in $\Ind(\C)$.
\begin{lem}
The above conditions are equivalent.
\end{lem}
\begin{proof}
If $\al:I\to\C$ is a functor with $I$ small and filtrant, then the obvious square 
$$
\begin{tikzcd}
\col\Hom_\C(X,\al)\ar[d,"\sim"']\ar[rr]&&\Hom_\C(X,\col\al)\ar[d,"\sim"]\\ 
\Hom_{\Ind(\C)}(X,\ic\al)\ar[rr,"\ee_{\ic\al}"']&&\Hom_{\Ind(\C)}(X,\col\al)
\end{tikzcd}
$$ 
commutes.
\end{proof} 
\begin{df}
We say that $X$ is of \emph{finite presentation} if the above conditions are satisfied.
\end{df}
\end{s}

%

\begin{s}\lb{634}
P. 140, proof of Proposition 6.3.4. The authors construct a bijection 

$\ds\Hom_{\Ind(\J)}(\ic_j\bt(j),\ic_i\al(i))$

\hfill$\ds\xr\sim\Hom_\C(JF(\ic_j\bt(j)),JF(\ic_i\al(i)))$.
 
We leave it to the reader to check that this bijection coincides with the natural map 

$\ds\Hom_{\Ind(\J)}(\ic_j\bt(j),\ic_i\al(i))$

\hfill$\ds\to\Hom_\C(JF(\ic_j\bt(j)),JF(\ic_i\al(i)))$. 

Here is a consequence of Proposition 6.3.4 (see Corollary 6.3.5 p.~141 in the book):

\nn\emph{Let $\C$ be a category admitting small filtrant inductive limits, and let $\C'$ be the full subcategory of $\C$ whose objects are isomorphic to small filtrant inductive limits of objects of $\C^{fp}$. Then $\C'$ is equivalent to $\Ind(\C^{fp})$. In particular $\C'$ admits small filtrant inductive limits. Moreover the inclusion $\C'\to\C$ commutes with such limits.}

\begin{proof} 
Let $\iota:\C^{fp}\to\C$ be the inclusion functor. By Corollary 6.3.2 and Proposition 6.3.4 p.~140 of the book, the functor $J\iota:\Ind(\C^{fp})\to\C$ is fully faithful and commutes with small filtrant inductive limits, and $\iota$ factors through $J\iota$. By Lemma 1.3.11 p.~21 of the book, $J\iota$ induces an equivalence $\Ind(\C^{fp})\xr\sim\C'$. The claims above follow easily from these observations. 
\end{proof}
\end{s}

%

\begin{s}
P. 143, proof of Proposition 6.4.1. The authors construct a bijection 
$$
\col_i\Hom_{\Fct(K,\Ind(\C))}(\psi,\al(i))\xr\sim\Hom_{\Fct(K,\Ind(\C))}\left(\psi,\col_i\al(i)\right).
$$
We leave it to the reader to check that this bijection coincides with the natural map 
$$
\col_i\Hom_{\Fct(K,\Ind(\C))}(\psi,\al(i))\to\Hom_{\Fct(K,\Ind(\C))}\left(\psi,\col_i\al(i)\right).
$$
\end{s}

%

\begin{s} 
P.~142, proof of Corollary 6.3.7. Let us check the isomorphism 
\begin{equation}\lb{k}
\kappa(X)\iso\ic\rho\ci\xi.
\end{equation}
Recall the setting:
$$
\begin{tikzcd}
I\ar{r}{\xi}&\C^{\text fp}\ar{d}[swap]{\iota_\C}\ar{r}{\rho}&\C\ar[yshift=-.9ex]{dl}{\kappa'}\ar{d}{\iota_\C}\\ 
{}&\Ind(\C^{\text fp})\ar{ur}{J\rho}\ar{r}[swap]{I\rho}&\Ind(\C),
\end{tikzcd}
$$ 
$\kappa'$ being quasi-inverse to $J\rho$ (for more details, see p.~141 of the book), $\kappa$ is defined by $\kappa:=I\rho\ci\kappa'$, and $X\iso\col\rho\ci\xi$. We have 
$$
\kappa(X)\iso I\rho(\kappa'(\col\rho\ci\xi))\iso I\rho(\ic\xi)
$$

$$
\iso\ic(I\rho\ci\iota_\C\ci\xi)\iso\ic(\rho\ci\xi), 
$$ 

\nn the second, third and fourth isomorphisms being respectively justified by \qr{140} p.~\pr{140}, \qr{133ii} p.~\pr{133ii} and \qr{133i} p.~\pr{133i}. This proves \qr{k}. 

Parts (ii) and (iii) of Corollary 6.3.7 are equivalent by Proposition 1.5.6 (ii) p.~29 of the book. To prove (ii) note that we have 
$$
\sigma(\kappa(\col\rho\ci\xi))\iso\sigma(\ic\rho\ci\xi)\iso\col\rho\ci\xi
$$ 
by Corollary 6.3.7 (i) p.~141 and Proposition 6.3.1 (i) p.~139 of the book. 
\end{s}

%%

\sbs{Theorem 6.4.3 p.~144}

Notational convention for this section, {\em and for this section only!} Superscripts will {\em never} be used to designate a category of the form $\C^{X'}$ attached to a functor $\C\to\C'$ and to an object $X'$ of $\C'$. Only two categories of the form $\C_{X'}$ (again attached to a functor $\C\to\C'$ and to an object $X'$ of $\C'$) will be considered in this section. As a lot of subscripts will be used, we shall denote these categories by 
\begin{equation}\lb{slice}
\C/G(a)\text{ and }L/a
\end{equation} 
instead of $\C_{G(a)}$ and $L_a$, to avoid confusion. Superscripts will {\em always} be used to designate categories of functors, like the category $\B^\A$ of functors from $\A$ to $\B$. 

Let $\C$ be a category and $K$ a small category. Recall that, by Corollary 6.3.2 p.~140 of the book (see \qr{140} p.~\pr{140} above), there is a functor $\Phi:\Ind(\C^K)\to\Ind(\C)^K$ such that, if $F:N\to\C^K$ is a functor defined on a small filtrant category and if $k$ is in $K$, then we have 
$$
\Phi(\ic F)(k)\iso\ic_{n\in N}(F(n)(k))=\ic(F(\ )(k)).
$$

\begin{thm}[Theorem 6.4.3 p.~144]\lb{643} 
If $\C$ is a category and if $K$ is a finite category such that $\Hom_K(k,k)=\{\id_k\}$ for all $k$ in $K$, then the functor 
$$
\Phi:\Ind(\C^K)\to\Ind(\C)^K,
$$ 
whose existence is recalled above, is an equivalence.
\end{thm} 

The key point is to check that 
\begin{equation}\lb{es} 
\Phi\text{ is essentially surjective.} 
\end{equation} 
(The fact that $\Phi$ is fully faithful is proved as Proposition 6.4.1 p.~142 of the book.) 

In the book \qr{es} is proved by an inductive argument. The limited purpose of this section is to attach, in an ``explicit'' way (in the spirit of the proof of Proposition 6.1.13 p.~134 of the book), to an object $G$ of $\Ind(\C)^K$ a small filtrant category $N$ and a functor $F:N\to\C^K$ such that 
$$ 
\Phi(\ic F)\iso G,
$$ 
that is, we want isomorphisms 
$$
\ic F(\ )(k)\iso G(k)
$$ 
functorial in $k\in K$.

As in the book we assume, as we may, that any two isomorphic objects of $K$ are equal. 

Let $\C,K$ and $G$ be as above. We consider $\C$ as being given once and for all, so that, in the notation below, the dependence on $\C$ will be implicit. For each $k$ in $K$, let $I_k$ be a small filtrant category and let 
$$
\al_k:I_k\to\C
$$ 
be a functor such that 
$$
G(k)=\ic\al_k.
$$ 

We define the category 
$$
N:=N\{K,G,(\al_k)\}
$$ 
as follows:

\nn[Beginning of the definition of the category $N:=N\{K,G,(\al_k)\}$.] An \emph{object} of $N$ is a pair $((i_k),P)$, where each $i_k$ is in $I_k$ and $P$ is a functor from $K$ to $\C$, subject to the conditions 

\nn$\bu\ \al_k(i_k)=P(k)$ for all $k$, 

\nn$\bu$ the coprojections $u_k(i_k):\al_k(i_k)=P(k)\to G(k)$ induce a morphism of functors 
%
\begin{equation}\lb{u':}
u':P\to G.
\end{equation}

\nn(We regard $\C$ as a subcategory of $\Ind(\C)$.) The picture is very similar to the second diagram of p.~135 of the book: For each morphism $f:k\to\ell$ in $K$ we have the commutative square 
$$ 
\begin{tikzcd} 
\al_k(i_k)\ar[equal]{r}&P(k)\ar{r}{P(f)}\ar{d}[swap]{u_k(i_k)}&P(\ell)\ar{d}{u_\ell(i_\ell)}\ar[equal]{r}&\al_\ell(i_\ell)\\ 
{}&G(k)\ar{r}[swap]{G(f)}&G(\ell) 
\end{tikzcd} 
$$ 
in $\Ind(\C)$. 

A \emph{morphism} from $((i_k),P)$ to $((j_k),Q)$ is a pair $((f_k),\theta)$, where each $f_k$ is a morphism $f_k:i_k\to j_k$ in $I_k$, and $\theta:P\to Q$ is a morphism of functors, subject to the condition $\theta_k=\al_k(f_k)$ for all $k$: 
$$ 
\begin{tikzcd} 
\al_k(i_k)\ar{r}{\al_k(f_k)}\ar[equal]{d}&\al_k(j_k)\ar[equal]{d}\\ 
P(k)\ar{r}[swap]{\theta_k}&Q(k).
\end{tikzcd} 
$$ 
[End of the definition of the category $N:=N\{K,G,(\al_k)\}$.] 

Let $p_k:N\to I_k$ be the natural projection. Then the functor %$F':K\to\C^N$ corresponding to 
$F:N\to\C^K$ is given by 
$$
F(\ )(k)=\al_k\ci p_k\quad\forall\ k\in K:
$$ 

$$
N\xr{p_k}I_k\xr{\al_k}\C.
$$ 
In other words, we set
$$
F\big((i_k),P\big)(k_0):=\al_{k_0}(i_{k_0}).
$$ 

\begin{lem}\lb{npk}
The category $N$ is small and filtrant, and the functor $p_k$ is cofinal.
\end{lem} 

Clearly, Lemma~\ref{npk} implies Theorem~\ref{643}.  

\begin{proof}[Proof of Lemma~\ref{npk}]
We start as in the proof of Theorem 6.4.3 p.~144 of the book: 

We order $\Ob(K)$ be decreeing $k\le\ell$ if and only if $\Hom_K(k,\ell)\neq\vi$, and argue by induction on the cardinal $n$ of $\Ob(K)$. 

If $n=0$ the result is clear.

Otherwise, let $a$ be a maximal object of $K$; let $L$ be the full subcategory of $K$ such that 
$$
\Ob(L)=\Ob(K)\setminus\{a\}; 
$$ 
let $G_L:L\to\Ind(\C)$ be the restriction of $G$ to $L$; let  
$$
\widetilde{\al_a}:I_a\to\C/G(a)
$$ 
(see \qr{slice} p.~\pr{slice} for the definition of $\C/G(a)$) be the functor defined by 
$$
\widetilde{\al_a}(i_a):=\Big(u'(a):\al_a(i_a)\to G(a)\Big);
%
%\widetilde{\al_a}(i_a):=\Big(u(i_a):\al_a(i_a)\to G(a)\Big);
$$
and put 
$$
N':=N\{L,G_L,(\al_\ell)\}.
$$ 

We define the functor 
$$ 
\pp:N'\to\big(\C/G(a)\big)^{L/a} 
$$ 
(see \qr{slice} p.~\pr{slice} for the definition of $L/a$) as follows. Let $((i_\ell),Q)$ be in $N'$. In particular, $Q$ is a functor from $L$ to $\C$, and we have, for each $\ell$ in $L$, a morphism 
$$
Q(\ell)=\al_\ell(i_\ell)\xr{u'(\ell)}\icolim\al_\ell=G(\ell) 
$$ 
in $\C$ (see \qr{u':} p.~\pr{u':}). Letting $\ell\xr fa$ be a morphism in $K$ viewed as an object of $L/a$, we put  
$$
\pp\big((i_\ell),Q\big)\left(\ell\xr fa\right):=\left(Q(\ell)\xr{u'(\ell)}G(\ell)\xr{G(f)}G(a)\right)\in\C/G(a).
$$

Letting 
$$
\DT:\C/G(a)\to\big(\C/G(a)\big)^{L/a}
$$ 
be the diagonal functor (see Notation~\ref{diag} p.~\pr{diag}), we can form the category 
$$
M:=M\left[N'\xr{\pp}\big(\C/G(a)\big)^{L/a}\xl{\ \DT\ci\widetilde{\al_a}}I_a\right].
$$ 
Concretely, an object of $M$ is a triple 
%
\begin{equation}\lb{il}
\Big(\big((i_\ell),Q\big),i_a,\big(\xi_f:Q(\ell)\to\al_a(i_a)\big)_{f:\ell\to a}\Big),
\end{equation}
%  
where $((i_\ell),Q)$ is an object of $N'$, where $i_a$ is an object of $I_a$, where $f$ runs over the morphisms from $\ell$ to $a$ in $K$, and where $\xi_f$ is a morphism from $Q(\ell)$ to $\al_a(i_a)$ which makes the square  
$$
\begin{tikzcd}
Q(\ell)\ar{r}{\xi_f}\ar{d}[swap]{u'(\ell)}&\al_a(i_a)\ar{d}{u'(a)}\\ %{u(i_a)}\\ 
G(\ell)\ar{r}[swap]{G(f)}&G(a) 
\end{tikzcd}
$$ 
in $\C$ commute, and a morphism from \qr{il} to 
$$
\Big(\big((i'_\ell),Q'\big),i'_a,\big(\xi'_f:Q'(\ell)\to\al_a(i'_a)\big)_{f:\ell\to a}\Big)
$$ 
is given by a family $(f_k:i_k\to i'_k)_{k\in K}$ of morphisms in $I_k$ making the squares 
$$
\begin{tikzcd}
Q(\ell)\ar{r}{\al_\ell(f_\ell)}\ar{d}[swap]{\xi_f}&Q'(\ell)\ar{d}{\xi'_f}\\ 
\al_a(i_a)\ar{r}[swap]{\al_a(f_a)}&\al_a(i'_a) 
\end{tikzcd}
$$ 
in $\C$ commute. (Recall $Q(\ell)=\al_\ell(i_\ell)$, $Q'(\ell)=\al_\ell(i'_\ell)$.)

We shall define functors 
$$
\begin{tikzcd}
N\ar[yshift=0.7ex]{r}{\ld}&M\ar[yshift=-0.7ex]{l}{\mu}
\end{tikzcd}
$$ 
and leave it to the reader to check that they are mutually inverse isomorphisms. (In fact, we shall only define the effect of $\ld$ and $\mu$ on objects, leaving also to the reader the definition of the effect of these functors on morphisms.)

We shall define maps 
$$
\begin{tikzcd}
\Ob(N)\ar[yshift=0.7ex]{r}{\ld}&\Ob(M).\ar[yshift=-0.7ex]{l}{\mu}
\end{tikzcd}
$$ 

To define $\ld$ let $((i_k),P)$ be in $N$, and let $Q$ be the restriction of $P$ to $L$. Then $\ld((i_k),P)$ will be of the form 
$$
\Big(\big((i_\ell),Q\big),i_a,\big(\xi_f:Q(\ell)\to\al_a(i_a)\big)_{f:\ell\to a}\Big).
$$ 
As $Q(\ell)=P(\ell)$ and $\al_a(i_a)=P(a)$, we can (and do) put $\xi_f:=P(f)$. 

To define $\mu$ let 
$$
\Xi:=\Big(\big((i_\ell),Q\big)\ ,\ i_a\ ,\ \pp((i_\ell),Q)\to\DT\widetilde{\al_a}(i_a)\Big)
$$ 
be in $M$. The object $\mu(\Xi)$ of $N$ will be of the form $((i_k),P)$, so that we must define a functor $P:K\to\C$. 

We define $P(k)$ by putting $P(\ell):=Q(\ell)$ for $\ell$ in $L$ and $P(a):=\al_a(i_a)$. 

If $f:\ell\to m$ is a morphism in $L$, then we set $P(f):=Q(f):P(\ell)\to P(m)$. Let $\ell$ be in $L$. There is at most one morphism $f:\ell\to a$. If this morphism does exist, then we put $P(f):=\xi_f$. 

We leave it to the reader to check that $\ld$ and $\mu$ are mutually inverse bijections. 

We also leave it to the reader to check that the set of morphisms in $M$ from $\ld((i_k),P)$ to $\ld((i'_k),P')$ is \emph{equal} (in the strictest sense of the word) to the set of morphisms in $N$ from $((i_k),P)$ to $((i'_k),P')$, so that we get an isomorphism 
$$ 
N\iso M\left[N'\xr{\pp}(\C/G(a))^{L/a}\xl{\ \DT\ci\widetilde{\al_a}}I_a\right]. 
$$

By induction hypothesis, 
%
\begin{equation}\lb{n'saf}
N'\text{ is small and filtrant}
\end{equation} 
%
and the projection $N'\to I_\ell$ is cofinal for all $\ell$ in $L$. It follows from Proposition 2.6.3 (ii) p.~61 of the book that $\widetilde{\al_a}$ is cofinal. By assumption $\C/G(a)$ is filtrant, and Lemma~\ref{delta} below will imply that $\DT$ is cofinal. Thus, 
\begin{equation}\lb{daa}
\DT\ci\widetilde{\al_a}\text{ is cofinal.}
\end{equation} 
Taking Lemma~\ref{delta} below for granted, Lemma~\ref{npk} p.~\pr{npk} now follows from \qr{n'saf}, \qr{daa} and Proposition 3.4.5 p.~89 of the book. 
\end{proof} 

As already observed, Lemma~\ref{npk} implies Theorem~\ref{643} p.~\pr{643}. The only remaining task is to prove 

\begin{lem}\lb{delta}
If $I$ is a finite category and $\C$ a filtrant category, then the diagonal functor $\DT:\C\to\C^I$ is cofinal.
\end{lem}

\begin{proof}
It suffices to verify Conditions (a) and (b) of Proposition 3.2.2 (iii) p.~78 of the book. Condition (b) is clear. To check Condition (a), let $\al$ be in $\C^I$. We must show that there is pair $(X,\ld)$, where $X$ is in $\C$ and $\ld$ is a morphism of functors from $\al$ to $\DT X$. Let $S$ be a set of morphisms in $I$. It is easy to prove 
$$
(\exists\ Y\in\C)\left(\exists\ \mu\in\prod_{i\in I}\Hom_\C(\al(i),Y)\right)\Big(\forall\ (s:i\to j)\in S\Big)\Big(\mu_{j}\ci\al(s)=\mu_i\Big) 
$$ 
by induction on the cardinal of $S$, and to see that this implies the existence of $(X,\ld)$.
\end{proof} 

%%

\sbs{Exercise 6.8 p.~146} 

Recall the statement: 

Let $R$ be a ring.

\nn(i) Prove that $M\in\Mod(R)$ is of finite presentation in the sense of Definition~\ref{633} p.~\pr{633} if and only if it is of finite presentation in the classical sense (see Examples 1.2.4 (iv)), that is, if there exists an exact sequence $R^m\to R^n\to M\to0$.

\nn(ii) Prove that any $R$-module $M$ is a small filtrant inductive limit of modules of finite presentation. (Hint: consider the full subcategory of $(\Mod(R))_M$ consisting of modules of finite presentation and prove it is essentially small and filtrant.)

\nn(iii) Deduce that the functor $J\rho$ defined in Diagram (6.3.1) induces an equivalence $J\rho:\Ind(\Mod^{fp}(R))\to\Mod(R)$.

\nn\textbf{Solution.} We shall freely use Proposition 3.1.3 p.~73 of the book, which describes the inductive limit of a set-valued functor defined on a small filtrant category, as well as Corollary 3.1.5 (same page), which says that the forgetful functor $\Mod(R)\to\Set$ commutes with small filtrant inductive limits.

\nn(i) (a) Let $R^m\to R^n\to M\to0$ be exact, and let us show that $M\in\Mod(R)$ is of finite presentation in the sense of Definition~\ref{633} p.~\pr{633}. 

Let $(N_i)_{i\in I}$ be an inductive system in $\Mod(R)$ indexed by a small filtrant category $I$, let $N$ be its inductive limit, and, for each $i$, let %$p_i:N_i\to N$ 
$$
p_i:N_i\to N\quad\text{and}\quad q_i:\Hom_R(M,N_i)\to\col_i\Hom_R(M,N_i)
$$ 
be the coprojections, and consider the map
\begin{equation}\lb{pc}
\col\Hom_R(M,N_i)\to\Hom_R(M,N)
\end{equation}  
induced by the 
$$
p_i\ci:\Hom_R(M,N_i)\to\Hom_R(M,N).
$$ 
(i) (a1) The map \qr{pc} is injective. (This part of the proof also works if $M$ is just finitely \emph{generated}, without being finitely presented.) Let $i$ be an object of $I$ and $f:M\to N_i$ an $R$-linear map such that $q_i(f)$ is in the kernel of \qr{pc}. It suffices to show $q_i(f)=0$. Let $F$ be the subset of $M$ formed by the images of the elements of the canonical basis of $R^n$. For each $x$ in $F$, the element $f(x)$ is annihilated by $p_i$. As $F$ is finite, there is a $j$ in $I$ and a morphism $s:i\to j$ such that $N_s(f(x))=0$ for all $x$ in $F$, and thus for all $x$ in $M$. This implies $q_i(f)=0$, as required. This ends the proof of the injectivity of \qr{pc}.

\nn(i) (a2) The map \qr{pc} is surjective. Let $f:M\to N$ be $R$-linear. It suffices to show that $f$ factors through $p_i:N_i\to N$ for some $i$ in $I$. Let $a_j\in M$ be the image of the $j$-th element of the canonical basis of $R^n$, and let $(\ld_{jk})$ be the matrix of our map $R^m\to R^n$, so that we have 
$$
\sum_{j=1}^n\ \ld_{jk}\,a_j=0\quad\text{for}\quad k=1,\dots,m.
$$ 
There is an $i'$ in $I$ and there are $b_1,\dots,b_n$ in $N_{i'}$ such that $p_{i'}(b_j)=f(a_j)$ for all $j$. This yields 
$$
p_{i'}\left(\sum_{j=1}^n\ \ld_{jk}\,b_j\right)=0\quad\text{for}\quad k=1,\dots,m.
$$ 
As a result, there is a $i$ in $I$ and there are $c_1,\dots,c_n$ in $N_i$ such that $p_i(c_j)=f(a_j)$ for all $j$ and 
$$
\sum_{j=1}^n\ \ld_{jk}\,c_j=0\quad\text{for}\quad k=1,\dots,m.
$$ 
Hence there is an $R$-linear map $g:M\to N_i$ such that $g(a_j)=c_j$ for all $j$, and thus $p_i\ci g=f$. This ends the proof of the surjectivity of \qr{pc}, and also the proof of the fact that any $R$-module which is of finite presentation in the classical sense is of finite presentation in the sense of Definition~\ref{633} p.~\pr{633}. 

\nn(i) (b) We assume now that $M\in\Mod(R)$ is of finite presentation in the sense of Definition~\ref{633} p.~\pr{633}, and we prove that $M$ is of finite presentation in the classical sense.

\nn(i) (b1) The $R$-module $M$ is finitely generated. Let $I$ be the set of all finitely generated submodules of $M$. Then $I$ is a small filtrant ordered set. For each $N$ in $I$ let $q_N:\Hom_R(M,N)\to\col_N\Hom_R(M,N)$ be the coprojection. Then the identity of $M$ is the image of $q_N(f)$ for some $N$ in $I$ and some $R$-linear $f:M\to N$. This implies $N=M$. 

\nn(i) (b2) The $R$-module $M$ is finitely presented in the classical sense. The argument is similar to the one in (i) (b1) above. There is a small set $K$, a positive integer $n$ and an exact sequence $R^{\oplus K}\xr f R^n\to M\to0$. Let $I$ be the set of the finite subsets of $K$. Then $I$ is a small filtrant ordered set. For each $F$ in $I$ set $M_F:=R^n/f(R^{\oplus F})$. Then $(M_F)_{F\in I}$ is, in a natural way, an inductive system whose colimit is $M$, the coprojections being the obvious maps $p_F:M_F\to M$. Let $q_F:\Hom_R(M,M_F)\to\Hom_R(M,M)$ be the coprojections. The identity of $M$ factors through $p_F:M_F\to M$ for some $F$ in $I$. This implies $M\iso M_F$, ending the proof that $M$ is finitely presented in the classical sense, and thus the proof of (i).

\nn(ii) (I don't understand the hint.) Clearly any $R$-module is a small filtrant inductive limit of finitely generated $R$-modules. Hence, in view of \S\ref{634} p.~\pr{634}, it suffices to show that any finitely generated $R$-module is a small filtrant inductive limit of finitely presented $R$-modules. But this follows from (i) (b2) above.

There is also a more direct way to prove that any $R$-module $M$ is a small filtrant inductive limit of finitely presented $R$-modules: Let $F$ be a finite subset of $M$, let $K_F$ be the kernel the natural map $R^{\oplus F}\to M$, let $G$ be a finite subset of $K_F$, and let $M_{F,G}$ be the cokernel of the obvious map $R^{\oplus G}\to R^{\oplus F}$. Then the $M_{F,G}$ form a small filtrant inductive system of $R$-modules whose colimit is $M$.

\nn(iii) Claim (iii) follows from \S\ref{634} p.~\pr{634}. %Corollary 6.3.5 p.~141 of the book.

%%

\sbs{Exercise 6.11 p. 147} 

We prove the following slightly more precise statement:

\begin{prop}\lb{myprop1}
Let $F:\C\to\C'$ be a fully faithful functor, let $A'$ be in $\Ind(\C')$, and let $S$ be the set of objects $A$ of $\Ind(\C)$ such that $IF(A)\iso A'$. Then the following conditions are equivalent: 

\nn{\em(a)} $S\neq\vi$, 

\nn{\em(b)} all morphism $X'\to A'$ in $\Ind(\C')$ with $X'$ in $\C'$ factorizes through $F(X)$ for some $X$ in $\C$, 

\nn{\em(c)} the natural functor $\C_{A'\ci F}\to\C'_{A'}$ is cofinal, 

\nn{\em(d)} $A'\ci F$ is in $S$.
\end{prop}

\begin{proof}\ 

\nn(a)$\then$(b). Let $f:X'\to IF(A)$ be a morphism in $\Ind(\C')$ with $X'$ in $\C'$ and $A$ in $\Ind(\C)$, let $\bt_0:I\to\C$ be a functor with $I$ small and filtrant and $\ic\bt_0\iso A$; in particular $\ic(F\ci\bt_0)\iso IF(A)$. By Proposition 6.1.13 p.~134 of the book there is a functor $\bt:J\to\C$ and a morphism of functors $\pp:\DT X'\to F\ci\bt$, where $\DT X':J\to\C'$ is the constant functor equal to $X'$, such that 

$J$ is small and filtrant, 

$\ic(F\ci\bt)\iso IF(A)$, 

$\ic\pp\iso f$. 

\nn Then $f$ factorizes as $X'\xr{\pp_j}F(\bt(j))\xr{p_j}IF(A)$, where $p_j$ is the coprojection.

\nn(b)$\then$(c). This follows from Proposition~\ref{comb} p.~\pr{comb}. 

\nn(c)$\then$(d). This follows from Remark~\ref{cof} p.~\pr{cof} and Proposition~\ref{f} p.~\pr{f}. 

\nn(d)$\then$(a). This is obvious.
\end{proof}

%%%

\section{About Chapter 7}

\sbs{Brief comments}

\begin{s}
P. 149, Definition 7.1.1. 
\begin{prop}\lb{p711}
If $\C$ is a category with disjoint hom-sets and $\SSS$ is a set of morphisms in $\C$, then the localization $\C_\SSS$ of $\C$ at $\SSS$ exists.
\end{prop}
\begin{lem}\lb{l711}
If $\C$ is a category with disjoint hom-sets and $\SSS$ is a set of morphisms in $\C$, then there is a category $\C'$ and a functor $Q:\C\to\C'$ such that any functor $\C\to\A$ turning the morphisms of $\SSS$ into isomorphisms factors uniquely through $Q$. %functor $G:\C'\to\A$ such that $G\ci Q=F$.
\end{lem} 
\begin{proof}[Proof of Lemma \ref{l711}] 
We define the objects of $\C'$ as being the objects of $\C$, and we construct the morphisms of $\C'$ by inverting formally the elements of $\SSS$. The details are left to the reader.
\end{proof}
\begin{proof}[Proof of Proposition \ref{p711}] 
It is straightforward to check that the functor $Q:\C\to\C'$ in Lemma~\ref{l711} defines a localization $\C_\SSS$ of $\C$ at $\SSS$ in the sense of Definition 7.1.1. (The key point is to verify Condition~(c) of Definition 7.1.1 p.~149; see \S\ref{7116} p.~\pr{7116} below.)
\end{proof}
%\begin{uspb}P. 149, Definition 7.1.1, condition (b). Is it important that one requires an isomorphism, as opposed to an equality, between $F$ and $F_\SSS\ci Q$? \end{uspb}
\end{s}

% https://docs.google.com/document/d/1XSIMsAb8ChL6VxN-VQ3HoC5sg1C4Yzu0Gmq87ebsLdY/edit

%\begin{s}P.~150, Lemma 7.1.3. Proposition 1.5.6 (ii) is implicitly used the statement of the lemma.\end{s}

\begin{s} 
P. 150, Definition 7.1.1. Condition (c) says that, in terms of \S\ref{s232} (c) p.~\pr{s232}, for all functor $G:\C_\SSS\to\A$ the functor $Q^\ddg Q_*G$ exists and is isomorphic to $G$ via the identity of $G\ci Q$. (Recall that $v\star Q$ denotes the horizontal composition of $v$ and $Q$; see Definition~\ref{dil1} p.~\pr{dil1}.)
\end{s}

%

\begin{s}\lb{713} 
P. 150, Lemma 7.1.3. The last sentence says that, in terms of \S\ref{s232} (c) p.~\pr{s232}, the functor $Q^\ddg Q_*G$ exists and is isomorphic to $G$ via the identity of $G\ci Q$, or, more explicitly, that for all $F:\C'\to\A$, the map 
$$
\Hom_{\A^{\C'}}(F,G)\to\Hom_{\A^\C}(F\ci Q,G\ci Q),\quad v\mt v\star Q
$$ 
is bijective. (Recall that $v\star Q$ denotes the horizontal composition of $v$ and $Q$; see Definition~\ref{dil1} p.~\pr{dil1}.)
\end{s}

%

\begin{s}
P. 151, last sentence of the proof of Lemma 7.1.3. Omitting most of the parenthesis, we have 
\begin{align*}
Gf\ci\widetilde\theta X_1&=(Gs_2)^{-1}\ci(GQt_2)^{-1}\ci GQt_1\ci Gs_1\ci\widetilde\theta X_1\\ \\ 
&=(Gs_2)^{-1}\ci(GQt_2)^{-1}\ci \theta Y_3\ci FQt_1\ci Fs_1\\ \\ 
&=(Gs_2)^{-1}\ci(GQt_2)^{-1}\ci \theta Y_3\ci FQt_2\ci Fs_2\ci Ff\\ \\ 
&=\widetilde\theta X_2\ci Ff.
\end{align*}
\end{s}

%

\begin{s} 
P.~154. Below the statement of Lemma 7.1.13 it is written: ``The verification is left to the reader. // Hence, we get a big category ...''. One might add between the two sentences something like: We also leave it to the reader to define the identity $\id_X$ of $X$ viewed as an object of $\C^r_\SSS$, and to check the equalities $f\ci\id_X=f$, $\id_X\ci g=g$ for $f$ in $\Hom_{\C^r_\SSS}(X,Y)$ and $g$ in $\Hom_{\C^r_\SSS}(Y,X)$.
\end{s}

%

\begin{s} 
P.~155. In the text between Lemma 7.1.15 and Theorem 7.1.16, one might add the following observation. The inverse of $(s:X\to X')\in\SSS$ is given by 
$$
X'\xr gY'\os{t}{\leftarrow}X,
$$
where $g$ and $t$ are obtained by applying S3 with $f=\id_X$:
$$
\begin{tikzcd}
X\ar{r}{\id_X}\ar{d}[swap]{s}&X\ar[dashed]{d}{t}\\ X'\ar[dashed]{r}[swap]{g}&Y'.
\end{tikzcd}
$$
\end{s}

%

%\begin{s}\lb{155} 
%P.~155. Here is a corollary to Theorem 7.1.16. (The proof is obvious.)

%If $\SSS$ is a right multiplicative system in a category $\C$, if $\C\xr F\A\xr G\B$ are functors, and if $F(s)$ is an isomorphism for all $s$ in $\SSS$, then $F_\SSS$ and $(G\ci F)_\SSS$ exist. Moreover $G\ci F_\SSS$ and $(G\ci F)_\SSS$ are isomorphic. 
%\end{s}

%

\begin{s}\lb{7116}
P. 156, proof of Theorem 7.1.16, part (c). I think one can avoid Lemma 7.1.3 by arguing as follows:

Recall that $\C_\SSS$ has the same object as $\C$ and that $Q(X)=X$ for all $X$ in $\C$. Let $F,G:\C_\SSS\parar\A$. We have a natural map 
\begin{equation}\lb{7116c1}
\Hom_{\A^{\C_\SSS}}(F,G)\to\Hom_{\A^\C}(F\ci Q,G\ci Q).
\end{equation} 
We wish to define 
\begin{equation}\lb{7116c2}
\Hom_{\A^\C}(F\ci Q,G\ci Q)\to\Hom_{\A^{\C_\SSS}}(F,G),\quad\ld\mt\mu
\end{equation} 
by $\mu_X:=\ld_X$ for all $X$ in $\C$. To do this, we must check that the $\ld_X$ commute (in the obvious sense of the term) with all morphisms of $\C_\SSS$, even if \emph{a priori} they only commute with the morphism of $\C$ (or, more precisely, with the images in $\C_\SSS$ of the morphism of $\C$). Any morphism of $\C_\SSS$ being of the form $Q(s)^{-1}\ci Q(f)$ with $s$ in $\SSS$ and $f$ a morphism in $\C$, it suffices to check that the $\ld_X$ commute with the $Q(s)^{-1}$, which is easy. This shows that the map \eqref{7116c2} is well-defined, and it is straightforward to check that it is inverse to \eqref{7116c1}. (This argument was implicit in the proof of Proposition~\ref{p711} p.~\pr{p711}.)
\end{s}

%

\begin{s}\lb{778}% old version
% https://docs.google.com/document/d/1v9ahntIb_GRudtAonRtBkkQeCONr4j5W4CrT5YxZJA8/edit
%
About the proof of Remark 7.1.18 (ii) p.~156. The following is almost a copy and paste of the display in the proof of Remark 7.1.18 (ii): 
$$
\Hom_{\C_\SSS^\ell}(X,Y)\iso\col_{(X'\to X)\in\SSS_X}\Hom_\C(X',Y)
$$ 
$$
\xr\sim\col_{(X'\to X)\in\SSS_X,(Y\to Y')\in\SSS^Y}\Hom_\C(X',Y')
$$ 

$$\xl\sim\col_{(Y\to Y')\in\SSS^Y}\Hom_\C(X,Y')\iso\Hom_{\C_\SSS^r}(X,Y).$$ 

Let us describe the implicit map 
\begin{equation}\lb{x}
\col_{(X'\to X)\in\SSS_X,(Y\to Y')\in\SSS^Y}\Hom_\C(X',Y')\to\col_{(X'\to X)\in\SSS_X}\Hom_\C(X',Y).
\end{equation} 
Given a diagram $X\xl sX'\xr fY'\xl tY$ with $s$ and $t$ in $\SSS$, we must first concoct a diagram $X\xl uX''\xr gY$ with $u$ in $\SSS$. As $\SSS$ is a left multiplicative system, the solid diagram 
$$
\begin{tikzcd}
X''\ar[dashed]{d}[swap]{v}\ar[dashed]{r}{g}&Y\ar{d}{t}\\ 
X'\ar{r}[swap]{f}&Y'
\end{tikzcd}
$$ 
can be completed to a commutative square as indicated, with $v$ in $\SSS$, and it suffices to set $u:=s\ci v$. We leave the proof of the fact that the element in the right hand side of \qr{x} so obtained does not depend on the choice of the object $X''$ and the morphisms $g$ and $v$. From this point the proof of Remark 7.1.18 (ii) is straightforward.
\end{s}

%

\begin{s}  
\begin{uspb} 
P. 157, proof of Proposition 7.1.20. In the sentence\index{unsolved problem} 

\nn``Since $t\ci s\in\SSS$, we have thus proved that, for $f:X\to Y$ in $\C$, if $Q(f)$
is an isomorphism, then there exists $g:Y\to Z$ such that $g\ci f\in\SSS$'',

\nn I don't understand why $g\ci f\in\SSS$. (Proposition 7.1.20 doesn't seem to be used elsewhere in the book.)
\end{uspb}
\end{s}

%

\begin{s}\lb{durl} 
P.~159, Definition 7.3.1 (i). Recall the definition: 

Let $\C$ be a $\U$-small category, let $\SSS$ be a right multiplicative system, and let $Q:\C\to\C_\SSS$ be the right the localization of $\C$ by $\SSS$. A functor $F:\C\to\A$ is said to be {\em right localizable} if $Q^\dg F$ exists, in which case we say that $Q^\dg F$ a {\em right localization} of $F$, and denote this functor by $R_\SSS F$. 

% The comment below was here 
% https://docs.google.com/document/d/1WwEuQko13SpCaJ53JDskmaGyiC3F0HQuxLvMMLLubJI/edit

In terms of \S\ref{s232} (a) p.~\pr{s232}, the condition is that there is some morphism of functors $\tau:F\to R_\SSS F\ci Q$ such that for all $G:\C_\SSS\to\A$ and all $w:F\to G\ci Q$ there is a unique $v:R_\SSS F\to G$ such that $(v\star Q)\ci\tau=w$: 
$$
\begin{tikzcd}
F\ar[r,"\tau"]\ar[dr,"w"']&R_\SSS F\ci Q\ar[d,"v\star Q"]&R_\SSS F\ar[d,dashed,"v"]\\ 
&G\ci Q&G.
\end{tikzcd}
$$ 
(Recall that $\star$ denotes the horizontal composition of morphisms of functors; see Definition~\ref{dil1} p.~\pr{dil1}.)
\end{s}

%

\begin{s}%\lb{731b} 
%P. 159, Definition 7.3.1 (ii). It is written: 

\begin{comment}

``We say that $F$ is universally right localizable if for any functor $K:\A\to\A'$, the functor $K\ci F$ is localizable and $R_\SSS(K\ci F)\xr\sim K\ci R_\SSS F$.'' 

This suggests that there is a natural morphism of functors from $R_\SSS(K\ci F)$ to $K\ci R_\SSS F$, and that the condition is that this morphism is an isomorphism. Unfortunately I don't see this morphism. As a result I prefer to be more prudent and set 

\begin{df}%\lb{o1}
In the above situation, $F$ is universally right localizable if for any functor $K:\A\to\A'$, the functor $K\ci F$ is right localizable and there is an isomorphism $\rho:R_\SSS(K\ci F)\xr\sim K\ci R_\SSS F$ such that the diagram 
$$
\begin{tikzcd}
K\ci F\ar[rd,"K\star\tau"']\ar[r,"\sigma"]&R_\SSS(K\ci F)\ci Q\ar[d,"\rho\star Q"]\\ 
&K\ci R_\SSS F\ci Q
\end{tikzcd}
$$
commutes, where $\tau:F\to R_\SSS F\ci Q$ and $\sigma:K\ci F\to R_\SSS(K\ci F)\ci Q$ are the structural morphisms. (Recall that $\star$ denotes horizontal composition; see Definition~\ref{dil1} p.~\pr{dil1}.)
%In the above situation, $F$ is universally right localizable if for any functor $K:\A\to\A'$, the functor $K\ci F$ is right localizable and there is an isomorphism $R_\SSS(K\ci F)\iso K\ci R_\SSS F$.
\end{df}

%``We say that $F$ is universally right localizable if for any functor $K:\A\to\A'$, the functor $K\ci F$ is right localizable and there is an isomorphism $R_\SSS(K\ci F)\iso K\ci R_\SSS F$.'' 

Another option would be to set 

\begin{df}%\lb{o2}
In the above situation, $F$ is universally right localizable if 
$$
\col_{(X\to X')\in\SSS^X}F(X')
$$ 
exists universally in $\A$ in the sense of Definition~\ref{uil} p.~\pr{uil}.
\end{df}

\end{comment}

P. 159, Definition 7.3.1 (ii). The following proposition is obvious:

\begin{prop}\lb{rsf}
In the setting of \S\ref{durl}, assume that $F(s)$ is an isomorphism for all $s$ in $\SSS$. Recall that $\al_X:\SSS^X\to\C$ is the forgetful functor $(X',s)\mt X'$ (see Definition 7.1.9 p.~153 in the book). Define $p:F\ci\al_X\to\DT F(X)$ by $p_{X',s}:=F(s)^{-1}$. Then $(F(X),p)$ is a universal inductive limit of $F\ci\al_X$ in $\A$ in the sense of Definition~\ref{uil} p.~\pr{uil}. In particular $F$ is universally right localizable in the sense of the two above definitions, $R_\SSS F\iso F_\SSS$ (for the definition of $F_\SSS$, see condition (b) in Definition 7.1.1 p.~149 in the book), and for any functor $K:\A\to\A'$ the diagram below commutes up to isomorphism
$$
\begin{tikzcd}
\C\ar{rr}{F}\ar{d}[swap]{Q}&&\A\ar{d}{K}\\
\C_{\SSS}\ar{rr}[swap]{(K\ci F)_{\SSS}}\ar{rru}[swap]{F_{\SSS}}&&\A'.
\end{tikzcd}
$$ 
\end{prop}
\end{s}

%\begin{s} P.~159, Definition of universally right localizable functor (Definition 7.3.1 (ii)): see \S\ref{durl} p.~\pr{durl}.\end{s}

\begin{s}\lb{732} 
P.~160, Proposition 7.3.2. 

\nn(a) The second sentence of the proof reads: ``By hypothesis (i) and Corollary 7.2.2, $\iota_Q:\I_\T\to\C_\SSS$ is an equivalence''. It is also worth noting that $\I$ is a right multiplicative system in $\I$.

\nn(b) The third sentence of the proof reads: ``By hypothesis (ii) the localization $F_\T$ of $F\ci\iota$ exists''. %The functor of $F_\SSS$ is defined in Definition 7.1.1 p.~149, but it is not called a localization. 
See Proposition~\ref{rsf}. By condition (b) in Definition 7.1.1 we have
\begin{equation}\lb{ft}
F_\T\ci Q_\T\iso F\ci\iota.
\end{equation}

\nn(c) The book attaches, to each functor $G:\C_\SSS\to\A$, a bijection 
$$
f_G:\Hom_{\A^{\C_\SSS}}(RF,G)\to\Hom_{\A^\C}(F,G\ci Q_\SSS).
$$ 
We must verify that we have 
\begin{equation}\lb{fgm}
f_G(\mu)=(\mu\star Q_\SSS)\ci f_{RF}(\id_{RF})
\end{equation} 
for all $G:\C_\SSS\to\A$ and all $\mu:RF\to G$ (see Definition~\ref{dil1} p.~\pr{dil1} and \S~\ref{durl} p.~\pr{durl}). We can assume that we have the following equalities (not only isomorphisms, see \S\ref{l711} p.~\pr{l711}) between functors: 
$$
RF=F_\T\ci\iota_Q^{-1},\quad\iota_Q^{-1}\ci\iota_Q=\id_{\I_\T},\quad F\ci\iota=F_\T\ci Q_\T,\quad Q_\SSS\ci\iota=\iota_Q\ci Q_\T. 
$$ 
Recall that, by assumption, all $X$ in $\C$ comes with a morphism $s:X\to\iota(Y)$ in $\SSS$ with $Y$ in $\I$. It is easy to see that \qr{fgm} follows then from the commutativity of 
$$
\begin{tikzcd}
F_\T\iota_Q^{-1}Q_\SSS X\ar[d,"\mu(Q_\SSS X)"']\ar[rr,"F_\T\iota_Q^{-1}(s)"]&&F_\T\iota_Q^{-1}Q_\SSS\iota Y\ar[r,equal]&F\iota Y\ar[d,"\mu(\iota_Q Q_\T Y)"]\\ 
GQ_\SSS X\ar[rrr,"FQ_\SSS (s)"']&&&GQ_\SSS\iota Y.
\end{tikzcd} 
$$ 
(We have omitted many parenthesis.) 

\nn(d) The last claim in Proposition 7.3.2 is the existence of an isomorphism 
$$
RF\ci Q_\SSS\ci\iota\iso F\ci\iota.
$$ 
This can be proved as follows: 
$$
RF\ci Q_\SSS\ci\iota\iso RF\ci\iota_Q\ci Q_\T\iso F_\T\ci\iota_Q^{-1}\ci\iota_Q\ci Q_\T\iso F_\T\ci Q_\T\iso F\ci\iota,
$$ 
the last isomorphism following from \qr{ft}.

%\nn(d) If, in the setting of Proposition 7.3.2, any $t$ in $\T$ is an isomorphism, then the right localization $(\C_\SSS,Q)$ of $\C$ is universal in the sense of Definition~\ref{url2}.

% removed:
% https://docs.google.com/document/d/18qXG0GqcLJD33QRv2VQFE2Yn41AbyXqzHMTKzuDZsR4/edit
\end{s}

%

\begin{s}\lb{737}
P.~161. We paste Display (7.3.7), which appears in Proposition 7.3.3 (iii) p.~161 of the book: 
\begin{equation}\lb{(7.3.7)}
(R_\SSS F)(Q(X))\iso\col_{(X\to Y)\in\SSS^X}F(Y).
\end{equation} 
Let $\C$ be a $\U$-category (Definition~\ref{myucat} p.~\pr{myucat}), and let $\V$ be a universe such that $\U\in\V$ and $\C$ is a $\V$-small category (Definition~\ref{myuscat} p.~\pr{myuscat}). Writing $\A$ for the category of $\V$-sets, Proposition 7.3.3 (iii) of the book implies the following:

Let $X$ and $Y$ be two objects of $\C$. 

If $\SSS$ is a right multiplicative system in $\C$, then the functor 
$$
R_\SSS\Hom_\C(X,\ \ )
$$ 
exists and is isomorphic to $\Hom_{\C_\SSS^r}(X,\ \ )$. 

Similarly, if $\SSS$ is a left multiplicative system in $\C$, then the functor 
$$
R_{\SSS^{\op}}\Hom_\C(\ \ ,Y)
$$ 
exists and is isomorphic to $\Hom_{\C_\SSS^\ell}(\ \ ,Y)$. 
\end{s}

%%

\begin{s}
P. 161. We prove the isomorphism at the bottom of p.~161.  

Recall the setting: $\SSS$ is a right multiplicative system in a category $\C$ such that $\SSS^X$ is cofinally small for all $X$ in $\C$. Let $X$ and $Y$ be in $\C$. It is claimed in the book that there is a natural isomorphism 
\begin{equation}\lb{clc}
\col_{(Y\to Y')\in\SSS^Y}\Hom_\C(X,Y')\xr\sim\lim_{(X\to X')\in\SSS^X}\col_{(Y\to Y')\in\SSS^Y}\Hom_\C(X',Y').
\end{equation} 
We can rewrite \qr{clc} as 
\begin{equation}\lb{clc2}
\Hom_{\C_\SSS}(Q(X),Q(Y))\xr\sim\lim_{(X\to X')\in\SSS^X}\Hom_{\C_\SSS}(Q(X'),Q(Y)).
\end{equation} 

For $X\to X'$ in $\SSS$ let 
$$
p[X\to X']:\lim_{(X\to X'')\in\SSS^X}\Hom_{\C_\SSS}(Q(X''),Q(Y))\to\Hom_{\C_\SSS}(Q(X'),Q(Y))
$$ 
be the projection, and define the maps 
$$
\begin{tikzcd} 
\ds\Hom_{\C_\SSS}(Q(X),Q(Y))\ar[yshift=0.7ex]{r}{f}&\ds\lim_{(X\to X')\in\SSS^X}\Hom_{\C_\SSS}(Q(X'),Q(Y))\ar[yshift=-0.7ex]{l}{g}
\end{tikzcd}
$$ 
as follows: We define $f$ by
$$
p[X\to X']\Big(f\big(Q(X)\to Q(Y)\big)\Big):=\big(Q(X')\to Q(X)\to Q(Y)\big)
$$ 
for $X\to X'$ in $\SSS$, where $Q(X')\to Q(X)$ is the inverse of $Q(X\to X')$, and we define $g$ by
$$
g:=p[X\xr{\id}X].
$$ 
To show that $f\ci g$ is the identity of the right-hand side of \qr{clc2}, note that we have in the above notation 
$$
p[X\to X']\Bigg(f\bigg(g\Big(\big(Q(X'')\to Q(Y)\big)_{X\to X''}\Big)\bigg)\Bigg)
$$ 
$$
=p[X\to X']\Big(f\big(Q(X)\to Q(Y)\big)\Big)
$$ 
$$
=\big(Q(X')\to Q(X)\to Q(Y)\big)=p[X\to X']\Big(\big(Q(X'')\to Q(Y)\big)_{X\to X''}\Big).
$$ 
The proof that $g\ci f$ is the identity of the left-hand side of \qr{clc2} is similar and easier.
\end{s} 

%%

\sbs{Proof of (7.4.3) p.~162}

Recall that $\SSS$ is a right multiplicative system in $\C$ such that $\SSS^X$ is cofinally small for all $X$ in $\C$. We have the (non-commutative) diagram
$$
\begin{tikzcd}
\C\ar{rr}{F}\ar{d}[swap]{Q}\ar{dr}{\iota_\C}&&\A\ar{d}{\iota_\A}\\ 
\C_\SSS\ar{r}[swap]{\al_\SSS}&\Ind(\C)\ar{r}[swap]{IF}&\Ind(\A).
\end{tikzcd}
$$
Let $X$ be in $\C$. We must prove that there is an isomorphism 
$$
R_\SSS(\iota_\A\ci F)(Q(X))\iso IF(\al_\SSS(Q(X)))
$$ 
in $\Ind(\C)$. Recall the following facts: 

\nn$\bu$ Proposition 7.4.1 p.~162 of the book implies
$$
A:=\al_\SSS(Q(X))=\col_{(X\to X')\in\SSS^X}\iota_\C(X')\in\Ind(\C).
$$ 
$\bu$ Display~\qr{(7.3.7)} p.~\pr{(7.3.7)} implies
$$
B:=R_\SSS(\iota_\A\ci F)(Q(X))=\col_{(X\to X')\in\SSS^X}\iota_\A(F(X'))\in\Ind(\A).
$$
$\bu$ The definition of $IF$ p.~133 of the book implies
$$
C:=IF(A)=\col_{(U\to A)\in\C_A}\iota_\A(F(U))\in\Ind(\A).
$$

We want to prove $B\iso C$.

%\nn{\em Notation.} If $\al:I\to\B$ is a functor whose inductive limit is $X\in\B$, then we write $p(X,i):\al(i)\to X$ for the coprojection. (Of course this morphism depends on $\al$.) 

In the rest of this section we use the following convention: 

\nn If $\al:I\to\B$ is a functor whose inductive limit is $Y\in\B$, then we write 
$$
p[Y,i]:\al(i)\to Y
$$ 
for the corresponding coprojection. 

As the inductive systems we shall consider have different colimits, this convention should create no confusion.

We leave the proof of the following claims to the reader:

Claim 1: There is a unique morphism $f:B\to C$ such that the diagram 
$$
\begin{tikzcd}
B\ar[r,"f"]&C\\ 
\iota_\A(F(X'))\ar{u}{p[B,X\to X']}\ar{ur}[swap]{p[C,p[A(X'),X\to X'](\id_{X'})]}
\end{tikzcd}
$$ 
commutes for all $X\to X'$ in $\SSS^X$.

Claim 2: There is a unique morphism $g:C\to B$ such that the diagram 
$$
\begin{tikzcd}
C\ar[rrr,"g"]&&&B\\ 
\iota_\A(F(U))\ar{u}{p[C,p[A(U),X\to X'](U\to X')]}\ar[rrr,"\iota_\A(F(U\to X'))"']&&&\iota_\A(F(X'))\ar{u}[swap]{p[B,X\to X']}
\end{tikzcd}
$$ 
commutes for all $X\to X'$ in $\SSS^X$ and all $U\to X'$ in $\C_{X'}$. 

Claim 3: The morphisms $f$ and $g$ are mutually inverse.

%

\begin{s}
P. 162, proof of Lemma 7.4.3. Recall the statement
\begin{lem}[Lemma 7.4.3 p. 162]
If $G:\A\to\A'$ is a functor and $F$ is right localizable at $X$, then $G\ci F$ is right localizable at $X$.
\end{lem}
\begin{proof}
By (7.4.3) p. 162 the assumption that $F$ is right localizable at $X$ is equivalent to the existence of an object $Y$ of $\A$ satisfying $$IF(\al_\SSS(X))\iso\iota_\A(Y).$$ Applying $IG$ we get $$IG(IF(\al_\SSS(X)))\iso IG(\iota_\A(Y)),$$ and thus $$I(G\ci F)(\al_\SSS(X))\iso\iota_{\A'}(G(Y)).$$ This shows that $G\ci F$ is right localizable at $X$. 
\end{proof}
\end{s}

%%

\sbs{Brief comments}

%\begin{s}P. 163, proof of implication (i) $\then$ (ii) in Proposition 7.4.4. It seems to me the claim follows from the straightforward lemma below (coupled with Theorem 2.3.3 p.~52 in the book).

%\begin{lem}If $\iota:\C\to\C'$ is fully faithful, if $\al:I\to\C$ is a functor, and if $\col\iota\ci\al\iso\iota(X)$ for some $X$ in $\C$, then we have $\col\al\iso X$. \end{lem}\end{s}

\begin{s}
P. 163, proof of Proposition 7.4.34, implication (i)$\then$(ii). Here are some of the tasks implicitly left to the reader: 

\nn$\bu$ Define a morphism of functors $\tau$ from $F$ to $H\ci Q$.

\nn$\bu$ Define the three bijections in the display.

\nn$\bu$ Check that the composition of these three bijections coincides with the map $v\mt(v\star Q)\ci\tau$. (See \S\ref{durl} p.~\pr{durl}.)
\end{s}

%

\begin{s}\lb{745}
P. 163, Remark 7.4.5. In this \S\ we adhere to Convention 11.7.1 of the book, according to which, paradoxically, in the expression $\Hom_\C(X,Y)$, the variable $Y$ is considered as the \emph{first} variable and $X$ as the \emph{second} variable.

Let $\SSS$ be a left and right multiplicative system in $\C$, and let $X$ and $Y$ be two objects of $\C$. \S\ref{737} p.~\pr{737} implies that the functors 
$$
R_\SSS\Hom_\C(X,\ \ ),\quad R_{\SSS^{\op}}\Hom_\C(\ \ ,Y),\quad R_{\SSS\tm\SSS^{\op}}\Hom_\C
$$
exist and satisfy 
$$
\Hom_{\C_\SSS}(X,Y)\iso R_\SSS\big(\Hom_\C(X,\ \ )\big)(Y)
$$

$$
\iso R_{\SSS^{\op}}\big(\Hom_\C(\ \ ,Y)\big)(X)\iso R_{\SSS\tm\SSS^{\op}}\Hom_\C(X,Y).
$$

\nn More precisely, if, in the diagram 
\begin{equation}\lb{rccs}
\begin{tikzcd}
R_\SSS\oo H_\C(X,\ \ )(Y)\ar[leftrightarrow]{dr}\ar{r}&R_{\SSS\tm\SSS^{\op}}\oo H_\C(X,Y)\ar[leftrightarrow]{d}&R_{\SSS^{\op}}\oo H_\C(\ \ ,Y)(X)\ar{l}\ar[leftrightarrow]{dl}\\ 
&\oo H_{\C_\SSS}(X,Y),
\end{tikzcd}
\end{equation} 
where we have written $\oo H$ for $\Hom$ to save space, the horizontal arrows are the natural maps, and the other arrows are the above bijections, then \qr{rccs} commutes and all its arrows are bijective.
\end{s} 

%

\begin{s} 
Exercise 7.4 p.~164. 

Statement: In a category endowed with a right multiplicative system $\SSS$, if there is a diagram 
$$
\begin{tikzcd}
Z\ar[d,"c"']\ar[r,"a"]&Y&X\ar[l,"s"']\ar[d,"d"]\\ 
W\ar[r,"b"']&V&U\ar[l,"t"]
\end{tikzcd}
$$ 
with $s,t\in\SSS$ and $Q(d)\ci Q(s)^{-1}\ci Q(a)=Q(t)^{-1}\ci Q(b\ci c)$, then there is a commutative diagram 
$$
\begin{tikzcd}
Z\ar[d,"c"']\ar[r,"a"]&Y\ar[d,"e"]&X\ar[l,"s"']\ar[d,"d"]\\ 
W\ar[r,"f"']&T&U\ar[l,"u"]
\end{tikzcd}
$$ 
with $u\in\SSS$ and $Q(u)^{-1}\ci Q(f)=Q(t)^{-1}\ci Q(b)$. [This statement solves clearly the exercise.]

Proof: We build a commutative diagram 
$$
\begin{tikzcd}
Z\ar[ddd,"c"']\ar[r,"a"]&Y\ar[d,"g"']&X\ar[l,"s"']\ar[ddd,"d"]\\ 
&S\ar[d,"w"']\\ 
&T\\ 
W\ar[ur,"f"]\ar[r,"b"']&V\ar[u,"h"]&U\ar[l,"t"]\ar[ul,"u"]\ar[uul,"v"']
\end{tikzcd}
$$ 
with $u,v,w\in\SSS$ by forming firstly $g$ and $v$, secondly $h$ and $w$, and thirdly $f$ and $u$. 
\end{s} 

%%%%

\section{About Chapter 8} 

\sbs{About Section 8.1}

The following definition of a commutative group object is much less general and much less elegant than the one in the book (p.~169), but it is slightly simpler and seems sufficient in this context. 

Let $\C$ be a category with finite products; let $0$ be the terminal object of $\C$; let $X$ be in $\C$; let $p_1,p_2:X\tm X\to X$ be the projections; and let $v:X\tm X\to X\tm X$ be defined by the equalities 
$$
p_i\ci v=p_j
$$ 
for all $i,j$ such that $\{i,j\}=\{1,2\}$. 

A structure of \emph{commutative group object} on an object $X$ of $\C$ is a triple $(\al,e,a)$ satisfying the following conditions:

We have 
$$
\al:X\tm X\to X,\quad e:0\to X,\quad a:X\to X,
$$ 
and the following diagrams commute:
$$
\begin{tikzcd}
X\tm X\tm X\ar{d}[swap]{\al\tm\id}\ar{r}{\id\tm\al}&X\tm X\ar{d}{\al}\\ 
X\tm X\ar{r}[swap]{\al}&X,
\end{tikzcd}
$$ 
$$
\begin{tikzcd}
X\ar{rd}[swap]{\id}\ar{r}{(\id,e)}&X\tm X\ar{d}{\al}\\ 
&X,
\end{tikzcd}
$$ 
$$
\begin{tikzcd}
X\ar{rd}[swap]{\id}\ar{r}{(e,\id)}&X\tm X\ar{d}{\al}\\ 
&X,
\end{tikzcd}
$$
$$
\begin{tikzcd}
X\ar{d}\ar{r}{(\id,a)}&X\tm X\ar{d}{\al}\\ 
0\ar{r}[swap]{e}&X,
\end{tikzcd}
$$
$$
\begin{tikzcd}
X\ar{d}\ar{r}{(a,\id)}&X\tm X\ar{d}{\al}\\ 
0\ar{r}[swap]{e}&X,
\end{tikzcd}
$$ 
$$
\begin{tikzcd}
X\tm X\ar{rd}[swap]{\al}\ar{r}{v}&X\tm X\ar{d}{\al}\\ 
&X.
\end{tikzcd}
$$ 

%%%

\sbs{About Section 8.2}

\subsubsection{Definition 8.2.1 p. 169}

The proposition and lemma below are obvious. 

\begin{prop}\lb{payp} 
Let $\C$ be a pre-additive category, let $\A$ be the category of additive functors from $\C^{\op}$ to $\Mod(\bb Z)$, let $h:\C\to\A$ be the obvious functor satisfying $h(X)(Y)=\Hom_\C(Y,X)$ for all $X$ and $Y$ in $\C$, let $X$ be in $\C$ and $A$ in $\A$, and let 
$$
\begin{tikzcd}
\Hom_\A(h(X),A)\ar[yshift=.7ex]{r}{\Phi}&A(X)\ar[yshift=-.7ex]{l}{\Psi}
\end{tikzcd}
$$
be defined by 
$$
\Phi(\theta)=\theta_X(\id_X),\quad\Psi(x)(f)=A(f)(x).
$$
Then $\Phi$ and $\Psi$ are mutually inverse abelian group isomorphisms.
\end{prop} 

\nn(See Theorem \ref{yol} p.~\pr{yol}.)

\begin{conv}\lb{payc}
In the above setting we denote $\A$ by $\C^\wg$ and $h$ by $\hy_\C$. (This abuse is justified by Proposition~\ref{payp}.) 
\end{conv} 

\begin{lem}\lb{payl} 
Let $\C$ and $\C'$ be pre-additive categories, let $\A$ be the category of additive functors from $\C$ to $\C'$, and let $\al:I\to\A$ be a functor such that $\col\al(X)$ exists in $\C'$ for all $X$ in $\C$. Then $\col\al$ exists in $\A$ and satisfies 
$$
(\col\al)(X)\iso\col\al(X)
$$ 
for all $X$ in $\C$. (There is a similar statement for projective limits.)
\end{lem} 

%%

\subsubsection{Lemma 8.2.3 p. 169} 

Here is a statement contained in Lemma 8.2.3: 

\begin{cor}\lb{823} 
Let $\C$ be a pre-additive category, let $X_1$ and $X_2$ be two objects of $\C$ such that the product $X=X_1\tm X_2$ exists in $\C$, let $p_a:X\to X_a$ be the projection, and define $i_a:X_a\to X$ by 
$$
p_a\ci i_b=\begin{cases}\id_{X_a}&\text{if }a=b\\0&\text{if }a\not=b.\end{cases}
$$ 
Then $X$ is a coproduct of $X_1$ and $X_2$ by $i_1$ and $i_2$. Moreover we have 
$$
i_1\ci p_1+i_2\ci p_2=\id_{X_1\tm X_2}.
$$
\end{cor}

For the reader's convenience we reproduce the statement and the proof of Lemma 8.2.3 (ii) p.~169 of the book:

\begin{lem}[Lemma 8.2.3 (ii) p. 169]\lb{823ii}
Let $\C$ be a pre-additive category; let $X,X_1,$ and $X_2$ be objects of $\C$; and, for $a=1,2$, let $X_a\xr{i_a}X\xr{p_a}X_a$ be morphisms satisfying 
$$
p_a\ci i_b=\delta_{ab}\ \id_{X_a},\quad i_1\ci p_1+i_2\ci p_2=\id_X.
$$
Then $X$ is a product of $X_1$ and $X_2$ by $p_1$ and $p_2$ and a coproduct of $X_1$ and $X_2$ by $i_1$ and $i_2$. 
\end{lem}

\begin{proof}
For any $Y$ in $\C$ we have 
$$
\Hom_\C(Y,p_a)\ci\Hom_\C(Y,i_b)=\delta_{ab}\ \id_{\Hom_\C(Y,X_a)},
$$ 
$$
\Hom_\C(Y,i_1)\ci\Hom_\C(Y,p_1)+\Hom_\C(Y,i_2)\ci\Hom_\C(Y,p_2)=\id_{\Hom_\C(Y,X)}.
$$ 

\nn This implies that $\Hom_\C(Y,X)$ is a product of $\Hom_\C(Y,X_1)$ and $\Hom_\C(Y,X_2)$ by $\Hom_\C(Y,p_1)$ and $\Hom_\C(Y,p_2)$, and thus, $Y$ being arbitrary, that $X$ is a product of $X_1$ and $X_2$ by $p_1$ and $p_2$, and we conclude by applying this observation to the opposite category.
\end{proof}

Note also the following corollary to Lemma 8.2.3 (ii) (stated above as Lemma \ref{823ii}). 

\begin{cor}\lb{823b}
Let $F:\C\to\C'$ be an additive functor of pre-additive categories; let $X,X_1$ and $X_2$ be objects of $\C$; and, for $a=1,2$, let $X_a\xr{i_a}X\xr{p_a}X_a$ be morphisms such that $X$ is a product of $X_1$ and $X_2$ by $p_1,p_2$ and a coproduct of $X_1$ and $X_2$ by $i_1,i_2$. Then $F(X)$ is a product of $F(X_1)$ and $F(X_2)$ by $F(p_1),F(p_2)$ and a coproduct of $F(X_1)$ and $F(X_2)$ by $F(i_1),F(i_2)$. 
\end{cor}

%%

\subsubsection{Brief comments}

\begin{s}
P.~170, Corollary 8.2.4. Recall the statement: 

\begin{cor}[Corollary 8.2.4 p.~170] 
Let $\C$ be a pre-additive category and let $X_1,X_2\in\C$. If $X_1\tm X_2$ exists in $\C$, then $X_1\sqcup X_2$ also exists. Moreover denoting by $i_j:X_j\to X_1\sqcup X_2$ and $p_j:X_1 \tm X_2\to X_j$ the $j$-th co-projection and projection, the morphism $r:X_1\sqcup X_2\to X_1\tm X_2$ given by 
$$
p_j\ci r\ci i_k=\begin{cases}\id_{X_k}&\text{if }j=k\\0&\text{if }j\not=k.\end{cases}
$$ 
is an isomorphism. 
\end{cor} 

\begin{conv}\lb{oplus} 
Let $X_1$ and $X_2$ be two objects of a category $\C$. Assume that the product $X_1\tm X_2$ and the coproduct $X_1\sqcup X_2$ exist in $\C$ and are isomorphic. %, and that there is an isomorphism $r$ from $X_1\sqcup X_2$ to $X_1\tm X_2$ is an isomorphism. 
In such a situation, we make a new exception to Convention~\ref{term} p.~\pr{term}: we set $X_1\oplus X_2:=X_1\tm X_2$, we transport the coprojections of $X_1\sqcup X_2$ to $X_1\oplus X_2$ and redefine $X_1\sqcup X_2$ by setting 
$$
X_1\sqcup X_2:=X_1\oplus X_2:=X_1\tm X_2,
$$ 
so that $X_1\oplus X_2$ is at the same time a product and a coproduct of $X_1$ and $X_2$.
%In the above situation, we make a new exception to Convention~\ref{term} p.~\pr{term}: we set $X_1\oplus X_2:=X_1\tm X_2$, we define $i'_j:X_j\to X_1\oplus X_2$ by $i'_j:=r\ci i_j$, and we observe that $X_1\oplus X_2$ is a coproduct of $X_1$ and $X_2$ by $i'_1$ and $i'_2$. 
\end{conv} 
\end{s} 

% 

The following lemma, whose proof is left to the reader, is implicit in the book. 

\begin{lem}
For $a=1,2$ let $f_a:X_a\to Y_a$ be a morphism in a pre-additive category $\C$. Assume that $X_1\oplus X_2$ and $Y_1\oplus Y_2$ exist in $\C$ (see Convention~\ref{oplus} above). Then we have 
$$
f_1\tm f_2=f_1\sqcup f_2
$$ 
(equality in $\Hom_\C(X_1\oplus X_2,Y_1\oplus Y_2)$). 
\end{lem} 

We denote this morphism by $f_1\oplus f_2$.%\medskip 

%\begin{proof} Put $X:=X_1\oplus X_2,\ Y:=Y_1\oplus Y_2$ and write $$X_a\xr{i_a}X\xr{p_a}X_a,\quad Y_a\xr{j_a}Y\xr{q_a}Y_a$$ for the projections and coprojections. We have $q_a\ci(f_1\tm f_2)=f_a\ci p_a$ for all $a$, and we must show $q_b\ci (f_1\tm f_2)\ci i_a=q_b\ci j_a\ci f_a$ for all $a,b$. This follows immediately from Corollary~\ref{823}.\end{proof}

%

\begin{s}
P.~171, Corollary 8.2.6. Recall the statement: 

\begin{cor}[Corollary 8.2.6 p.~171] 
Let $\C$ be a pre-additive category, $X,Y\in\C$ and $f_1,f_2\in\Hom_\C(X,Y)$. Assume that the direct sums $X\oplus X$ and $Y\oplus Y$ exist (see Convention~\ref{oplus} p.~\pr{oplus}). Then $f_1+f_2\in\Hom_\C(X,Y)$ coincides with the composition 
$$
X\xr{\delta_X}X\oplus X\xr{f_1\oplus f_2}Y\oplus Y\xr{\sigma_Y}Y.
$$ 
Here $\delta_X:X\to X\tm X=X\oplus X$ is the diagonal morphism and $\sigma_Y:Y\oplus Y=Y\sqcup Y\to Y$ is the codiagonal morphism.
\end{cor}

\begin{proof} 
For $a=1,2$ let 
$$
\begin{tikzcd}
X\oplus X\ar[r,yshift=4pt,"p_a"]&X\ar[l,yshift=-4pt,"i_a"]&Y\oplus Y\ar[r,yshift=4pt,"q_a"]&Y\ar[l,yshift=-4pt,"j_a"]
\end{tikzcd}
$$ 
be the projections and coprojections. Writing $xy$ for $x\ci y$ we have 
$$
\sigma_Y\ (f_1\oplus f_2)\ \delta_X=\sum_{a,b}\ \sigma_Y\ j_a\ q_a\ (f_1\oplus f_2)\ i_b\ p_b\ \delta_X
$$ 
$$ 
=\sum_{a,b}\ q_a\ (f_1\oplus f_2)\ i_b=\sum_a\ q_a\ (f_1\oplus f_2)\ i_a=f_1+f_2,
$$ 
the second equality following from the definitions of $\sigma_Y$ and $\delta_X$, and the third and fourth equalities following from the definitions of $f_1\oplus f_2$. (The justification of the first equality is left to the reader.)
\end{proof}
\end{s}

%

\begin{s} 
P.~172, Lemma 8.2.9. Recall the statement:

\begin{lem}[Lemma 8.2.9 p. 172] 
Let $\C$ be a pre-additive category which admits finite products. Then $\C$ is additive.
\end{lem}

Let us check that $\C$ has a zero object. (This part of the proof is left to the reader by the authors.) 

Let $X$ and $Y$ be in $\C$. By Lemma 8.2.3 p.~169 of the book, the product $X\tm Y$ is also a coproduct of $X$ and $Y$. Let us denote this object by $X\oplus Y$. Let $T$ be a terminal object of $\C$. For any $X$ in $\C$ we have a natural isomorphism $X\oplus T\iso X$. In particular $T$ can be viewed as $T\sqcup T$ via the morphisms $T\xr{\id}T\xl{\id}T$. This implies successively that, for $X$ in $\C$, the diagonal map 
$$
\Hom_\C(T,X)\to\Hom_\C(T,X)\tm\Hom_\C(T,X)
$$ 
is bijective, that the set $\Hom_\C(T,X)$ has at most one element, and that it has exactly one element. As $X$ is arbitrary, this entails that $T$ is a zero object. q.e.d.

Also note that Corollary 8.2.4 p.~170 of the book is useful to prove Lemma 8.2.9.
\end{s}

%

\begin{s} 
P.~172, Lemma 8.2.10. Let me state the result in a more explicit way: 

\begin{lem}[Lemma 8.2.10 p.~172]
If $X$ is an object of an additive category $\C$, then the morphism %composition 
$$
X\tm X=X\sqcup X\xr{\ \sigma_X\ }X
$$ 
%$$\begin{tikzcd}X\tm X&X\sqcup X\ar[l,"r","\sim"']\ar[r,"\sigma_X"]&X\end{tikzcd}$$ 
defines a structure of a commutative group object on $X$.
\end{lem} 

The associativity of the addition can also be proved as follows:

Put $X^n:=X\oplus\cdots\oplus X$ ($n$ factors), and let $X\xr{i_a}X^n\xr{\sigma_n}X$ be respectively the $a$-th coprojection and the codiagonal morphism. It clearly suffices to show that the composition 
$$
X^3\xr{\sigma_2\oplus X}X^2\xr{\sigma_2}X
$$ 
is equal to $\sigma_3$. This follows from the fact that the composition 
$$
X\xr{i_a}X^3\xr{\sigma_2\oplus X}X^2
$$ 
is equal to $i_b$ with 
$$
b=\begin{cases}1&\text{if }a=1,2\\2&\text{if }a=3.\end{cases}
$$ 
q.e.d.
\end{s}

%

\begin{s} 
P.~173, proof of Proposition 8.2.13. The fact that any $X$ in $\C$ has a structure of commutative group object follows from Lemma 8.2.10 p.~172 of the book.
\end{s} 

%

\begin{s} 
Proposition 8.2.13 p.~173. It seems to me the proposition can be stated as follows: %Here is a slightly different wording: 
\begin{prop}[Proposition 8.2.13 p.~173]
If $\C$ is an additive category, if $\C'$ is the category of finite product preserving functors from $\C$ to $\Mod(\bb Z)$, if $\C''$ is the category of finite product preserving functors from $\C$ to $\Set$, and if $U$ is the forgetful functor from $\Mod(\bb Z)$ to $\Set$, then the functor $U\ci:\C'\to\C''$ is an \emph{isomorphism} (not just an equivalence!). 
%let $F:\C\to\Set$ be a functor commuting with finite products, and for each $X$ in $\C$ denote the composition $$F(X)\tm F(X)\xleftarrow\sim F(X\oplus X)\xr{F(\sigma_X)}F(X)$$ by $+_X$. Then we have \nn\emph{(a)} The pair $(F(X),+_X)$ is an abelian group. \nn\emph{(b)} There is a unique functor $G:\C\to\Mod(\bb Z)$ such that $G(X)=(F(X),+_X)$ for all $X$ in $\C$. \nn\emph{(c)} If $H:\C\to\Mod(\bb Z)$ is a functor such that $U\ci H=F$, where $U:\Mod(\bb Z)\to\Set$ is the forgetful functor, then there is a unique morphism $\theta:G\to H$ such that $U\star\theta=F$ (recall that $\star$ denotes the horizontal composition defined in Definition~\ref{dil1} p.~\pr{dil1}). Moreover this morphism $\theta$ is an isomorphism. 
\end{prop} 
\end{s}  

% 

\begin{s}
% https://docs.google.com/document/d/19UsvOfHLs3aEap6kzitWQo-ciyyiBHJgVxqyMVZfTbI/edit
% was removed from here
P.~173, Theorem 8.2.14. Let me state the result in a more explicit way: 
\begin{thm}[Theorem 8.2.14]\lb{8214}
If $\C$ is an additive category, then $\C$ has a unique structure of a pre-additive category. More precisely, for $f$ and $g$ in $\Hom_\C(X,Y)$, the morphism $f+g\in\Hom_\C(X,Y)$ is given by the composition 
$$
\begin{tikzcd}
X\ar[r,"\delta_X"]&X\tm X\ar[d,"\sim"]\ar[r,"f\tm g"]&Y\tm Y\ar[d,"\sim"]\\ 
&X\sqcup X\ar[r,"f\sqcup g"']&Y\sqcup Y\ar[r,"\sigma_Y"]&Y.
\end{tikzcd}
$$ 
\end{thm} 
Let me also try to rewrite the beginning of the proof: 

Let $X\in\C$. By applying Proposition 8.2.13 p.~173 of the book to the functor $F:=\Hom_\C(X,\ )$, we obtain that $\Hom_\C(X,Y)$ has a structure of an additive group for all $Y$ in $\C$. Then Lemma 8.2.11 p.~172 of the book implies that the addition on $\Hom_\C(X,Y)$ is given by the above commutative diagram. 

We complete the proof by showing as in the book that this addition does define a pre-additive structure on $\C$. 
\end{s}

%

\begin{s}
P.~173, Theorem 8.2.14 (stated above as Theorem~\ref{8214}). Consider the following claims:

\nn(a) the fields $\bb Q$ and $\bb F_3(X)$ have isomorphic multiplicative groups,

\nn(b) there is a category $\C$ admitting two pre-additive structures $p$ and $q$ such that there is no additive equivalence from $(\C,p)$ to $(\C,q)$. 

We leave it to the reader to prove (a) and to show that (a) implies (b).
\end{s}

%

\begin{s} 
P.~173, Proposition 8.2.15. Recall the setting: $F:\C\to\C'$ is a functor between additive categories, and the claim is: 
$$
F\text{ is additive }\ssi\ F\text{ commutes with finite products}.
$$ 

I think the authors forgot to prove implication $\then$. Let us do it. It suffices to show that $F$ commutes with $n$-fold products for $n=0$ or $n=2$. 

Case $n=0$: Put $X:=F(0)$. We must prove $X\iso 0$. The equality $0=1$ holds in the ring $\Hom_\C(X,X)$ because it holds in the ring $\Hom_\C(0,0)$. As a result, the morphisms $0\to X$ and $X\to 0$ are mutually inverse isomorphisms. 

Case $n=2$: Let $X_1,X_2$ be in $\C$. The natural morphisms 
$$
F(X_1\oplus X_2)\rightleftarrows F(X_1)\oplus F(X_2)
$$ 
are mutually inverse isomorphisms by Corollary~\ref{823b} p.~\pr{823b} above. q.e.d.
\end{s}

%%

\sbs{About Section 8.3}

\subsubsection{Proposition 8.3.4 p. 176}

Here are a few more details about the proof of Proposition 8.3.4. Recall the setting: We have a morphism $f:X\to Y$ in an abelian category $\C$. Let $P$ be the fiber product $X\tm_YX$; let $p_1,p_2:P\parar X$ be the projections; let $p$ be the morphism $p_1-p_2$ from $P$ to $X$; and consider the diagram 
$$
\begin{tikzcd}
\Ker f\ar{r}{h} &X\ar[equal]{d}\ar{r}{a}&\Coker h\\ 
P\ar{r}[swap]{p}&X\ar{r}[swap]{b}       &\Coker p\ar[equal]{r}&\Coim f,
\end{tikzcd}
$$ 
where $h,a,$ and $b$ are the natural morphisms. 

We claim $b\ci h=0$. Indeed, we define $c:\Ker f\to P$ by the condition $p_1\ci c=h,p_2\ci c=0$: 
$$
\begin{tikzcd}
{}&X\ar{dr}{f}\\ 
\Ker f\ar{ur}{h}\ar{dr}[swap]{0}\ar[dashed]{r}{c}&P\ar{u}[swap]{p_1}\ar{d}{p_2}&Y\\ 
{}&X\ar{ur}[swap]{f},
\end{tikzcd}
$$
and we get $b\ci h=b\ci p\ci c=0\ci c=0$. This proves the claim. Hence, there is a unique morphism $d:\Coker h\to\Coim f$ making the diagram 
$$
\begin{tikzcd}
\Ker f\ar{r}{h}&X\ar[equal]{d}\ar{r}{a}&\Coker h\ar{d}{d}\\ 
P\ar{r}[swap]{p}&X\ar{r}[swap]{b}&\Coim f\ar[equal]{r}&\Coker p
\end{tikzcd}
$$ 
commute. 

As $p$ factors through $h$, we have $a\ci p=0$, and there is a unique morphism $e:\Coim f\to\Coker h$ making the diagram 
$$
\begin{tikzcd}
\Ker f\ar{r}{h}&X\ar[equal]{d}\ar{r}{a}&\Coker h\\ 
P\ar{r}[swap]{p}&X\ar{r}[swap]{b}&\Coim f\ar{u}[swap]{e}\ar[equal]{r}&\Coker p
\end{tikzcd}
$$ 
commute. 

It is easy to see that $d$ and $e$ are mutually inverse isomorphisms. In short, there is a natural isomorphism $\Coker h\iso\Coim f$ which makes the diagram
\begin{equation}\lb{834a}
\begin{tikzcd}
\Ker f\ar{r}{h}&X\ar[equal]{d}\ar{r}{a}&\Coker h\ar[leftrightarrow]{d}{\sim}\\ 
P\ar{r}[swap]{p}&X\ar{r}[swap]{b}&\Coim f\ar[equal]{r}&\Coker p
\end{tikzcd}
\end{equation} 
commute. 

Dually, let $S$ (for ``sum'') be the fiber coproduct $Y\oplus_XY$, let $i_a:Y\to S$ be the coprojection, let $i$ be the morphism $i_1-i_2$ from $Y$ to $S$, and consider the diagram 
$$
\begin{tikzcd}
\Ima f\ar{r}{a}&Y\ar[equal]{d}\ar{r}{i}&S\\ 
\Ker k\ar{r}{b}&Y\ar{r}[swap]{k}&\Coker f
\end{tikzcd}
$$ 
where $a,b,$ and $k$ are the natural morphisms. Then there is a natural isomorphism $\Ima f\iso\Ker k$ which makes the diagram 
\begin{equation}\lb{834b}
\begin{tikzcd}
\Ima f\ar[leftrightarrow]{d}[swap]{\sim}\ar{r}&Y\ar[equal]{d}\ar{r}{i}&S\\ 
\Ker k\ar{r}&Y\ar{r}[swap]{k}&\Coker f
\end{tikzcd}
\end{equation} 
commute. Let us record these observations:
\begin{prop}\lb{p834}
In the above setting there are natural isomorphisms 
$$
\Coker h\iso\Coim f,\quad\Ima f\iso\Ker k
$$ 
which make Diagrams \qr{834a} and \qr{834b} commute.
\end{prop}

Note that we can splice Diagrams \qr{834a} and \qr{834b}:
$$
\begin{tikzcd}
\Ker f\ar{r}{h} &X\ar[equal]{d}\ar{r}&\Coker h\ar[leftrightarrow]{d}{\sim}\\ 
P\ar{r}[swap]{p}&X\ar{r}             &\Coim f\ar{d}{\sim}\\ 
{}&{}&\Ima f\ar[leftrightarrow]{d}[swap]{\sim}\ar{r}&Y\ar[equal]{d}\ar{r}{i}&S\\ 
{}&{}&\Ker k\ar{r}&Y\ar{r}[swap]{k}&\Coker f.
\end{tikzcd}
$$ 

%%

\subsubsection{Definition 8.3.5 p.~177}

The following definitions and observations are implicit in the book. Let $\A$ be a subcategory of a pre-additive category $\B$, and let $\iota:\A\to\B$ be the inclusion. If $\A$ is pre-additive and $\iota$ is additive, we say that $\A$ is a {\em pre-additive subcategory} of $\B$. If, moreover, $\A$ and $\B$ are additive (resp. abelian), we say that $\A$ is {\em an additive (resp. abelian) subcategory} of $\B$. Now let $\A$ and $\B$ be categories. If $\B$ is pre-additive (resp. additive, abelian), then so is the category $\C:=\B^\A$ of functors from $\A$ to $\B$. Assume furthermore that $\A$ is pre-additive. If $\B$ is pre-additive (resp. additive, abelian), then the full subcategory $\DD:=\Ad(\A,\B)$ of $\C$ whose objects are the additive functors from $\A$ to $\B$ is a pre-additive (resp. additive, abelian) subcategory of $\C$.

%%

\subsubsection{The Complex (8.3.3) p.~178}

Let us just add a few details about the proof of the isomorphisms
\begin{equation}\lb{834}
\begin{split}
\Ima u\iso\Coker(\pp:\Ima f\to\Ker g)\iso\Coker(X'\to\Ker g)\\ 
\iso\Ker(\psi:\Coker f\to\Ima g)\iso\Ker(\Coker f\to X''),
\end{split}
\end{equation}
labeled (8.3.4) in the book. Recall that the underlying category $\C$ is abelian, and that the complex in question is denoted 
%
\begin{equation}\lb{833}
X'\xr{f}X\xr{g}X''.
\end{equation}
%  
We shall freely use the isomorphism between image and coimage, as well as the abbreviations 
$$
K_v:=\Ker v,\quad K'_v:=\Coker v,\quad I_v:=\Ima v.
$$ 
Let us also write ``$A\os{\sim}{\to}B$'' for ``the natural morphism $A\to B$ is an isomorphism''. 

Proposition \ref{p834} p.~\pr{p834} can be stated as follows. 
%
\begin{prop}\lb{p834b}
Let $f:X\to Y$ be a morphism, and consider the commutative diagram 
$$
\begin{tikzcd}
K_f\ar[tail]{rr}{h}&&X\ar{rr}{f}\ar[two heads]{dl}\ar[two heads]{dr}&&Y\ar[two heads]{rr}{k}&&K'_f\\ 
&K'_h\ar{rr}&&I_f\ar[tail]{ur}\ar{rr}&&K_k.\ar[tail]{ul}
\end{tikzcd}
$$ 
Then the bottom arrows are isomorphisms.
\end{prop}
%
Going back to our complex \qr{833} p.~\pr{833}, let us introduce the notation 
$$
\begin{tikzcd}
X'\ar{rrr}{f}\ar[equal]{d}&&&X\ar[equal]{d}\ar{rrr}{g}&&&X''\ar[equal]{d}\\ 
X'\ar[two heads]{r}{a}&I_f\ar[tail]{r}{\pp}&K_g\ar[equal]{d}\ar[tail]{r}{b}&X\ar[two heads]{r}{c}&K'_f\ar[equal]{d}\ar[two heads]{r}{\psi}&I_g\ar[tail]{r}{d}&X''\\ 
K_u\ar[tail]{rr}{e}&&K_g\ar[two heads]{dl}\ar[two heads]{dr}\ar{rr}{u}&&K'_f\ar[two heads]{rr}{h}&&K'_u\\ 
&K'_e\ar{rr}{\sim}[swap]{i}&&I_u\ar[tail]{ur}\ar{rr}{\sim}[swap]{j}&&K_h.\ar[tail]{ul}
\end{tikzcd}
$$ 
By Proposition~\ref{p834} p.~\pr{p834} 
\begin{equation}\lb{ijisos}
i\text{ and }j\text{ are isomorphisms.}
\end{equation}

We shall prove 
$$
\begin{tikzcd}
K'_{\pp\ci a}\ar{r}{k}[swap]{\sim}&K'_\pp\ar{r}{\ell}[swap]{\sim}&K'_e\ar{r}{i}[swap]{\sim}&I_u\ar{r}{j}[swap]{\sim}&K_h\ar{r}{m}[swap]{\sim}&K_\psi\ar{r}{n}[swap]{\sim}&K_{d\ci\psi}.
\end{tikzcd}
$$
This will imply \qr{834} p.~\pr{834}. 

The morphisms $k$ and $n$ are isomorphisms because $a$ is an epimorphism and $d$ a monomorphism. Thus, in view of \qr{ijisos}, it only remains to prove that 
\begin{equation}\lb{lmisos}
\ell\text{ and }m\text{ are isomorphisms.}
\end{equation}

There is a natural monomorphism from $I_f$ to $K_u$. Indeed, we have 
$$
u\ci\pp\ci a=c\ci f=0.
$$ 
As $a$ is an epimorphism, this implies $u\ci\pp=0$. 

It is easy to see that there is a natural monomorphisms from $K_u$ to $K_c$. By Proposition~\ref{p834} p.~\pr{p834}, we have $I_f\xr\sim K_c$, and it is easy to see that this implies $I_f\xr\sim K_u$. Similarly we prove $K'_u\xr\sim I_g$. 

We can thus complete our previous diagram as follows: 
$$
\begin{tikzcd}
X'\ar{rrr}{f}\ar[equal]{d}&&&X\ar[equal]{d}\ar{rrr}{g}&&&X''\ar[equal]{d}\\ 
X'\ar[two heads]{r}{a}&I_f\ar[dashed]{dl}[swap]{\sim}\ar[tail]{r}{\pp}&K_g\ar[equal]{d}\ar[tail]{r}{b}&X\ar[two heads]{r}{c}&K'_f\ar[equal]{d}\ar[two heads]{r}{\psi}&I_g\ar[tail]{r}{d}&X''\\ 
K_u\ar[tail]{rr}{e}&&K_g\ar[two heads]{dl}\ar[two heads]{dr}\ar{rr}{u}&&K'_f\ar[two heads]{rr}{h}&&K'_u\ar[dashed]{ul}[swap]{\sim}\\ 
&K'_e\ar{rr}{\sim}[swap]{i}&&I_u\ar[tail]{ur}\ar{rr}{\sim}[swap]{j}&&K_h.\ar[tail]{ul}
\end{tikzcd}
$$ 
(The two dashed arrows have been added.) Now \qr{lmisos} is clear.

%%

\subsubsection{Brief comments}

\begin{s} 
For the reader's convenience we state Lemma 8.3.11 p.~180. Consider the commutative square 
\begin{equation}\lb{837}
\begin{tikzcd}
X'\ar{d}[swap]{g'}\ar{r}{f'}&Y'\ar{d}{g}\\ 
X\ar{r}[swap]{f}&Y
\end{tikzcd}
\end{equation} 
in the abelian category $\C$. 

\begin{lem}[Lemma 8.3.11 p.~180]\lb{8311}
We have:

\nn\emph{(a)} Assume that \qr{837} is cartesian. 

\emph{(i)} We have $\Ker f'\xr\sim\Ker f$. 

\emph{(ii)} If $f$ is an epimorphism, then \qr{837} is cocartesian and $f'$ is an epimorphism.

\nn\emph{(b)} Assume that \qr{837} is cocartesian.

\emph{(i)} We have $\Coker f'\xr\sim\Coker f$.

\emph{(ii)} If $f'$ is a monomorphism, then \qr{837} is cartesian and $f$ is a
monomorphism.
\end{lem} 
\end{s} 

%

\begin{s}
P.~180, Lemma 8.3.12. Here is a minor variant:

\begin{lem}\lb{8312}
For a complex $Z\to Y\to X$ in some abelian category, the following conditions are equivalent:

\nn{\em(a)} the complex is exact,

\nn{\em(b)} any commutative diagram of solid arrows
$$
\begin{tikzcd}
V\ar[dashed]{d}\ar[dashed, two heads]{r}&W\ar{dr}{0}\ar{d}\\ 
Z\ar{r}&Y\ar{r}&X
\end{tikzcd}
$$ 
can be completed as indicated ($V\to W$ being an epimorphism),

\nn{\em(c)} any commutative diagram of solid arrows
$$
\begin{tikzcd}
Z\ar{dr}[swap]{0}\ar{r}&Y\ar{d}\ar{r}&X\ar[dashed]{d}\\ 
{}&W\ar[tail,dashed]{r}&V
\end{tikzcd}
$$ 
can be completed as indicated ($W\to V$ being a monomorphism).
\end{lem}

\begin{proof}
Equivalence (a)$\ssi$(b) is proved in the book, and Equivalence (a)$\ssi$(c) follows by reversing arrows.
\end{proof}
\end{s}

%

\begin{s} 
Page 181, the Five Lemma (minor variant of the proof). 

\begin{thm}[Lemma 8.3.13 p.~181, Five Lemma]\lb{5l} 
Consider the commutative diagram of complexes 
$$
\begin{tikzcd}
X^0\ar[two heads]{d}[swap]{f^0}\ar{r}{a^0}&
X^1\ar[tail]{d}[swap]{f^1}\ar{r}{a^1}&
X^2\ar{d}{f^2}\ar{r}{a^2}&
X^3\ar[tail]{d}{f^3}\\ 
Y^0\ar{r}[swap]{b^0}&
Y^1\ar{r}[swap]{b^1}&
Y^2\ar{r}[swap]{b^2}&
Y^3,
\end{tikzcd}
$$
where $f^0$ is an epimorphism, $f^1$ and $f^3$ are monomorphisms, and $X^1\to X^2\to X^3$ and $Y^0\to Y^1\to Y^2$ are exact. Then $f^2$ is a monomorphism. 
\end{thm} 

\begin{proof}
Note that Equivalence (a)$\ssi$(b) in Lemma~\ref{8312} p.~\pr{8312} can be stated as follows: 

\nn$(*)\ f:X\to Y$ is an epimorphism if and only if any subobject of $Y$ is the image of some subobject of $X$. 

We write $fx$ for the image of a subobject $x$ of $X$, and $fg$ for $f\ci g$.

Put $x^2:=\Ker f^2$. Using $(*)$ we see that there is: 

\nn$\bu$ a subobject $x^1$ of $X^1$ such that $x^2=a^1x^1$ (because $f^3$ is a monomorphism, $f^3a^2x^2=0$, and $X^1\xr{a^1}X^2\xr{a^2}X^3$ is exact), 

\nn$\bu$ a subobject $y^0$ of $Y^0$ such that $f^1x^1=b^0y^0$ (because $b^1f^1x^1=0$ and $Y^0\xr{b^0}Y^1\xr{b^1}Y$ is exact), and 

\nn$\bu$ a subobject $x^0$ of $X^0$ such that $y^0=f^0x^0$ (because $f^0$ is an epimorphism). 

This yields 
$$
f^1a^0x^0=b^0f^0x^0=b^0y^0=f^1x^1,
$$
implying $a^0x^0=x^1$ (because $f^1$ is a monomorphism), and thus 
$$
0=a^1a^0x^0=a^1x^1=x^2.
$$ 
\end{proof}
\end{s}

%

\begin{s} 
P.~181, Lemma 8.3.13 (Five Lemma). We spell out the dual of Theorem~\ref{5l} above.
\begin{thm}\lb{5ld} 
Consider the commutative diagram of complexes 
$$
\begin{tikzcd}
X^0\ar[two heads]{d}[swap]{f^0}\ar{r}{a^0}&
X^1\ar{d}[swap]{f^1}\ar{r}{a^1}&
X^2\ar[two heads]{d}{f^2}\ar{r}{a^2}&
X^3\ar[tail]{d}{f^3}\\ 
Y^0\ar{r}[swap]{b^0}&
Y^1\ar{r}[swap]{b^1}&
Y^2\ar{r}[swap]{b^2}&
Y^3,
\end{tikzcd}
$$
where $f^0$ and $f^2$ are epimorphisms, $f^3$ is a monomorphism, and $X^0\to X^1\to X^2$ and $Y^1\to Y^2\to Y^3$ are exact. Then $f^1$ is an epimorphism. 
\end{thm}
\end{s}

% https://docs.google.com/document/d/1jB8iAaLYhMJ3lRUk7lY_gFxdUTgktM8Libu7Pa6ONac/edit

\begin{s} 
P.~182, proof of the equivalence (iii)$\ssi$(iv) in Proposition 8.3.14. Here is the statement of the proposition:

\begin{prop}[Proposition 8.3.14 p. 182] 
Let $0\to X'\xr fX\xr gX''\to0$ be a short exact sequence in an abelian category $\C$. Then the conditions below are equivalent:
\begin{itemize}
\item[\em(i)] there exits $h:X''\to X$ such that $g\ci h=\id_{X''}$,
\item[\em(ii)] there exits $k:X\to X'$ such that $k\ci f=\id_{X'}$,
\item[\em(iii)] there exits $h:X''\to X$ and $k:X\to X'$ such that $\id_X=f\ci k+h\ci g$,
\item[\em(iv)] there exits $\pp=(k,g)$ and $\psi=(f,h)$ such that $X\xr\pp X'\oplus X''$ and $X'\oplus X''\xr\psi X$ are mutually inverse isomorphisms,
\item[\em(v)] for any $Y$ in $\C$, the map $\Hom_\C(Y,X)\xr{g\ci}\Hom_\C(Y,X'')$ is surjective,
\item[\em(vi)] for any $Y$ in $\C$, the map $\Hom_\C(X,Y)\xr{\ci f}\Hom_\C(X',Y)$ is surjective.
\end{itemize}
\end{prop}

The authors say that the equivalence (iii)$\ssi$(iv) is obvious. I agree, but here are a few more details. Implication (iv)$\then$(iii) is indeed obvious in the strongest sense of the word. Implication (iii)$\then$(iv) can be proved as follows. 

Assume (iii), that is, we have morphisms $h:X''\to X$ and $k:X\to X'$ such that 
\begin{equation}\lb{fk+hg} 
f\ci k+h\ci g=\id_X.
\end{equation} 
As $g\ci f=0$, this implies 
$$
g\ci h\ci g=g\ci f\ci k+g\ci h\ci g=g\ci\id_X=g. 
$$ 
Since $g$ is an epimorphism, this entails $g\ci h=\id_{X''}$. We prove similarly $k\ci f=\id_{X'}$. Let us record the two above equalities: 
\begin{equation}\lb{gh,kf} 
g\ci h=\id_{X''},\quad k\ci f=\id_{X'}.
\end{equation} 
Now \qr{fk+hg} and \qr{gh,kf} imply 
$$
k\ci h=k\ci(f\ci k+h\ci g)\ci h=k\ci f\ci k\ci h+k\ci h\ci g\ci h=k\ci h+k\ci h,
$$ 
and thus 
\begin{equation}\lb{kh} 
k\ci h=0, 
\end{equation} 
and (iv) follows from \qr{fk+hg}, \qr{gh,kf} and \qr{kh}. q.e.d.
\end{s}

%

\begin{s} 
P.~183. Here is an example showing that filtrant and cofiltrant small projective limits of $R$-modules are not exact in general: 
$$
\lim_{n\in\bb N}\big(\bb Z\to\bb Z/2^n\bb Z\to0\big)=\big(\bb Z\to\bb Z_2\to0\big).
$$
\end{s} 

% 

\begin{s} 
P. 184, Definitions 8.3.21 (v) and (vi). See \S~\ref{dg2} p.~\pr{dg2}.
\end{s} 

%%

\subsubsection{Proof of Lemma 8.3.23 p. 184}

In the book, the proofs of the two lemmas below are left to the reader. 

\begin{lem}\lb{184a}
If 
$$
\begin{tikzcd}
&&Y\ar[d,"c"']\ar[r,"u"]&Y''\ar[d,"\id"]\ar[r]&0\\ 
0\ar[r]&Y'\ar[r,"a"']&X\ar[r,"b"']&Y''\ar[r]&0
\end{tikzcd}
$$ 
is an exact commutative diagram in an abelian category, then $Y\oplus Y'\xr{(c,a)}X$ is an epimorphism. 
\end{lem}

\begin{proof}
In this proof below we write $\psi\pp$ for $\psi\ci\pp$, and we tacitly use Lemma~\ref{8312} p.~\pr{8312}. 

Let $x:Z\to X$ and let us show that the solid diagram 
$$
\begin{tikzcd}
W\ar[d,"e"',dashed]\ar[r,"d",dashed,two heads]&Z\ar[d,"x"]\\ 
Y\oplus Y'\ar{r}[swap]{(c,a)}&X
\end{tikzcd}
$$ 
may be completed as indicated. We get a commutative square 
$$
\begin{tikzcd}
V\ar[d,"y"']\ar[r,"f",two heads]&Z\ar[d,"bx"]\\ 
Y\ar[r,"bc"',two heads]&Y''
\end{tikzcd}
$$ 
and then a commutative diagram 
$$
\begin{tikzcd}
W\ar[d,"y'"']\ar[r,"g",two heads]&V\ar[d,"xf-cy"]\\ 
Y'\ar[r,"a"',rightarrowtail]&X\ar[r,"b"',two heads]&Y''.
\end{tikzcd}
$$ 
Setting $d:=fg,e:=(yg,y')$ yields %$xd=xfg$ and 
$$
(c,a)e=(c,a)(yg,y')=cyg+ay'=cyg+xfg-cyg=xfg=xd.
$$ 
\end{proof}

\begin{lem}\lb{184b}
If
$$
\begin{tikzcd}
&&0\ar[d]&0\ar[d]\\ 
&&Z\ar[d,"b"']\ar[r,"a"]&Y\ar[d,"c"]\\ 
0\ar[r]&X\ar[d,equal]\ar[r,"d"]&W\ar[d,"f"']\ar[r,"e"]&V\ar[d,"g"]\ar[r]&0\\ 
0\ar[r]&X\ar[r,"h"']&U\ar[d]\ar[r,"i"']&T\ar[d]\ar[r]&0\\ 
&&0&0
\end{tikzcd}
$$ 
is an exact commutative diagram in an abelian category, then $a$ is an isomorphism. 
\end{lem}

\begin{proof}
In this proof below we write $\psi\pp$ for $\psi\ci\pp$, and we tacitly use Lemma~\ref{8312} p.~\pr{8312}. 

We claim
\begin{equation}\lb{8323m}
a\text{ is a monomorphism}.
\end{equation}
Let $z:S\to Z$ satisfy $az=0$, and let us show $z=0$. We have $ebz=0$, and the solid diagram 
$$
\begin{tikzcd}
R\ar[d,"x"',dashed]\ar[r,"j",dashed,two heads]&S\ar[d,"bz"]\\ 
X\ar[r,"d"']&W\ar[r,"e"']&V
\end{tikzcd}
$$ 
may be completed as indicated, yielding successively $hx=fdx=fbzj=0$, $x=0$, $bzj=0$, $bz=0$, $z=0$. This proves \qr{8323m}. 

We claim
\begin{equation}\lb{8323e}
a\text{ is an epimorphism}.
\end{equation} 
Let $y:S\to Y$ and let us show that the solid diagram 
$$
\begin{tikzcd}
R\ar[d,"z"',dashed]\ar[r,"j",dashed,two heads]&S\ar[d,"y"]\\ 
Z\ar[r,"a"']&Y
\end{tikzcd}
$$ 
may be completed as indicated. We get successively: a commutative square 
$$
\begin{tikzcd}
Q\ar[d,"w"']\ar[r,"k",two heads]&S\ar[d,"cy"]\\ 
W\ar[r,"e"',two heads]&V;
\end{tikzcd}
$$ 
equalities $ifw=gew=gcyk=0$; an exact commutative diagram 
$$
\begin{tikzcd}
P\ar[d,"x"']\ar[r,"\ell",two heads]&Q\ar[d,"fw"]\\ 
W\ar[r,"h"']&V\ar[r,"i"']&T;
\end{tikzcd}
$$ 
equalities $fdx=hx=fw\ell$, $f(w\ell-dx)=0$; an exact commutative diagram 
$$
\begin{tikzcd}
R\ar[d,"z"']\ar[r,"m",two heads]&P\ar[d,"w\ell-dx"]\\ 
Z\ar[r,"b"']&W\ar[r,"f"']&U;
\end{tikzcd}
$$ 
equalities $caz=ebz=ew\ell m-edxm=ew\ell m=cyk\ell m$, $az=yk\ell m$; and it suffices to set $j:=k\ell m$. This proves \qr{8323e}, and thus our lemma. %\qr{8323}.
\end{proof}

%%

\subsubsection{Brief comments}

\begin{s} 
On p. 185 we read:

``Recall (see Proposition 5.2.4) that in an abelian category, the conditions
below are equivalent:

(i) $G$ is a generator, that is, the functor $\pp_G=\Hom_\C(G,\ )$ is conservative,

(ii) The functor $\pp_G$ is faithful.

\nn\dots

Moreover, if C admits small inductive limits, the conditions above are
equivalent to:

(iii) for any $X\in\C$, there exist a small set $I$ and an epimorphism $G^{\sqcup I}\epi X$.''

It would be better (I think) to refer to Proposition 2.2.3 p.~45 of the book for the equivalence between (i) and (ii).
\end{s}

%

\begin{s} 
P.~186, Definition 8.3.24 (definition of a Grothendieck category). The condition that small filtrant inductive limits are exact is not automatic. I know no entirely elementary proof of this important fact. Here is a proof using a little bit of sheaf theory. To show that there is an abelian category where small filtrant inductive limits exist but are not exact, it suffices to prove that there is an abelian category $\C$ where small filtrant {\em projective} limits exist but are not exact. It is even enough to show that small products are not exact in $\C$. Let $X$ be a topological space, and let $U_0\supset U_1\supset\cdots$ be a decreasing sequence of open subsets whose intersection is a non-open closed singleton $\{a\}$. We can take for $\C$ the category of small abelian sheaves on $X$. To see this, let $G$ be the abelian presheaf over $X$ such that $G(U)$ is $\mathbb Z$ if $a$ is in $U$ and 0 otherwise, and, for each $n$ in $\mathbb N$, let $F_n$ be the abelian presheaf over $X$ such that $F_n(U)$ is $\mathbb Z$ if $U\subset U_n$ and 0 otherwise. These presheaves are easily seen to be sheaves. For each $n$ in $\mathbb N$ and each open set $U$ let $F_n(U)\to G(U)$ be the identity if $a$ is in $U\subset U_n$ and 0 otherwise. This family of morphisms defines, when $U$ varies, an epimorphism $\pp_n:F_n\epi G$. Put 
$$
F:=\prod_{n\in\mathbb N}F_n,\quad H:=\prod_{n\in\mathbb N}G,\quad\pp:=\prod_{n\in\mathbb N}\pp_n:F\to H.
$$ 
It suffices to show that the morphism $\pp(a):F(a)\to H(a)$ between the stalks at $a$ induced by $\pp$ is not an epimorphism. This is clear because $\pp(a)$ is the natural morphism 
$$
\bigoplus_{n\in\mathbb N}\mathbb Z\to\prod_{n\in\mathbb N}\mathbb Z.
$$
q.e.d.
\end{s}

%

\subsubsection{Corollary 8.3.26 p.~186}
 
Recall the statement:

Let $\C$ be a Grothendieck category and let $X\in\C$. Then the set of quotients of $X$ and the the set of subobjects of $X$ are small. 

The proof is phrased as follows: ``Apply Proposition 5.2.9''. One could add ``\dots and Proposition 5.2.3 (v)''.

%

\subsubsection{Proposition 8.3.27 (i) p.~186}\lb{gcsbc}

By Proposition 8.3.27 (i) and Lemma 3.3.9 p.~83 of the book, in a Grothendieck $\U$-category\index{Grothendieck category} $\U$-small filtrant inductive limits are stable by base change (Definition 2.2.6 p.~47 of the book; see \S\ref{parsbc} p.~\pr{parsbc}).

%% 

\sbs{About Section 8.4} 

This is about Proposition 8.4.7 p. 187. Let us just rewrite in a slightly less concise way the part of the proof on p.~188 which starts with the sentence ``Define $Y:=Y_0\tm_XG_i$'' at the fifth line of the last paragraph of the proof, and goes to the end of the proof. 

It suffices to show that there is a morphism $a_0:G_i\to Y_0$ satisfying $l_0\ci a_0=\pp$:
$$
\begin{tikzcd}
X'\ar{d}[swap]{h}\ar[tail]{r}{k_0}&Y_0\ar{dl}{g_0}\ar[tail]{r}{l_0}&X\\ 
Z&&G_i.\ar[dashed]{ul}{a_0}\ar{u}[swap]{\pp}
\end{tikzcd}
$$ 
Form the cartesian square 
$$
\begin{tikzcd}
Y\ar{r}{b}\ar[swap]{d}{c}&Y_0\ar[tail]{d}{l_0}\\
G_i\ar[swap]{r}{\pp}&X,
\end{tikzcd}
$$
and the cocartesian square 
$$
\begin{tikzcd}
Y\ar{r}{b}\ar[swap]{d}{c}&Y_0\ar{d}{\ld}\\
G_i\ar[swap]{r}{a_1}&Y_1.
\end{tikzcd}
$$ 
Let $l_1:Y_1\to X$ be the morphism which makes the diagram 
$$
\begin{tikzcd}
{}&Y_0\ar{d}[swap]{\ld}\ar{dr}{l_0}\\ 
Y\ar{ur}{b}\ar{dr}[swap]{c}&Y_1\ar[dashed]{r}{l_1}&X\\ 
{}&G_i\ar{u}{a_1}\ar{ru}[swap]{\pp}
\end{tikzcd}
$$ 
commutative. By Lemma \ref{8311} (a) (i) p.~\pr{8311}, $c$ is a monomorphism, and, by Part (b) (ii) of the same lemma, $\ld$ is also a monomorphism. As $Z$ is injective, there is a morphism $d:G_i\to Z$ satisfying $d\ci c=g_0\ci b$: 
$$
\begin{tikzcd}
Y\ar{r}{b}\ar[tail]{d}[swap]{c}&Y_0\ar{d}{g_0}\\ 
G_i\ar[dashed]{r}[swap]{d}&Z.
\end{tikzcd}
$$ 
By the definition of $Y_1$ there is a morphism $g_1:Y_1\to Z$ such that 
$$
\begin{tikzcd}
Y\ar{r}{b}\ar[swap]{d}{c}&Y_0\ar{d}[swap]{\ld}\ar[bend left]{ddr}{g_0}\\
G_i\ar{r}{a_1}\ar[bend right]{rrd}[swap]{d}&Y_1\ar{dr}{g_1}\\ 
{}&{}&Z
\end{tikzcd}
$$ 
commutes. We get the commutative diagram
$$
\begin{tikzcd}
{}&Y_0\ar[equal]{d}\ar{rr}{l_0}&&X\ar[equal]{d}\\ 
X'\ar{d}[swap]{h}\ar[tail]{r}{k_0}&Y_0\ar{dl}[swap]{g_0}\ar[tail]{r}{\ld}&Y_1\ar{dll}{g_1}\ar[tail]{r}{l_1}&X\\ 
Z&&&G_i.\ar{u}[swap]{\pp}\ar{lll}{d}\ar{ul}{a_1}
\end{tikzcd}
$$ 
As $\ld$ is an isomorphism by maximality of $(Y_0,g_0,l_0)$, we can set $a_0:=\ld^{-1}\ci a_1$, and we get 
$$
l_0\ci a_0=l_0\ci\ld^{-1}\ci a_1=l_1\ci\ld\ci\ld^{-1}\ci a_1=l_1\ci a_1=\pp.
$$ 
q.e.d.
% https://docs.google.com/document/d/1r1eBAIdNxQNO4uJ7N1Ki3ewyPQBoYIxLLq5I54ciVa8/edit 

%% 

\sbs{About Section 8.5}

\subsubsection{Brief comments}

\begin{s} P.~190, Proposition 8.5.5. It might be worth writing explicitly the formulas (for $X,Y\in\Mod(R,\C)$):
$$%\begin{equation}\lb{rop}
\Hom_{R^{\op}}(N,\Hom_\C(X,Y))\iso\Hom_\C\left(N\otimes_RX,Y\right),
$$%\end{equation} 
$$
\Hom_R(M,\Hom_\C(Y,X))\iso\Hom_\C\left(Y,\Hom_R(M,X)\right),
$$
$$
R^{\op}\otimes_RX\iso X,
$$
$$
\Hom_R(R,X)\iso X.
$$
One could also mention explicitly the adjunctions
$$
\begin{tikzcd}
\Mod(R^{\op})\ar[xshift=-0.7ex]{d}[swap]{-\otimes_RX}&&&
\Mod(R)^{\op}\ar[xshift=-0.7ex]{d}[swap]{\Hom_\C(-,X)}\\
\C\ar[xshift=0.7ex]{u}[swap]{\Hom_\C(X,-)}&&&\C,\ar[xshift=0.7ex]{u}[swap]{\Hom_R(-,X)}
\end{tikzcd}
$$
where, we hope, the notation is self-explanatory.

%The fact that the functor $$-\otimes_R-:\Mod(R^{\op})\tm\Mod(R,\C)\to\C$$ is right exact in each variable follows from \qr{rop} and \S\ref{ex} p.~\pr{ex}.
\end{s}

%

\begin{s} P.~191, proof of Theorem 8.5.8 (iii) (minor variant). Recall the statement: 

\begin{prop}[Theorem 8.5.8 (iii) p.~191]\lb{858iii}
Let $\C$ be a Grothendieck category, let $G$ be a generator, let $R$ be the ring $\oo{End}_\C(G)^{\op}$, put $\M:=\Mod(R)$, let $\pp:\C\to\M$ be the functor defined by $\pp(X):=\Hom_\C(G,X)$. Then $\pp$ is fully faithful. 
\end{prop}

\begin{proof}
Let $\psi:\M\to\C$ be the functor defined by $\psi(M):=G\otimes_RM$, let $\C_0$ be the full subcategory of $\C$ whose objects are 
$$
0,\quad G,\quad G\oplus G,\quad G\oplus G\oplus G,\quad\dots,
$$
and let $\M_0$ be the full subcategory of $\M$ whose objects are 
$$
0,\quad R,\quad R\oplus R,\quad R\oplus R\oplus R,\quad\dots
$$
Then $\pp$ and $\psi$ induce mutually quasi-inverse equivalences 
$$
\begin{tikzcd}
\C_0\ar[yshift=.7ex]{r}{\pp_{{}_0}}&\M_0.\ar[yshift=-.7ex]{l}{\psi_{{}_0}}
\end{tikzcd}
$$ 
We can assume that $\C_0$ and $\M_0$ are small. If $\ld:\C\to(\C_0)^\wg$ and $\ld':\M\to(\M_0)^\wg$ are the obvious functors, then the diagram 
$$
\begin{tikzcd}
\C\ar{r}{\pp}\ar{d}[swap]{\ld}&\M\ar{d}{\ld'}\\
(\C_0)^\wg\ar{r}[swap]{\widehat\pp_{{}_0}}&(\M_0)^\wg
\end{tikzcd}
$$ 
quasi-commutes. The functors $\ld$ and $\ld'$ are fully faithful by Section~\ref{gcsbc} p.~\pr{gcsbc} and Theorem~\ref{536} p.~\pr{536} above. As $\widehat\pp_{{}_0}$ is an equivalence (a quasi-inverse being $\widehat\psi_{{}_0}$, see Proposition 2.7.1 p.~62 in the book), the proof is complete.
\end{proof}
\end{s}

%%

\subsubsection{Theorem 8.5.8 (iv) p.~191} 

Here is a minor variant of Step~(a) of the proof of Theorem 8.5.8 (iv). Recall the statement: 

\begin{lem}
In the setting of Proposition~\ref{858iii}, assume that there is a finite set $F$, an epimorphism $R^F\epi M$ in $\M$, a small set $S$, and a monomorphism $M\rightarrowtail R^{\oplus S}$. Let $\psi:\M\to\C$ be the functor defined by $\psi(M):=G\otimes_RM$. Then $\psi(M)\to\psi(R^{\oplus S})$ is a monomorphism. 
\end{lem}

\begin{proof}
There is a finite subset $F'$ of $S$ such that $M\rightarrowtail R^{\oplus S}$ factors as 
$$
M\rightarrowtail R^{F'}\rightarrowtail R^{\oplus S}.
$$ 
As $R^{F'}$ is a direct summand of $R^{\oplus S}$, the morphism $\psi(R^{F'})\to\psi(R^{\oplus S})$ is a monomorphism. In other words, we may assume $S=F'$, and it suffices to check that $\psi(M)\to\psi(R^{F'})$ is a monomorphism, or, more explicitly, that 
\begin{equation}\lb{fpsi}
f:\psi(M)\to G^{F'}\text{ is a monomorphism.}
\end{equation}

Applying the right exact functor $\psi$ to 
$$
R^F\epi M\rightarrowtail R^{F'},
$$
we get 
$$
\begin{tikzcd}
K\ar[tail]{r}{i}\ar[bend right]{rrr}{0}&G^F\ar[two heads]{r}{p}&\psi(M)\ar{r}{f}&G^{F'},
\end{tikzcd}
$$
where $K:=\Ker(f\ci p)$. Applying $\pp$ we obtain
$$
\begin{tikzcd}
\pp(K)\ar{r}{\pp(i)}\ar[bend right]{rrr}{0}&R^F\ar{r}{\pp(p)}&\pp(\psi(M))\ar{r}{\pp(f)}&R^{F'}.
\end{tikzcd}
$$
The commutative diagram
$$
\begin{tikzcd}
\pp(K)\ar[equal]{d}\ar{rrr}{0}&&&R^{F'}\ar[equal]{d}\\
\pp(K)\ar{r}{\pp(i)}&R^F\ar[equal]{d}\ar{r}{\pp(p)}&\pp(\psi(M))\ar{r}{\pp(f)}&R^{F'}\ar[equal]{d}\\
&R^F\ar{r}[swap]{a}&M\ar[tail]{r}[swap]{b}\ar{u}&R^{F'}
\end{tikzcd}
$$ 
yields $b\ci a\ci\pp(i)=0$. As $b$ is a monomorphism, we get $a\ci\pp(i)=0$, and thus $\pp(p)\ci\pp(i)=0$. Since $\pp$ is faithful by Proposition~\ref{858iii} p.~\pr{858iii}, this implies 
\begin{equation}\lb{pi=0}
p\ci i=0.
\end{equation} 

Let us prove \qr{fpsi}. Let $x:X\to\psi(M)$ be a morphism in $\C$ satisfying $f\ci x=0$. It suffices to prove 
\begin{equation}\lb{x=0}
x=0.
\end{equation} 
As $p$ is an epimorphism, the diagram of solid arrows 
$$
\begin{tikzcd}
Y\ar[dashed]{d}[swap]{y}\ar[dashed, two heads]{r}{c}&X\ar{d}{x}\\ 
G^F\ar[two heads]{r}[swap]{p}&\psi(M)
\end{tikzcd}
$$ 
can be completed, by Lemma \ref{8311} (b) (i) p.~\pr{8311}, to a commutative square as indicated, $c$ being an epimorphism. The commutative diagram of solid arrows 
$$
\begin{tikzcd}
{}&Y\ar[dashed]{dl}[swap]{z}\ar{d}{y}\ar[two heads]{r}{c}&X\ar{d}[swap]{x}\ar{dr}{0}\\ 
K\ar{r}[swap]{i}&G^F\ar[two heads]{r}[swap]{p}&\psi(M)\ar{r}[swap]{f}&G^F
\end{tikzcd}
$$ 
can in turn be completed to a commutative diagram as indicated, and we get 
$$
x\ci c=p\ci i\ci z=0 
$$  
by \qr{pi=0}. As $c$ is an epimorphism, this implies successively \qr{x=0}, \qr{fpsi} and the lemma. 
\end{proof}

%

\sbs{About Section 8.6} 

\begin{s}
P.~193. Just before the statement of Proposition 8.6.2 it is claimed that the inclusion 
$$
\Ind(\C)\subset\C^{\wg,add}
$$ 
holds. This inclusion follows from Propositions 6.1.7 p.~132 and 8.2.15 p.~173 in the book.
\end{s}

%

%\begin{s} P. 194, proof of Theorem 8.6.5. \nn(ii) The exactness of $\C\to\Ind(\C)$ follows from Corollaries 6.1.6 and 6.1.17 (i). The left exactness of $\Ind(\C)\to\C^{\wg,add}$ follows from \S\ref{869} above and Proposition 6.1.16 (i) p.~136 in the book. \nn(iii) $\Ind(\C)$ admits small inductive limits by Proposition 6.1.18 (iii) p.~136 in the book.\end{s} \begin{s}\lb{869} P. 195, proof of Proposition 8.6.9. We give additional details: \nn(i) $\Ind(\C)$ is additive: For $A,B$ in $\Ind(\C)$ we define $A\oplus B\in\C^\wg$ by the formula $$(A\oplus B)(X):=A(X)\oplus B(X),$$ and easily check that $A\oplus B$ is additive and left exact, and thus belongs to $\Ind(\C)$. It is then obvious that $A\oplus B$ is a product and a coproduct in $\Ind(\C)$. $\Ind(\C)$ admits kernels and cokernels: $\Ind(\C)$ admits finite inductive and projective limits by Corollary 6.1.17 (i) and Proposition 6.1.18 (iii) p.~136.\nn(ii) $\Ind(\C)$ admits small filtrant inductive limits by Theorem~\ref{618} p.~\pr{618}. \nn(iii) The natural embedding $\Ind(\C)\to\C^{\wg,add}$ commutes with kernels: this follows from Proposition 6.1.16 (i) p.~136. \end{s} 

%

\begin{s} 
Proof of Lemma 8.6.7 p.~195. The proof uses the following lemma: 

\begin{lem} 
Let $f:X\to Y$ be a morphism in an abelian category, define $f':X\to X\oplus Y$ and $f'':X\oplus Y\to Y$ by $f':=\bigl[\begin{smallmatrix}1\\ f\end{smallmatrix}\bigr],\ f'':=[f\ -1]$ (obvious notation). Then the sequence $X\xr{f'}X\oplus Y\xr{f''}Y$ is exact. 
\end{lem} 

\begin{proof} 
It suffices to show that an arbitrary solid commutative diagram 
$$
\begin{tikzcd}
W\ar[dashed]{d}[swap]{i}\ar[dashed, two heads]{r}{h}&Z\ar{dr}{0}\ar{d}{g}\\ 
X\ar{r}[swap]{f'}&X\oplus Y\ar{r}[swap]{f''}&Y
\end{tikzcd}
$$ 
may be completed as indicated. Let the above square be cartesian. Note that $g$ is of the form $\bigl[\begin{smallmatrix}g_1\\ f\ci g_1\end{smallmatrix}\bigr]$ with $g_1:Z\to X$. It suffices to show that $h$ is an epimorphism. Let $j:Z\to V$ be a morphism such that $j\ci h=0$. It suffices to prove $j=0$. The solid commutative diagram 
$$
\begin{tikzcd}
&X\ar[rd,"f'"]\\ 
Z\ar[r,dashed,"k"]\ar[ru,"g_1"]\ar[rd,"\id"']&W\ar[u,"i"']\ar[d,"h"]&X\oplus Y\\ 
&Z\ar[ru,"g"']
\end{tikzcd}
$$ 
may be completed as indicated, yielding $0=j\ci h\ci k=j$. 
\end{proof} 
\end{s} 

%

\begin{s} 
P. 197, proof of Proposition 8.6.12. The existence of the epimorphism $X_1\to Y''$ follows from Proposition 8.6.9 in the book, and the statement ``Since the top square on the left is co-Cartesian, the middle row is exact'' follows from the lemma below.

\begin{lem}
If, in the commutative diagram 
$$
\begin{tikzcd}
0\ar[r]&A\ar[d]\ar[r]&B\ar[d]\ar[r]&C\ar[d,equal]\ar[r]&0\\ 
0\ar[r]&D\ar[r]&E\ar[r]&C\ar[r]&0,
\end{tikzcd}
$$ 
the left square is co-cartesian and the top row is exact, then the bottom row is exact.
\end{lem} 
\begin{proof}
In this proof we write $gf$ for $g\ci f$. Let us verify that the bottom row is exact at $E$. By Lemma \ref{8312} p.~\pr{8312}, (c)$\then$(a), it suffices to check that the solid diagram 
$$
\begin{tikzcd}
0\ar[r]&A\ar[d]\ar[r]&B\ar[d,"v"]\ar[r,"t"]&C\ar[d,equal]\ar[r]&0\\ 
0\ar[r]&D\ar[rd,"0"']\ar[r,"u"]&E\ar[d,"y"]\ar[r,"w"]&C\ar[d,"x",dashed]\ar[r]&0\\ 
&&F\ar[r,"z"',dashed,rightarrowtail]&G
\end{tikzcd}
$$ 
can be completed as indicated. The top row being exact, 
$$
\begin{tikzcd}
0\ar[r]&A\ar[rd,"0"']\ar[r]&B\ar[d,"yv"]\ar[r,"t"]&C\ar[d,"x",dashed]\ar[r]&0\\ 
&&F\ar[r,"z"',dashed,rightarrowtail]&G
\end{tikzcd}
$$ 
can be completed as indicated. It suffices to prove $zy=xw$. The square $ABDE$ being co-cartesian, it suffices to prove $zyu=xwu$ and $zyv=xwv$, which is easy. Thus shows that $0\to D\to E\to C$ is exact at $E$. Using Lemma~\ref{8311} p.~\pr{8311} it is straightforward to prove the exactness at $D$ and $C$.
\end{proof}
\end{s}

%% 

\sbs{About Section 8.7}

\subsubsection{Lemma 8.7.3 p. 198}\lb{873}

Let us spell out the proof of the fact that $K(\al)$ is a monomorphism. Consider the commutative diagram 
$$
\begin{tikzcd} 
Z\ar[ddd,"f"']\ar[r,"a"]&Y\ar[d,"c"']\ar[rr,"b"]&&X\ar[d,"d"]\\ 
&W\ar[dd,"g"']\ar[rd,two heads,"h"]\ar[rr,"e"]&&V\ar[dd,two heads,"i"]\\ 
&&U\ar[d,"j"]\\ 
T\ar[r,"k"']&S\ar[r,two heads,"\ell"']&R\ar[r,"m"']&Q,
\end{tikzcd}
$$ 
where the five rectangles are cartesian, and the three sequences 
$$
T\to S\to R\to 0,\quad X\to V\to Q\to 0,\quad Z\to W\to U\to 0
$$ 
are exact. 

We must check the $j$ is a monomorphism. 

(In the proof below we omit the composition symbols $\ci$ and most of the parenthesis. We shall freely use the equivalence (a)$\ssi$(b) in Lemma \ref{8312} p.~\pr{8312}.)

Let $u:P\to U$ satisfy $ju=0$, and let us show $u=0$. 

There is a commutative square 
$$
\begin{tikzcd} 
N\ar[d,"w"']\ar[r,two heads,"n"]&P\ar[d,"u"]\\ 
W\ar[r,two heads,"h"']&U.
\end{tikzcd}
$$ 
As $\ell gw=jhw=jun=0$, there is a commutative diagram 
$$
\begin{tikzcd} 
M\ar[d,"t"']\ar[r,two heads,"p"]&N\ar[d,"gw"']\ar[rd,"0"]\\ 
T\ar[r,"k"']&S\ar[r,"\ell"']&R.
\end{tikzcd}
$$ 
As $iewp=m\ell gwp=m\ell kt=0$, there is a commutative diagram 
$$
\begin{tikzcd} 
L\ar[d,"x"']\ar[r,two heads,"q"]&M\ar[d,"ewp"']\ar[rd,"0"]\\ 
X\ar[r,"d"']&V\ar[r,"i"']&Q.
\end{tikzcd}
$$

As $Z\iso T\oplus X$, we can introduce the coprojections $a'$ and $f'$ indicated below:
$$
\begin{tikzcd} 
Z\ar[ddd,xshift=0.7ex,"f"]\ar[r,"a"]&Y\ar[d,"c"']\ar[rr,"b"]&&X\ar[d,"d"]\ar[lll,bend right,"a'"']\\ 
&W\ar[dd,"g"']\ar[rd,two heads,"h"]\ar[rr,"e"]&&V\ar[dd,two heads,"i"]\\ 
&&U\ar[d,"j"]\\ 
T\ar[uuu,xshift=-0.7ex,"f'"]\ar[r,"k"']&S\ar[r,two heads,"\ell"']&R\ar[r,"m"']&Q.
\end{tikzcd}
$$ 

Define $z:L\to Z$ by $z:=f'tq+a'x$. We claim 
\begin{equation}\lb{caz}
caz=wpq.
\end{equation} 
It suffices to verify 
$$
gcaz=gwpq,\quad ecaz=ewpq.
$$ 
The proof of the two above equalities is straightforward and left to the reader, so that we consider that \qr{caz} has been proved. We get 
$$
unpq=hwpq=hcaz=0.
$$ 
As $pq$ is an epimorphism, this implies $u=0$, as desired. q.e.d.

%

\subsubsection{Lemma 8.7.5 (i) p.~199}

Let us spell out the proof of the fact that $\Coker(u)\to\Coker(v)$ is a monomorphism. We shall use the same notation and arguments as in Section~\ref{873} p.~\pr{873}.

In the commutative diagram 
$$
\begin{tikzcd} 
Y'\ar[d,two heads,"a"']\\ 
Z\ar[d,"p"']\ar[r,"q"]&Y\ar[d,"u"]\\ 
X'\ar[d,two heads,"c"']\ar[r,two heads,"b"]&X\ar[d,two heads,"d"]\\ 
W'\ar[r,"e"']&W,
\end{tikzcd}
$$ 
the square $ZYXX'$ is cartesian and the sequences 
$$
Y\to X\to W\to0,\quad Y'\to X'\to W'\to0
$$ 
are exact. 

We must show that $e$ is an isomorphism. Clearly $e$ is an epimorphism. It suffices to prove that $e$ is a monomorphism. Let $w':V\to W'$ satisfy $ew'=0$. It suffices to prove $w'=0$. 

Form the commutative square
$$
\begin{tikzcd} 
U\ar[d,"x'"']\ar[r,two heads,"f"]&V\ar[d,"w'"]\\ 
X'\ar[r,two heads,"c"']&W'.
\end{tikzcd}
$$ 
As we have $dbx'=ecx'=ew'f=0$, we can form the commutative diagram
$$
\begin{tikzcd} 
T\ar[d,"y"']\ar[r,two heads,"g"]&U\ar[d,"bx'"']\ar[rd,"0"]\\ 
Y\ar[r,"u"']&X\ar[r,"d"']&W.
\end{tikzcd}
$$ 
As we have $bx'g=uy$, we get a morphism $z:T\to Z$ such that $pz=x'g$ and $qz=y$, and we can form the commutative square
$$
\begin{tikzcd} 
S\ar[d,"y'"']\ar[r,two heads,"h"]&T\ar[d,"z"]\\ 
Y'\ar[r,two heads,"a"']&Z.
\end{tikzcd}
$$ 

This yields $w'fgh=cx'gh=cpzh=cpay'=0$. As $f,g$ and $h$ are epimorphisms, this implies $w'=0$, as desired. q.e.d. 

Here is a second version: 

% this second version is available at
% https://docs.google.com/document/d/1DTe3T39VIZf1zVzdAXUHG_K5euJ1EYuQzpil6QEtZjM/edit

P. 199, proof of Lemma 8.7.5 (i). As we have 
$$
\Coker(Y\tm_XX'\to X')\xr\sim\Coker(u)
$$ 
by Lemma~\ref{837} p.~\pr{837}, we can assume $X\in\J$. Let $b:Y'\epi Y$ be an epimorphism with $Y'\in\J$, and set $v:=ub,\ W:=\Coker(v),\ Z:=\Coker(u)$: 
$$
\begin{tikzcd}
Y'\ar[d,two heads,"b"']\ar[r,"v"]&X\ar[d,equal]\ar[r,"a"]&W\ar[d,"c"]\ar[r]&0\\ 
Y\ar[r,"u"']&X\ar[r,"d"']&Z\ar[r]&0.
\end{tikzcd}
$$ 
(The above diagram commutes and the rows are exact.) We shall use Lemma~\ref{8312} p.~\pr{8312}. We must show that $c$ is an isomorphism. Clearly $c$ is an epimorphism. It suffices to prove that $c$ is a monomorphism. Let $w:T\to W$ satisfy $cw=0$. It suffices to show $w=0$. There are commutative diagrams with exact rows and equalities 
$$
\begin{tikzcd}
R\ar[d,"x"']\ar[r,two heads,"t"]&T\ar[d,"w"]\\ 
X\ar[r,"a"']&W\ar[r]&0,
\end{tikzcd}
$$ 
$$
0=cwt=cax=dx,
$$ 
$$
\begin{tikzcd}
Q\ar[d,"y"']\ar[r,two heads,"r"]&R\ar[d,"x"]\\ 
Y\ar[r,"u"']&X\ar[r,two heads,"d"']&Z,
\end{tikzcd}
$$ 
$$
\begin{tikzcd}
P\ar[d,"y'"']\ar[r,two heads,"q"]&Q\ar[d,"y"]\\ 
Y'\ar[r,"b"']&Y\ar[r]&0,
\end{tikzcd}
$$ 
$$
wtrq=axrq=auyq=auby'=avy'=0.
$$ 
As $t,r$ and $q$ are epimorphisms, this implies $w=0$, as desired. q.e.d.

%

\subsubsection{Proof of (8.7.3) p. 200}

Right after (8.7.4) we read

``The condition that $K(\al)$ is an isomorphism is equivalent to the fact that the
sequence $Y\to X\oplus Y'\to X'\to0$ is exact.''

It seems to me we get a counterexample by setting $0\iso Y\iso X\iso X'\not\iso Y'$, but we can prove

\nn(8.7.3) for $\al:u\to v$ in $\Mor(\DD_0)$, if $K(\al)$ is an isomorphism, then $A'(\al)$ is an isomorphism

\nn as follows: 

Let $\al:u\to v$ in $\Mor(\DD_0)$ be such that $K(\al)$ is an isomorphism. By Proposition 8.3.18 p.~183, the diagram below commutes: 
$$
\begin{tikzcd} 
A'(u)\ar[d,equal]\ar[rr,"A'(\al)"]&&A'(v)\ar[d,equal]\\ 
\Coker A(u)\ar[d,"\sim"']\ar[rr,"\Coker A(\al)"]&&\Coker A(v)\ar[d,"\sim"]\\ 
A(K(u))\ar[rr,"A(K(\al))","\sim"']&&A(K(v)).
\end{tikzcd}
$$

%

\subsubsection{Proof of (8.7.2) p. 200}

Let us spell out the proof of the claim 

``The condition $K(\al)=0$ implies that $X\tm_{X'}Y'\to X$ is an epimorphism.''

\nn(This is the third sentence of the last paragraph.)

We shall use the same notation and arguments as in Section~\ref{873} p.~\pr{873}.

We have the commutative square with exact columns 
$$
\begin{tikzcd} 
Y\ar[d,"u"']\ar[r,"\al_0"]&Y'\ar[d,"v"]\\ 
X\ar[d,"u'"']\ar[r,"\al_1"]&X'\ar[d,"v'"]\\ 
Ku\ar[d]\ar[r,"0"]&Kv\ar[d]\\ 
0&0.
\end{tikzcd}
$$ 

Set $Z:=X\tm_{X'}Y'$ and write $p:Z\to X$ and $q:Z\to Y'$ for the projections, so that we must show that $p$ is an epimorphism. Let $x:W\to X$ be given. It suffices to complete the solid diagram 
$$
\begin{tikzcd} 
V\ar[d,dashed,"z"']\ar[r,dashed,two heads,"a"]&W\ar[d,"x"]\\ 
Z\ar[r,"p"']&X
\end{tikzcd}
$$ 
as indicated. As $v'\al_1x=0$, we get the commutative diagram 
$$
\begin{tikzcd} 
V\ar[d,"y'"']\ar[r,two heads,"a"]&W\ar[d,"\al_1x"']\ar[rd,"0"]\\ 
Y'\ar[r,"v"']&X'\ar[r,"v'"']&Kv.
\end{tikzcd}
$$ 
As $\al_1xa=vy'$ we get the commutative diagram 
$$
\begin{tikzcd} 
V\ar[d,"xa"']\ar[r,"y'"]\ar[rd,"z"]&Y'\\ 
X&Z\ar[l,"p"]\ar[u,"q"'].
\end{tikzcd}
$$ 

%

\subsubsection{Commutativity of the last diagram p. 200}

I failed to prove that the triangle $Y_1Y'X_1$ commutes, but it seems to me that this is not needed.

% https://docs.google.com/document/d/1M17qN1VK4E_zpNvlS6bNHYr32iVugc2GZFJJ7Nl8m9c/edit
% Lemma 8.7.4 (ii) p. 199

\subsubsection{Proof of Lemma 8.7.7 p. 201}

The last sentence of the proof of Lemma 8.7.7 uses Exercise 8.19 p.~204 (see Section \ref{819} p.~\pr{819} below).

(In the rest of this section we omit the composition symbols $\ci$ and most of the parenthesis, we freely use Lemma~\ref{8311} p.~\pr{8311} and Lemma~\ref{8312} p.~\pr{8312}, and we let the setting of Lemma 8.7.7 of the book be in force.)

\begin{lem}\lb{877}
If $Z\xr aY\xr bX\to0$ is an exact sequence in $\C$, then there is an exact commutative diagram 
\begin{equation}\lb{zyx}
\begin{tikzcd} 
T\ar[d,"h"']\ar[r,"k"]&S\ar[d,"i"]\ar[r,"\ell"]&R\ar[d,"j"]\ar[r]&0\\ 
W\ar[d,"c"']\ar[r,"f"]&V\ar[d,"d"]\ar[r,"g"]&U\ar[d,"e"]\ar[r]&0\\ 
Z\ar[d]\ar[r,"a"']&Y\ar[d]\ar[r,"b"']&X\ar[d]\ar[r]&0\\ 
0&0&0
\end{tikzcd}
\end{equation} 
with $R,S,T,U,V,W$ in $\J$.
\end{lem}

\begin{lem}\lb{877a}
The solid diagram in $\C$ below can be completed as indicated to a commutative diagram in $\C$ with $Z$ in $\J$: 
$$
\begin{tikzcd} 
\Ker b\ar[d,dashed,two heads,"c"']\ar[r,dashed,rightarrowtail,"a"]&Z\ar[d,dashed,two heads,"d"]\ar[r,dashed,two heads,"b"]&Y\ar[d,two heads,"e"]\\ 
\Ker g\ar[r,rightarrowtail,"f"']&X\ar[r,two heads,"g"']&W.
\end{tikzcd}
$$
\end{lem}

\begin{proof}[Proof of Lemma \ref{877a}]
Form the cartesian square 
$$
\begin{tikzcd} 
V\ar[d,"i"']\ar[r,"h"]&Y\ar[d,two heads,"e"]\\ 
X\ar[r,,two heads,"g"']&W.
\end{tikzcd}
$$ 
Note that $h$ and $i$ are epimorphisms. Let $\begin{tikzcd}Z\ar[r,two heads,"j"]&V\end{tikzcd}$ be an epimorphism in $\C$ with $Z$ in $\J$. We get the commutative diagram 
$$
\begin{tikzcd} 
\Ker b\ar[dd,"c"']\ar[r,rightarrowtail,"a"]&Z\ar[d,two heads,"j"']\ar[dr,two heads,"b"]\\ 
&V\ar[d,two heads,"i"']\ar[r,two heads,"h"']&Y\ar[d,two heads,"e"]\\ 
\Ker g\ar[r,rightarrowtail,"f"']&X\ar[r,two heads,"g"']&W.
\end{tikzcd}
$$ 
Set $d:=ij$. It only remains to check that $c$ is an epimorphism. Let $x:U\to\Ker g$. It suffices to complete the solid diagram 
$$
\begin{tikzcd} 
T\ar[d,dashed,"z"']\ar[r,dashed,two heads,"k"]&U\ar[d,"x"]\\ 
\Ker b\ar[r,"c"']&\Ker g
\end{tikzcd}
$$ 
as indicated. There is a morphism $v:U\to V$ such that 
$$
\begin{tikzcd} 
&Y\\ 
U\ar[ru,"0"]\ar[r,"v"]\ar[rd,"fx"']&V\ar[u,two heads,"h"']\ar[d,two heads,"i"]\\ 
&X
\end{tikzcd}
$$ 
commutes, and there are morphisms $k$ and $z'$ such that 
$$
\begin{tikzcd} 
T\ar[d,"z'"']\ar[r,two heads,"k"]&U\ar[d,"v"]\\ 
Z\ar[r,two heads,"j"']&V
\end{tikzcd}
$$ 
commutes. As we have $bz'=hjz'=hvk=0$, there is a morphism $z:T\to\Ker b$ such that $az=z'$, and we get $fcz=ijaz=ijz'=ivk=fxk$. Since $f$ is a monomorphism, this yields $cz=xk$, as desired. 
\end{proof}

The above proof shows that we have in fact:

\begin{lem}\lb{877b}
The solid diagram in $\C$ below can be completed as indicated to a commutative diagram in $\C$ with $Z$ in $\J$: 
$$
\begin{tikzcd} 
&\Ker b\ar[d,dashed,rightarrowtail]\ar[r,dashed,two heads]&\Ker c\ar[d,rightarrowtail]\\ 
\Ker a\ar[d,dashed,two heads]\ar[r,dashed,rightarrowtail]&Z\ar[d,dashed,two heads,"b"']\ar[r,dashed,two heads,"a"]&Y\ar[d,two heads,"c"]\\ 
\Ker d\ar[r,rightarrowtail]&X\ar[r,two heads,"d"']&W.
\end{tikzcd}
$$
\end{lem}

\begin{proof}[Proof of Lemma \ref{877}] 
Let $e:U\epi X$ be an epimorphism in $\C$ with $U$ in $\J$. A first application of Lemma~\ref{877b} gives a commutative diagram 
$$
\begin{tikzcd} 
&&\Ker d\ar[d,rightarrowtail]\ar[r,two heads,"m"]&\Ker e\ar[d,rightarrowtail]\\ 
&\Ker g\ar[d,two heads]\ar[r,rightarrowtail]&V\ar[d,two heads,"d"]\ar[r,two heads,"g"]&U\ar[d,two heads,"e"]\\ 
Z\ar[r,two heads]&\Ker b\ar[r,rightarrowtail]&Y\ar[r,two heads,"b"']&X
\end{tikzcd}
$$ 
with $V$ in $\J$. A second application of Lemma~\ref{877b} gives a commutative diagram 
$$
\begin{tikzcd} 
&&\Ker d\ar[d,rightarrowtail]\ar[r,two heads,"m"]&\Ker e\ar[d,rightarrowtail]\\ 
W\ar[d,two heads,"c"']\ar[r,two heads]&\Ker g\ar[d,two heads]\ar[r,rightarrowtail]&V\ar[d,two heads,"d"]\ar[r,two heads,"g"]&U\ar[d,two heads,"e"]\\ 
Z\ar[r,two heads]&\Ker b\ar[r,rightarrowtail]&Y\ar[r,two heads,"b"']&X
\end{tikzcd}
$$ 
with $W$ in $\J$. Let $R\epi\Ker e$ be an epimorphism in $\C$ with $R$ in $\J$. A third application of Lemma~\ref{877b} gives a commutative diagram 
$$
\begin{tikzcd} 
&\Ker\ell\ar[d,two heads]\ar[r,rightarrowtail]&S\ar[d,two heads]\ar[r,two heads,"\ell"]&R\ar[d,two heads]\\ 
&\Ker m\ar[r,rightarrowtail]&\Ker d\ar[d,rightarrowtail]\ar[r,two heads,"m"]&\Ker e\ar[d,rightarrowtail]\\ 
W\ar[d,two heads]\ar[r,two heads]&\Ker g\ar[d,two heads]\ar[r,rightarrowtail]&V\ar[d,two heads,"d"]\ar[r,two heads,"g"]&U\ar[d,two heads,"e"]\\ 
Z\ar[r,two heads]&\Ker b\ar[r,rightarrowtail]&Y\ar[r,two heads,"b"']&X
\end{tikzcd}
$$ 
with $S$ in $\J$. Let $\begin{tikzcd}\Ker c&T\ar[l,two heads]\ar[r,two heads]&\Ker\ell\end{tikzcd}$ be a diagram in $\C$ with $T$ in $\J$. We finally get 
$$
\begin{tikzcd} 
T\ar[d,two heads]\ar[r,two heads]&\Ker\ell\ar[d,two heads]\ar[r,rightarrowtail]&S\ar[d,two heads]\ar[r,two heads,"\ell"]&R\ar[d,two heads]\\ 
\Ker c\ar[d,rightarrowtail]&\Ker m\ar[r,rightarrowtail]&\Ker d\ar[d,rightarrowtail]\ar[r,two heads,"m"]&\Ker e\ar[d,rightarrowtail]\\ 
W\ar[d,two heads,"c"']\ar[r,two heads]&\Ker g\ar[d,two heads]\ar[r,rightarrowtail]&V\ar[d,two heads,"d"]\ar[r,two heads,"g"]&U\ar[d,two heads,"e"]\\ 
Z\ar[r,two heads]&\Ker b\ar[r,rightarrowtail]&Y\ar[r,two heads,"b"']&X,
\end{tikzcd}
$$ 
as required. 
\end{proof}

%% 

\sbs{About the exercises} 

\subsubsection{Exercise 8.4 p.~202}

Recall the statement: 

\emph{Let $\C$ be an additive category and $\SSS$ a right multiplicative system. Prove that the localization $\C_\SSS$ is an additive category and $Q:\C\to\C_\SSS$ is an additive functor.} 

It is easy to equip $\C_\SSS$ with a pre-additive structure making $Q$ additive. Then the result follows from Corollary~\ref{823b} p.~\pr{823b}. 

The pre-additive structure on $\C_\SSS$ is described in a very detailed way at the beginning of the following text of Dragan Mili\v{c}i\'c:  
%
\begin{center}\href{http://www.math.utah.edu/~milicic/Eprints/dercat.pdf}{http://www.math.utah.edu/$\sim$milicic/Eprints/dercat.pdf}
\end{center} 

%%%

\subsubsection{Exercise 8.17 p.~204}\lb{817} 

\paragraph{Preliminaries} 

\begin{lem} 
If  
%
\begin{equation}\lb{e817}
X\xr fY\xr gZ 
\end{equation}
% 
are morphisms in an abelian category $\C$ (we do not assume $g\ci f=0$), then the commutative diagram 
$$
\begin{tikzcd}
\Ker(g\ci f)\ar{r}&X\ar{d}\ar{r}&\Ima(g\ci f)\ar[dashed]{dl}\ar{d}\ar{r}&0\\ 
0\ar{r}&\Ima g\ar{r}&Z\ar{r}&\Coker g
\end{tikzcd}
$$ 
of solid arrows, whose rows are exact sequences, can be completed as indicated. The situation can also be represented as follows: 
$$
\begin{tikzcd}
X\ar[two heads]{dd}\ar{rr}{f}&&Y\ar[two heads]{dl}\ar{dd}{g}\\ 
{}&\Ima g\ar[tail]{dr}\\ 
\Ima(g\ci f)\ar[dashed,tail]{ur}\ar[tail]{rr}&&Z.
\end{tikzcd}
$$ 
In particular $\Ima(g\ci f)\to\Ima g$ is a monomorphism. 
\end{lem} 

\begin{proof} 
We claim that the diagram of solid arrows 
$$
\begin{tikzcd}
{}&X\tm_ZX\ar{r}{a}&X\ar{d}[swap]{f}\ar{r}{d}&\Coim(g\ci f)\ar{r}\ar[dashed]{ddl}{b}&0\\ 
{}&{}&Y\ar{d}[swap]{g}\\
0\ar{r}&\Ima g\ar{r}&Z\ar{r}[swap]{c}&Z\oplus_YZ,
\end{tikzcd}
$$ 
whose rows are exact sequences, can be completed as indicated. Indeed, the existence of $b$ follows from the equality %ies 
$g\ci f\ci a=%g\ci0=
0$. To prove the lemma, it is enough to check that $b$ factors through $\Ima g$, or, equivalently, that $c\ci b=0$. As $d$ is an epimorphism, the vanishing of $c\ci b$ is equivalent to the vanishing of $c\ci b\ci d$. But we have $c\ci b\ci d=c\ci g\ci f=0\ci f=0$. 
\end{proof}

\begin{lem}\lb{817b1} 
If, in the setting of Lemma~\ref{817a}, $f$ is an epimorphism, then 
$$
\Ima(g\ci f)\to\Ima g
$$ 
is an isomorphism. 
\end{lem} 

\begin{proof} 
Consider the commutative square  
$$
\begin{tikzcd}
X\ar{d}\ar{r}{f}&Y\ar{d}{a}\\ 
\Ima(g\ci f)\ar{r}[swap]{b}&\Ima g,
\end{tikzcd}
$$ 
where $a$ and $b$ are the natural morphisms. As $f$ and $a$ are epimorphisms, so is $b$. 
\end{proof} 

%%

\paragraph{Exercise 8.17}

The exercise follows easily from Lemmas \ref{817a} and \ref{817b} below.

Let us denote the cokernel of any morphism $h:Y\to Z$ in any abelian category by $Z/\Ima h$. 

Recall that, by Proposition 8.3.18 p.~183 of the book, an additive functor between abelian categories $F:\C\to\C'$ is left exact if and only if
\begin{equation}\lb{sex1}
\left.
\begin{matrix}
0\to X'\xr fX\xr gX''\text{ exact }\\ 
\then\\ 
0\to F(X')\os{F(f)\ }{\longrightarrow}F(X)\os{F(g)\ }{\longrightarrow}F(X'')\text{ exact}
\end{matrix}
\right\}
\end{equation}

Consider the condition
\begin{equation}\lb{sex2}
\left.
\begin{matrix}
0\to X'\xr fX\xr gX''\to0\text{ exact }\\ 
\then\\ 
0\to F(X')\os{F(f)\ }{\longrightarrow}F(X)\os{F(g)\ }{\longrightarrow}F(X'')\text{ exact}
\end{matrix}
\right\}
\end{equation}

\begin{lem}\lb{817a}
We have (\ref{sex1})$\ssi$(\ref{sex2}). 
\end{lem}

\begin{proof}
Implication $\then$ is clear. To prove $\si$, let 
$$
0\to X'\xr fX\xr gX''
$$
be exact. We must check that 
\begin{equation}\lb{0fx'fxfx''}
0\to F(X')\to F(X)\to F(X'')
\end{equation} 
is exact. Let $I$ be the image of $g$. The sequence 
$$ 
0\to X'\to X\to I\to0
$$ 
being exact, so is 
\begin{equation}\lb{0fx'fxfi}
0\to F(X')\to F(X)\to F(I).
\end{equation} 
This implies that \qr{0fx'fxfx''} is exact at $F(X')$. The sequence 
$$ 
0\to I\to X''\to X''/I\to0
$$  
being exact, so is 
$$ 
0\to F(I)\to F(X''),
$$  
and we have 
\begin{equation}\lb{kfxfx'}
\Ker\big(F(X)\to F(I)\big)\xr\sim\Ker\big(F(X)\to F(X'')\big).
\end{equation}
The exactness of \qr{0fx'fxfi} implies 
\begin{equation}\lb{kfxfi}
\Ima\big(F(X')\to F(X)\big)\xr\sim\Ker\big(F(X)\to F(I)\big), 
\end{equation} 
and the exactness of \qr{0fx'fxfx''} at $F(X)$ follows from \qr{kfxfx'} and \qr{kfxfi}. 
\end{proof} 

Consider the conditions below on our additive functor $F:\C\to\C'$:
\begin{equation}\lb{ex1}
\left.
\begin{matrix}
0\to X'\xr fX\xr gX''\to0\text{ exact }\\ 
\then\\ 
0\to F(X')\os{F(f)\ }{\longrightarrow}F(X)\os{F(g)\ }{\longrightarrow}F(X'')\to0\text{ exact}
\end{matrix}
\right\}
\end{equation}
%
\begin{equation}\lb{ex2}
\left.
\begin{matrix}
X'\xr fX\xr gX''\text{ exact }\\ 
\then\\ 
F(X')\os{F(f)\ }{\longrightarrow}F(X)\os{F(g)\ }{\longrightarrow}F(X'')\text{ exact}
\end{matrix}
\right\}
\end{equation}

\begin{lem}\lb{817b}
We have (\ref{ex1})$\ssi$(\ref{ex2}).
\end{lem}

\begin{proof}
Implication $\si$ is clear. To prove $\then$, let 
$$
X'\xr fX\xr gX''
$$
be exact. We must show that 
\begin{equation}\lb{fx'fxfx''}
F(X')\to F(X)\to F(X'')
\end{equation} 
is exact. Let $K_g,K_f$ and $I_g$ denote the indicated kernels and image. The sequence 
$$ 
0\to I_g\to X''\to X''/I_g\to 0
$$ 
being exact, so is 
$$ 
0\to F(I_g)\to F(X''), 
$$ 
and we get 
\begin{equation}\lb{kfxfx''}
\Ker\big(F(X)\to F(I_g)\big)\xr\sim\Ker\big(F(X)\to F(X'')\big). 
\end{equation} 
The sequence 
$$ 
0\to K_g\to X\to I_g\to 0
$$ 
being exact, so is 
$$ 
F(K_g)\to F(X)\to F(I_g), 
$$ 
and we get 
\begin{equation}\lb{kfxfig}
\Ima\big(F(K_g)\to F(X)\big)\xr\sim\Ker\big(F(X)\to F(I_g)\big). 
\end{equation} 
The sequence 
$$ 
0\to K_f\to X'\to K_g\to 0 
$$ 
being exact, so is 
$$ 
F(X')\to F(K_g)\to0,  
$$ 
and the isomorphism 
\begin{equation}\lb{ifkg}
\Ima\big(F(X')\to F(X)\big)\xr\sim\Ima\big(F(K_g)\to F(X)\big)  
\end{equation} 
results from Lemma~\ref{817b1} p.~\pr{817b1} with $F(X')\to F(K_g)\to F(X)$ instead of \qr{e817} p.~\pr{e817}. The exactness of \qr{fx'fxfx''} follows from \qr{kfxfx''}, \qr{kfxfig} and \qr{ifkg}.
\end{proof}

%%

\subsubsection{Exercise 8.19 p. 204}\lb{819}

Let 
$$
\begin{tikzcd} 
&0\ar[d]&0\ar[d]&0\ar[d]\\ 
0\ar[r]&Z\ar[d,"c"']\ar[r,"a"]&Y\ar[d,"d"]\ar[r,"b"]&X\ar[d,"e"]\\ 
0\ar[r]&W\ar[d,"h"']\ar[r,"f"]&V\ar[d,"i"]\ar[r,"g"]&U\\ 
&T\ar[r,"j"']&S
\end{tikzcd}
$$ 
be a commutative diagram in an abelian category. If the first two rows and the last two columns are exact, then the first column is exact.  %the row containing $a$, the row containing $f$, and the column containing $d$ are exact, then the column containing $c$ is exact. 

\begin{proof}
(In this proof we omit the composition symbols $\ci$ and most of the parenthesis, and we freely use Lemma~\ref{8311} p.~\pr{8311} and Lemma~\ref{8312} p.~\pr{8312}.)

\nn Exactness at $Z$: If $z:R\to Z$ satisfies $cz=0$, we get $daz=fcz=0$, and thus $z=0$.

\nn Exactness at $W$: Let $w:Q\to W$ satisfy $hw=0$: 
$$
\begin{tikzcd}
&Q\ar[dr,"0"]\ar[d,"w"']\\ 
Z\ar[r,"c"']&W\ar[r,"h"']&T.
\end{tikzcd}
$$ 
The equalities $ifw=jhw=0$ yield the commutative diagram 
$$
\begin{tikzcd} 
P\ar[d,"y"']\ar[r,two heads,"k"]&Q\ar[d,"fw"']\ar[dr,"0"]\\ 
Y\ar[r,"d"']&V\ar[r,"i"']&S,
\end{tikzcd}
$$ 
and the equalities $eby=gdy=gfwk=0$ and thus $by=0$ yield the commutative diagram 
$$
\begin{tikzcd} 
N\ar[d,"z"']\ar[r,two heads,"\ell"]&P\ar[d,"y"']\ar[dr,"0"]\\ 
Z\ar[r,"a"']&Y\ar[r,"b"']&X.
\end{tikzcd}
$$ 
This implies $fcz=daz=dy\ell=fwk\ell$, and thus $cz=wk\ell$: 
$$
\begin{tikzcd}
N\ar[d,"z"']\ar[r,two heads,"k\ell"]&Q\ar[dr,"0"]\ar[d,"w"']\\ 
Z\ar[r,"c"']&W\ar[r,"h"']&T.
\end{tikzcd}
$$ 
\end{proof} 

For the reader's convenience we spell out the dual statement: 

Let 
$$
\begin{tikzcd} 
&U\ar[d,"g"']\ar[r,"e"]&X\ar[d,"b"]\ar[r]&0\\ 
S\ar[d,"j"']\ar[r,"i"]&V\ar[d,"f"']\ar[r,"d"]&Y\ar[d,"a"]\ar[r]&0\\ 
T\ar[r,"h"']&W\ar[d]\ar[r,"c"']&Z\ar[d]\ar[r]&0\\ 
&0&0
\end{tikzcd}
$$ 
be a commutative diagram in an abelian category. If the last two columns and the first two rows are exact, then the last row is exact. %the column containing $f$, the column containing $a$, and the row containing $d$ are exact, then the row containing $c$ is exact. 

%%%%

\section{About Chapter 9}

I find Chapter 9 especially beautiful!

\sbs{Brief comments}

\begin{s}
P.~217, beginning of Section 9.2. 

\begin{prop}\lb{ppifil}
Let $\pi$ be an infinite cardinal. The following conditions on a small category $I$ are equivalent:

\nn{\em(a)} For any category $J$ with $\card(\Mor(J))<\pi$ and any functor 
$$
\al:I\tm J^{\op}\to\Set
$$ 
the natural map 
$$
\col_{i\in I}\lim_{j\in J}\al(i,j)\to\lim_{j\in J}\col_{i\in I}\al(i,j)
$$ 
is bijective.

\nn{\em(b)} For any category $J$ with $\card(\Mor(J))<\pi$ and any functor 
$$
\al:I\tm J^{\op}\to\Set
$$ 
the natural map 
$$
\col_{i\in I}\lim_{j\in J}\al(i,j)\to\lim_{j\in J}\col_{i\in I}\al(i,j)
$$ 
is surjective.

\nn{\em(c)} The following conditions hold:

{\em(c1)} for any $A\subset\Ob(I)$ such that $\card(A)<\pi$ there is a $j$ in $J$ such that for any $a$ in $A$ there is a morphism $a\to j$ in $I$,

{\em(c2)} for any $i$ and $j$ in $I$ and for any $B\subset\Hom_I(i,j)$ such that $\card(B)<\pi$ there is a morphism $j\to k$ in $I$ such that the composition $i\xr sj\to k$ does not depend on $s\in B$.

\nn{\em(d)} For any category $J$ such that $\card(\Mor(J))<\pi$ and any functor $\pp:J\to I$ there is an $i$ in $I$ such that $\lim\Hom_I(\pp,i)\neq\vi$. 
\end{prop}

\begin{proof}
Implications (c) $\ssi$ (d) $\then$ (a) are proved in Proposition 9.2.1 p.~217 and Proposition 9.2.9 p.~219 of the book. Implication (a)$\then$(b) is obvious. The proof of Implication (b)$\then$(d) is the same as the proof of Implication (b)$\then$(a) in Theorem 3.1.6 p.~74 of the book.
\end{proof}

\begin{df}[$\pi$-filtrant category] 
Let $\pi$ be an infinite cardinal and $I$ a category. Then $I$ is $\pi$-{\em filtrant}\index{$\pi$-filtrant} if (and only if) the equivalent conditions of Proposition~\ref{ppifil} are satisfied.
\end{df}
\end{s}

%

\begin{s}\lb{922}
P.~218. One can make the following observation after Definition 9.2.2: 

\emph{If $I$ admits inductive limits indexed by categories $J$ such that $\card(\Mor(J))<\pi$, then $I$ is $\pi$-filtrant.} 

\begin{proof}
For $\pp:J\to I$ we have
$$
\lim\Hom_\C(\pp,\col\pp)\xl\sim\Hom_\C(\col\pp,\col\pp)\neq\vi.
$$
\end{proof}
\end{s}

%

\begin{s}
P. 218, Example 9.2.3. In fact $b$ is the least element of $J\setminus A'$, which exists because $J$ is well-ordered and $J\setminus A'$ is nonempty. We get $A\subset J_b\cup\{b\}$; in particular $b$ is an upper bound for $A$. (The inclusion $A'\subset J_b$, and thus the equality $A'=J_b$, are true, but not needed.)
\end{s}

%

\begin{s}
P.~218, Lemma 9.2.5. 

\begin{lem}[Lemma 9.2.5 p.~218] 
Let $\pp:J\to I$ be a cofinal functor. If $J$ is $\pi$-filtrant, so is $I$.
\end{lem}

Clearly, $I$ satisfies conditions (a) and (b) in Proposition~\ref{ppifil} p.~\pr{ppifil}. 
\end{s}

%

\begin{s}
P.~219, proof of Remark 9.2.6. To prove that $I'$ is $\pi$-filtrant it is straightforward to check that $I'$ satisfies Conditions (c1) and (c2) in Proposition~\ref{ppifil} p.~\pr{ppifil}.
\end{s} 

%

\begin{s}
P.~220, proof of Proposition 9.2.9. We add a few details to the argument in the book. Recall the statement: 
\begin{prop}
Let $\pi$ be an infinite cardinal. Let $J$ be a category such that $\card(\Mor(J))<\pi$ and let $I$ be a small $\pi$-filtrant category. Consider a functor $\al:I\tm J^{\op}\to\Set$, $(i,j)\mt\al_{ij}$. Then the natural map $\ld$ below is bijective: 
$$
\ld:\col_i\lim_j\al_{ij}\to\lim_j\col_i\al_{ij}. 
$$ 
\end{prop} 

\begin{proof}
Let 
$$
\begin{tikzcd}
\ds\col_i\lim_j\al_{ij}\ar[rr,"\ld"]&&\ds\lim_j\col_i\al_{ij}\ar[d,"q'_j"]\\ 
\ds\lim_j\al_{ij}\ar[u,"p_i"]\ar[r,"p'_{ij}"']&\al_{ij}\ar[r,"q_{ij}"']&\ds\col_i\al_{ij}
\end{tikzcd}
$$ 
be the obvious commutative diagram. 

\nn(i) Injectivity. Let $i$ be in $I$ and let $x,y\in\lim_j\al_{ij}$ satisfy $q_{ij}\,x_j=q_{ij}\,y_j$ for all $j$. (In this proof we omit almost all parenthesis.) It suffices to prove $p_i\,x=p_i\,y$. For each $j$ there is a morphism $i\to i(j)$ in $I$ such that $\al_{i\to i(j),j}\,x_j=\al_{i\to i(j),j}\,y_j$. Since $I$ is $\pi$-filtrant and $\card(J)<\pi$, there is an $i'$ in $I$ and there are morphisms $i(j)\to i'$ in $I$ such that the composition $i\to i(j)\to i'$ does not depend on $j$. We get $\al_{i\to i',j}\,x_j=\al_{i\to i',j}\,y_j$ for all $j$, and thus $p_i\,x=p_i\,y$. 

\nn(i) Surjectivity. Let $y$ be in $\lim_j\col_i\al_{ij}$. Each $y_j$ is of the form $q_{ij}\,z_j$. \emph{A priori} $i$ depends on $j$, but it is easy to see that we can assume that $i$ independent of $j$. Let $j\to j'$ be a morphism in $J^{\op}$. We have $q_{ij'}\,z_{j'}=q_{ij'}\,\al_{i,j\to j'}\,z_j$, and there is a morphism $i\to i(j\to j')$ in $I$ such that 
$$
\al_{i\to i(j\to j'),j'}\,z_{j'}=\al_{i\to i(j\to j'),j'}\,\al_{i,j\to j'}\,z_j.
$$ 
Since $\card(\Mor(J))<\pi$, there are morphisms $i(j\to j')\to i'$ in $I$ such that the composition $i\to i(j\to j')\to i'$ does not depend on $j\to j'$. For each $j$ set $x_j:=\al_{i\to i',j}\,z_j\in\al_{i'j}$. 

We claim $x\in\lim_j\al_{ij}$. Indeed, for any morphism $j\to j'$ in $J^{\op}$ we have 
$$\al_{i,j\to j'}\,x_j=\al_{i,j\to j'}\,\al_{i\to i',j}\,z_j=\al_{i(j\to j')\to i',j'}\,\al_{i\to i(j\to j')\to i',j'}\,\al_{i,j\to j'}\,z_j
$$ 
$$
=\al_{i(j\to j')\to i',j'}\,\al_{i\to i(j\to j'),j'}\,z_{j'}=\al_{i\to i',j'}\,z_{j'}=\,x_{j'}.
$$ 

We claim $\ld\,p_i\,x=y$. Let $j$ be in $J$. It suffices to show $q_{i'j}\,x_j=q_{ij}\,z_j$. We have $q_{i'j}\,x_j=q_{i'j}\,\al_{i\to i',j}\,z_j=q_{ij}\,z_j$. 
\end{proof} 
\end{s} 

%

\begin{s}
For the reader's convenience we state and prove Proposition 9.2.10 p. 220. 

\begin{prop}[Proposition 9.2.10 p. 220] 
If $\C$ admits small $\pi$-filtrant inductive limits, if $J$ is a category satisfying $\card(\Mor(J))<\pi$, if $\bt:J\to\C_\pi$ is a functor, and if $\col\bt$ exists in $\C$, then it belongs to $\C_\pi$. 
\end{prop}

\begin{proof}
Let $\al:I\to\C$ be a functor with $I$ small and $\pi$-filtrant, and consider the commutative diagram
$$
\begin{tikzcd}
\col_i\Hom_\C(\col_j\bt(j),\al(i))\ar{r}{a}\ar{d}{b}[swap]{\sim}&\Hom_\C(\col_j\bt(j),\col\al)\ar{dd}{e}[swap]{\sim}\\ 
\col_i\lim_j\Hom_\C(\bt(j),\al(i))\ar{d}{c}[swap]{\sim}\\ 
\lim_j\col_i\Hom_\C(\bt(j),\al(i))\ar{r}{d}[swap]{\sim}&\lim_j\Hom_\C(\bt(j),\col\al).
\end{tikzcd}
$$ 
The maps $b$ and $e$ are bijective for obvious reasons. The map $c$ is bijective because of our assumptions on $I$ and $J$. The map $d$ is bijective because $\bt(j)$ is in $\C_\pi$ for all $j$. Thus, the map $a$ is bijective. 
\end{proof}
\end{s}

%

\begin{s} 
P.~220, proof of Corollary 9.2.11. 

\begin{cor}[Corollary 9.2.11 p.~220] 
If $\C$ admits small inductive limits and if $X$ is an object of $\C$, then $\C_\pi$ and $(\C_\pi)_X$ are $\pi$-filtrant. 
\end{cor}

This follows from \S\ref{922}. Note that it suffices to assume that $\C$ admits inductive limits indexed by categories $J$ such that $\card(\Mor(J))<\pi$. (For the case of $(\C_\pi)_X$, see Lemma~\ref{bcl} p.~\pr{bcl} above.) %2.1.12 p.~41 in the book.)
\end{s}

%

%\begin{s} P.~221, Lemma 9.2.15. In the first sentence of the proof it is written ``$\C_A$ is cofinally small by Proposition 6.1.5 (see Remark 9.2.6).'' More simply, $\C_A$ is cofinally small because the identity of $A$ is a terminal object of $\C_A$ (recall that $A$ is in $\C$).\end{s} 

%

\begin{s} 
P.~222, Proposition 9.2.17, proof of implication (ii)$\then$(i). I suspect that the argument of the book is better than the one given here, but, unfortunately, I don't understand it. Here is a less concise wording:

Recall the setting: $\C$ is a category admitting inductive limits indexed by any category $J$ such that $\card(\Mor(J))<\pi$, and $A$ is in $\Ind(\C)$. Conditions (i) and (ii) are as follows: 

\nn(i) $\C_A$ is $\pi$-filtrant, 

\nn(ii) for any category $J$ such that $\card(\Mor(J))<\pi$ and any functor $\pp:J\to\C$, the natural map $A(\col\pp)\to\lim A(\pp)$ is surjective. 

To prove (ii)$\then$(i), let $J$ be a category satisfying 
$$
\card(\Mor(J))<\pi,
$$ 
and let $\psi:J\to\C_A$ be a functor. It suffices find a $\xi$ in $\C_A$ satisfying 
$$
\lim\Hom_{\C_A}(\psi,\xi)\neq\vi
$$ 
(see Condition~(d) in Proposition~\ref{ppifil} p.~\pr{ppifil}). Let $\pp:J\to\C$ be the composition of $\psi$ with the forgetful functor $\C_A\to\C$, and write 
$$
\psi(j)=\left(\pp(j),\pp(j)\xr{y_j}A\right)\in\C_A.
$$ 
In particular the family $(y_j)$ belongs to $\lim A(\pp)$. Our assumption about $\C$ implies that $\col\pp$ exists in $\C$. Let $p_j:\pp(j)\to\col\pp$ be the coprojection. By surjectivity of the map $A(\col\pp)\to\lim A(\pp)$ in (ii), there is an $x:\col\pp\to A$ such that $x\ci p_j=y_j$ for all $j$. Setting 
$$
\xi:=\left(\col\pp,\col\pp\xr x A\right)\in\C_A,
$$ 
and letting $f_j:\psi(j)\to\xi$ be the obvious morphism, we get $(f_j)\in\lim\Hom_{\C_A}(\psi,\xi)$. q.e.d. 
\end{s}

%%

\sbs{Section 9.3 pp 223--228}\lb{s934}

Here is a slightly different wording. 

\subsubsection{Conditions (9.3.1) p. 223}\lb{931}

Recall Conditions (9.3.1) of the book: $\C$ is a category satisfying  

(i) $\C$ admits small inductive limits,

(ii) $\C$ admits finite projective limits,

(iii) small filtrant inductive limits are exact, 

(iv) there exists a generator $G$,

(v) epimorphisms are strict.

\subsubsection{Summary of Section 9.3}

The main purpose of Section 9.3 of the book is to prove Corollaries 9.3.7 and 9.3.8 p.~228 of the book, and these corollaries could be stated immediately after Conditions (9.3.1) above. For the reader's convenience we recall the definition of a regular cardinal and state Corollary 9.3.7:

\begin{df}[regular cardinal]\lb{rc} 
A cardinal $\pi$ is \emph{regular}\index{regular cardinal} if for any family of sets $(B_i)_{i\in I}$ we have 
$$
\card(I)<\pi,\quad\card(B_i)<\pi\ \forall\ i\quad\then\quad\card\left(\bigsqcup_iB_i\right)<\pi.
$$
\end{df}

\begin{cor}[Corollary 9.3.7 p. 228]
Assume (9.3.1). Then for any small subset $S$ of $\Ob(\C)$ there exists an infinite cardinal $\pi$ such that $S\subset\Ob(\C_\pi)$.
\end{cor}

We make a few comments about Corollary 9.3.8. Firstly, it would be simpler (I think) to replace $\SSS$ with $\C_\pi$ in the statement, since in the first sentence of the proof one sets $\SSS:=\C_\pi$. Secondly, in view of the way Theorem 9.6.1 p.~235 of the book is phrased, it would be better, even if it is a repetition, to incorporate Part~(iv) of Corollary 9.3.5 (which says that $\C_\pi$ is closed by finite projective limits) into Corollary 9.3.8. Then, Corollary 9.3.8 would read as follows:

\begin{cor}[Corollary 9.3.8 p.~227]\lb{938}
Assume (9.3.1) and let $\kappa$ be a cardinal. Then there exists an infinite regular cardinal $\pi>\kappa$ such that 

\nn{\em(i)} $\C_\pi$ is essentially small,

\nn{\em(ii)} if $X\epi Y$ is an epimorphism and $X$ is in $\C_\pi$, then $Y$ is in $\C_\pi$,

\nn{\em(iii)} if $X\mono Y$ is a monomorphism and $Y$ in $\C_\pi$, then $X$ is in $\C_\pi$,

\nn{\em(iv)} $G$ is in $\C_\pi$,

\nn{\em(v)} for any epimorphism $f:X\epi Y$ in $\C$ with $Y$ in $\C_\pi$, there exists $Z$ in $\C_\pi$ and a monomorphism $g:Z\mono X$ such that $f\ci g:Z\to Y$ is an epimorphism,

\nn{\em(vi)} $\C_\pi$ is closed by inductive limits indexed by categories $J$ which satisfy $$\card(\Mor(J))<\pi,$$

\nn{\em(vii)} $\C_\pi$ is closed by finite projective limits.
\end{cor}

See also Theorem \ref{961} p. \pr{961} below.

%

\subsubsection{Lemma 9.3.1 p. 224}

For the reader's convenience we state the lemma:

\begin{lem}[Lemma 9.3.1 p. 224]\lb{l331}
Assume that Conditions (9.3.1) p.~223 of the book (see \S\ref{931} p.~\pr{931}) hold, let $\pi$ be an infinite regular cardinal, let $I$ be a $\pi$-filtrant small category, let $\al:I\to\C$ be a functor, and let $\col\al\to Y$ be an epimorphism in $\C$. Assume either $\card(Y(G))<\pi$ or $Y\in\C_\pi$. Then there is an $i_0$ in $I$ such that the obvious morphism $\al(i_0)\to Y$ is an epimorphism.
\end{lem}

The proof of Lemma~\ref{l331} in the book uses twice the following lemma:

\begin{lem}\lb{ppi} 
Let $\C$ be a category, let $\pi$ be an infinite cardinal, and let $\al:I\to\C$ be a functor admitting an inductive limit $X$ in $\C$. Assume that the coprojections $p_i:\al(i)\to X$ are monomorphisms, and consider the conditions below:

\nn{\em(a)} $I$ is $\pi$-filtrant and $X$ is $\pi$-accessible,

\nn{\em(b)} the identity of $X$ factors through the coprojection $p_i$ for some $i$,

\nn{\em(c)} the coprojection $p_i$ is an isomorphism for some $i$,

\nn{\em(d)} there is an $i$ in $I$ such that $\al(s):\al(i)\to\al(j)$ is an isomorphism for all morphism $s:i\to j$ in $I$.

\nn Then we have {\em(a)} $\then$ {\em(b)} $\ssi$ {\em(c)} $\then$ {\em(d)}. 
\end{lem}

\begin{proof}
This follows immediately from Exercise 1.7 p. 31 of the book.
\end{proof}

%We give a slightly more detailed writing of the second sentence in Step~(a) of the proof of Lemma 9.3.1 p.~224 of the book. This second sentence is 

%``Set $S:=\col_iY_i(G)\subset Y(G)$.'' 

%Here is the rewriting: 

%The coprojection $Y_i\to Y$ being a monomorphism, so is $Y_i(G)\to Y(G)$. As small filtrant inductive limits are exact in $\Set$ (Proposition 3.3.7 (iv) p. 83 of the book), $S:=\col_iY_i(G)\to Y(G)$ is also a monomorphism.

We add a few details to the beginning of the proof of Lemma 9.3.1. 

Set $X_i=\al(i)$ and $Y_i=\Ima(X_i\to Y)=\Ker(Y\parar Y\sqcup_{X_i}Y)$. In particular the natural morphism $Y_i\to Y$ is a monomorphism. Since small filtrant inductive limits are exact, 
\begin{equation}\lb{col_iY_i}
\col_iY_i\xr\sim\Ker(Y\parar Y\sqcup_{\col_iX_i}Y)\xr\sim\Ima(\col_iX_i\to Y)\xr\sim Y,
\end{equation} 
where the last isomorphism follows from the hypothesis that $\col_iX_i\to Y$ is an epimorphism together with Proposition 5.1.2 (iv). It is easy to see that the chain of isomorphisms \qr{col_iY_i} coincides with the natural morphism $\col_iY_i\to Y$, and that the coprojections $Y_i\to Y$ are monomorphisms. In particular the maps $Y_i(G)\to Y(G)$ are injective by Proposition~\ref{333} p.~\pr{333} and Proposition~\ref{34i} p.~\pr{34i}. This is easily seen to imply that $\col_iY_i(G)\to Y(G)$ is also injective. 
%\begin{uspb}The book claims that $\col_iY_i(G)\to Y(G)$ is an injection, but I'm unable to prove it.\index{unsolved problem}\end{uspb}
%Small filtrant inductive limits being exact in $\Set$ (Proposition 3.3.7 (iv) p. 83 of the book), $S:=\col_iY_i(G)\to Y(G)$ is also a monomorphism. %the $Y_i(G)$, which can be viewed as subsets of $Y(G)$ by Proposition~\ref{333} p.~\pr{333} and Proposition~\ref{34i} p.~\pr{34i}, satisfy $$\bigcup_i\ Y_i(G)=\col_iY_i(G)\xr\sim Y(G),\quad Y(G)=\bigcup_i\ Y_i(G).$$

\nn(a) Assume that $\card(Y(G))<\pi$. %Set $S=\col_iY_i(G)\subset Y(G)$. Then $\card(S)\le \card(Y(G))<\pi$. 
Then $\card(Y(G))<\pi$. By Corollary 9.2.12, $Y(G)\in\Set_\pi$ and this implies 
$$
\col_{i\in I}\Hom_\Set(Y(G),Y_i(G))\xr\sim\Hom_\Set(Y(G),Y(G)).
$$ 
Hence, there exist $i_0$ and a map $Y(G)\to Y_{i_0}(G)$ such that the composition $Y(G)\to Y_{i_0}(G)\mono Y(G)$ is the identity. Therefore $Y_{i_0}(G)=Y(G)$ and hence, $Y_{i_0}(G)\to Y_i (G)$ is bijective for any $i_0\to i$ by Lemma~\ref{ppi}. Hence $Y_{i_0}\to Y_i$ is an isomorphism, which implies, again by Lemma~\ref{ppi}, that $Y_{i_0}\to Y$ is an isomorphism. Applying Proposition 5.1.2 (iv), we find that $X_{i_0}\to Y$ is an epimorphism. 

%

\subsubsection{Proposition 9.3.2 p. 224}

\begin{prop}[Proposition 9.3.2 p.~224]\lb{932} 
Let $\C$ be a category satisfying Conditions (9.3.1) of the book, conditions stated in Section~\ref{931} p.~\pr{931} above. If $\pi$ is an infinite regular cardinal, if $A$ is in $\C$, and if 
$$
\card(A(G))<\pi,\quad\card\big(G^{\sqcup A(G)}(G)\big)<\pi,
$$ 
then $A$ is in $\C_\pi$.
\end{prop}

Here is a rewriting of the proof with a few more details:

\begin{proof}[Proof of Proposition~\ref{932}]${}$ 

\nn$\bu$ Step 1. Note that $\Set\ni S\mt G^{\sqcup S}\in\C$ is a well-defined covariant functor. Also note that $\card(G^{\sqcup S}(G))<\pi$ for any $S\subset A(G)$. Indeed, there are maps 
$$
S\to A(G)\to S
$$ 
whose composition is the identity. Hence, the composition 
$$
G^{\sqcup S}(G)\to G^{\sqcup A(G)}(G)\to G^{\sqcup S}(G)
$$ 
is the identity.

\nn$\bu$ Step 2. Let $I$ be a small $\pi$-filtrant category, let $(X_i)_{i\in I}$ be an inductive system in $\C$, and let $X$ be its inductive limit. Claim~\ref{lbij} below will imply Proposition~\ref{932}. 

\begin{claim}\lb{lbij} 
The map 
$$
\ld_A:\col_{i\in I}\Hom_\C(A,X_i)\to\Hom_\C(A,X).
$$ 
is bijective. 
\end{claim}

\begin{claim}\lb{linj} 
The map $\ld_A$ is injective. 
\end{claim} 

\begin{proof}[Proof of Claim~\ref{linj}] 
(We shall only use $\card(A(G))<\pi$.) Suppose that $f,g:A\parar X_{i_0}$ have same image in $\Hom_\C(A,X)$. This just means that the two compositions 
$$
A\parar X_{i_0}\to X
$$ 
coincide. We must show that $f$ and $g$ have already same image in 
$$
\col_{i\in I}\Hom_\C(A,X_i),
$$ 
that is, we must show that there is a morphism $s_1:i_0\to i_1$ in $I$ such that the two compositions $A\parar X_{i_0}\to X_{i_1}$ coincide. For each $s:i_0\to i$, set 
$$
N_s:=\Ker(A\parar X_i).
$$ 
By Corollary 3.2.3 (i) p.~79 of the book, $I^{i_0}$ is filtrant and the forgetful functor $I^{i_0}\to I$ is cofinal. One of our assumptions, namely Condition (9.3.1) (iii) in Section~\ref{931} p.~\pr{931}, says that small filtrant inductive limits are exact in $\C$. In particular, $\col_{s\in I^{i_0}}$ is exact in $\C$, and we get  
$$
\col_{s\in I^{i_0}}N_s\iso\Ker\left(A\parar\col_{s\in I^{i_0}}X_i\right)\iso\Ker(A\parar X)\iso A. 
$$ 
As $\card(A(G))<\pi$ by assumption, Lemma~\ref{l331} p.~\pr{l331} implies that there is a morphism $s_1:i_0\to i_1$ in $I$ such that $N_s\to A$ is an epimorphism. Hence, the two compositions $A\parar X_{i_0}\to X_{i_1}$ coincide, as was to be shown. This proves Claim~\ref{linj}. 
\end{proof} 

It only remains, in order to prove Proposition~\ref{932}, to check that $\ld_A$ is surjective. Let $f:A\to X$ be a morphism. Claim~\ref{pig=f} below will imply the surjectivity of $\ld_A$, and thus the truth of Proposition~\ref{932}. 

\begin{claim}\lb{pig=f} 
There is a morphism $g:A\to X_i$ such that $p_i\ci g=f$, where $p_i:X_i\to X$ is the coprojection.
\end{claim} 

\nn$\bu$ Step 3. Consider the following conditions: 

\nn(a) the diagram of solid arrows 
$$
\begin{tikzcd}
B\ar[dashed,two heads]{r}\ar[dashed]{d}&A\ar{d}{f}\\ 
X_{i_0}\ar{r}[swap]{p_{i_0}}&X
\end{tikzcd}
$$
can be completed to a commutative diagram as indicated (the morphism $B\to A$ being an epimorphism),

\nn(b) the diagram of solid arrows 
\begin{equation}\lb{fa}
\begin{tikzcd}
C\ar[dashed,two heads]{r}{a}\ar[dashed]{d}[swap]{x}&A\ar{d}{f}\\ 
X_{i_0}\ar{r}[swap]{p_{i_0}}&X
\end{tikzcd}
\end{equation} 
can be completed to a commutative diagram as indicated, with $\card(C(G))<\pi$ (the morphism $C\to A$ being an epimorphism).

We shall show that (a) holds, that (a) implies (b), and that (b) implies Claim \ref{pig=f}, and thus Proposition~\ref{932} p.~\pr{932}.% Claim~\ref{pig=f} creates an overful

\nn$\bu$ Step 4: (a) holds. For each $i$ in $I$ define $Y_i:=A\tm_XX_i$. As $\col_i$ is exact in $\C$, we have $\col_iY_i\iso A$. As $\card(A(G))<\pi$, Lemma 9.3.1 p.~224 of the book (stated above as Lemma~\ref{l331} p.~\pr{l331}) implies that there is an $i_0$ in $I$ such that $B:=Y_{i_0}\to A$ is an epimorphism.

\nn$\bu$ Step 5: (a) implies (b). Assuming (a), we build the commutative square 
%
\begin{equation}\lb{s5a}
\begin{tikzcd}
B\ar[two heads]{r}\ar{d}&A\ar{d}{f}\\ 
X_{i_0}\ar{r}[swap]{p_{i_0}}&X,
\end{tikzcd}
\end{equation}
% 
and we put $S:=\Ima(B(G)\to A(G))\subset A(G)$ and $C:=G^{\sqcup S}$, so that we have maps $B(G)\to S\to A(G)$. By Step~1 this implies $\card(C(G))<\pi$. The vertical arrows of the commutative diagram 
\begin{equation}\lb{s5b}
\begin{tikzcd}
G^{\sqcup B(G)}\ar{r}\ar{d}&C\ar{r}&G^{\sqcup A(G)}\ar{d}\\ 
B\ar[two heads]{rr}&&A
\end{tikzcd}
\end{equation} 
being epimorphisms by Proposition 5.2.3 (iv) p.~118 of the book, so is $C\to A$. From the commutative diagram 
$$
\begin{tikzcd}
S\ar[equal]{rr}\ar[equal]{d}&&S\ar[equal]{d}\\ 
S\ar{r}&B(G)\ar{r}&S,
\end{tikzcd}
$$ 
we get, by Step 1, the commutative diagram 
\begin{equation}\lb{s5c}
\begin{tikzcd}
C\ar[equal]{rr}\ar[equal]{d}&&C\ar[equal]{d}\\ 
C\ar{r}&G^{\sqcup B(G)}\ar{r}&C.
\end{tikzcd}
\end{equation} 
Splicing \qr{s5a}, \qr{s5b} and \qr{s5c} gives 
$$
\begin{tikzcd}
C\ar[equal]{rr}\ar[equal]{d}&&C\ar[equal]{d}\\ 
C\ar{r}\ar{rdd}[swap]{x}&G^{\sqcup B(G)}\ar{r}\ar{d}&C\ar[two heads]{d}{a}\\ 
{}&B\ar{r}\ar{d}&A\ar{d}\\ 
{}&X_{i_0}\ar{r}[swap]{p_{i_0}}&X.
\end{tikzcd}
$$ 
This proves (b).

\nn$\bu$ Step 6: (b) implies Claim~\ref{pig=f} p.~\pr{pig=f}, and thus Proposition~\ref{932} p.~\pr{932}. Assuming (b), form the cartesian square 
$$
\begin{tikzcd}
P\ar{r}\ar{d}&C\ar[two heads]{d}{a}\\ 
C\ar[two heads]{r}[swap]{a}&A.
\end{tikzcd}
$$
Epimorphisms in $\C$ being strict, the sequence $P\parar C\xr aA$ is exact. As 
$$
P(G)\le\card(C(G))^2<\pi,
$$ 
Claim~\ref{linj} implies that the natural map 
$$
\ld_P:\col_{i\in I}\Hom_\C(P,X_i)\to\Hom_\C(P,X)
$$ 
is injective. Consider the commutative diagram 
$$
\begin{tikzcd}
P\ar[yshift=0.7ex]{r}\ar[yshift=-0.7ex]{r}&C\ar{r}{a}\ar{d}[swap]{x}&A\ar{d}{f}\\ 
{}&X_{i_0}\ar{r}[swap]{p_{i_0}}&X.
\end{tikzcd}
$$ 
As $\ld_P$ is injective, and as the compositions $P\parar C\xr xX_{i_0}\xr{p_{i_0}}X$ are equal, there is a morphism $s:i_0\to i$ such that the compositions $P\parar C\xr xX_{i_0}\xr{X_s} X_i$ are equal. The exactness of $P\parar C\xr aA$ implies the existence of a morphism $g:A\to X_i$ such that 
%
\begin{equation}\lb{ga}
X_s\ci x=g\ci a.
\end{equation} 
% 

\begin{proof}[Proof of Claim~\ref{pig=f}]
It suffices to show that the above morphism $g$ satisfies $f=p_i\ci g$. Consider the diagram 
$$
\begin{tikzcd}
P\ar[yshift=0.7ex]{r}\ar[yshift=-0.7ex]{r}&C\ar{d}[swap]{x}\ar[two heads]{rr}{a}&&A\ar{dl}[swap]{g}\ar{d}{f}\\ 
{}&X_{i_0}\ar{r}[swap]{X_s}\ar[equal]{d}&X_i\ar{r}[swap]{p_i}&X\ar[equal]{d}\\ 
{}&X_{i_0}\ar{rr}[swap]{p_{i_0}}&&X.
\end{tikzcd}
$$ 
We have 
\begin{align*}
f\ci a&=p_{i_0}\ci x&\text{by \qr{fa} p.~\pr{fa}}\\ 
&=p_i\ci X_s\ci x\\ 
&=p_i\ci g\ci a&\text{by \qr{ga} p.~\pr{ga}}.
\end{align*} 
As $a$ is an epimorphism, this forces $f=p_i\ci g$, and the proof of Claim~\ref{pig=f} is complete. 
\end{proof} 

As already indicated, Claim~\ref{pig=f} implies Proposition~\ref{932} p.~\pr{932}. 
\end{proof}  

%%

\subsubsection{Definition of two infinite regular cardinals}\lb{tirg}

(See (9.3.4) p. 226 of the book. We shall modify slightly the definition of $\pi_1$.) Let $\C$ be a category satisfying Conditions (9.3.1) in Section~\ref{931} p.~\pr{931} above. Let $\pi_0$ be an infinite regular cardinal such that 
$$
\card\big(G(G)\big)<\pi_0,\quad\card\big(G^{\sqcup G(G)}(G)\big)<\pi_0.
$$ 
Now choose a cardinal $\pi_1\ge\pi_0$ such that we have for all set $A$ with $\card(A)<\pi_0$: 

$\card\big(G^{\sqcup A}(G)\big)<\pi_1$, 

if $X$ is a quotient of $G^{\sqcup A}$, then $\card\big(X(G)\big)<\pi_1$. 

\nn(Since the set of quotients of $G^{\sqcup A}$ is small by Proposition 5.2.9 p.~121 of the book, such a cardinal $\pi_1$ exists.) In the sequel of Section~\ref{s934} we assume 

\begin{cond}\lb{ass}
Conditions (i)--(v) of Section~\ref{931} p.~\pr{931} hold; $\pi_0$ and $\pi_1$ are as above; and $\pi$ is the successor of $2^{\pi_1}$.
\end{cond}

\nn The cardinals $\pi$ and $\pi_0$ satisfy 

(a) $\pi$ and $\pi_0$ are infinite regular cardinals,

(b) $G$ is in $\C_{\pi_0}$,

(c) $\pi'^{\pi_0}<\pi$ for any $\pi'<\pi$, 

(d) if $X$ is a quotient of $G^{\sqcup A}$ with $\card(A)<\pi_0$, then $\card\big(X(G)\big)<\pi$, 

(e) if $A$ is a set with $\card(A)<\pi_0$, then $\card\big(G^{\sqcup A}(G)\big)<\pi$.

\nn Condition~(a) holds because $\pi_0$ is infinite regular by assumption, and $\pi$ is infinite regular by Statement~(iv) p.~217 of the book. Condition~(b) holds by Proposition \ref{932} p.~\pr{932}. Condition~(c) is proved as follows: if $\pi'<\pi$, then $\pi'\le2^{\pi_1}$ and 
$$
\pi'^{\pi_0}\le(2^{\pi_1})^{\pi_0}=2^{\pi_0\pi_1}=2^{\pi_1}<\pi.
$$ 
Conditions (d) and (e) are clear. 

\subsubsection{Lemma 9.3.3 p. 226}

We state Lemma 9.3.3 for the reader's convenience:

\begin{lem}[Lemma 9.3.3 p.~226]\lb{933}
If Condition \ref{ass} holds, if $A$ is a set of cardinal $<\pi$, and if $X$ is a quotient of $G^{\sqcup A}$, then $\card(X(G))<\pi$.
\end{lem}

The beginning of the proof of Lemma 9.3.3 in the book uses implicitly the following two lemmas, which we prove for the sake of completeness. 

\begin{lem}\lb{ord}
If $\al$ is a cardinal, then the cardinal of the set of those cardinals $\bt$ such that $\bt<\al$ does not exceed $\al$.
\end{lem} 

\begin{lem}\lb{cardipi}
If $\pi_0,\pi$ and $A$ are as above, and if $I:=\{B\subset A\,|\,\card(B)<\pi_0\}$, then we have $\card(I)<\pi$. 
\end{lem} 

\begin{proof}[Proof of Lemma~\ref{ord}]
Recall that a subset $S$ of an ordered set $X$ is a {\em segment} if $x<s\in S$ with $x\in X$ implies $x\in S$. In particular $X_{<x}$ (obvious notation) is a segment of $X$ for any $x$ in $X$. We take for granted the following well-known facts:

\nn$\bu$ every set can be well-ordered,

\nn$\bu$ if $T$ is a set of two non-isomorphic well-ordered sets, then there is a unique triple $(W_1,W_2,S)$ such that $T=\{W_1,W_2\}$ and $S$ is a proper segment of $W_2$ isomorphic to $W_1$,

\nn$\bu$ if $W$ is a well-ordered set, then the assignment $w\mt W_{<w}$ is an isomorphism of well-ordered sets from $W$ onto the set of proper segments of $W$.

Let $A$ be a well-ordered set of cardinal $\al$, and, for each cardinal $\bt$ with $\bt<\al$, let $B$ be a well-ordered set of cardinal $\bt$. Then $B$ is isomorphic to $A_{<a}$ for a unique $a$ in $A$, and the map $\bt\mt a$ is injective.
\end{proof} 

\begin{proof}[Proof of Lemma~\ref{cardipi}] 
Putting $\al:=\card(A)$ we have
$$
\card(I)=\sum_{\pi'<\pi_0}\ \binom{\al}{\pi'}\le\sum_{\pi'<\pi_0}\ \al^{\pi_0}<\pi,
$$ 
the last inequality following from Lemma~\ref{ord}, (c) and (a). 
\end{proof} 

\subsubsection{Theorem 9.3.4 p. 227}

\begin{thm}[Theorem 9.3.4 p.~227]\lb{934}
Assume Condition~\ref{ass} p.~\pr{ass} holds and let $X$ be an object of $\C$. Then we have 
$$
X\in\C_\pi\ssi\card(X(G))<\pi.
$$ 
\end{thm}

\begin{proof}[Proof of Theorem \ref{934}]${}$

\nn$\then:$ We prove this implication as in the book. For the reader's convenience we reproduce the argument: Set $I:=\{A\subset X(G)\ |\ \card(A)<\pi\}$. By Example 9.2.4 p.~218 of the book, $I$ is $\pi$-filtrant. We get the morphisms 
$$
G^{\sqcup A}\to G^{\sqcup X(G)}\to X
$$ 
for $A$ in $I$, and 
$$
\col_{A\in I}G^{\sqcup A}\xr\sim G^{\sqcup X(G)}\to X.
$$ 
Then we see that $G^{\sqcup X(G)}\to X$ is an epimorphism by Proposition 5.2.3 (iv) p.~118 of the book, that $G^{\sqcup A}\to X$ is an epimorphism for some $A$ in $I$ by Lemma~\ref{l331} p.~\pr{l331}, and that $\card(X(G))<\pi$ by Lemma~\ref{933} p.~\pr{933}.

\nn$\si:$ In view of Proposition~\ref{932} p.~\pr{932}, it suffices to prove  

\begin{equation}\lb{934b}
\card\big(G^{\sqcup X(G)}(G)\big)<\pi.
\end{equation} 

To verify this inequality, we argue as in the proof of Lemma 9.3.3 p.~226 of the book (stated on p.~\pr{933} above as Lemma~\ref{933}). (Conditions (b), (c) and (e) referred to below are stated in Section~\ref{tirg} p.~\pr{tirg}.)

Let $I$ be the ordered set of all subsets of $X(G)$ of cardinal $<\pi_0$. Then $I$ is $\pi_0$-filtrant by Example 9.2.4 p.~218 of the book, and we have 
$$
G^{\sqcup X(G)}\iso\col_{B\in I}G^{\sqcup B}.
$$ 
As $G$ is $\pi_0$-accessible by (b), we get 
$$
G^{\sqcup X(G)}(G)\iso\col_{B\in I}\ G^{\sqcup B}(G).
$$ 
By Lemma~\ref{cardipi} p.~\pr{cardipi} we have $\card(I)<\pi$. Since $\card(G^{\sqcup B}(G))<\pi$ for all $B$ in $I$ by (e), this implies \qr{934b}.
\end{proof}

\subsubsection{Brief comments}

\nn$*$ P.~227, Corollary 9.3.5. In the proof of (i) we use Propositions 5.2.3 (iv) p.~118 and 5.2.9 p.~121 of the book. As already pointed out, in the proof of (iv), $\C$ should be $\C_\pi$. 

\nn$*$ P.~228, Corollary 9.3.6. As already pointed out, $\ilim$ in the statement should be $\sigma_\pi$. As for the proof, Conditions (i), (ii) and (iii) of Proposition 9.2.19 p.~223 of the book follow respectively from (9.3.1) (i) (see (i) at the beginning of Section~\ref{s934} p.~\pr{s934}), (9.3.4) (b) (see (b) right after Condition~\ref{ass} p.~\pr{ass}), and Corollary 9.3.5 (i) p.~227 of the book. 

\nn$*$ P.~228, Corollary 9.3.7. As $\{\card(X(G))\,|\,X\in S\}$ is a small set of cardinals, we may assume in Condition~\ref{ass} p.~\pr{ass} that we have $\pi>\card(X(G))$ for all $X$ in $S$, and apply Theorem~\ref{934} p.~\pr{934}. 

\nn$*$ P.~228, Corollary 9.3.8. The proof uses implicitly Proposition 5.2.3 (iv) p.~118 of the book and Example 9.2.4 p.~218 of the book. 

%%%

\sbs{Quasi-Terminal Object Theorem \index{Quasi-Terminal Object Theorem}}

Recall the following result:

\begin{thm}[Zorn's Lemma]\lb{zorn}
If $X$ is an ordered set such that each well-ordered subset of $X$ has an upper bound, then $X$ has a maximal element.
\end{thm}

The purpose of this section is to prove a common generalization of Theorem~\ref{zorn} above and of Theorem 9.4.2 p.~229 of the book, stated below as Theorem~\ref{942}. We start with a reminder:

\begin{df}[Definition 9.4.1 p.~228, quasi-terminal object] 
An object $X$ of a category $\C$ is {\em quasi-terminal}\index{quasi-terminal object} if any morphism $u:X\to Y$ admits a left inverse.
\end{df}

\begin{thm}[Theorem 9.4.2 p.~229]\lb{942} 
Any essentially small nonempty category admitting small filtrant inductive limits has a quasi-terminal object.
\end{thm}

Here is a weakening of the notion of inductive limit:

\begin{df}[small well-ordered upper bounds] 
Let $I$ be a nonempty well-ordered small set and $\al:I\to\C$ a functor. An {\em upper bound} for $\al$ is a morphism of functors $a:\al\to\DT X$ (see Notation~\ref{diag} p.~\pr{diag}). If $\C$ has the property that any such functor admits some upper bound, we say that $\C$ {\em admits small well-ordered upper bounds}\index{small well-ordered upper bounds}. 
\end{df}

\begin{df}[special well-ordered small set] 
Let $\C$ be a category. A nonempty well-ordered small set $I$ is $\C$-{\em special}\index{special well-ordered small set} if it has no largest element and if, for any functor $\al:I\to\C$, there is some upper bound $(a_i:\al(i)\to X)_{i\in I}$ and some element $i_0$ in $I$ such that $a_{i_0}$ is an epimorphism. 
\end{df}

Our goal is to prove:

\begin{thm}[Quasi-Terminal Object Theorem]\lb{qtot}
If $\C$ is a nonempty essentially small category $\C$ admitting small well-ordered upper bounds and a $\C$-special well-ordered set, then $\C$ has a quasi-terminal object.
\end{thm}

Theorem~\ref{qtot} clearly implies Zorn's Lemma (Theorem~\ref{zorn}). Lemma~\ref{945} below will show that Theorem~\ref{942} follows also from Theorem~\ref{qtot}. Theorem~\ref{942} will be used in the book to prove Theorem 9.5.5 p.~233.

The proof of Theorem~\ref{qtot} is essentially the same as the proof of Theorem~\ref{942} given in the book. For the reader's convenience, we spell out the details. 

\begin{lem}\lb{943}
If $\C$ is a nonempty small category admitting small well-ordered upper bounds, then there is an $X$ in $\C$ such that, for all morphism $X\to Y$, there is a morphism $Y\to X$.
\end{lem}

\begin{proof}
Let $\F$ be the set of well-ordered subcategories of $\C$. For $I$ and $J$ in $\F$ we decree that $I\le J$ if and only if $I$ is an initial segment of $J$. This order is clearly inductive. Let $\SSS$ be a maximal element of $\F$. As $\SSS$ is small, it admits an upper bound $(a_S:S\to X)_{S\in\SSS}$. 

We shall prove that $X$ satisfies the conditions in the statement. Let $u:X\to Y$ be a morphism in $\C$. 

\nn(i) The object $Y$ is in $\SSS$. Otherwise, we can form the well-ordered subcategory $\widetilde\SSS$ of $\C$ by appending the element $Y$ to $\SSS$ and making it the largest element of $\widetilde\SSS$, the morphism $S\to Y$ being $u\ci a_S$. We have $\widetilde\SSS\in\F$ and $\SSS<\widetilde\SSS$, contradicting the maximality of $\SSS$. 

\nn(ii) As $Y$ is in $\SSS$, there is a morphism $Y\to X$, namely $a_Y$.
\end{proof} 

\begin{df}[Property~$(P)$]\lb{pofa} 
We say that a morphism $a:A\to B$ in a given category has \emph{Property}~$(P)$ if for any morphism $b:B\to C$ there is a morphism $c:C\to B$ satisfying $c\ci b\ci a=a$.  
\end{df}  

\begin{lem}[Sublemma 9.4.4 p. 229]\lb{944} 
If $\C$ is a small nonempty category admitting small well-ordered upper bounds, and if $X$ is an object of $\C$, then there is a morphism $f:X\to Y$ having Property~$(P)$.
\end{lem}

\begin{proof}
The category $\C^X$ is again nonempty, small, and admits small well-ordered upper bounds, so that Lemma~\ref{943} applies to it. Let $f:X\to Y$ be to $\C^X$ what $X$ is to $\C$ in Lemma~\ref{943}. Then it is easy to see that $f$ has Property~$(P)$. 
\end{proof}

We recall the notion of {\em construction by transfinite induction\index{transfinite induction}}. 

\begin{thm}[Construction by Transfinite Induction]\lb{meta} 
Let $\U$ be a universe, let $F:\U\to\U$ be a map, and let $I$ be a well-ordered $\U$-set. Then there is a unique pair $(S,f)$ such that $S$ is a $\U$-set, $f:I\to S$ is a surjection, and we have 
$$
f(i)=F(f(j)_{j<i})
$$ 
for all $i$ in $I$, where $f(j)_{j<i}$ is viewed as a family of elements of $\{f(j)\,|\,j\in I,j<i\}$ indexed by $\{j\in I\,|\,j<i\}$. %(In particular, $S$ is a $\U$-set.)
\end{thm}

\begin{proof}
Uniqueness: Assume that $(S,f)$ and $(T,g)$ have the indicated properties. It suffices to prove $f(i)=g(i)$ for all $i$ in $I$. Suppose this is false, and let $i$ be the least element of $I$ such that $f(i)\neq g(i)$. We have 
$$
f(i)=F(f(j)_{j<i})=F(g(j)_{j<i})=g(i),
$$ 
a contradiction. 

Existence: Recall that a subset $J$ of $I$ is called a {\em segment} if $I\ni i<j\in J$ implies $i\in J$. Let $Z$ be the set of all triples $(J,S_J,f_J)$, where $J$ is a segment of $I$, where $f:J\to S_J$ is a surjection, and where we have $f_J(j)=F(f_J(k)_{k<j})$ for all $j$ in $J$. Decree that 
$$
Z\ni(J,S_J,f_J)\le(K,S_K,f_K)\in Z
$$ 
if and only if $J$ is a segment of $K$. By the uniqueness part, $(Z,\le)$ is inductive. Let $(J,S_J,f_J)$ be a maximal element of $Z$. It suffices to assume that $J$ is a proper segment of $I$ and to derive a contradiction. Let $k$ be the minimum of $I\setminus J$, put 
$$
K:=J\cup\{k\},\quad f_K(j):=f_J(j)\ \forall\ j\in J,
$$
$$
f_K(k):=F(f_K(j)_{j<k}),\quad S_K:=S_J\cup\{f_K(k)\}.
$$ 
Then $(K,S_K,f_K)$ contradicts the maximality if $(J,S_J,f_J)$. 
\end{proof}

\begin{proof}[Proof of the Quasi-Terminal Object Theorem (Theorem \ref{qtot} p.~\pr{qtot})] 
Let $\C$ be as in the statement. We assume (as we may) that $\C$ is small. Let us choose a $\C$-special well-ordered set $I$, and let us define an inductive system $(X_i)_{i\in I}$ by transfinite induction as follows: For the least element $0$ of $I$ we choose an arbitrary object $X_0$ of $\C$. Let $i>0$ and assume that $X_j$ and $u_{jk}:X_k\to X_j$ have been constructed for $k\le j<i$. 

\nn(a) If $i=j+1$ for some $j$, take $u_{ij}:X_j\to X_i$ with Property~$(P)$, and put $u_{ik}:=u_{ij}\ci u_{jk}$ for any $k\le j$. 

\nn(b) If $i=\sup\,\{j\,|\,j<i\}$, let $(a_j:X_j\to X_i)_{j<i}$ be some upper bound for $(X_j)_{j<i}$ and put $u_{ij}:=a_j$. 

(Recall that, by Definition~\ref{pofa} p.~\pr{pofa}, the condition ``$u_{ij}:X_j\to X_i$ has Property~$(P)$'' means that for any morphism $b:X_i\to C$ there is a morphism $c:C\to X_i$ satisfying $c\ci b\ci u_{ij}=u_{ij}$. Recall also that such a $u_{ij}$ exists by Lemma~\ref{944} p.~\pr{944}.) 

Then $(X_i)_{i\in I}$ is indeed an inductive system in $\C$. As $I$ is $\C$-special, there is an upper bound $(b_i:X_i\to X)_{i\in I}$ for $(X_i)_{i\in I}$, and there is an $i_0$ in $I$ such that $b_{i_0}:X_{i_0}\to X$ is an epimorphism. 

We claim that $X$ is quasi-terminal. Let $u:X\to Y$ be a morphism. It suffices to prove the claim below: 
%
\begin{claim}\lb{cqt}
There is a morphism $v:Y\to X$ such that $v\ci u=\id_X$. 
\end{claim} 

Consider the morphisms 
$$
\begin{tikzcd}
X_{i_0}\ar{rr}{u_{i_0+1,i_0}}&&X_{i_0+1}\ar{rr}{u\ci b_{i_0+1}}&&Y.
\end{tikzcd}
$$
As $u_{i_0+1,i_0}$ has Property~$(P)$, there is a morphism $w:Y\to X_{i_0+1}$ satisfying  
%
\begin{equation}\lb{wu}
w\ci u\ci b_{i_0+1}\ci u_{i_0+1,i_0}=u_{i_0+1,i_0}.
\end{equation}  
% 
Put
%
\begin{equation}\lb{vb}
v:=b_{i_0+1}\ci w:Y\to X.
\end{equation}
%
It suffices to show that $v$ satisfies the equality $v\ci u=\id_X$ in Claim~\ref{cqt} p.~\pr{cqt}. We have 
%
\begin{align*}
v\ci u\ci b_{i_0}&=b_{i_0+1}\ci w\ci u\ci b_{i_0}&\text{by \qr{vb}}\\ 
&=b_{i_0+1}\ci w\ci u\ci b_{i_0+1}\ci u_{i_0+1,i_0}\\ 
&=b_{i_0+1}\ci u_{i_0+1,i_0}&\text{by \qr{wu}}\\ 
&=b_{i_0}\\ 
&=\id_X\ci b_{i_0}. 
\end{align*}
%
As $b_{i_0}$ is an epimorphism, this implies $v\ci u=\id_X$, proving Claim~\ref{cqt} p.~\pr{cqt}, and thus the Quasi-Terminal Object Theorem (Theorem \ref{qtot} p.~\pr{qtot}). 
\end{proof}

Here is a diagrammatic illustration of the above computation:
$$
\begin{tikzcd}
X_{i_0}\ar[equal]{d}\ar{rr}{b_{i_0}}&&X\ar[equal]{d}\ar{r}{u}&Y\ar[equal]{d}\ar{rr}{v}&&X\ar[equal]{d}\\ 
X_{i_0}\ar[equal]{d}\ar{r}[swap]{u_{i_0+1,i_0}}&X_{i_0+1}\ar{r}[swap]{b_{i_0+1}}&X\ar{r}[swap]{u}&Y\ar{r}[swap]{w}&X_{i_0+1}\ar[equal]{d}\ar{r}[swap]{b_{i_0+1}}&X\ar[equal]{d}\\ 
X_{i_0}\ar[equal]{d}\ar{rrrr}[swap]{u_{i_0+1,i_0}}&&&&X_{i_0+1}\ar{r}[swap]{b_{i_0+1}}&X\ar[equal]{d}\\ 
X_{i_0}\ar{rrrrr}[swap]{b_{i_0}}&&&&&X.
\end{tikzcd}
$$ 

For the reader's convenience we state and prove Sublemma 9.4.5 p.~229 of the book.

\begin{lem}[Sublemma 9.4.5 p.~229]\lb{945}
If $\C$ is a small nonempty category admitting small filtrant inductive limits, if $\pi$ is an infinite regular cardinal such that $\card(\Mor(\C))<\pi$, if $I$ is a $\pi$-filtrant small category, and if $(X_i)_{i\in I}$ is an inductive system in $\C$ indexed by $I$, then there is an $i_0$ in $I$ such that the coprojection $X_{i_0}\to\col_iX_i$ is an epimorphism. 
\end{lem}

\begin{proof}
Set $X:=\col_iX_i$ and let $a_i:X_i\to X$ be the coprojection. For any $Y$ in $\C$ let 
$$
b^Y_i:\Hom_\C(Y,X_i)\to\col_j\Hom_\C(Y,X_j)
$$ 
be the coprojection, let $F(Y)$ be the image of the natural map 
$$
\col_j\Hom_\C(Y,X_j)\to\Hom_\C(Y,X),
$$ 
and define $\pp^Y$ by the commutative diagram 
$$
\begin{tikzcd}
\col_j\Hom_\C(Y,X_j)\ar[two heads]{r}{\pp^Y}&F(Y)\ar[hook]{r}&\Hom_\C(Y,X)\\ 
\Hom_\C(Y,X_i).\ar{u}{b^Y_i}\ar{ur}[swap]{\pp^Y_i:=a_i\ci}
\end{tikzcd}
$$ 

Claim: There is an $i_0$ in $I$ such that $\pp^Y_{i_0}:=a_{i_0}\ci:\Hom_\C(Y,X_{i_0})\to F(Y)$ is surjective for all $Y$ in $\C$.

As $\card(\Hom_\C(Y,X))<\pi$, we have $F(Y)\in\Set_\pi$ by Corollary 9.2.12 p.~221 of the book. By Lemma 9.3.1 p.~224 of the book (stated above as Lemma~\ref{l331} p.~\pr{l331}), there is an $i_Y$ in $I$ such that 
$$
a_{i_Y}\ci:\Hom_\C(Y,X_{i_Y})\to F(Y)
$$ 
is surjective. As $\card(\{i_Y\,|\,Y\in\Ob(\C)\})<\pi$ and $I$ is $\pi$-filtrant, there is an $i_0$ in $I$ such that, for any $Y$ in $\C$, there is a morphism $i_Y\to i_0$. This implies the claim. 

Let $i$ be in $I$. In particular, $a_i=\pp^{X_i}_i(\id_{X_i})$ is in $F(X_i)$. As 
$$
\pp^{X_i}_{i_0}:=a_{i_0}\ci:\Hom_\C(X_i,X_{i_0})\to F(X_i)
$$ 
is surjective by the claim, there is a morphism $h_i:X_i\to X_{i_0}$ such that $a_{i_0}\ci h_i=a_i$. 

Let us show that $a_{i_0}:X_{i_0}\to X$ is an epimorphism. Let $f_1,f_2:X\parar Y$ be a pair of parallel arrows such that $f_1\ci a_{i_0}=f_2\ci a_{i_0}$. Then, for any $i$ in $I$, we have 
$$
f_1\ci a_i=f_1\ci a_{i_0}\ci h_i=f_2\ci a_{i_0}\ci h_i=f_2\ci a_i.
$$ 
This implies $f_1=f_2$.
\end{proof}

We give again a diagrammatic illustration of the above computation:
$$
\begin{tikzcd}
X_i\ar[equal]{d}\ar{rr}{a_i}&&X\ar[equal]{d}\ar{r}{f_1}&Y\ar[equal]{d}\\ 
X_i\ar[equal]{d}\ar{r}{h_i}&X_{i_0}\ar[equal]{d}\ar{r}{a_{i_0}}&X\ar{r}{f_1}&Y\ar[equal]{d}\\ 
X_i\ar[equal]{d}\ar{r}{h_i}&X_{i_0}\ar{r}{a_{i_0}}&X\ar[equal]{d}\ar{r}{f_2}&Y\ar[equal]{d}\\ 
X_i\ar{rr}[swap]{a_i}&&X\ar{r}[swap]{f_2}&Y.
\end{tikzcd}
$$ 

%%%

\sbs{Lemma 9.5.3 p. 231}

We give more details about the proof, but first let us recall the setting:

Let $\C$ be a $\U$-category, let $\C_0$ be a subcategory of $\C$, and assume 

\nn(9.5.2) (i) $\C_0$ admits small filtrant inductive limits and $\C_0\to\C$ commutes with them.

\nn(9.5.2) (ii) Any diagram of solid arrows 
\begin{equation}\lb{952ii}
\begin{tikzcd}
X\ar{d}[swap]{f}\ar{r}{u}&Y\ar[dashed]{d}{g}\\ 
X'\ar[dashed]{r}[swap]{u'}&Y',
\end{tikzcd}
\end{equation} 
with $u$ in $\Mor(\C_0)$ and $f$ in $\Mor(\C)$, can be completed to a commutative diagram with dashed arrows $u'$ in $\Mor(\C_0)$ and $g$ in $\Mor(\C)$.

\begin{lem}[Lemma 9.5.3 p. 231] 
If $X'$ is in $\C_0$, if $I$ is a small set, and if  
$$
(u_i:X_i\to Y_i)_{i\in I},\quad(f_i:X_i\to X')_{i\in I}
$$ 
are families of morphisms in $\C_0$ and $\C$ respectively, then there is an object $Y'$ of $\C_0$, a morphism $u':X'\to Y'$ in $\C_0$, and a family $(g_i:Y_i\to Y')_{i\in I}$ of morphisms in $\C$ such that $g_i\ci u_i=u'\ci f_i$ for all $i$:
$$
\begin{tikzcd}
X_i\ar{d}[swap]{f_i}\ar{r}{u_i}&Y_i\ar[dashed]{d}{g_i}\\ 
X'\ar[dashed]{r}[swap]{u'}&Y'.
\end{tikzcd}
$$ 
\end{lem} 

\begin{proof}
We assume, as we may, that $I$ is nonempty, well-ordered, and admits a maximum $m$. Let $0$ be the least element of $I$. We shall complete the following Task $(T_i)$ by transfinite induction on $i\in I$ (see Theorem~\ref{meta} p.~\pr{meta}): 

\nn[Beginning of the description of Task $(T_i)$.] Construct, for each $j\le i$ in $I$, a commutative diagram 
$$
\begin{tikzcd}
X_j\ar{rr}{u_j}\ar{d}[swap]{f_j}&&Y_j\ar[dashed]{d}{h_j}\\
X'\ar[dashed]{r}[swap]{v_j}&Y'_{<j}\ar[dashed]{r}[swap]{w_j}&Y'_j,
\end{tikzcd}
$$ 
with $v_j,w_j$ in $\Mor(\C_0)$, and construct, for each $(i,j,k)$ in $I^3$ with $i\ge j>k$, a commutative diagram 
$$
\begin{tikzcd}
X'\ar{r}{v_j}\ar[equal]{d}&Y'_{<j}\ar{r}{w_j}&Y'_j\\
X'\ar{r}[swap]{v_k}&Y'_{<k}\ar{r}[swap]{w_k}&Y'_k,\ar[dashed]{ul}[swap]{p_{jk}}
\end{tikzcd}
$$
with $p_{jk}$ in $\Mor(\C_0)$, in such a way that we have
%
\begin{equation}\lb{pijk}
p_{ij}\ci w_j\ci p_{jk}=p_{ik}\quad\forall\quad i>j>k,
\end{equation}
%
\begin{equation}\lb{w0}
w_0=\id_{Y'_0}.
\end{equation}
%
Here is a diagrammatic illustration of \qr{pijk}: 
$$
\begin{tikzcd}
Y'_{<i}\ar[equal]{rr}&&Y'_{<i}\\
Y'_{<j}\ar{r}{w_j}&Y'_j\ar{ul}[swap]{p_{ij}}\\ 
{}&Y'_k.\ar{ul}{p_{jk}}\ar{uur}[swap]{p_{ik}}
\end{tikzcd}
$$

\nn[End of the description of Task $(T_i)$.] 

\nn[Beginning of the accomplishment of Task $(T_i)$ for all $i$.] To handle Task $(T_0)$, we define $Y'_0,v_0$ and $h_0$ by (9.5.2) (ii):
$$
\begin{tikzcd}
X_0\ar{d}[swap]{f_0}\ar{r}{u_0}&Y_0\ar[dashed]{d}{h_0}\\
X'\ar[dashed]{r}[swap]{v_0}&Y'_0,
\end{tikzcd}
$$ 
and we define $Y'_{<0}$ and $w_0$ by the commutative diagram
$$
\begin{tikzcd}
X_0\ar{rr}{u_0}\ar{d}[swap]{f_0}&&Y_0\ar{d}{h_0}\\
X'\ar{r}{v_0}\ar[equal]{d}&Y'_{<0}\ar{r}{w_0}\ar[equal]{d}&Y'_0\ar[equal]{d}\\
X'\ar{r}[swap]{v_0}&Y'_0\ar{r}[swap]{\id}&Y'_0.
\end{tikzcd}
$$

Let $i$ in $I$ satisfy $i>0$, and let us tackle Task $(T_i)$. 

We assume (as we may) that Task $(T_j)$ has already been achieved for $j<i$, {\em i.e.} that the $Y'_{<j},Y'_j,h_j,v_j,w_j$ have already been constructed for $j<i$, that the $p_{jk}$ have already been constructed for $k<j<i$, and that all these morphisms satisfy the required conditions. 

It suffices to define $Y'_{<i},Y'_i,h_i,v_i,w_i,$ and $p_{ij}$ for $j<i$, in such a way that the required conditions are still satisfied. 

For $k<j<i$ we define $u_{jk}:Y'_k\to Y'_j$ by 
\begin{equation}\lb{ujk}
u_{jk}:=w_j\ci p_{jk}.
\end{equation} 
By \qr{pijk} we have $u_{jk}\ci u_{k\ell}=u_{j\ell}$ for all $\ell<k<j<i$. In particular, 
\begin{equation}\lb{y'j}
(Y'_j)_{j<i}\text{ is an inductive system in }\C_0.
\end{equation} 
We denote its limit (which exists in $\C_0$ thanks to (9.5.2) (i)) by $Y'_{<i}$, and we write $p_{ij}$ for the coprojection $Y'_j\to Y'_{<i}$. We also set 
\begin{equation}\lb{vi}
v_i:=p_{i0}\ci v_0,
\end{equation} 
and we define
$$
Y'_{<i}\xr{w_i}Y'_i\xl{h_i}Y_i
$$
by (9.5.2) (ii):
$$
\begin{tikzcd}
X_i\ar{d}[swap]{v_i\ci f_i}\ar{r}{u_i}&Y_i\ar[dashed]{d}{h_i}\\
Y'_{<i}\ar[dashed]{r}[swap]{w_i}&Y'_i,
\end{tikzcd}
$$
so that we have  
\begin{equation}\lb{hiui2}
h_i\ci u_i=w_i\ci v_i\ci f_i. 
\end{equation} 
We must check 
\begin{equation}\lb{pikwk}
p_{ik}\ci w_k\ci v_k=v_i\ \forall\ k<i,
\end{equation} 
\begin{equation}\lb{pijwj}
p_{ij}\ci w_j\ci p_{jk}=p_{ik}\ \forall\ k<j<i.
\end{equation}  
To prove \qr{pikwk}, first note that we have 
$$v_k=p_{k0}\ci w_0\ci v_0
$$ 
by induction hypothesis, $w_0=\id_{Y'_0}$ by \qr{w0}, and thus 
\begin{equation}\lb{vk}
v_k=p_{k0}\ci v_0.
\end{equation} 
We get 
\begin{align*}
p_{ik}\ci w_k\ci v_k&=p_{ik}\ci w_k\ci p_{k0}\ci v_0&\text{by \qr{vk}}\\ 
&=p_{i0}\ci v_0&\text{by \qr{pijk}}\\ 
&=v_i&\text{by \qr{vi}}.
\end{align*} 
This proves \qr{pikwk}. We have 
\begin{align*}
p_{ij}\ci w_j\ci p_{jk}&=p_{ij}\ci u_{jk}&\text{by \qr{ujk}}\\ 
&=p_{ik}&\text{by \qr{y'j}}.
\end{align*} 
This proves \qr{pijwj}. 

Task $(T_i)$ has been performed for the specific $i$ we have been considering, and thus Task $(T_i)$ has been completed for all $i$ in $I$. [End of the accomplishment of Task $(T_i)$ for all $i$.]

In particular Task $(T_m)$, where, remember, $m$ is the maximum of $I$, has also been achieved. Putting $Y':=Y'_m$ and 
\begin{equation}\lb{gi}
g_i:=u_{mi}\ci h_i\ \forall\ i<m,
\end{equation} 
\begin{equation}\lb{gm}
g_m:=h_m,
\end{equation} 
\begin{equation}\lb{u'}
u':=w_m\ci v_m,
\end{equation} 
we get 
\begin{align*}
g_i\ci u_i&=u_{mi}\ci h_i\ci u_i&\text{by \qr{gi}}\\ 
&=u_{mi}\ci w_i\ci v_i\ci f_i&\text{by \qr{hiui2}}\\ 
&=w_m\ci p_{mi}\ci w_i\ci v_i\ci f_i&\text{by \qr{ujk}}\\ 
&=w_m\ci v_m\ci f_i&\text{by \qr{pikwk}}\\ 
&=u'\ci f_i&\text{by \qr{u'}}
\end{align*} 
for $i<m$, and 
\begin{align*}
g_m\ci u_m&=h_m\ci u_m&\text{by \qr{gm}}\\ 
&=w_m\ci v_m\ci f_m&\text{by \qr{hiui2}}\\ 
&=u'\ci f_m&\text{by \qr{u'}}.
\end{align*}
\end{proof}

%%

\sbs{Theorems 9.5.4 and 9.5.5 pp 232-234}\lb{954955}

The purpose of this section is to give a combined statement of Theorems 9.5.4 and 9.5.5. 

Let $\C$ be a $\U$-category, let $\F\subset\C_0$ be subcategories of $\C$ such that $\F$ is essentially small (see Remark~\ref{954} below), let $\pi$ be an infinite cardinal such that $X$ is in $\C_\pi$ for any $X\to Z$ in $\F$, and assume 

\nn(9.5.2) (i) $\C_0$ admits small filtrant inductive limits and $\C_0\to\C$ commutes with them;

\nn(9.5.2) (ii) any diagram of solid arrows
$$
\begin{tikzcd}
X\ar{d}[swap]{f}\ar{r}{u}&Y\ar[dashed]{d}{g}\\ 
X'\ar[dashed]{r}[swap]{u'}&Y',
\end{tikzcd}
$$ 
with $u$ in $\Mor(\C_0)$ and $f$ in $\Mor(\C)$, can be completed as indicated to a commutative diagram with dashed arrows $u'$ in $\Mor(\C_0)$ and $g$ in $\Mor(\C)$; 

\nn(9.5.6) for any $X$ in $\C_0$, the category $(\C_0)_X$ is essentially small;

\nn(9.5.7) any cartesian square 
$$
\begin{tikzcd}
X'\ar{d}[swap]{u}\ar{r}{f'}&Y'\ar{d}{v}\\ 
X\ar{r}[swap]{f}&Y
\end{tikzcd}
$$ 
in $\C$ with $f,f'$ in $\Mor(\C_0)$ decomposes into a commutative diagram 
$$
\begin{tikzcd}
X'\ar{d}[swap]{u}\ar{r}{f'}&Y'\ar{d}\ar{rd}{v}\\ 
X\ar{r}[swap]{g}&Z\ar{r}[swap]{h}&Y
\end{tikzcd}
$$ 
such that the square $(X',Y',Z,X)$ is cocartesian, $g$ and $h$ are in $\Mor(\C_0)$, and $f=h\ci g$; 

\nn(9.5.8) if a morphism $f:X\to Y$ in $\C_0$ is such that any cartesian square of solid arrows
$$
\begin{tikzcd}
U\ar{d}[swap]{u}\ar{r}{s}&V\ar{d}{v}\ar[dashed]{ld}[swap]{\xi}\\ 
X\ar{r}[swap]{f}&Y
\end{tikzcd}
$$ 
can be completed as indicated to a commutative diagram in $\C$ with the dashed arrow $\xi$, then $f$ is an isomorphism. 
%
\begin{thm} 
If the above conditions hold, then, for any $X$ in $\C_0$, there is a $\Mor(\C_0)$-injective object $Y$ of $\C$, and morphism $f:X\to Y$ in $\C_0$. If (9.5.2) holds, but (9.5.6), (9.5.7) and (9.5.8) do not necessarily hold, then there is an $\F$-injective object $Y$ of $\C$, and a morphism $f:X\to Y$ in $\C_0$.
\end{thm}
%
\begin{rk}\lb{954}
In the book $\F$ is supposed to be small, but the proof clearly works if $\F$ is essentially small. (See \S\ref{962} below.)
\end{rk}

%%

\sbs{Brief comments}

\begin{s} 
P.~235, Theorem 9.6.1. In view of the comments made before Corollary~\ref{938} p.~\pr{938}, Theorem 9.6.1 could be stated as follows:

\begin{thm}[Theorem 9.6.1 p. 235]\lb{961}
Let $\C$ be a Grothendieck category. Then, for any small subset $E$ of $\Ob(\C)$, there exists an infinite cardinal $\pi$ such that

\nn{\em(i)} $\Ob(\C_\pi)$ contains $E$,

\nn{\em(ii)} $\C_\pi$ is a fully abelian subcategory of $\C$,

\nn{\em(iii)} $\C_\pi$ is essentially small,

\nn{\em(iv)} $\C_\pi$ contains a generator of $\C$,

\nn{\em(v)} $\C_\pi$ is closed by subobjects and quotients in $\C$,

\nn{\em(vi)} for any epimorphism $f:X\epi Y$ in $\C$ with $Y$ in $\C_\pi$, there exists $Z$ in $\C_\pi$ and a monomorphism $g:Z\mono X$ such that $f\ci g:Z\to Y$ is an epimorphism,

\nn{\em(vii)} $\C_\pi$ is closed by countable direct sums.
\end{thm}
\end{s}

%

\begin{s}\lb{962}
P.~236, proof of Theorem 9.6.2. 

Line 3: One could change ``Let $\F$ be the set of monomorphisms $N\incl G$. This is a small set by Corollary 8.3.26'' to ``Let $\F$ be the set of monomorphisms $N\incl G$. This is an essentially small subcategory by Corollary 8.3.26''. In view of Remark~\ref{954}, we can still apply Theorem 9.5.4.

Line 6: Condition (9.5.2) (i) (see Section~\ref{954955} p.~\pr{954955} above) follows from 
\begin{lem}\lb{952i}
Let $\C$ be a category. Assume that small filtrant inductive limits exist in $\C$ and are exact. Let $\al:I\to\C$ be a functor such that $I$ is small and filtrant, and $\al(s):\al(i)\to\al(j)$ is a monomorphism for all morphism $s:i\to j$ in $I$. Then the coprojection $p_i:\al(i)\to\col\al$ is a monomorphism.
\end{lem} 
\begin{proof}
By Corollary 3.2.3 p. 79 of the book, $I^i$ is filtrant and the forgetful functor $\pp:I^i\to I$ is cofinal. Define the morphism of functors 
$$
\theta\in\Hom_{\C^{I^i}}(\DT\al(i),\al\ci\pp)
$$ 
(see Notation~\ref{diag} p.~\pr{diag}) by 
$$
\theta_{(s:i\to j)}:=\big(\al(s):\al(i)\to\al(j)\big). 
$$ 
As $\theta$ is a monomorphism, Proposition~\ref{34i} p.~\pr{34i} implies that $\col\theta$ is also a monomorphism. Then the conclusion follows from the commutativity of the diagram 
$$
\begin{tikzcd}
\col\DT\al(i)\ar{d}[swap]{\sim}\ar[tail]{rr}{\col\theta}&&\col\al\ci\pp\ar{d}{\sim}\\ 
\al(i)\ar{rr}[swap]{p_i}&&\col\al.
\end{tikzcd}
$$
\end{proof}
\end{s}

%%

%\begin{s}% Corollary 9.6.4 (i) p.~237 has already been stated as Proposition 8.3.27 (ii) p.~186. (See \S\ref{964i} p.~\pr{964i}.) \end{s}

%

\begin{s}
Pp 237-239. For the reader's convenience we first reproduce (with minor changes) two corollaries with their proof. 

\begin{cor}[Corollary 9.6.5 p. 237]
If $\C$ is a small abelian category, then $\Ind(\C)$ admits an injective cogenerator.
\end{cor}

\begin{proof}
Apply Theorem 8.6.5 (vi) p.~194 and Theorem 9.6.3 p.~236 of the book.
\end{proof}

\begin{cor}[Corollary 9.6.6 p. 237]
Let $\C$ be a Grothendieck category. Denote by $\I$ the full additive subcategory of $\C$ consisting of injective objects, and by $\iota:\I\to\C$ the inclusion functor. Then there exist a (not necessarily additive) functor $\Psi:\C\to\I$ and a morphism of functors $\id_\C\to\iota\ci\Psi$ such that $X\to\Psi(X)$ is a monomorphism for any $X$ in $\C$.
\end{cor}

\begin{proof}
The category $\C$ admits an injective cogenerator $K$ by Theorem 9.6.3 p.~236 of the book, and admits small products by Proposition 8.3.27 p.~186 of the book. Consider the (non additive) functor 
$$
\Psi:\C\to\I,\quad X\mt K^{\Hom_\C(X,K)}.
$$ 
The identity of 
$$
\Hom_{\Set}(\Hom_\C(X,K),\Hom_\C(X,K))\iso\Hom_\C(X,K^{\Hom_\C(X,K)})
$$ 
defines a morphism $X\to\iota(\Psi(X))=K^{\Hom_\C(X,K)}$, and this morphism is a monomorphism by Proposition 5.2.3 (iv) p.~118 of the book. 
\end{proof}

%We now add three parenthetical points: 

The first sentence of the proof of Lemma 9.6.8 p.~238 of the book follows from Proposition 5.2.3 (iv) p.~118 of the book. 

The third sentence of the proof of Lemma 9.6.9 p.~238 of the book follows from Proposition 5.2.3 (i) p.~118 of the book. The end of the proof of Lemma 9.6.9 uses the following lemma:

\begin{lem} 
If 
$$
\begin{tikzcd}
0\ar[d]\ar[r]&Z\ar[d,"c"]\ar[r,"a"]&Y\ar[d,"d","\sim"']\ar[r,"b"]&X\ar[d,rightarrowtail,"e"]\\ 
0\ar[r,]&W\ar[r,"f"']&V\ar[r,"g"']&U
\end{tikzcd}
$$ 
is an exact commutative diagram in an abelian category, and if $d$ is an isomorphism and $e$ a monomorphism, then $c$ is an isomorphism.
\end{lem} 

\begin{proof}
(In this proof we omit the composition symbols $\ci$ and most of the parenthesis, and we freely use Lemma~\ref{8311} p.~\pr{8311} and Lemma~\ref{8312} p.~\pr{8312}.) As $c$ is clearly a monomorphism, it suffices to show that $c$ is an epimorphism. Let $w:T\to W$ be a morphism. It suffices to prove that the solid diagram 
$$
\begin{tikzcd} 
S\ar[d,dashed,"z"']\ar[r,dashed,two heads,"h"]&T\ar[d,"w"]\\ 
Z\ar[r,"c"']&W
\end{tikzcd}
$$ 
can be completed as indicated. We have $ebd^{-1}fw=gfw=0$, and thus $bd^{-1}fw=0$. This implies the existence of a commutative diagram 
$$
\begin{tikzcd} 
S\ar[d,"z"']\ar[r,two heads,"h"]&T\ar[d,"d^{-1}fw"]\\ 
Z\ar[r,"a"']&Y\ar[r,"b"']&X,
\end{tikzcd}
$$ 
yielding $fwh=dd^{-1}fwh=daz=fcz$, and thus $wh=cz$. 
\end{proof} 

In the proof of Theorem 9.6.10 p.~238 of the book, the exactness of $\C\to\oo{Pro}(\C)$ follows from Theorem 8.6.5 (ii) p.~194 of the book.
\end{s}

%%

\section{About Chapter 10}

\sbs{Definition of a triangulated category\index{triangulated category}\index{May (Peter May)}}

The purpose of this Section is to spell out the observation made by J. P. May that, in the definition of a triangulated category, Axiom TR4 of the book (p.~243) follows from the other axioms. See Section~1 of {\em The axioms for triangulated categories} by J. P. May: 
\begin{center}\href{http://www.math.uchicago.edu/~may/MISC/Triangulate.pdf}{http://www.math.uchicago.edu/$\sim$may/MISC/Triangulate.pdf} 
\end{center} 
Various related links are given in the document \href{http://goo.gl/df2Xw}{http://goo.gl/df2Xw}. 

To make things as clear as possible, we remove TR4 from the definition of a triangulated category and prove that any triangulated category satisfies TR4:

\begin{df}[triangulated category]\lb{tr} 
A {\em triangulated category}\index{triangulated category} is an additive category $(\DD,T)$ with translation endowed with a set of triangles satisfying Axioms {\em TR0, TR1, TR2, TR3} and {\em TR5} on p.~243 of the book.
\end{df}

Let $(\DD,T)$ be a triangulated category. In the book the theorem below is stated as Exercise 10.6 p.~266 and is used at the top of p.~251 within the proof of Theorem 10.2.3 p.~249.

\begin{thm}\lb{mayt}
Let 
$$
\begin{tikzcd}
X^0\ar{r}{u}\ar{d}[swap]{f}&X^1\ar{d}\ar{r}{v}&X^2\ar[dashed]{d}\ar{r}{w}&TX^0\ar{d}{Tf}\\ 
Y^0\ar{r}\ar{d}[swap]{g}&Y^1\ar{d}\ar{r}&Y^2\ar[dashed]{d}\ar{r}&TY^0\ar{d}{Tg}\\ 
Z^0\ar[dashed]{r}\ar{d}[swap]{h}&Z^1\ar{d}\ar[dashed]{r}&Z^2\ar[dashed]{d}\ar[dashed]{r}&TZ^0\ar{d}{-Th}\\ 
TX^0\ar{r}[swap]{Tu}&TX^1\ar{r}[swap]{Tv}&TX^2\ar{r}[swap]{-Tw}&T^2X^0 
\end{tikzcd}
$$ 
be a diagram of solid arrows in $\DD$. Assume that the first two rows and columns are distinguished triangles, and the top left square commutes\footnote{I think the assumption that the top left square commutes is implicit in the book.}. Then the dotted arrows may be completed in order that the bottom right square anti-commutes, the eight other squares commute, and all rows and columns are distinguished triangles. 
\end{thm}

\begin{cor}\lb{may}
Any category which is triangulated in the sense of Definition~\ref{tr} satisfies {\em TR4}.
\end{cor} 

Recall Axiom TR5: If the diagram 
$$
\begin{tikzcd}
U\ar[equal]{dd}\ar{r}&V\ar[equal]{d}\ar{r}&W'\ar{r}&TU\\
&V\ar{r}&W\ar[equal]{d}\ar{r}&U'\ar{r}&TV\\
U\ar{rr}&&W\ar{rr}&&V'\ar{rr}&&TU
\end{tikzcd}
$$
commutes, and if the rows are distinguished triangles, then there is a distinguished triangle $W'\to V'\to U'\to TW'$ such that the diagram 
$$
\begin{tikzcd}
U\ar{r}\ar[equal]{d}&V\ar{d}\ar{r}&W'\ar{d}\ar{r}&TU\ar[equal]{d}\\
U\ar{d}\ar{r}&W\ar{r}\ar[equal]{d}&V'\ar{d}\ar{r}&TU\ar{d}\\
V\ar{d}\ar{r}&W\ar{d}\ar{r}&U'\ar{r}\ar[equal]{d}&TV\ar{d}\\
W'\ar{r}&V'\ar{r}&U'\ar{r}&TW' 
\end{tikzcd}
$$ 
commutes. 

\begin{proof}[Proof of Theorem~\ref{mayt}]
From 
$$
\begin{tikzcd}
X^0\ar[equal]{dd}\ar{r}&X^1\ar[equal]{d}\ar{r}&X^2\ar{r}&TX^0\\
&X^1\ar{r}&Y^1\ar[equal]{d}\ar{r}&Z^1\ar{r}&TX^1\\
X^0\ar{rr}&&Y^1\ar{rr}&&W\ar{rr}&&TX^0,
\end{tikzcd}
$$
where the last row is obtained by TR2, we get by TR5
\begin{equation}\lb{v1}
\begin{tikzcd}
X^0\ar{r}\ar[equal]{d}&X^1\ar{d}\ar{r}&X^2\ar{d}{a}\ar{r}[swap]{w}&TX^0\ar[equal]{d}\\
X^0\ar{d}\ar{r}&Y^1\ar{r}\ar[equal]{d}&W\ar{d}{b}\ar{r}{d}&TX^0\ar{d}\\
X^1\ar{d}\ar{r}&Y^1\ar{d}\ar{r}&Z^1\ar{r}\ar[equal]{d}&TX^1\ar{d}\\
X^2\ar{r}[swap]{a}&W\ar{r}[swap]{b}&Z^1\ar{r}[swap]{c}&TX^2.
\end{tikzcd}
\end{equation}
%
From 
$$
\begin{tikzcd}
X^0\ar[equal]{dd}\ar{r}&Y^0\ar[equal]{d}\ar{r}&Z^0\ar{r}&TX^0\\
&Y^0\ar{r}&Y^1\ar[equal]{d}\ar{r}&Y^2\ar{r}&TY^0\\
X^0\ar{rr}&&Y^1\ar{rr}&&W\ar{rr}&&TX^0,
\end{tikzcd}
$$
we get by TR5
\begin{equation}\lb{v2}
\begin{tikzcd}
X^0\ar{r}\ar[equal]{d}&Y^0\ar{d}\ar{r}&Z^0\ar{d}{e}\ar{r}[swap]{h}&TX^0\ar[equal]{d}\\
X^0\ar{d}\ar{r}&Y^1\ar{r}\ar[equal]{d}&W\ar{d}\ar{r}{d}&TX^0\ar{d}\\
Y^0\ar{d}\ar{r}&Y^1\ar{d}\ar{r}&Y^2\ar{r}\ar[equal]{d}&TY^0\ar{d}\\
Z^0\ar{r}[swap]{e}&W\ar{r}&Y^2\ar{r}&TZ^0.
\end{tikzcd}
\end{equation}
%
%We define $Z^0\to Z^1$ as the composition of $Z^0\xr eW$ in \qr{v2} with $W\xr bZ^1$ in \qr{v1}. 
From 
$$
\begin{tikzcd}
Z^0\ar[equal]{dd}\ar{r}{e}&W\ar[equal]{d}\ar{r}&Y^2\ar{r}&TZ^0\\
&W\ar{r}{b}&Z^1\ar[equal]{d}\ar{r}{c}&TX^2\ar{r}{-Ta}&TW\\
Z^0\ar{rr}&&Z^1\ar{rr}&&Z^2\ar{rr}{\ell}&&TZ^0,
\end{tikzcd}
$$
where the second row is obtained from $X^2\xr aW\xr bZ^1\xr cTX^2$ in \qr{v1} by TR3 and TR0, and 
\begin{equation}\lb{r3}
\text{the last row is obtained by TR2,} 
\end{equation} 
we get by TR5 
\begin{equation}\lb{v3}
\begin{tikzcd}
Z^0\ar{r}\ar[equal]{d}&W\ar{d}{b}\ar{r}&Y^2\ar{d}{j}\ar{r}&TZ^0\ar[equal]{d}\\
Z^0\ar{d}\ar{r}&Z^1\ar{r}\ar[equal]{d}&Z^2\ar{d}{k}\ar{r}[swap]{\ell}&TZ^0\ar{d}[swap]{Te}\\
W\ar{d}\ar{r}{b}&Z^1\ar{d}\ar{r}{c}&TX^2\ar{r}{-Ta}\ar[equal]{d}&TW\ar{d}\\
Y^2\ar{r}[swap]{j}&Z^2\ar{r}[swap]{k}&TX^2\ar{r}[swap]{-Ti}&TY^2,
\end{tikzcd}
\end{equation}
where 
\begin{equation}\lb{c3} 
X^2\xr iY^2\xr jZ^2\xr kTX^2\text{ is a distinguished triangle.} 
\end{equation} 
We want to prove that the bottom right square of 
\begin{equation}\lb{v4}
\begin{tikzcd}
X^0\ar{r}{u}\ar{d}[swap]{f}&X^1\ar{d}\ar{r}{v}&X^2\ar{d}{i}\ar{r}{w}&TX^0\ar{d}{Tf}\\ 
Y^0\ar{r}\ar{d}[swap]{g}&Y^1\ar{d}\ar{r}&Y^2\ar{d}{j}\ar{r}&TY^0\ar{d}{Tg}\\ 
Z^0\ar{r}\ar{d}[swap]{h}&Z^1\ar{d}\ar{r}&Z^2\ar{d}{k}\ar{r}{\ell}&TZ^0\ar{d}{-Th}\\ 
TX^0\ar{r}[swap]{Tu}&TX^1\ar{r}[swap]{Tv}&TX^2\ar{r}[swap]{-Tw}&T^2X^0
\end{tikzcd}
\end{equation} 
anti-commutes, that the eight other squares commute, and that all rows and columns are distinguished triangles.

We list the nine squares of each of the diagrams (\ref{v1}), (\ref{v2}), (\ref{v3}), (\ref{v4}) as follows:
$$
\begin{matrix}1&2&3\\ 4&5&6\\ 7&8&9
\end{matrix}
$$ 
and we denote the $j$-th square of Diagram $(i)$ by $(i)j$. 

The commutativity of (\ref{v1})2 and (\ref{v2})5 implies that of (\ref{v4})2. 

The commutativity of (\ref{v1})3 and (\ref{v2})6 implies that of (\ref{v4})3.

The commutativity of (\ref{v2})7 and (\ref{v3})1 implies that of (\ref{v4})4.

The commutativity of (\ref{v2})8 and (\ref{v3})2 implies that of (\ref{v4})5. 

The commutativity of (\ref{v2})9 and (\ref{v3})3 implies that of (\ref{v4})6. 

The commutativity of (\ref{v2})3 and (\ref{v1})6 implies that of (\ref{v4})7. 

The commutativity of (\ref{v1})9 and (\ref{v3})8 implies that of (\ref{v4})8. 

To prove the anti-commutativity of the bottom right square of \qr{v4}, note 
%
\begin{align*}
Th\ci\ell&=Td\ci Te\ci\ell&\text{by \qr{v2}}\\ 
&=-Td\ci Ta\ci k&\text{by \qr{v3}}\\  
&=-Tw\ci k&\text{by \qr{v1}}.
\end{align*} 

The third row and column are distinguished triangles by \qr{r3} and \qr{c3} respectively. It is easy to check that the other rows and columns are distinguished triangles too.
\end{proof}

%%

\sbs{Brief comments}

\begin{s} 
P.~250, proof of Theorem 10.2.3 (iii). In view of Corollary~\ref{may} p.~\pr{may}, it is not necessary to prove TR4.
\end{s}

%

\begin{s} 
P.~253, Definition 10.3.1. One of the key ingredients justifying the second sentence is Display~\qr{(7.3.7)} p.~\pr{(7.3.7)}.
\end{s}

%

\begin{s} P.~263, last sentence of the proof of Lemma 10.5.8. Consider the commutative diagram  
$$
\begin{tikzcd}
\oplus_i\,\pp(Z_i)\ar[equal]{d}\ar{r}&\oplus_i\,\pp(Y_i)\ar[equal]{d}\ar{r}&\oplus_i\,\widetilde\pp(X_i)\ar{d}\ar{r}&0\\ 
\oplus_i\,\pp(Z_i)\ar{r}&\oplus_i\,\pp(Y_i)\ar{r}&\widetilde\pp(\oplus_i\,X_i)\ar{r}&0, 
\end{tikzcd}
$$ 
whose rows are complexes. We already know that the bottom row is exact. The exactness of the top row follows (as in the proof of Lemma 10.5.7 (ii) p.~261 of the book) from the isomorphisms 
$$
\Coker(\oplus_i\,\pp(Z_i)\to\oplus_i\,\pp(Y_i))\iso\oplus_i\,\Coker(\pp(Z_i)\to\pp(Y_i))\iso\oplus_i\,\widetilde\pp(X_i).
$$
\end{s}

%%

\begin{s} P.~263, proof of Lemma 10.5.9. Before the sentence ``Since $Z_n$ and $X_n$ belong to $\K$, $X_{n+1}$ also belongs to $\K$'', one could add ``We may, and do, assume that $\K$ is saturated''.

Recall the Yoneda isomorphisms 
$$
\Hom_{\SSS^{\wg,\text{prod}}}(\pp(X),H_0)\iso H(X)\iso\Hom_{\DD^\wg}(X,H)
$$ 
for $X$ in $\SSS$.

Note that Convention~\ref{payc} p.~\pr{payc} can be applied.
\end{s}

%%

\sbs{Exercise 10.11 p.~266} 

Recall the statement: 

\nn(i) Let $\N$ be a null system in a triangulated category $\DD$, let $Q:\DD\to\DD/\N$ be the localization functor, and let $f:X\to Y$ be a morphism in $\DD$ satisfying $Q(f)=0$. Then $f$ factors through some object of $\N$. 

\nn(ii) The following conditions on $X$ in $\DD$ are equivalent: 

\nn(a) $Q(X)\iso0$,\quad(b) $X\oplus Y\in\N$ for some $Y\in\DD$,\quad(c) $X\oplus TX\in\N$.

\begin{proof}\ 

\nn(i) The definition of $\DD/\N$ and the assumption $Q(f)=0$ imply the existence of a morphism $s:Y\to Z$ in $\N Q$ such that $s\ci f=0$ (see (7.1.5) p.~155 of the book), and thus, in view of the definition of $\N Q$ (see (10.2.1) p.~249 of the book), the existence of a d.t. $W\to Y\to Z\to TW$ with $W$ in $\N$, and the conclusion follows from the fact that $\Hom_\DD(X,\ )$ is cohomological (see Proposition 10.1.13 p.~245 of the book). 

\nn(ii)

\nn(a)$\then$(b): As $Q(\id_X)=0$, the first part of the exercise implies that $\id_X$ factors as $X\xr fZ\xr g X$ with $Z$ in $\N$. By TR2 there is a d.t. 
$$
X\xr fZ\xr hY\xr kTX.
$$ 
Since $g\ci f=\id_X$, the morphism $f$ is a monomorphism, and so is $Tf$. As $Tf\ci k=0$ by Proposition 10.1.11 p.~245 of the book, this implies $k=0$. Hence we have a morphism of d.t. 
$$
\begin{tikzcd}
X\ar[equal]{d}\ar{r}{f}&Z\ar{d}{(g,h)}\ar{r}{h}&Y\ar[equal]{d}\ar{r}{0}&TX\ar[equal]{d}\\ 
X\ar{r}&X\oplus Y\ar{r}&Y\ar{r}&TX
\end{tikzcd}
$$
(the bottom is a d.t. by Corollary 10.1.20 (ii) p.~248 of the book) and Proposition 10.1.15 p.~246 of the book implies that $(g,h)$ is an isomorphism.\bigskip 

\nn(b)$\then$(c): Let $\DT_1,\dots,\DT_5$ be the triangles
$$
\begin{tikzcd}
X\ar{r}&0\ar{r}&TX\ar{r}{=}&TX\\ 
Y\ar{r}{=}&Y\ar{r}&0\ar{r}&TY\\ 
X\oplus Y\ar{r}&Y\ar{r}&TX\ar{r}&TX\oplus TY\\ 
0\ar{r}&X\ar{r}&X\ar{r}&0\\ 
X\oplus Y\ar{r}&X\oplus Y\ar{r}&X\oplus TX\ar{r}&TX\oplus TY,
\end{tikzcd}
$$ 
with $\DT_3:=\DT_1\oplus\DT_2$ and $\DT_5:=\DT_3\oplus\DT_4$. It is easy to see that $\DT_1,\DT_2$ and $\DT_4$ are distinguished. Then $\DT_3$ and $\DT_5$ are distinguished by Proposition 10.1.19 p.~247 of the book, and, as $X\oplus Y$ is in $\N$, Condition N'3 of Lemma 10.2.1 (b) p.~249 of the book implies that $X\oplus TX$ is in $\N$.

\nn(c)$\then$(a): Follows from Theorem 10.2.3 (iv) p.~249 of the book.
\end{proof} 

%%%

\section{About Chapter 11}

\begin{s}
P.~270. Recall that $(\A,T)$ is an additive category with translation. Let 
\begin{equation}\lb{iliad}
(d_{X,i}:X_i\to TX_i)_{i\in I}
\end{equation} 
be an inductive system in $\A_d$. Assume that $X:=\col_iX_i$ exists in $\A$. Then the natural morphism $\col_id_{X,i}:X\to TX$ is an inductive limit of \qr{iliad} in $\A_d$. There are analogous statements with ``projective'' instead of ``inductive'' and $\A_c$ instead of $\A_d$.
\end{s}

%

\begin{s}
P.~270, Definition 11.1.3. Here is a ``picture'' of the mapping cone of $f:X\to Y$:
$$
\begin{tikzcd}
TX\ar{rr}{-T(d_X)}\ar{rrdd}{T(f)}&&T^2X\\ 
\oplus&&\oplus\\ 
Y\ar{rr}[swap]{d_Y}&&TY.
\end{tikzcd}
$$
\end{s}

%

\begin{s}
P.~271, Remark 11.1.5. We have:
$$
d_{\Mc(T(f))}=
\begin{pmatrix}
d_{T^2X}&0\\ \\ 
T^2(f)&d_{TY}
\end{pmatrix},\quad 
d_{T(\Mc(f))}=
\begin{pmatrix}
d_{T^2X}&0\\ \\ 
-T^2(f)&d_{TY}
\end{pmatrix},
$$ 
and 
$$
T(\Mc(f))=T^2X\oplus TY\xr{\begin{pmatrix}-1&0\\ 0&1\end{pmatrix}}T^2X\oplus TY=\Mc(T(f))
$$ 
is a differential isomorphism.
\end{s}

%%

\sbs{Theorem 11.2.6 p~273} 

Here is a minor comment about the verification of Axiom TR5 in the proof of Theorem 11.2.6. We stated Axiom~TR5 right after Corollary~\ref{may} p.~\pr{may} above. For the reader's convenience we restate it. 

If the diagram 
$$
\begin{tikzcd}
U\ar[equal]{dd}\ar{r}&V\ar[equal]{d}\ar{r}&W'\ar{r}&TU\\
&V\ar{r}&W\ar[equal]{d}\ar{r}&U'\ar{r}&TV\\
U\ar{rr}&&W\ar{rr}&&V'\ar{rr}&&TU
\end{tikzcd}
$$
commutes, and if the rows are distinguished triangles, then there is a distinguished triangle $W'\to V'\to U'\to TW'$ such that the diagram below commutes:
$$
\begin{tikzcd}
U\ar{r}\ar[equal]{d}&V\ar{d}\ar{r}&W'\ar{d}\ar{r}&TU\ar[equal]{d}\\
U\ar{d}\ar{r}&W\ar{r}\ar[equal]{d}&V'\ar{d}\ar{r}&TU\ar{d}\\
V\ar{d}\ar{r}&W\ar{d}\ar{r}&U'\ar{r}\ar[equal]{d}&TV\ar{d}\\
W'\ar{r}&V'\ar{r}&U'\ar{r}&TW'.
\end{tikzcd}
$$ 

Going back to the proof of TR5 on p.~275, we consider the commutative diagram
$$
\begin{tikzcd}
X\ar[equal]{dd}\ar{r}{f}&Y\ar[equal]{d}\ar{r}{\al(f)}&TX\oplus Y\ar{r}{\bt(f)}&TX\\
&Y\ar{r}{g}&Z\ar[equal]{d}\ar{r}{\al(g)}&TY\oplus Z\ar{r}{\bt(g)}&TY\\
X\ar[swap]{rr}{g\ci f}&&Z\ar[swap]{rr}{\al(g\ci f)}&&TX\oplus Z\ar[swap]{rr}{\bt(g\ci f)}&&TX.
\end{tikzcd}
$$ 
The goal of the proof is then to construct a commutative diagram
$$
\begin{tikzcd}
X\ar{r}{f}\ar[equal]{d}&Y\ar{r}{\al(f)}\ar{d}{g}&TX\oplus Y\ar{r}{\bt(f)}\ar{d}{u}&TX\ar[equal]{d}\\ 
X\ar{r}{g\ci f}\ar{d}[swap]{f}&Z\ar{r}{\al(g\ci f)}\ar[equal]{d}&TX\oplus Z\ar{r}{\bt(g\ci f)}\ar{d}{v}&TX\ar{d}{T(f)}\\ 
Y\ar{r}{g}\ar{d}[swap]{\al(f)}&Z\ar{r}{\al(g)}\ar{d}[swap]{\al(g\ci f)}&TY\oplus Z\ar{r}{\bt(g)}\ar[equal]{d}&TY\ar{d}{T(\al(f))}\\ 
TX\oplus Y\ar{r}[swap]{u}&TX\oplus Z\ar{r}[swap]{v}&TY\oplus Z\ar{r}[swap]{w}&T^2X\oplus TY.
\end{tikzcd}
$$ 

%%

\sbs{Example 11.6.2 (i) p.~290} 

(As already stated, there is a typo; see \S\ref{290} p.~\pr{290}.) Let $\C,\C'$ and $\C''$ be additive categories with translation. If $F:\C\tm\C'\to\C''$ is a bifunctor of additive categories with translation and if $\C''$ admits countable direct sums, then, as explained in the book, $F$ induces a bifunctor of additive categories with translation 
$$
F_\oplus:\oo{C}(\C)\tm\oo{C}(\C')\to\oo{C}(\C'').
$$ 
If $\C''$ admits countable products instead of direct sums, then $F$ induces a bifunctor of additive categories with translation 
$$
F_\pi:\oo{C}(\C)\tm\oo{C}(\C')\to\oo{C}(\C'').
$$ 
The precise formulas are given in the book. If $F:\C\tm\C'^\op\to\C''$ is a bifunctor of additive categories with translation and if $\C''$ admits countable products, then $F$ induces again a bifunctor of additive categories with translation 
$$
F_\pi:\oo{C}(\C)\tm\oo{C}(\C')^{\op}\to\oo{C}(\C'').
$$ 
The formulas defining $F_\pi$ in this setting are almost the same as in the previous setting, and we give them without further comments:
$$
F_\pi(Y,X)^{n,m}=F(Y^n,X^{-m}),
$$
$$
d'^{n,m}=F(d_Y^n,X^{-m}),
$$
$$
d''^{n,m}=(-1)^{m+1}F(Y^n,d_X^{-m-1}),
$$
$$
\theta_{Y,X}:F_\pi(TY,X)\to TF_\pi(Y,X),
$$
$$
\theta'_{Y,X}:F_\pi(Y,T^{-1}X)\to TF_\pi(Y,X).
$$
$$
\theta_{Y,X}^{i+j}:F_\pi(TY,X)^{i+j}\to(TF_\pi(Y,X))^{i+j},
$$
$$
\theta_{Y,X}^{i,j}:F_\pi((TY)^i,X^{-j})=F(Y^{i+1},X^{-j})\to F_\pi(Y,X)^{i+j+1}=(TF_\pi(Y,X))^{i+j},
$$
$$
{\theta'}_{Y,X}^{i+j}:F_\pi(Y,T^{-1}X)^{i+j}\to(TF_\pi(Y,X))^{i+j},
$$
$$
{\theta'}_{Y,X}^{i,j}:F_\pi(Y^i,(T^{-1}X)^{-j})=F(Y^i,X^{-j-1})\to F_\pi(Y,X)^{i+j+1}=(TF_\pi(Y,X))^{i+j},
$$
the morphism ${\theta'}_{Y,X}^{i,j}$ being $(-1)^i$ times the canonical embedding. 

%%%

\section{About Chapter 12}

\sbs{Avoiding the Snake Lemma p. 297} \index{Snake Lemma} \index{avoiding the Snake Lemma}

This is about Sections 12.1 and 12.2 of the book. I think the Snake Lemma can be avoided as follows: 

Let $\A$ be an abelian category. 

\begin{lem}\lb{sl1}
If 
$$
\begin{tikzcd}
{}&X'\ar{d}{u}\ar{r}{f}&X\ar{d}{v}\ar{r}{g}&X''\ar{d}{w}\ar{r}&0\\ 
0\ar{r}&Y'\ar{r}[swap]{f'}&Y\ar{r}[swap]{g'}&Y''.
\end{tikzcd}
$$ 
is a commutative diagram in $\A$ with exact rows, then the sequence 
$$
\Ker u\to\Ker v\to\Ker w\xr0\Coker u\to\Coker v\to\Coker w
$$
is exact at $\Ker v$ and $\Coker v$. If in addition $w$ is a monomorphism or $u$ is an epimorphism, then the whole sequence is exact.
\end{lem}
%
The proof is straightforward (and much easier than that of the Snake Lemma). 

Let $(\A,T)$ be an abelian category with translation. 

\begin{lem}[see Theorem 12.2.4 p.~301]\lb{sl2}
If $0\to X\xr fY\xr g Z\to0$ is an exact sequence in $\A_c$, then the sequence $H(X)\to H(Y)\to H(Z)$ is exact. If, in addition, $H(T^nX)\iso0$ (respectively $H(T^nZ)\iso0$) for all $n$, then $T^nY\to T^nZ$ (respectively $T^nX\to T^nY$) is a qis for all $n$. 
\end{lem}

\begin{proof}
Taking into account Display (12.2.1) p.~300 of the book, apply Lemma~\ref{sl1} to the commutative diagram 
$$
\begin{tikzcd}
{}&\Coker T^{-1}d_X\ar{d}{d_X}\ar{r}{f}&\Coker T^{-1}d_Y\ar{d}{d_Y}\ar{r}{g}&\Coker T^{-1}d_Z\ar{d}{d_Z}\ar{r}&0\\ 
0\ar{r}&\Ker Td_X\ar{r}[swap]{f}&\Ker Td_Y\ar{r}[swap]{g}&\Ker Td_Z.
\end{tikzcd}
$$ 
\end{proof}

\begin{prop}[Corollary 12.2.5 p.~301]\lb{sl3}
The functor 
$$
H:\text K_c(\A)\to\A
$$ 
is cohomological.  
\end{prop}

\begin{proof}
Let $X\to Y\to Z\to TX$ be a d.t. in $K_c(\A)$. It is isomorphic to 
$$
V\xr{\al(u)}\Mc(u)\xr{\bt(u)}TU\to TV
$$ 
for some morphism $u:U\to V$. Since the sequence 
$$
0\to V\to\Mc(u)\to TU\to0
$$ 
in $\A_c$ is exact, it follows from Lemma~\ref{sl2} that the sequence 
$$ 
H(V)\to H(\Mc(u))\to H(TU)
$$ 
is exact.
\end{proof}
%
\begin{prop}[Corollary 12.2.6 p.~302]\lb{sl4}
Let $0\to X\xr f Y\xr g Z\to0$ be an exact sequence in $\A_c$ and define $\pp:\Mc(f)\to Z$ by $\pp:=(0,g)$. Then $\pp$ is a morphism in $\A_c$, and this morphism is a qis. In particular, there are natural morphisms $H(T^nZ)\to H(T^{n+1}X)$ such that the sequence 
$$
\cdots\to H(X)\to H(Y)\to H(Z)\to H(TX)\to\cdots
$$
is exact. 
\end{prop}

\begin{proof}
The commutative diagram in $\A_c$ with exact rows 
$$
\begin{tikzcd}
0\ar{r}&X\ar{d}[swap]{\id_X}\ar{r}{\id_X}&X\ar{d}{f}\ar{r}&0\ar{d}\ar{r}&0\\ 
0\ar{r}&X\ar{r}[swap]{f}&Y\ar{r}[swap]{g}&Z\ar{r}&0
\end{tikzcd}
$$ 
yields the exact sequence 
$$
0\to\Mc(\id_X)\to\Mc(f)\xr\pp\Mc(0\to Z)\to0
$$
in $\A_c$. As $H(\Mc(\id_X))\iso0$, $\pp$ is a qis by Lemma~\ref{sl2}.
\end{proof}

%%%

\section{About Chapter 13}

\sbs{Brief comments}

\begin{s}
P.~337. Theorem 13.4.1 suggests the following question: 

Let $\C$ be an abelian category and let $X$ and $Y$ be in $\oo K(\C)$. Is the natural morphism 
%
\begin{equation}\lb{spalt}
\col_{(X'\to X),(Y\to Y')\in\oo{Qis}}\Hom_{\oo K(\C)}(X',Y')\to\Hom_{\oo D(\C)}(X,Y)
\end{equation}
%
an isomorphism?

Theorem 13.4.1 and Corollary~\ref{1432} p.~\pr{1432} imply that the answer is yes if $\C$ is a Grothendieck category. 

(Observe that the Axiom of Universes is not necessary to define the morphism~\qr{spalt}.)
\end{s}

%

\begin{s}
P.~337. (See \S\ref{1341} p. \pr{1341}.) In view of \S\ref{745} p.~\pr{745} above and Theorem 13.4.1 p.~337 of the book, the functors 

$$
\Hom_\C^\bu:\oo K(\C)\tm\oo K(\C)^{\op}\to\oo K(\Mod(\mathbb Z))
$$ 

\nn and 

$$
\Hom_{\oo K(\C)}:\oo K(\C)\tm\oo K(\C)^{\op}\to\Mod(\mathbb Z)
$$ 

\nn give rise to the commutative diagram 
\begin{equation}\lb{r0hom} 
\begin{tikzcd} 
R^0\oo H_\C^\bu(X,\ \ )(Y)\ar[leftrightarrow]{d}\ar{r}&
R^0\oo H_\C^\bu(X,Y)\ar[leftrightarrow]{d}&
R^0\oo H_\C^\bu(\ \ ,Y)(X)\ar{l}\ar[leftrightarrow]{d}\\ 
%
R\oo H_{\oo K(\C)}(X,\ \ )(Y)\ar[leftrightarrow]{dr}\ar{r}&
R\oo H_{\oo K(\C)}(X,Y)\ar[leftrightarrow]{d}&
R\oo H_{\oo K(\C)}(\ \ ,Y)(X)\ar{l}\ar[leftrightarrow]{dl}\\ 
%
&\oo H_{\oo D(\C)}(X,Y),
\end{tikzcd}
\end{equation} 
where we have written $\oo H$ for $\Hom$ to save space, and where the horizontal arrows are the natural maps and the other arrows are the natural bijections, and where 
$$
R(\Hom_{\oo K(\C)}(X,\ \ ))(Y),\quad R\Hom_{\oo K(\C)}(X,Y),\quad R(\Hom_{\oo K(\C)}(\ \ ,Y))(X)
$$ 
are defined by Notation 10.3.8 p.~255 of the book. Then \qr{r0hom} commutes, and all its arrows are bijective. This implies 

$$
R\Hom_\C^\bu(X,Y)\iso R(\Hom_\C^\bu(X,\ \ ))(Y)\iso R(\Hom_\C^\bu(\ \ ,Y))(X).
$$
\end{s}

%%

\sbs{Exercise 13.15 p. 342}

Here is a partial solution. Let $\C$ be an abelian category. 

\begin{lem}\lb{738}
Let $Z\to Y\to X\to W\to0$ be an exact sequence and $Y\to V$ a morphism in $\C$, let $U$ be the fiber coproduct $V\oplus_YX$, and let $U\to W$ be the morphism which makes 
$$
\begin{tikzcd}
Z\ar{d}\ar{r}&Y\ar{d}\ar{r}&X\ar{d}\ar{r}&W\ar[equal]{d}\ar{r}&0\\ 
0\ar{r}&V\ar{r}&U\ar{r}&W\ar{r}&0
\end{tikzcd}
$$ 
a commutative diagram of complexes. Then the bottom row is exact.
\end{lem}

\begin{proof}
We shall use Lemma~\ref{8312} p.~\pr{8312}.

\nn Exactness at $V$: Let $V\to T$ be a morphism. By Lemma~\ref{8312}, (c)$\then$(a), it suffices to show that the diagram of solid arrows 
$$
\begin{tikzcd}
V\ar{r}\ar{d}&U\ar[dashed]{d}\\ 
T\ar[dashed,tail]{r}&S
\end{tikzcd}
$$ 
can be completed to a commutative diagram as indicated. To do this, we decree that the above completed square is cocartesian, and note that $T\to S$ is a monomorphism by Lemma \ref{8311} (b) (ii) p.~\pr{8311}.

\nn Exactness at $U$: Let $U\to T$ be a morphism whose composition with $V\to U$ is zero. By Lemma~\ref{8312}, (c)$\then$(a), it suffices to show that the commutative diagram of solid arrows 
$$
\begin{tikzcd}
V\ar{r}\ar{dr}[swap]{0}&U\ar{r}\ar{d}&W\ar[dashed]{d}\\ 
{}&T\ar[dashed,tail]{r}&S
\end{tikzcd}
$$ 
can be completed as indicated. This follows from the fact that, by Lemma~\ref{8312}, (a)$\then$(c), the commutative diagram of solid arrows 
$$
\begin{tikzcd}
Y\ar{r}\ar{dr}[swap]{0}&X\ar{r}\ar{d}&W\ar[dashed]{d}\\ 
{}&T\ar[dashed,tail]{r}&S
\end{tikzcd}
$$ 
can itself be completed as indicated.

\nn Exactness at $W$: Let $T\to W$ be a morphism. By Lemma~\ref{8312}, (b)$\then$(a), it suffices to show that the diagram of solid arrows 
$$
\begin{tikzcd}
S\ar[dashed,two heads]{r}\ar[dashed]{d}&T\ar{d}\\ 
U\ar{r}&W
\end{tikzcd}
$$ 
can be completed to a commutative diagram as indicated. This follows from the fact that, by Lemma~\ref{8312}, (a)$\then$(b), the commutative diagram of solid arrows 
$$
\begin{tikzcd}
S\ar[dashed,two heads]{r}\ar[dashed]{d}&T\ar{d}\\ 
X\ar{r}&W
\end{tikzcd}
$$ 
can itself be completed to a commutative diagram as indicated.
\end{proof}

Let $X$ and $Y$ be in $\C$, let $E$ be the set of short exact sequences 
$$
0\to Y\to Z\to X\to0,
$$ 
and let $\sim$ be the following equivalence relation on $E$: the exact sequences 
$$
0\to Y\to Z\to X\to0
$$ 
and 
$$
0\to Y\to W\to X\to0
$$ 
are equivalent if and only if there is a commutative diagram 
$$
\begin{tikzcd}
0\ar{r}&Y\ar[equal]{d}\ar{r}&Z\ar{d}\ar{r}&X\ar[equal]{d}\ar{r}&0\\ 
0\ar{r}&Y\ar{r}&W\ar{r}&X\ar{r}&0.
\end{tikzcd}
$$ 
(This is easily seen to be indeed an equivalence relation.) To the element 
$$
0\to Y\to Z\to X\to0
$$ 
of $E$ we attach the morphism in 
$$
\Hom_{\oo D(\C)}(X,Y[1])=\Ext^1_\C(X,Y)
$$ 
suggested by the diagram 
$$
\begin{tikzcd}
{}&X\\ 
Y\ar[equal]{d}\ar{r}&Z\ar{u}\\ 
Y,
\end{tikzcd}
$$ 
where each row is a complex (viewed as an object of $\oo D(\C)$), with the convention that only the possibly nonzero terms are indicated (the top morphism being a qis). 

We claim: 

(a) this process induces a map from $E/\!\!\sim$ to $\Ext^1_\C(X,Y)$, 

(b) this map (a) is bijective. 

Claim (a) is left to the reader. To prove (b) we construct the inverse map. To this end, we start with a complex $Z^\bullet$, a qis $Z^\bullet\to X$, and a morphism $Z^\bullet\to Y[1]$ representing our given element of $\Ext^1_\C(X,Y)$. The natural morphism $\tau^{\le0}Z^\bullet\to Z^\bullet$ being a qis, we can replace $Z^\bullet$ with $\tau^{\le0}Z^\bullet$, or, in other words, we may, and will, assume $Z^n\iso0$ for $n>0$. Letting $Z$ be the fiber coproduct $Y\oplus_{Z^{-1}}Z^0$, Lemma~\ref{738} p.~\pr{738} yields an exact sequence $0\to Y\to Z\to X\to0$. It is easy to see that this process defines a map from $\Ext^1_\C(X,Y)$ to $E/\!\!\sim$, and that this map is inverse to the map constructed before. q.e.d.

%%%

\section{About Chapter 14}

\sbs{Proposition 14.1.6 p.~349}

Here are some additional details about Step~(ii) of the proof of Proposition 14.1.6. 

We refer the reader to the book for a precise description of the setting. The following facts can be easily verified: 

The morphisms 
$$
f:X\to Y,\quad\pp:X\to I,\quad g:Y\to Z
$$ 
in $\A$ satisfy 
\begin{equation}\lb{minac}
f,\pp,g\text{ are in fact morphisms in }\A_c.
\end{equation}
We also have morphisms 
$$
\widetilde h:T^{-1}\to I,\quad\widetilde\psi:Y\to I,\quad\psi:Y\to I,\quad h:T^{-1}Y\to I,\quad\xi:Z\to I
$$ 
in $\A$, and we have the equalities 
\begin{equation}\lb{ppf}
\pp=\psi\ci f,
\end{equation} 
\begin{equation}\lb{hpm1}
h=T^{-1}d_I\ci T^{-1}\psi-\psi\ci T^{-1}d_Y,
\end{equation} 
\begin{equation}\lb{hpm2}
h=T^{-1}d_I\ci T^{-1}\psi+\psi\ci d_{T^{-1}Y},
\end{equation} 
\begin{equation}\lb{hhg}
h=\widetilde h\ci T^{-1}g,
\end{equation} 
\begin{equation}\lb{htilde}
\widetilde h=T^{-1}d_I\ci T^{-1}\xi-\xi\ci T^{-1}d_Z,
\end{equation} 
\begin{equation}\lb{psitildef}
\widetilde\psi=\psi-\xi\ci g. 
\end{equation}

To prove $h\ci T^{-1}f=0$, we note: 
\begin{align*}
h\ci T^{-1}f&=T^{-1}d_I\ci T^{-1}\psi\ci T^{-1}f+\psi\ci d_{T^{-1}Y}\ci T^{-1}f&\text{by \qr{hpm2}}\\ 
&=T^{-1}d_I\ci T^{-1}\pp+\psi\ci d_{T^{-1}Y}\ci T^{-1}f&\text{by \qr{ppf}}\\ 
&=T^{-1}d_I\ci T^{-1}\pp+\psi\ci f\ci d_{T^{-1}X}&\text{by \qr{minac}}\\ 
&=T^{-1}d_I\ci T^{-1}\pp+\pp\ci d_{T^{-1}X}&\text{by \qr{ppf}}\\ 
&=0&\text{by \qr{minac}}. 
\end{align*}

To prove that $\widetilde\psi$ is a morphism in $\A_c$, we note: 
\begin{align*}
d_I\ci\widetilde\psi-T\widetilde\psi\ci d_Y&=d_I\ci\psi-d_I\ci\xi\ci g-T\psi\ci d_Y+T\xi\ci Tg\ci d_Y&\text{by \qr{psitildef}}\\ 
&=(d_I\ci\psi-T\psi\ci d_Y)-(d_I\ci\xi\ci g-T\xi\ci Tg\ci d_Y)\\ 
&=Th-(d_I\ci\xi\ci g-T\xi\ci Tg\ci d_Y)&\text{by \qr{hpm1}}\\ 
&=Th-(d_I\ci\xi\ci g-T\xi\ci d_Z\ci g)&\text{by \qr{minac}}\\ 
&=Th-T\widetilde h\ci g&\text{by \qr{htilde}}\\ 
&=0&\text{by \qr{hhg}}. 
\end{align*}

%%

\sbs{Brief comments}

\begin{s} 
P.~350, last paragraph. In view of the comments made about Corollary~\ref{938} p.~\pr{938} and Theorem~\ref{961} p.~\pr{961}, one could replace ``there exists an essentially small full subcategory $\SSS$ of $\A_c$ such that \dots'' with ``there exists an infinite cardinal $\pi$ such that $(\A_c)_\pi$ is essentially small and satisfies \dots'', and replace $\SSS$ with $(\A_c)_\pi$ in (14.1.4) p.~351 of the book.
\end{s}

%

\begin{s}\lb{14112}
P.~352, Corollary 14.1.12. We also have the following corollary: 

Let $\U_0\subset\U$ be universes, let $(\A,T)$ be a Grothendieck $\U$-category with translation, and let $\A_0\subset\A$ be a fully abelian subcategory with translation. Assume that $\A_0$ is a Grothendieck $\U_0$-category. Then the natural functor $\D_c(\A_0)\to\D_c(\A)$ is fully faithful.

This follows immediately from Corollary 14.1.12 (i). 
\end{s}

%

\begin{s} 
P.~352, Corollary 14.1.12 (iv). Here are slightly more precise statements:

\nn(iii) the functor $Q:\oo K_c(\A)\to\oo D_c(\A)$ admits a right adjoint $R_q:\oo D_c(\A)\to\oo K_c(\A)$, this right adjoint is triangulated, satisfies $Q\ci R_q\iso\id_{\oo D_c(\A)}$, and is isomorphic to the composition of $\iota:\oo K_{c,\oo{hi}}(\A)\to\oo K_c(\A)$ and a quasi-inverse of $Q\ci\iota$,

\nn(iv) the right localization $(\oo D_c(\A),Q)$ of $\oo K_c(\A)$ exists. %is universal in the sense of Definition~\ref{url2} p.~\pr{url2}. 
(See \S\ref{732} p.~\pr{732}.)
\end{s}

%

\begin{s}
Corollary 14.3.2 p.~356. Let us add one sentence to the statement:
%
\begin{cor}\lb{1432}
Let $k$ be a commutative ring and let $\C$ be a Grothendieck $k$-abelian category. Then $(\oo K_{\oo{hi}}(\C),\oo K(\C)^{\op})$ is $\Hom_\C$-injective, and the functor $\Hom_\C$ admits a right derived functor 
$$
\oo{RHom}_\C:\oo D(\C)\tm\oo D(\C)^{\op}\to\oo D(k).
$$ 
If $X$ and $Y$ are in $\oo K(\C)$, then for any qis $Y\to I$ with $I$ in $\oo K_{\oo{hi}}(\C)$ (such exist) we have 
$$
\oo{RHom}_\C(X,Y)\xr\sim\Hom_{\oo K(\C)}(X,I)\xr\sim\Hom_{\oo D(\C)}(X,I).
$$ 
Moreover, $H^0(\oo{RHom}_\C(X,Y))\iso\Hom_{\oo D(\C)}(X,Y)$ for $X,Y$ in $\oo D(\C)$.
\end{cor}
\end{s}

%

\begin{s}%\lb{s144}
P.~357, Section 14.4. Having been unable to solve Part (iii) of Exercise 8.37 p.~211, I suggest the following changes to Section 14.4. (I might be missing something. If so, thank you for letting me know.)

\nn(a) Replace Assumption (14.4.1) p.~358 with: ``$\C$ admits inductive limits indexed by the ordered set $\bb N$, and such limits are exact''.

\nn(b) Say (only for the duration of this comment) that a full saturated subcategory $\A$ of a category $\B$ is {\em closed by coproducts} if the coproduct of any family of objects of $\A$ which exists in $\B$ belongs to $\A$.

\nn(c) In Lemma 14.4.2 p. 359, replace ``full triangulated'' with ``full saturated triangulated'', and ``closed by small direct sums'' with ``closed by direct sums (in the sense of the above definition)''. 

There are analog observations for the other statements of Section 14.4.
\end{s}

%

\begin{s}%\lb{s1448}
Statement of Theorem 14.4.8 p.~361. I know that the statement is already very long, but I shall consider here a minor variant which would make it even longer! More precisely, (14.4.5) could be stated as follows:

Let $X_i$ be in $\oo K(\C_i)$ for $i=1,2,3$, and let $P_i\to X_i$ ($i=1,2$) and $X_3\to I$ be qis with $X_i$ in $\widetilde\PP_i$ and $I$ in $\oo K_{\oo{hi}}(\C_3)$ (such exist). Consider the functorial morphisms of abelian groups
\begin{equation}\lb{1448}
\begin{split}
\Hom_{\oo D(\C_3)}(LG(X_1,X_2),X_3)&\xr a\Hom_{\oo D(\C_3)}(G(P_1,P_2),I)\xl b\\ 
\Hom_{\oo K(\C_3)}(G(P_1,P_2),I)&\iso\Hom_{\oo K(\C_1)}(P_1,F_1(P_2,I))\xr c\\ 
\Hom_{\oo D(\C_1)}(P_1,F_1(P_2,I))&\xl d\Hom_{\oo D(\C_1)}(X_1,RF_1(X_2,X_3)),
\end{split}
\end{equation}
where the middle isomorphism is the obvious one. Then $a,b,c,d$ are isomorphisms. There is an analogous statement for $F_2$.
\end{s}

%

\begin{s}
Step~(f) of the proof of Theorem 14.4.8 p.~364. We already know that $a,b,c,d$ in \qr{1448} are isomorphisms. As explained in the book, we have morphisms  
\begin{equation}\lb{f1}
\begin{split}
\oo{RHom}_{\C_3}(LG(X_1,X_2),X_3)&\to\\ 
\oo{RHom}_{\C_1}(RF_1(X_2,LG(X_1,X_2),RF_1(X_2,X_3))&\to\\ 
\oo{RHom}_{\C_1}(X_1,RF_1(X_2,X_3)).
\end{split}
\end{equation}
Applying $H^0$ we get, in view of Theorem 13.4.1 p.~337 of the book, morphisms 
\begin{equation}\lb{f2}
\begin{split}
\oo{Hom}_{\oo D(\C_3)}(LG(X_1,X_2),X_3)&\to\\ 
\oo{Hom}_{\oo D(\C_1)}(RF_1(X_2,LG(X_1,X_2),RF_1(X_2,X_3))&\to\\ 
\oo{Hom}_{\oo D(\C_1)}(X_1,RF_1(X_2,X_3)).
\end{split}
\end{equation}
By (1.5.7) p. 29 of the book, Composition~\qr{f2} coincides with Composition~\qr{1448}, and is, thus, an isomorphism. This implies that Composition~\qr{f1} is also an isomorphism.
\end{s}

%%%

\section{About Chapter 16}

\sbs{Sieves and local epimorphisms}\index{sieve}\index{local epimorphism}

This section is about the beginning of Section 16.1 p.~389 of the book. Let $\C$ be a category whose hom-sets are disjoint, let $M$ be the set of morphisms of $\C$, and for each $U$ in $\C$ let $M_U\subset M$ be the set of morphisms whose target is $U$. A subset $S$ of $M_U$ is a \textbf{sieve} \index{sieve} over $U$ if it is a right ideal of $M$, in the sense that $S$ contains all morphism of the form $s\ci f$ with $s$ in $S$. If $S$ is a sieve over $U$ and $f:V\to U$ is a morphism, we put
$$
S\tm_UV:=\{W\to V\ |\ (W\to V\to U)\in S\}.
$$
One easily checks that this is a sieve over $V$. 

To a sieve $S$ over $U$ we attach the subobject $A_S$ of $U$ in $\C^\wg$ by the formula
$$
A_S(V):=S\cap\Hom_\C(V,U).
$$ 
Conversely, to an object $A\to U$ of $(\C^\wg)_U$ we attach the sieve $S_{A\to U}$ over $U$ by putting 
$$
S_{A\to U}:=\{V\to A\to U\}.
$$ 

Let $\Sigma_U$ be the set of sieves over $U$. Let $(\Gamma_U)_{U\in\C}$ be a subfamily of the family $(\Sigma_U)_{U\in\C}$ and consider the following conditions:

\begin{cond}\lb{gt}
$\ $

\nn GT1: for all $U$ in $\C$ we have: $M_U\in\Gamma_U$,

\nn GT2: for all $U$ in $\C$ we have: $\Gamma_U\ni S\subset S'\in\Sigma_U\implies S'\in\Gamma_U$,

\nn GT3: for all $U$ in $\C$ we have: $S\in\Gamma_U,\ (V\to U)\in M\implies S\tm_UV\in\Gamma_V$,

\nn GT4: for all $U$ in $\C$ we have: 
$$
S\in\Gamma_U,\ S'\in\Sigma_U,\ S'\tm_UV\in\Gamma_V\ \forall\ (V\to U)\in S\implies S'\in\Gamma_U.
$$
\end{cond}

Consider the following conditions on a set $\E$ of morphisms in $\C^\wg$:

\nn LE1: $\id_U$ is in $\E$ for all $U$ in $\C$,

\nn LE2: if the composition of two elements of $\E$ exists, it belongs to $\E$,

\nn LE3: if the composition $v\ci u$ of two morphisms of $\C^\wg$ exists and is in $\E$, then $v$ is in $\E$,

\nn LE4: a morphism $A\to B$ in $\C^\wg$ is in $\E$ if and only if, for any morphism $U\to B$ in $\C^\wg$ with $U$ in $\C$, the projection $A\tm_BU\to U$ is in $\E$.

As proved in the book, 
\begin{equation}\lb{ecae}
\E\text{ contains all epimorphisms.}
\end{equation}

For the reader's convenience we paste the proof of \qr{ecae} (see p.~391 in the book):

Assume that $u:A\to B$ is an epimorphism in $\C^\wg$. If $w:U\to B$ is a morphism with $U$ in $\C$, there exists $v:U\to A$ such that $w=u\ci v$ by %Proposition 3.3.3 p.~82 of the book, stated above as 
Proposition~\ref{333} p.~\pr{333}, and Exercise 3.4 (i) p.~90 of the book, stated above as Proposition~\ref{34i} p.~\pr{34i}. Hence, $\id_U:U\to U$ factors as $U\to A\tm_BU\to U$. Therefore $A\tm_BU\to U$ is a local epimorphism by LE1 and LE3, and this implies that $A\to B$ is a local epimorphism by LE4. q.e.d.

The elements of $\E$ are called \textbf{local epimorphisms}.

Let $(\Gamma_U)_{U\in\C}$ be a subfamily of the family $(\Sigma_U)_{U\in\C}$ satisfying GT1\--GT4, let $\U$ be a universe such that $\C$ is $\U$-small, and let 
%
\begin{equation}\lb{cce}
\E=\E(\Gamma,\U)
\end{equation}
%
be the set of those morphisms $A\to B$ in $\C^\wg$ such that, for any morphism $U\to B$ in $\C^\wg$ with $U$ in $\C$, the sieve $S_{A\tm_BU\to U}$ is in $\Gamma_U$. 

\begin{lem}\lb{1613i}
A morphism $A\to U$ in $\C^\wg$ with $U$ in $\C$ is in $\E$ if and only if $S_{A\to U}$ is in $\Gamma_U$. 
\end{lem}

\begin{proof}
Observe first that, in the setting 
$$
A\to U\leftarrow V
$$ 
(obvious notation), we have 
\begin{equation}\lb{28a}
S_{A\tm_UV\to V}=S_{A\to U}\tm_UV.
\end{equation} 
Let $A\to U$ be a morphism in $\C^\wg$ with $U$ in $\C$. Consider the conditions (with obvious notation) 
\begin{equation}\lb{28b}
(A\to U)\in\E,
\end{equation}
\begin{equation}\lb{28c}
S_{A\tm_UV\to V}\in\Gamma_V\quad\forall\quad V\to U,
\end{equation}
\begin{equation}\lb{28d}
S_{A\to U}\tm_UV\in\Gamma_V\quad\forall\quad V\to U,
\end{equation}
\begin{equation}\lb{28e}
S_{A\to U}\in\Gamma_U.
\end{equation} 
We have \qr{28b} $\ssi$ \qr{28c} by definition of $\E$, and \qr{28c} $\ssi$ \qr{28d} by Lemma~\ref{1613i}, \qr{28d} $\then$ \qr{28e} by GT1 and GT4, and \qr{28e} $\then$ \qr{28d} by GT3.
\end{proof}

Let us check that $\E$ satisfies LE1-LE4:

\nn LE1 follows immediately from GT1.

\nn LE2: Let $A\to B\to C$ be a diagram in $\C^\wg$, and assume that the two arrows are in $\E$. Consider the diagram of solid arrows with cartesian squares 
$$
\begin{tikzcd}
F\ar[dashed]{r}\ar[dashed]{d}&V\ar[dashed]{d}\\ 
D\ar{r}\ar{d}&E\ar{r}\ar{d}&U\ar{d}\\ 
A\ar{r}&B\ar{r}&C
\end{tikzcd}
$$
in $\C^\wg$ (with $U$ in $\C$). We have that $S_{E\to U}$ is in $\Gamma_U$ (because $B\to C$ is in $\E$) and we must prove that $S_{D\to U}$ is in $\Gamma_U$. Let $V\to U$ be in $S_{E\to U}$, and let us complete the diagram with cartesian squares as indicated. By GT4 it suffices to check that $S_{F\to V}$ is in $\Gamma_V$. But this follows from the assumption that $A\to B$ is in $\E$ (together with a transitivity property of cartesian squares which has already been tacitly used).

\nn LE3 follows immediately from GT2.

\nn LE4 follows immediately from Lemma~\ref{1613i}. %Let $A\to B$ be a morphism in $\C^\wg$. We must check: 
%\begin{center}
%$A\to B$ is in $\E$\\ $\iff$\\ for any morphism $U\to B$ in $\C^\wg$ with $U$ in $\C$, the projection $A\tm_BU\to U$ is in $\E$.
%\end{center}
%Implication $\then$ is obvious, and Implication $\si$ follows from Lemma~\ref{1613i}. 

Conversely, given an object $U$ of $\C$ and a set $\E$ of morphisms in $\C^\wg$ satisfying LE1-LE4, put
$$
\Gamma_U:=\{S\in\Sigma_U\ |\ (A_S\to U)\in\E\}.
$$
Let us check that $\Gamma(\E):=(\Gamma_U)_{U\in\C}$ satisfies GT1-GT4.

GT1 follows from LE1 and the equality $(A_{M_U}\to U)=\id_U$. 

GT2 follows from LE3 and the fact that, in the setting 
$$
\Gamma_U\ni S\subset S'\in\Sigma_U\implies S'\in\Gamma_U,
$$ 
the morphism $A_S\to U$ factors as $A_S\to A_{S'}\to U$. 

To prove GT3, note that if $S$ is a sieve over $U$ and $V\to U$ is a morphism in $\C$, then we have 
%
\begin{equation}\lb{asuv}
A_{S\tm_UV}=A_S\tm_UV.
\end{equation}
%
In view of LE4, this implies GT3.  

The lemma below will helps us verify GT4. 

\begin{lem}\lb{prepagt4}
Let $s:V\to U$ be a morphism in $\C$ and $S$ a sieve over $U$. Then $s$ is in $S$ if and only if $s$ factors through the natural morphism $i:A_S\to U$.
\end{lem}

\begin{proof}
By the Yoneda Lemma (Lemma~\ref{yol} p.~~\pr{yol}), there is a bijection 
$$
S\cap\Hom_\C(V,U)\xr\pp\Hom_{\C^\wg}(V,A_S)
$$
such that $\pp(s)_W=s\ci$ for all $W$ in $\C$. 

Assume that $s$ is in $S$ and let us show that there is a morphism $v:V\to A_S$ satisfying $i\ci v=s$. It suffices to prove $i\ci\pp(s)=s$ and to put $v:=\pp(s)$. We have for all $W$ in $\C$
$$
(i\ci\pp(s))_W=i_W\ci\pp(s)_W=i_W\ci(s\ci)=s\ci=s_W.
$$ 

Conversely, assuming that $v$ is in $\Hom_{\C^\wg}(V,A_S)$, it suffices to prove that $i\ci v$ is in $S$. We have 
$$
i\ci v=(i\ci v)\ci\id_V=(i\ci v)_V(\id_V)=i_V(v_V(\id_V))=v_V(\id_V)\in A_S(V)\subset S. 
$$ 
\end{proof} 

\begin{lem}
Condition GT4 holds.
\end{lem}

\begin{proof} 
Let us assume 
%
\begin{equation}\lb{sgu}
S\in\Gamma_U,\ S'\in\Sigma_U,\ S'\tm_UV\in\Gamma_V\ \forall\ (V\to U)\in S.
\end{equation}
%
It suffices to check $S'\in\Gamma_U$, or, equivalently, 
%
\begin{equation}\lb{as'}
(A_{S'}\to U)\in\E.
\end{equation}

Form the cartesian square 
$$
\begin{tikzcd}
B\ar{r}\ar{d}&A_S\ar{d}\\ 
A_{S'}\ar{r}&U.
\end{tikzcd}
$$ 
As $A_S\to U$ is in $\E$ by assumption, it suffices, by LE2 and LE3, to check 
\begin{equation}\lb{bas}
(B\to A_S)\in\E.
\end{equation} 
Let $V\to A_S$ be a morphism in $\C^\wg$ with $V$ in $\C$, and let 
$$
\begin{tikzcd}
C\ar{r}\ar{d}&V\ar{d}\\ 
B\ar{r}&A_S
\end{tikzcd}
$$ 
be a cartesian square. By LE4 it is enough to verify 
\begin{equation}\lb{cv}
(C\to V)\in\E.
\end{equation} 
The morphism $V\to U$ being in $S$ by Lemma~\ref{prepagt4}, the sieve $S'\tm_UV$ is in $\Gamma_V$ by \qr{sgu}, and $A_{S'\tm_UV}\to V$ is in $\E$ by definition of $\Gamma_V$. We have 
$$
\E\ni(A_{S'\tm_UV}\to V)\iso(A_{S'}\tm_UV\to V)\iso(C\to V).
$$ 
Indeed, the first isomorphism holds by \qr{asuv} p.~\pr{asuv}, and the second one holds because the rectangle 
$$
\begin{tikzcd}
C\ar{r}\ar{d}&V\ar{d}\\ 
B\ar{r}\ar{d}&A_S\ar{d}\\ 
A_{S'}\ar{r}&U
\end{tikzcd}
$$ 
is cartesian. This proves successively \qr{cv}, \qr{bas}, \qr{as'} and the lemma.  
\end{proof}

We have proved that $(\Gamma_U)_{U\in\C}$ satisfies GT1-GT4. 

It is now easy to prove 

\begin{thm}
If $\C$ is a $\U$-small category, if $\Gamma$ is a subfamily of $(\Sigma_U)_{U\in\C}$ satisfying GT1-GT4, and if $\E$ is a set of morphisms in $\C^\wg_\U$ satisfying LE1-LE4, then the equalities $\E=\E(\Gamma,\U)$ and $\Gamma=\Gamma(\E)$ are equivalent.
\end{thm}

\begin{cor}\lb{leu}
Let $\U\subset\U'$ be universes, let $\C$ be a $\U$-small category, let $\Gamma$ be a subfamily of $(\Sigma_U)_{U\in\C}$ satisfying GT1-GT4 (see Conditions~\ref{gt} p.~\pr{gt}), let $u:A\to B$ be a morphism in $\C^\wg_\U$, and let $u':A'\to B'$ be the corresponding morphism in $\C^\wg_{\U'}$. Then $u$ is in $\E(\Gamma,\U)$ (see \qr{cce} p.~\pr{cce}) if and only if $u'$ is in $\E(\Gamma,\U')$.
\end{cor}

%%

\sbs{Brief comments}

\begin{s}\lb{390}
P.~390, Axioms LE1-LE4. The set of local epimorphisms attached to the natural Grothendieck topology associated with a small topological space $X$ can be described as follows. 

Let $f:A\to B$ be a morphism in $\C^\wg$, where $\C$ is the category of open subsets of $X$. For each pair $(U,b)$ with $U$ in $\C$ and $b$ in $B(U)$ let $\Sigma(U,b)$ be the set of those $V$ in $\C_U$ such that there is an $a$ in $A(V)$ satisfying $f_V(a)=b_V$, where $b_V$ is the restriction of $b$ to $V$. Then $f$ is a local epimorphism if and only if 
$$
U=\bigcup_{V\in\Sigma(U,b)}V
$$ 
for all $(U,b)$ as above.

Moreover, a morphism $u:A\to U$ in $(\oo{Op}_X)^\wg$ with $U$ in $\oo{Op}_X$ is a local epimorphism if and only if for all $x$ in $U$ there is a $V$ in $\oo{Op}_X$ such that $x\in V$ and $A(V)\ne\vi$. 
\end{s}

% https://docs.google.com/document/d/1vpjtCfl-qZCuuI-wkR5rbsav31DfDsZx1mQteXzu-UM/edit

\begin{s}
For any universe $\U$, any $\U$-small category $\C$ and any subfamily $\Gamma$ of $(\Sigma_U)_{U\in\C}$ satisfying GT1-GT4 (see Conditions~\ref{gt} p.~\pr{gt}), let $\M(\Gamma,\U)$ and $\I(\Gamma,\U)$ denote respectively the set of local monomorphisms and local isomorphisms attached to $\E(\Gamma,\U)$ (see \qr{cce} p.~\pr{cce}). Corollary \ref{leu} p. \pr{leu} implies:

Let $\U\subset\U'$ be universes, let $\C$ be a $\U$-small category, let $\Gamma$ be a subfamily of $(\Sigma_U)_{U\in\C}$ satisfying GT1-GT4, let $u:A\to B$ be a morphism in $\C^\wg_\U$, and let $u':A'\to B'$ be the corresponding morphism in $\C^\wg_{\U'}$. Then $u$ is in $\M(\Gamma,\U)$ (resp. in $\I(\Gamma,\U)$) if and only if $u'$ is in $\M(\Gamma,\U')$ (resp. in $\I(\Gamma,\U')$).
\end{s}

%

\begin{s}\lb{s1623}
P.~395, Lemma 16.2.3 (iii). Consider the conditions

\nn(b) for any diagram $C\parar A\to B$ such that $C$ is in $\C$ and the two compositions coincide, there exists a local epimorphism $D\to C$ such that the two compositions $D\to C\parar A$ coincide,

\nn(c) for any diagram $C\parar A\to B$ such that $C$ is in $\C^\wg$ and the two compositions coincide, there exists a local epimorphism $D\to C$ such that the two compositions $D\to C\parar A$ coincide,

\nn(d) for any diagram $C\parar A\to B$ such that $C$ is in $\C$ and the two compositions coincide, there exists a local isomorphism $D\to C$ such that the two compositions $D\to C\parar A$ coincide,

\nn(e) for any diagram $C\parar A\to B$ such that $C$ is in $\C^\wg$ and the two compositions coincide, there exists a local isomorphism $D\to C$ such that the two compositions $D\to C\parar A$ coincide.

Recall that (a) is the condition that $A\to B$ is a local monomorphism. Lemma 16.2.3 p.~395 of the book implies 
%
\begin{equation}\lb{e1623}
\text{Conditions (a), (b), (c), (d), (e) are equivalent.}
\end{equation}
%
Indeed, Part (iii) of the lemma says that (a), (b) and (c) are equivalent. Clearly (e) implies (c) and (d), and (d) implies (b). It suffices to check that (c) implies (e). Let $C\parar A\to B$ be as in the assumption (c), let $D\to C$ be the local epimorphism furnished by (c), and let $I$ be its image. The two compositions $I\to C\parar A$ coincide because $D\to I$ is an epimorphism, and $I\to C$ is a local isomorphism by Part (ii) of the lemma. q.e.d.
\end{s}

%

\begin{s} 
P.~397, Notation 16.2.5 (ii). The fact that 
\begin{equation}\lb{1625}
\text{such a $w$ is necessarily a local isomorphism}
\end{equation}
follows from Lemma 16.2.4 (vii) p. 396.
\end{s}

%

\begin{s} 
P.~398, proof of Lemma 16.2.7: see \S\ref{fpl} p.~\pr{fpl}.
\end{s}

% https://docs.google.com/document/d/1WVqlvFLqBG4k-afo0HVq-VV57s9WLch15mOq5CyTTg4/edit

\begin{s}
Right after Display (16.3.1) p.~399 of the book, in view of the natural isomorphism 
$$
A^a(U)\iso\Hom_{({\C^\wg})_{\mc{LI}}}(Q(U),Q(A)),
$$ 
the map $A^a(U')\to A^a(U)$ induced by a morphism $U\to U'$ can also be described by the diagram 
$$
Q(U)\to Q(U')\to Q(A).
$$ 
Similarly, the map $A(U)\to A^a(U)$ at the top of p.~400 of the book can also be described by the diagram 
$$
A(U)\iso\Hom_{\C^\wg}(U,A)\to\Hom_{({\C^\wg})_{\mc{LI}}}(Q(U),Q(A))\iso A^a(U).
$$ 
Then Lemma 16.3.1 can be stated as follows. 

If 
$$
U\xl sB\xr uA
$$ 
is a diagram in $\C^\wg$ with $U$ in $\C$ and $s$ a local isomorphism, and if 
$$
v=Q(u)\ci Q(s)^{-1}\in A^a(U)\iso\Hom_{({\C^\wg})_{\mc{LI}}}(Q(U),Q(A)), 
$$ 
then 
\begin{equation}\lb{vseu}
v\ci s=\ee(A)\ci u.
\end{equation}

Indeed, \qr{vseu} is equivalent to $v\ci Q(s)=Q(u)$.
\end{s}

%

\begin{s} 
P.~400, Step~(ii) in the proof of Lemma 16.3.2 (additional details):

We want to prove that $A\to A^a$ is a local monomorphism. In view of \qr{e1623} p.~\pr{e1623} it suffices to check that Condition~(b) of \S\ref{s1623} p.~\pr{s1623} holds. 

Recall that the functor 
$$
\al:(\mc{LI}_U)^{\op}\to\Set,\quad(B\xr sU)\mt\Hom_{\C^\wg}(B,A)
$$ 
satisfies $\col\al\iso A^a(U)$ (see (16.3.1) p.~399 of the book). Let $i(s):\al(s)\to A^a(U)$ be the coprojection, and let $f_1,f_2:U\parar A$ be two morphisms such that the compositions $U\parar A\to A^a$ coincide. By definition of the natural morphism $A\to A^a$, we have 
$$
i(\id_U)(f_1)=i(\id_U)(f_2).
$$ 
By the fact that $\mc{LI}_U$ is cofiltrant, and by Proposition 3.1.3 p.~73 of the book, there is a morphism 
$$
\pp:(B\xr sU)\to(U\xr{\id_U}U)
$$ 
in $\mc{LI}_U$ such that $\al(\pp)(f_1)=\al(\pp)(f_2)$. This means that the compositions $B\to U\parar A$ coincide. q.e.d.
\end{s}

%

\begin{s} P.~401, Step~(i) of the proof of Proposition 16.3.3. See \qr{e1623} p.~\pr{e1623} and \qr{1625} p.~\pr{1625}. (As already mentioned, $B''\to B$ should be $B''\to B'$.)
\end{s}

%%%

\section{About Chapter 17}

\sbs{Brief comments}

\begin{s}
P.~405, Chapter 17. It seems to me it would be more convenient to denote by $f^t$ the functor from $(\C_Y)^{\op}$ to $(\C_X)^{\op}$ (and {\em not} the functor from $\C_Y$ to $\C_X$) which defines $f$. To avoid confusion, we shall adopt here the following convention:

If $f:X\to Y$ is a morphism of presites, \index{morphism of presites} then we keep the notation $f^t$ for the functor from $\C_Y$ to $\C_X$, and we designate by $f^\tau$ \index{$f^\tau$} the functor from $(\C_Y)^{\op}$ to $(\C_X)^{\op}$:
\begin{equation}\lb{ttau}
f^t:\C_Y\to\C_X,\quad f^\tau:(\C_Y)^{\op}\to(\C_X)^{\op}.
\end{equation}
In other words, we set 
$$
\boxed{f^\tau:=(f^t)^{\op}}
$$ 
We keep the same definition of left exactness (based on $f^t$) of $f:X\to Y$ as in the book.

The motivation for introducing the functor $f^\tau$ can be described as follows: The diagram 
$$
\begin{tikzcd}
J\ar{dr}\ar{rr}{\pp}&&I\ar{dl}\\ 
{}&\C,
\end{tikzcd}
$$ 
representing the general setting of Section 2.3 p.~50 of the book, is now replaced by the commutative diagram 
$$
\begin{tikzcd}
(\C_Y)^{\op}\ar{dr}\ar{rr}{f^\tau}&&(\C_X)^{\op}\ar{dl}\\ 
{}&\A.
\end{tikzcd}
$$ 
(See also \S\ref{fhat} p.~\pr{fhat} and \S\ref{fdagger} p.~\pr{fdagger}.)
\end{s}

%

\begin{s}\lb{fhat}
P.~406. Recall that, in the first line of the second display, $(\C_Y)^\wg$ should be $\C_Y$ (twice). In notation \qr{ttau}, Formula \qr{275} p.~\pr{275} gives, for $B$ in $\C_Y^\wg$ and $U$ in $\C_X$, 
%
\begin{equation}\lb{efhat1}
\fthat(B)(U)\iso\col_{(V\to B)\in(\C_Y)_B}\Hom_{(\C_X)}(U,f^t(V))\iso\col_{(U\to f^t(V))\in(\C_Y)^U}B(V).
\end{equation}
%
For the sake of emphasis, we state: 

\begin{prop}\lb{p406}
The functor $\fthat$ commutes with small inductive limits (Proposition 2.7.1 p.~62 of the book, Remark~\ref{272} p.~\pr{272} above). Moreover, if $f$ is left exact, then $\fthat$ is exact (Corollary 3.3.19 p.~87 of the book).
\end{prop}

If $f:X\to Y$ is a continuous map of small topological spaces, if $B$ is in $(\oo{Op}_Y)^\wg$ and $U$ in $\oo{Op}_X$, then \qr{efhat1} gives 
%
\begin{equation}\lb{efhat2}
\fthat(B)(U)\iso\col_{f^{-1}(V)\supset U}B(V).
\end{equation}
%
\end{s}

%

\begin{s}\lb{fdagger} 
P.~407. Let $f:X\to Y$ be a morphism of presites and let $\A$ be a category admitting small inductive and projective limits. In the notation of \qr{ttau} p.~\pr{ttau}, we set 
$$
\boxed{f_*:=(f^\tau)_*}\quad,\quad\boxed{f^\dg:=(f^\tau)^\dg}\quad,\quad\boxed{f^\ddg:=(f^\tau)^\ddg}\quad,
$$ 
yielding
$$
f^\dg,f^\ddg:\PSh(X\A)\to\PSh(Y,\A).
$$
Then (17.1.3) and (17.1.4) follow respectively from (2.3.6) and (2.3.7) p.~52 of the book. For the sake of completeness, let us rewrite (17.1.3) and (17.1.4) (in the notation of \qr{ttau}):
%
\begin{equation}\lb{1713}
f^\dg(G)(U)=\col_{(f^\tau(V)\to U)\in((\C_Y)^{\op})_U}G(V),
\end{equation}
%
with $G$ in $\PSh(Y,\A),U$ in $\C_X$, $f^\tau(V)\to U$ being a morphism in $(\C_X)^{\op}$ (corresponding to a morphism $U\to f^t(V)$ in $\C_X$), 
%
\begin{equation}\lb{1714}
f^\ddg(G)(U)=\lim_{(U\to f^\tau(V))\in((\C_Y)^{\op})^U}G(V),
\end{equation}
%
with $G$ in $\PSh(Y,\A),U$ in $\C_X$, $U\to f^\tau(V)$ being a morphism in $(\C_X)^{\op}$ (corresponding to a morphism $f^t(V)\to U$ in $\C_X$).
\end{s}

%

\begin{s}\lb{s408}
P.~408, comment preceding Convention 17.1.6. Let us recall the comment: 

We extend presheaves over $X$ to presheaves over $\widehat X$ using the functor $\oo h_X^\ddg$ associated with the Yoneda embedding $\oo h_X^t=\oo h_{\C_X}$. Hence, for $F$ in $\PSh(X,\A)$ and $A$ in $\C_X^\wg$, we have 
$$
(\oo h_X^\ddg F)(A)=\lim_{(U\to A)\in(\C_X)_A}F(U).
$$ 
By Corollary 2.7.4 p.~63 of the book, the functor 
$$
\oo h_X^\ddg:\PSh(X,\A)\to\PSh(\widehat X,\A)
$$ 
induces an equivalence of categories between $\PSh(X,\A)$ and the full subcategory of $\PSh(\widehat X,\A)$ whose objects are the $\A$-valued presheaves over $\widehat X$ which commute with small projective limits. 

One can add that a quasi-inverse is given by 
$$
\oo h_{X*}:\PSh(\widehat X,\A)\to\PSh(X,\A). 
$$ 
\end{s}

%

\begin{s}
P.~408, Convention 17.1.6. Recall the convention: If $F$ is an $\A$-valued presheaf over $X$ and $A$ is a presheaf of sets over $X$, then we put 
%
\begin{equation}\lb{408}
F(A):=(\oo h_X^\ddg F)(A)=\lim_{(U\to A)\in(\C_X)_A}F(U).
\end{equation}
% 
(Note that the same comment is made at the beginning of Section 17.3 p.~414.) This convention of extending each presheaf $F$ over $X$ to a presheaf, still denoted by $F$, over $\widehat X$ which commutes with small projective limits implies that we have, for $A,B$ in $\C^\wg$, 
$$
B(A)\iso\Hom_{\C_X^\wg}(A,B).
$$ 

In the notation of \S\ref{opddagg} p.~\pr{opddagg}, Convention 17.1.6 can be described as follows:

If $X$ is a site, if $\C$ is the corresponding category, if $h:\C\to\C^\wg$ is the Yoneda embedding, if $F$ is an $\A$ valued sheaf over $X$, and if $A$ is an object of $\C^\wg$, then Convention 17.1.6 consists in putting 
$$
F(A):=(h^{\op})^\ddg(F)(A).
$$ 
\end{s}

%

\begin{s}\lb{prepa5}
P.~409, Proposition 17.1.9 follows immediately from \qr{prepa1} p.~\pr{prepa1}, \qr{prepa2} p.~\pr{prepa2} and \qr{prepa3} p.~\pr{prepa3}.
\end{s}

%

\begin{s}\lb{17115ps}
%$\frownie$\footnote{The symbol $\frownie$ (``frowny'') means that I'm not satisfied with this \S, and that I'm planning to rewrite it.} 
P.~410, Display (17.1.15): As already indicated in \S\ref{17115typo}, Display (17.1.15) p.~410 should read 
$$
\Hom_{\PSh(X,\A)}(F,G)\iso\lim_{U\in\C_X}\HOM_{\PSh(X,\A)}(F,G)(U).
$$ 
%\begin{uspb}I don't understand why $\HOM_{\PSh(X,\A)}(F,G)(U)$ depends functorially on $U\in\C_X$.\index{unsolved problem}\end{uspb}
%This follows from Theorem~\ref{fgd} p.~\pr{fgd}. %\qr{17115mod} p.~\pr{17115mod}. 
\end{s} 

%

%\begin{s}
%P.~411, Definition 17.2.1 of the notions of sites and of morphism of site. Proposition~\ref{yf} p.~\pr{yf} implies:

%If $\U$ is a universe, then there is a $\U$-category $\A$ (see Definition~\ref{myucat} p.~\pr{myucat}) whose objects are the $\U$-small categories (see Definition~\ref{myuscat} p.~\pr{myuscat}) and whose morphisms are are morphisms of sites.
%\end{s}

%

\begin{s}
P.~412, proof of Lemma 17.2.2 (ii), (b)$\then$(a), Step~(1): $\fthat$ is right exact by Proposition~\ref{333} p.~\pr{333} and Proposition~\ref{p406} p.~\pr{p406}.
\end{s}

%

\begin{s}
P.~412, proof of Lemma 17.2.2 (ii), (b)$\then$(a), Step~(3). See \S\ref{1722} p.~\pr{1722}. This is essentially a copy and paste of the book.

Claim: if a local isomorphism $u:A\to B$ in $\C_Y^\wg$ is either a monomorphism or an epimorphism, then $\fthat(u)$ is a local isomorphism in $\C_X^\wg$. 

Proof of the claim: Let $V\to B$ be a morphism in $\C_Y^\wg$ with $V$ in $\C_Y$. Then $u_V: A\tm_BV\to V$ is either a monomorphism or an epimorphism by Proposition~\ref{sbcs} p.~\pr{sbcs} and Proposition~\ref{34i} p.~\pr{34i}. Let us show that $\fthat(u_V)$ is a local isomorphism. 

If $u_V$ is a monomorphism, $\fthat(u_V)$ is a local isomorphism by assumption. 

If $u_V$ is an epimorphism, then $u_V$ has a section $s:V\to A\tm_BV$. Since $u_V$ is a local isomorphism by Lemma 16.2.4 (i) p.~395 of the book, $s$ is a local isomorphism. Since 
$$
\fthat(u_V)\ci\fthat(s)\iso\id_{f^t(V)}
$$ 
is a local monomorphism, and $\fthat(s)$ is a local epimorphism by Step~(2), Lemma 16.2.4 (vi) p.~396 of the book implies that $\fthat(u_V)$ is a local monomorphism. Since $\fthat(u_V)$ is an epimorphism by Step~(2), we see that $\fthat(u_V)$ is a local isomorphism. This proves the claim.

Taking the inductive limit with respect to $V\in(\C_Y)_B$, we conclude by Proposition 16.3.4 p.~401 of the book that $\fthat(u)$ is a local isomorphism.
\end{s}

%

\begin{s}\lb{1724ii}
P.~413, Definition 17.2.4 (ii): see  Remark~\ref{272b} p.~\pr{272b}.
\end{s}

%

\begin{s}
P.~413. Lemma 17.2.5 (ii) and Exercise 2.12 (ii) p.~66 of the book imply: If $f:X\to Y$ is weakly left exact, then $\fthat:\C_Y^\wg\to\C_X^\wg$ commutes with projective limits indexed by small connected categories.
\end{s} 

%

\begin{s}\lb{1725ii} 
P.~413, Lemma 17.2.5 (ii). Here is a corollary: 

Let $f:X\to Y$ be a weekly left exact morphism of sites such that $\fthat(u)$ is a local epimorphism if and only if $u$ is a local epimorphism. Then $\fthat(u)$ is a local monomorphism if and only if $u$ is a local monomorphism, and $\fthat(u)$ is a local isomorphism if and only if $u$ is a local isomorphism. 
\end{s} 

%

\begin{s}
P.~413, Example 17.2.7 (i). Recall that $f:X\to Y$ is a continuous map of small topological spaces. As explained in the book, to see that $f$ is a morphism of sites, it suffices to check that, if $u:B\to V$ is a local epimorphism in $(\oo{Op}_Y)^\wg$ with $V$ in $\oo{Op}_Y$, then $\fthat(B)\to f^{-1}(V)$ is a local epimorphism in $(\oo{Op}_X)^\wg$. This follows immediately from \S\ref{390} p.~\pr{390} and \qr{efhat2} p.~\pr{efhat2}. 
\end{s}

%

\begin{s} 
P.~414, Definition 17.2.8 (minor variant):

\begin{df}[Definition 17.2.8 p.~414, Grothendieck topology]\lb{1778} 
Let $X$ be a small presite. We assume, as we may, that the hom-sets of $\C_X$ are disjoint. A {\em Grothendieck topology}\index{Grothendieck topology} on $X$ is a set $\T$ of morphisms of $\C_X$ which satisfies Axioms LE1-LE4 p.~390. Let $\T'$ and $\T$ be Grothendieck topologies. We say that $\T$ is {\em stronger than} $\T'$, or that $\T'$ is {\em weaker than} $\T$, if $\T'\subset\T$. 
\end{df}

Let $(\T_i)$ be a family of Grothendieck topologies. We observe that $\bigcap\T_i$ is a Grothendieck topology, and we denote by $\bigvee\T_i$ the intersection of all Grothendieck topologies containing $\bigcup\T_i$.
\end{s}

%

\begin{s}
P.~415, Isomorphism (17.3.1). Recall briefly the setting. We have: 
$$
F\in\PSh(X,\A),\quad M\in\A,\quad U\in\C_X,
$$ 
and we claim 
%
\begin{equation}\lb{1731}
\HOM_{\PSh(X,\A)}(M,F)(U)\iso\Hom_\A(M,F(U)).
\end{equation} 
%
Here and in the sequel, we denote again by $M$ the constant presheaves over $X$ and $U$ attached to the object $M$ of $\A$. Note that, by \S\ref{s408} p.~\pr{s408}, this isomorphism can be written 
$$
\HOM_\A(M,F)\iso\Hom_\A(M,F(\ )).
$$ 
To prove \qr{1731}, observe that we have  
$$
\HOM_{\PSh(X,\A)}(M,F)(U)\iso\Hom_{\PSh(U,\A)}(\oo j_{U\to X*}M,\oo j_{U\to X*}F)
$$ 
$$
\iso\Hom_{\PSh(U,\A)}(M,\oo j_{U\to X*}F), 
$$ 
the two isomorphisms following respectively from the definition of $\HOM_{\PSh(X,\A)}$ given in (17.1.14) p.~410 of the book, and from the definition of the functor $\oo j_{U\to X*}$, so that we must show 
$$
\Hom_{\PSh(U,\A)}(M,\oo j_{U\to X*}F)\iso\Hom_\A(M,F(U)).
$$
We define maps 
$$
\begin{tikzcd}
\Hom_{\PSh(U,\A)}(M,\oo j_{U\to X*}F)\ar[yshift=0.7ex]{r}{\pp}&\Hom_\A(M,F(U))\ar[yshift=-0.7ex]{l}{\psi}
\end{tikzcd} 
$$ 
as follows: If $p:M\to\oo j_{U\to X*}F$ is a morphism in $\PSh(U,\A)$, given by morphisms $p(V\to U):M\to F(V)$ in $\A$, then we put $\pp(p):=p(U\xr{\id_U}U)$; if $a:M\to F(U)$ is a morphism in $\A$, then we put $\psi(a)(V\xr cU):=F(c)\ci a$; and we check that $\pp$ and $\psi$ are mutually inverse bijections. 
\end{s}

%

\begin{s} 
P.~418, proof of Lemma 17.4.2 (minor variant): Consider the natural morphisms 
$$
\col\al\xr f\col\al\ci\mu_u^{\op}\ci\ld_u^{\op}\xr g\col\al\ci\mu_u^{\op}\xr h\col\al.
$$
We must show that $g\ci f$ is an isomorphism. The equality $h\ci g\ci f=\id_{\col\al}$ is easily checked. Being a right adjoint, $\mu_u^{\op}$ is left exact, hence cofinal by Lemma 3.3.10 p.~84 of the book, and $h$ is an isomorphism. q.e.d.
\end{s}  

%

\begin{s}\lb{1744i}
P.~419, proof of Proposition 17.4.4: 

\nn First sentence of the proof: see \S\ref{fpl} p.~\pr{fpl}. 

%\begin{uspb}
%%$\frownie$ (``frowny'').
%Step~(i), Line~4: I don't know why $K^{\op}$ is cofinally small.\index{unsolved problem}
%%Step~(i), Line~4: The fact that $K^{\op}$ is cofinally small and filtrant results from \S\ref{poc} p.~\pr{poc} together with Lemma 16.2.8 p.~398, Lemma 16.2.7 p.~398 and Proposition 3.2.1 (iii) p.~78 of the book. 
%\end{uspb} 

\nn Step~(i), Line~4: The fact that $K^{\op}$ is %cofinally small and 
filtrant results from %\S\ref{poc} p.~\pr{poc} together with Lemma 16.2.8 p.~398, 
Lemma 16.2.7 p.~398 and Proposition 3.2.1 (iii) p.~78 of the book. 

The key ingredient to prove that $K^{\op}$ is cofinally small is Lemma~16.2.8 p.~398 which says that $K^{\op}$ is a product of cofinally small categories. To prove that $K^{\op}$ is cofinally small one must prove that a certain product $\prod P_i$ of connected categories is connected, but, as a product of connected categories is \emph{not} connected in general, some caution is needed. Going through the proof of Lemma~16.2.8, we see that each $P_i$ is \emph{filtrant}. This implies that $\prod P_i$ is filtrant, and thus that it is connected.

\nn Step~(i), additional details about the chain of isomorphisms at the bottom of p.~419 of the book: The chain reads 
$$
\prod_i\ F^b(A_i)
\os{(\text a)}{\iso}\prod_i\ \col_{(B_i\to A_i)\in\mc{LI}_{A_i}}F(B_i)
\os{(\text b)}{\iso}\col_{(B_i\to A_i)_{i\in I}\in K}\ \prod_i\ F(B_i)
$$
$$
\os{(\text c)}{\iso}\col_{(B_i\to A_i)_{i\in I}\in K}\ F\left(``\bigsqcup_i\!"B_i\right)
\os{(\text d)}{\iso}\col_{(B\to A)\in\mc{LI}_A}F(B)
\os{(\text e)}{\iso}F^b(A), 
$$ 
and the isomorphisms can be justified as follows: 

\nn(a) definition of $F^b$, 

\nn(b) $\A$ satisfies IPC,

\nn(c) $F$ commutes with small projective limits, 

\nn(d) an inductive limit of local isomorphisms is a local isomorphism by Proposition 16.3.4 p.~401 of the book, 

\nn(e) definition of $F^b$. 
\end{s} 

%

\begin{s} 
P.~419, proof of Proposition 17.4.4, Step~(ii). More details: The morphism $\ee_b(F^b)(A):F^b(A)\to F^{bb}(A)$ is obtained as the composition 
$$
F^b(A)\xr f\col_{(B\to A)\in\mc{LI}_A}F^b(A)\xr g\col_{(B\to A)\in\mc{LI}_A}F^b(B). 
$$ 
Moreover, $f$ is an isomorphism by Lemma 2.1.12 p.~41 of the book, and $g$ is an isomorphism by Lemma 17.4.2 p.~418 of the book.
\end{s}

%

\sbs{Proposition 17.4.4 p.~420}

We draw a few diagrams with the hope of helping the reader visualize the argument in Step~(ii) of the proof of Proposition 17.4.4. 

An object of the category 
$$
M\big[J\to K\leftarrow M[I\to K\leftarrow K]\big]
$$ 
can be represented by a diagram 
$$
\begin{tikzcd}
B'\ar{d}[swap]{\bt}&B'\tm_BA\ar{d}&A'\ar{l}[swap]{(u',\al)}\ar{r}{(v',\al)}\ar{d}{\al}&C'\tm_CA\ar{d}&C'\ar{d}{\gamma}\\ 
B&A\ar[equal]{r}&A\ar[equal]{r}&A&C, 
\end{tikzcd}
$$ 
and it is clear that this category is equivalent to $\E^{\op}$. 

Recall that $D:=B\ ``\!\sqcup\!"\!\!_A\ C$, let $E$ be one of the objects $A,B,C,$ or $D$, and consider the ``obvious'' functors 
$$
\begin{tikzcd}
\E\ar{r}{p_E}\ar{rd}[swap]{q_E}&\mc{LI}_E\ar{d}{j_E}\\ 
{}&\C_X^\wg
\end{tikzcd}
$$ 
($p_E$ is defined in the book, $j_E$ is the forgetful functor, and $q_E$ is the composition). We also define $r_E:\mc{LI}_E\to\E$ by mapping the object $E''\to E$ of $\mc{LI}_E$ to the object 
$$
\begin{tikzcd}
B\tm_EE''\ar{d}&A\tm_EE''\ar{l}\ar{r}\ar{d}&C\tm_EE''\ar{d}\\ 
B&A\ar{r}\ar{l}&C
\end{tikzcd}
$$ 
of $\E$. One checks that $(p_E,r_E)$ is a pair of adjoint functors. In particular $p_E$ is cocofinal. We have
$$
F^b(D)
\os{(\text a)}{\iso}\col_{y\in\mc{LI}_D}F(j_D(y))
\os{(\text b)}{\iso}\col_{x\in\E}F(q_D(x))
$$
$$
\os{(\text c)}{\iso}\col_{x\in\E}F\left(q_B(x)``\!\!\!\bigsqcup_{q_A(x)}\!\!\!"q_C(x)\right)
\os{(\text d)}{\iso}\col_{x\in\E}(F(q_B(x))\tm_{F(q_A(x))}F(q_C(x)))
$$
$$
\os{(\text e)}{\iso}\left(\col_{x\in\E}F(q_B(x))\right)\tm_{\col_{x\in\E}F(q_A(x))}\left(\col_{x\in\E}F(q_C(x))\right)
$$ 
$$
\os{(\text f)}{\iso}\left(\col_{y\in\mc{LI}_B}F(j_B(y))\right)\tm_{\col_{y\in\mc{LI}_A}F(j_A(y))}\left(\col_{y\in\mc{LI}_C}F(j_C(y))\right)
$$ 
$$
\os{(\text g)}{\iso}F^b(B)\tm_{F^b(A)}F^b(C).
$$ 
Indeed, the isomorphisms can be justified as follows: 

\nn(a) definition of $F^b$, 

\nn(b) cocofinality of $p_D$,

\nn(c) definition of $p_E$, 

\nn(d) left exactness of $F$, 

\nn(e) exactness of filtrant inductive limits in $\A$, 

\nn(f) cocofinality of $p_D$, 

\nn(g) definition of $F^b$. 

%%

\sbs{Brief comments}

\begin{s}\lb{fau}
P.~421, first display: 
$$
F^a(U)\iso\col_{(U\to A)\in\mc{LI}_U}F(A).
$$ 
Lemma 16.2.8 p.~398 of the book and its proof, show that $F^a$ does {\em not} depend on the universe such that $\C$ is a small category and $\A$ satisfies (17.4.1) p.~417 of the book.
\end{s}

%

\begin{s}P.~421, proof of Lemma 17.4.6 (i): The category $\mc{LI}_U$ is cofiltrant by Lemma 16.2.7 p.~398 of the book, small filtrant inductive limits are exact in $\A$ by Display (17.4.1) p.~417 of the book, exact functors preserve monomorphisms by Proposition~\ref{34i} p.~\pr{34i}.
\end{s}

%

\begin{s} 
P.~422. The first sentence of the proof of Theorem 17.4.7 (iv) follows from Corollary~\ref{bre} p.~\pr{bre}. One could add:

If $\A$ is abelian, then $\PSh(X,\A)$ and $\Sh(X,\A)$ are abelian, and $\iota:\Sh(X,\A)\to\PSh(X,\A)$ and $(\ )^a:\PSh(X,\A)\to\Sh(X,\A)$ are additive
\end{s}

%

\begin{s} 
P.~423, end of the proof of Theorem 17.4.9 (iv): the functor $(\ )^a$ is exact by Theorem 17.4.7 (iv) p.~421 of the book.
\end{s}

%

\begin{s} 
P.~424, proof of Theorem 17.5.2 (i). With the convention that a diagram of the form 
$$
\begin{tikzcd} 
\C_1\ar[xshift=-0.7ex]{d}[swap]{L}\\ \C_2\ar[xshift=0.7ex]{u}[swap]{R}
\end{tikzcd}
$$ 
means: ``$(L,R)$ is a pair of adjoint functors'', the proof of Theorem 17.5.2 (i) in the book can be visualized by the diagram 
$$
\begin{tikzcd} 
\PSh(Y,\A)\ar[xshift=-0.7ex]{d}[swap]{f^\dg}\\ 
\PSh(X,\A)\ar[xshift=0.7ex]{u}[swap]{f_*}\ar[xshift=-0.7ex]{d}[swap]{(\ )^a}\\ 
\Sh(X,\A).\ar[xshift=0.7ex]{u}[swap]{\iota}
\end{tikzcd}
$$ 
\end{s}

%

\begin{s} 
P.~424, proof of Theorem 17.5.2 (iv). As already mentioned, there is a typo: ``The functor $f^\dg$ is left exact'' should be ``The functor $f^\dg$ is exact''. 
\end{s}

%

\begin{s}\lb{1761}
P.~424, Definition 17.6.1. By Lemma 17.1.8 p.~409 of the book, a morphism 
$$
\begin{tikzcd} 
C\ar{rr}\ar{dr}&&B\ar{dl}\\ 
{}&A
\end{tikzcd}
$$ 
in $\C_A^\wg$ is a local epimorphism if and only if $C\to B$ is a local epimorphism in $\C_X^\wg$.
\end{s}

%

\begin{s}
P.~424, sentence following Definition 17.6.1: ``It is easily checked that we obtain a Grothendieck topology on $\C_A$''. The verification  of LE1, LE2 and LE3 is straightforward. Axiom LE4 follows from Parts (iii) and (ii) of Lemma 17.2.5 p.~413 of the book.
\end{s} 

%

\begin{s}\lb{1761b} 
P.~424, Definition 17.6.1. Here is an observation which follows from \S\ref{1725ii} p.~\pr{1725ii} and Lemma 17.2.5 (iii) p.~413 of the book: 

In the setting of Definition 17.6.1, let $B\to A$ be a morphism in $\C_X^\wg$, let $u:C\to B$ be a morphism in $\C_X^\wg$, and let $v:(C\to A)\to(B\to A)$ be the corresponding morphism in $\C_A^\wg$. Then $u$ is a local epimorphism if and only if $v$ is a local epimorphism, $u$ is a local monomorphism if and only if $v$ is a local monomorphism, and $u$ is a local isomorphism if and only if $v$ is a local isomorphism. 
\end{s} 

%

\begin{s}
P.~425, proof of Proposition 17.6.3: 

Step~(i): $\oo j_{A\to X}$ is weakly left exact by Lemma 17.2.5 (iii) p.~413 of the book, and $(\,\cdot\,)^a$ is exact by Theorem 17.4.7 (iv) p.~421 of the book.

Step~(ii): ``$f$ factors as $X\xr{\oo j_{A\to X}}A\xr gY$'': see Definition 17.2.4 (ii) p.~413 of the book and Remark~\ref{272b} p.~\pr{272b}. The isomorphism $f^{-1}\iso\oo j_{A\to X}^{-1}\ci g^{-1}$ follows from Proposition 17.5.3 p.~424 of the book.
\end{s} 

% 

\begin{s} 
P.~425, Display (17.6.1): Putting $j:=\oo j_{A\to X}$, we have the adjunctions 
$$
\begin{tikzcd}
\Sh(A,\A)\ar[xshift=-4ex]{d}[swap]{j^{-1}}\ar[xshift=4ex]{d}{j^\ddg}\\ 
\Sh(X,\A).\ar{u}{j_*}
\end{tikzcd}
$$ 

For the functor $j_*:\Sh(X,\A)\to\Sh(A,\A)$, see Proposition 17.5.1 p.~423 of the book. 

For the functor $j^{-1}:\Sh(A,\A)\to\Sh(X,\A)$, see last display of p.~423 of the book. 

For the functor $j^\ddg:\Sh(A,\A)\to\Sh(X,\A)$, see Proposition 17.6.2 p.~425 of the book.
\end{s}

%

\begin{s} 
P.~426, proof of Proposition 17.6.7 (i). The isomorphism 
\begin{equation}\lb{e406}
\fthat(V\tm B)\iso f^t(V)\tm\fthat(B)
\end{equation} 
follows from Proposition~\ref{p406} p.~\pr{p406}, and we have 
\begin{align*} 
%
\oo j_{B\to Y}^\ddg\big(f_{B*}(G)(V)\big)&\iso f_{B*}(G)(V\tm B\to B)&\text{by (17.1.12) p. 409}\\ \\ 
%
&\iso G\big(\fthat(V\tm B)\to\fthat(B)\big)&\text{by (17.1.6) p. 408}\\ \\ 
%
&\iso G\big(f^t(V)\tm A\to A\big)&\text{by \qr{e406}}, 
% 
\end{align*} 
as well as 
\begin{align*} 
%
f_*\big(\oo j_{A\to X}^\ddg(G)(V)\big)&\iso\oo j_{A\to X}^\ddg(G)(f^t(V))\\ \\ 
%
&\iso G\big(f^t(V)\tm A\to A\big)&\text{by (17.1.12) p. 409}. 
% 
\end{align*}  
\end{s} 

% 

\begin{s}\lb{1768}
P.~427, proof of Proposition 17.6.8, Step~(i). The isomorphism 
$$
\oo j_{A\to X}^\ddg(\oo j_{A\to X*}(G)(U))\iso\oo j_{A\to X*}(G)(U\tm A\to A)
$$ 
follows from (17.1.12) p.~409 of the book. The fact that $p:A\tm U\to U$ is a local isomorphism follows from the fact that the obvious square 
$$
\begin{tikzcd} 
A\tm U\ar{r}\ar{d}&U\ar{d}\\ 
A\ar{r}&\oo{pt}_X
\end{tikzcd}
$$ 
is cartesian and the bottom arrow is a local isomorphism by assumption. 
\end{s} 

%

\begin{s}
P.~427, proof of Proposition 17.6.8, Step~(ii). Let $v:V\to A$ be a morphism in $\C_X^\wg$. Here is a proof of the fact that 
\begin{equation}\lb{1768ii}
\begin{tikzcd} 
V\ar{rr}{(\id_V,v)}\ar{dr}&&V\tm A\ar{dl}\\ 
{}&A
\end{tikzcd}
\end{equation} 
is a local isomorphism in $\C_A^\wg$. 

As $V\tm A\to V$ is a local isomorphism in $\C_X^\wg$ by \S\ref{1768}, and $V\to V\tm A\to V$ is the identity of $V$, Lemma 16.2.4 (vii) p.~396 of the book implies that $V\to V\tm A$ is a local isomorphism in $\C_X^\wg$, and thus, by \S\ref{1761b} p.~\pr{1761b}, that \qr{1768ii} is a local isomorphism in $\C_A^\wg$. 
\end{s} 

%

\begin{s}\lb{a428}
P.~428, just after Definition 17.6.10: $((\ )_A,\Gamma_A(\ ))$ is a pair of adjoint functors: this follows from Theorem 17.5.2 (i) p.~424 of the book. 
\end{s} 

%

\begin{s}\lb{429}
P.~429, top. By \S\ref{phistar} p.~\pr{phistar} and Corollary~\ref{bre2} p.~\pr{bre2}, the functor $\Gamma(A;\ )$ commutes with small projective limits. 
\end{s} 

%

\begin{s} 
P.~430, first sentence of the proof of Proposition 17.7.1 (i). Let us make a general observation. 

Let $X$ be a site. In this \S, for any $A$ in $\C_X^\wg$, we denote the corresponding site by $A'$ instead of $A$. We also identify $\C_{A'}^\wg$ to $(\C_X^\wg)_A$ (see Lemma 17.1.8 p.~409 of the book). In particular, we get $\oo{pt}_{A'}\iso(A\xr{\id_A}A)\in\C_{A'}^\wg$. 

Let $A\to B$ be a local isomorphism in $\C_X^\wg$, and let us write $\omega$ for ``the'' terminal object $\oo{pt}_{B'}\iso(B\xr{\id_B}B)$ of $\C_{B'}^\wg$. We claim that 
\begin{equation}\lb{1771i1}
(A\to B)\to\omega 
\end{equation} 
is a local isomorphism in $\C_{B'}^\wg$.

Proof: \qr{1771i1} is a local epimorphism by \S\ref{1761} p.~\pr{1761}. It remains to check that 
\begin{equation}\lb{1771i2}
(A\to B)\to(A\to B)\tm_\omega(A\to B)\iso(A\tm_BA\to B)
\end{equation} 
is a local epimorphism. But this follows again from \S\ref{1761} p.~\pr{1761}. $\square$ 

Consider the morphism of presites $B'\to A'$ induced by $A\to B$ and note that the square
$$
\begin{tikzcd} 
X\ar{rr}{\oo j_{A\to X}}\ar[equal]{d}&&A'\\ 
X\ar{rr}[swap]{\oo j_{B\to X}}&&B'.\ar{u}
\end{tikzcd}
$$ 
commutes.
\end{s} 

%

\begin{s} 
P.~430, proof of Proposition 17.7.3. The third isomorphism follows, as indicated, from Proposition 17.6.7 (ii) p.~426 of the book. The fifth isomorphism follows from (17.6.2) (ii) p.~426 of the book. 
\end{s} 

%

\begin{s}\lb{175i}
P.~431, Exercise 17.5 (i). Put $PX:=\PSh(X,\A),\ SX:=\Sh(X,\A)$, and define $PY$ and $SY$ similarly. Let
$$
\begin{tikzcd} 
SY\ar[xshift=0.7ex]{d}{\iota_Y}\\ 
PY\ar[xshift=-0.7ex]{u}{a_Y}\ar[xshift=-0.7ex]{d}[swap]{f^\dg}\\ 
PX\ar[xshift=0.7ex]{u}[swap]{f_*}\ar[xshift=-0.7ex]{d}[swap]{a_X}\\ 
SX\ar[xshift=0.7ex]{u}[swap]{\iota_X}
\end{tikzcd}
$$ 
be the obvious diagram of adjoint functors. We must show 
$$
a_X\ci f^\dg\ci\iota_Y\ci a_Y\iso a_X\ci f^\dg. 
$$ 
Let $F$ be in $SX$ and $G$ be in $PY$. We have (omitting most of the parenthesis) 
$$
\Hom_{SX}(a_Xf^\dg\iota_Ya_YG,F)\iso
\Hom_{PX}(f^\dg\iota_Ya_YG,\iota_XF)\iso
\Hom_{PY}(\iota_Ya_YG,f_*\iota_XF)
$$
$$
\os{(\text a)}{\iso}
\Hom_{PY}(\iota_Ya_YG,\iota_Ya_Yf_*\iota_XF)\iso
\Hom_{SY}(a_Y\iota_Ya_YG,a_Yf_*\iota_XF)
$$
$$
\os{(\text b)}{\iso}
\Hom_{SY}(a_YG,a_Yf_*\iota_XF)\iso
\Hom_{PY}(G,\iota_Ya_Yf_*\iota_XF)\os{(\text c)}{\iso}
\Hom_{PY}(G,f_*\iota_XF)
$$ 
$$ 
\iso 
\Hom_{PX}(f^\dg G,\iota_XF)\iso 
\Hom_{SX}(a_Xf^\dg G,F)
$$ 
where (a) and (c) follow from the fact that the presheaf $f_*\iota_XF$ is actually a sheaf (Proposition 17.5.1 p.~423 of the book), (b) follows from the isomorphism 
$$
a_Y\ci\iota_Y\ci a_Y\iso a_Y,
$$ 
which holds by Lemma 17.4.6 (ii) p.~421 of the book, and the other isomorphisms hold by adjunction. 
\end{s} 

%

\begin{s} 
P.~431, Exercise 17.5 (ii). By \S\ref{1761b} p.~\pr{1761b} we have, for $U$ in $\C_X$ and $U\to A$ in $\C_A$, an isomorphism 
$$
\mc{LI}_{U\to A}\iso\mc{LI}_U.
$$
Exercise 17.5 (ii) follows immediately. 
\end{s} 

%%%

\section{About Chapter 18} 

\sbs{Brief comments}

%

\begin{s}
P.~437, Theorem 18.1.6 (v). If $X$ is a site, if $R$ a ring, if $F$ and $G$ are complexes of $R$-modules, then the complex of abelian groups $\RHom_R(F,G)$ (see Corollary 14.3.2 p.~356 of the book) does {\em not} depend on the universe chosen to define it (the universe in question being subject to the obvious conditions). This follows from \S\ref{14112} p.~\pr{14112} and \S\ref{fau} p.~\pr{fau}. 
\end{s} 

%

\begin{s}\lb{pshab}
P.~436, Lemma 18.1.4. Note that $\PSh(\R)$ is abelian, that $\Mod(\R)$ is an additive subcategory of $\PSh(\R)$, and that the functors
$$
\begin{tikzcd}
\Mod(\R)\ar[yshift=0.7ex,hook]{r}&\PSh(\R)\ar[yshift=-0.7ex]{l}{(\cdot)^a}\end{tikzcd}
$$ 
are additive.
\end{s} 

%

\begin{s} 
P.~437, proof of Theorem 18.1.6 (v). We prove 
$$
\Hom_\R(\R_U,F)\iso F(U).
$$ 
As 
$$
\Hom_\R(\R_U,F)\iso\Hom_\R(\oo j_{U\to X*}^{-1}(\R|U),F)\iso
\Hom_{\R|U}(\R|U,F|U), 
$$ 
we only need to verify 
$$
\Hom_{\R|U}(\R|U,F|U)\iso F(U).
$$ 
We shall define maps 
$$
\begin{tikzcd}
\Hom_{\R|U}(\R|U,F|U)\ar[yshift=0.7ex]{r}{\pp}&F(U)\ar[yshift=-0.7ex]{l}{\psi}
\end{tikzcd}
$$ 
and leave it to the reader to check that they are mutually inverse. 

Definition of $\pp$: Let $\theta$ be in $\Hom_{\R|U}(\R|U,F|U)$. In particular, for each morphism $f:V\to U$ in $\C_X$ we have a map $\theta(f):\R(V)\to F(V)$, and we put $\pp(\theta):=\theta(\id_U)(1)$. 

Definition of $\psi$: Let $x$ be in $F(U)$. For each morphism $f:V\to U$ in $\C_X$ we define $\psi(x)(f):\R(V)\to F(V)$ by $\psi(x)(f)(\ld):=\ld\,F(f)(x)$.
\end{s}

%

\begin{s}\lb{a438}
P.~438, end of Section 18.1: $\Gamma_A$ is left exact by \S\ref{a428} p.~\pr{a428}. Moreover, $\Gamma(A;\ )$ commutes with small projective limits by \S\ref{429} p.~\pr{429}, and is thus left exact by Proposition~\ref{333} p.~\pr{333}. 
\end{s} 

% 

\begin{s}\lb{homrr}
P.~438, bottom: One can add that we have $\HOM_\R(\R,F)\iso F$ for all $F$ in $\PSh(\R)$. 
\end{s} 

% 

\begin{s}\lb{neutral}
P.~439, after Definition 18.2.2: One can add that we have 
$$
\R\os{\text{\tiny psh}}{\otimes}_\R F\iso F
$$ 
for $F$ in $\PSh(\R)$ and 
$$
\R\otimes_\R F\iso F
$$ 
for $F$ in $\Mod(\R^{\op})$, as well as 
$$F\os{\text{\tiny psh}}{\otimes}_\R\R\iso F
$$ 
for $F$ in $\PSh(\R^{\op})$ and 
$$
F\otimes_\R\R\iso F
$$ 
for $F$ in $\Mod(\R^{\op})$. 
\end{s} 

% 

\begin{s}
P.~441. The proof of Proposition 18.2.5 uses Display (17.1.11) p.~409 of the book and \S\ref{175i} p.~\pr{175i}. 
\end{s} 

% 

\begin{s} 
P.~441. In the notation of Remark 18.2.6 we have 
$$
\Hom_{\R_3}({}_3M_2\otimes_{\R_2}{}_2M_1,{}_3M_4)\iso
\Hom_{\R_2}({}_2M_1,\HOM_{\R_3}({}_3M_2,{}_3M_4)),
$$
$$
\HOM_{\R_3}({}_3M_2\otimes_{\R_2}{}_2M_1,{}_3M_4)\iso
\HOM_{\R_2}({}_2M_1,\HOM_{\R_3}({}_3M_2,{}_3M_4)),
$$ 
$$
\Hom_{\R_3^{\op}}({}_1M_2\otimes_{\R_2}{}_2M_3,{}_4M_3)\iso
\Hom_{\R_2^{\op}}({}_1M_2,\HOM_{\R_3^{\op}}({}_2M_3,{}_4M_3)),
$$ 
$$
\HOM_{\R_3^{\op}}({}_1M_2\otimes_{\R_2}{}_2M_3,{}_4M_3)\iso
\HOM_{\R_2^{\op}}({}_1M_2,\HOM_{\R_3^{\op}}({}_2M_3,{}_4M_3)).
$$ 
More generally, if $\mc{R,S,T}$ are $\OO_X$-algebras, if $F$ is a $(\T\otimes_{\OO_X}\R^{\op})$-module, if $G$ is an $(\R\otimes_{\OO_X}\SSS)$-module, and if $H$ is an $(\SSS\otimes_{\OO_X}\T)$-module, then we have 
$$ 
\Hom_{\SSS\otimes_{\OO_X}\T}(F\otimes_\R G,H)\iso
\Hom_{\R\otimes_{\OO_X}\SSS}(G,\HOM_{\T}(F,H)), 
$$ 
\begin{equation}\lb{HOM}
\HOM_{\SSS\otimes_{\OO_X}\T}(F\otimes_\R G,H)\iso
\HOM_{\R\otimes_{\OO_X}\SSS}(G,\HOM_\T(F,H)). 
\end{equation}
\end{s} 

%

\begin{s}
P.~442, proof of Proposition 18.2.7. Here are additional details. 

Proof of (18.2.12): We must show 
\begin{equation}\lb{18212}
F_A\iso\R_A\otimes_\R F\iso k_{XA}\otimes_{k_X}F. 
\end{equation} 
We have 
$$
F_A\os{(\text a)}{\iso}
\oo j_{A\to X}^{-1}(F|_A)\os{(\text b)}{\iso}
\oo j_{A\to X}^{-1}(\R|_A\otimes_{\R|_A}F|_A)\os{(\text c)}{\iso}
(\oo j_{A\to X}^{-1}(\R|_A))\otimes_\R F\os{(\text d)}{\iso} 
\R_A\otimes_\R F.
$$ 
Indeed, (a) and (d) hold by Definition 17.6.10 (i) and Display (17.6.5) p.~428 of the book, (b) follows from \S\ref{neutral}, (c) follows from (18.2.6) p.~441 of the book. The isomorphism $F_A\iso k_{XA}\otimes_{k_X}F$ is a particular case of the isomorphism $F_A\iso\R_A\otimes_\R F$ just proved. 

Proof of (18.2.13): We must show 
\begin{equation}\lb{18213} 
\Gamma_A(F)\iso\HOM_\R(\R_A,F)\iso\HOM_{k_X}(k_{XA},F). 
\end{equation}
We have 
$$
\HOM_\R(\R_A,F)\os{(\text a)}{\iso}
\HOM_\R(\R\otimes_{k_X}k_{XA},F)
$$
$$
\os{(\text b)}{\iso}
\HOM_{k_X}(k_{XA},\HOM_\R(\R,F))\os{(\text c)}{\iso}
\HOM_{k_X}(k_{XA},F), 
$$ 
where (a) follows from \qr{18212}, (b) follows from Display (18.2.4) p.~439 of the book (which is a particular case of \qr{HOM}), and (c) follows from \S\ref{homrr}. Let us record the isomorphism 
\begin{equation}\lb{record}
\HOM_\R(\R_A,F)\iso\HOM_{k_X}(k_{XA},F). 
\end{equation} 
We also have for $G$ in $\Mod(\R)$ 
$$
\Hom_\R(G,\HOM_{k_X}(k_{XA},F))\os{(\text a)}{\iso}
\Hom_\R(G\otimes_{k_X}k_{XA},F)\os{(\text b)}{\iso}
\Hom_\R(\oo j_{A\to X}^{-1}\oo j_{A\to X*}G,F)
$$
$$
\os{(\text c)}{\iso}
\Hom_\R(G,\oo j_{A\to X}^\ddg\oo j_{A\to X*}F)\os{(\text d)}{\iso}
\Hom_\R(G,\Gamma_A(F)), 
$$ 
where (a) follows from \qr{HOM} with 
$$
(k_X;k_X,\R,k_X;k_{XA},G,F)
$$ 
instead of 
$$
(\OO_X;\mc{R,S,T};F,G,H),
$$ 
(b) follows from \qr{18212}, Definition 17.6.10 (i) and Display (17.6.5) p.~428 of the book, (c) follows by adjunction, and (d) by Definition 17.6.10 (ii) p.~428 of the book. 
\end{s} 

%%

\sbs{Lemma 18.5.3 p. 447} 

We give additional details about the proof of Lemma 18.5.3 of the book (stated below as Lemma~\ref{l1853} p.~\pr{l1853}) with the hope of helping the reader. We start with a technical lemma.

\begin{lem}\lb{techlem1}
Let $R$ be a ring, let $A$ be a right $R$-module, let $B$ be a left $R$-module, let $n$ be a positive integer, and let 
$$
(a_i)_{i=1}^n,\quad(b_i)_{i=1}^n
$$
be two families of elements belonging respectively to $A$ and $B$. Then Conditions \emph{(i)} and \emph{(ii)} below are equivalent:

\nn\emph{(i)} We have $\sum_{i=1}^n\,a_i\otimes b_i=0$ in $A\otimes_RB$. 

\nn\emph{(ii)} There are positive integers $\ell$ and $m$ with $\ell\ge n$, and there are three families 
$$
(a_i)_{i=n+1}^\ell,\quad(\ld_{ij})_{1\le i\le\ell,1\le j\le m},\quad(b'_j)_{j=1}^m
$$ 
of elements belonging respectively to $A$, $R$ and $B$, such that, if we set $b_i=0$ for $n<i\le\ell$, we have:
%
\begin{equation}\lb{lij1}
\sum_{j=1}^m\ \ld_{ij}\,b'_j=b_i\quad(\forall\ 1\le i\le\ell),
\end{equation}
%
\begin{equation}\lb{lij2}
\sum_{i=1}^\ell\ a_i\,\ld_{ij}=0\quad(\forall\ 1\le j\le m).
\end{equation}
\end{lem} 

\begin{proof} 
Implication (ii)$\then$(i) is clear. To prove Implication (i)$\then$(ii), we assume (i), and we choose a set $I$ containing $\{1,\dots,\ell\}$, where $\ell$ is an integer $\ge n$ to be determined later, such that there is a family $(a_i)_{i\in I}$ which completes the family $(a_i)_{1\le i\le n}$ and generates $A$. We write $C$ for the kernel of the epimorphism 
$$
f:R^{\oplus I}\epi A,\quad(\mu_i)\mt\sum_{i\in I}a_i\,\mu_i.
$$ 
In particular we have exact sequences 
$$
C\xr gR^{\oplus I}\xr fA\to0,\qquad C\otimes_RB\xr{g'}B^{\oplus I}\xr{f'}A\otimes_RB\to0,
$$ 
with 
$$
g'\big((\mu_i)\otimes b)\big)=(\mu_i\,b),\quad f'((b''_i))=\sum_{i\in I}a_i\otimes b''_i.
$$
Put $b_i:=0$ for $i$ in $I\setminus\{1,\dots,\ell\}$. The family $(b_i)_{i\in I}$ is in $\Ker f'$, and thus in $\Ima g'$. The condition $(b_i)\in\Ima g'$ means that there is a positive integer $m$, a family 
$$
(\ld_{ij})_{i\in I,1\le j\le m}
$$ 
of elements of $R$ such that 
$$
(\ld_{ij})_i\in C\subset R^{\oplus I}
$$ 
for $1\le j\le m$, and a family $(b'_j)_{1\le j\le m}$ of elements of $B$, such that 
$$
(b_i)_i=g'
\left(\sum_{j=1}^m(\ld_{ij})_i\otimes b'_j\right)=
\left(\sum_{j=1}^m\ld_{ij}\,b'_j\right)_i.
$$ 
As $(\ld_{ij})_i$ is in $R^{\oplus I}$ for all $j$, the set of those $i$ in $I$ for which there is a $j$ such that $\ld_{ij}\neq0$ is finite, and we can arrange the notation so that this set is contained in $\{1,\dots,\ell\}$ with $\ell\ge n$, and we get \qr{lij1}. As $(\ld_{ij})_i$ is in $C$ for all $j$, we also have \qr{lij2}. 
\end{proof} 

Here is another technical lemma:

\begin{lem}\lb{techlem2}
Let $R$ be a ring, let $\pp:A'\to A$ be a morphism of right $R$-modules, let $B$ be a left $R$-module, and let $s$ be an element of $\Ker(A'\otimes_RB\to A\otimes_RB)$. Then there exist 

\nn$\bu$ a commutative diagram 
$$
\begin{tikzcd}
{}&F'\ar{d}{f}\ar[equal]{r}&F'\ar{dd}{0}\\ 
F''\ar{r}{\psi}\ar{d}[swap]{g}&F\ar{d}{h}\\ 
A'\ar{r}[swap]{\pp}&A\ar[equal]{r}&A
\end{tikzcd}
$$  
of right $R$-modules such that $F,F'$ and $F''$ are free of finite rank,  

\nn$\bu$ elements $t\in F'\otimes_RB$, $u\in F''\otimes_RB$ such that the commutative diagram 
$$
\begin{tikzcd}
{}&F'\otimes_RB\ni t\ar{d}{f_1}\\ 
u\in F''\otimes_RB\ar{r}{\psi_1}\ar{d}[swap]{g_1}&F\otimes_RB\ar{d}{h_1}\\ 
s\in A'\otimes_RB\ar{r}[swap]{\pp_1}&A\otimes_RB
\end{tikzcd}
$$  
satisfies $g_1(u)=s$ and $\psi_1(u)=f_1(t)$. 
\end{lem} 

\begin{proof} 
Write 
$$
s=\sum_{i=1}^na'_i\otimes b_i
$$ 
with $a'_i$ in $A'$ and $b_i$ in $B$, and put $a_i:=\pp(a_i')\in A$, so that we have 
$$
\sum_{i=1}^n\,a_i\otimes b_i=0.
$$ 
By Lemma~\ref{techlem1} p.~\pr{techlem1} there are positive integers $\ell,m$ with $\ell\ge n$, and there are three families 
$$
(a_i)_{i=n+1}^\ell,\quad(\ld_{ij})_{1\le i\le\ell,1\le j\le m},\quad(b'_j)_{j=1}^m
$$ 
of elements belonging respectively to $A$, $R$ and $B$, such that, if we set $b_i=0$ for $n<i\le\ell$, we get \qr{lij1} and \qr{lij2} p.~\pr{lij1}. We have a commutative diagram of right $R$-modules 
\begin{equation}\lb{d447}
\begin{tikzcd}
{}&R^m\ar{d}{f}\\ 
R^n\ar[hook]{r}{\psi}\ar{d}[swap]{g}&R^\ell\ar{d}{h}\\ 
A'\ar{r}[swap]{\pp}&A
\end{tikzcd}
\end{equation} 
with 
$$
f(x)_i=\sum_{j=1}^m\ld_{ij}\,x_j,\quad g(x)=\sum_{i=1}^na'_i\,x_i,\quad h(x)=\sum_{i=1}^\ell a_i\,x_i.
$$ 
In particular, \qr{lij2} p.~\pr{lij2} implies $h\ci f=0$. 
\end{proof} 

\begin{lem}\lb{rff'}
If $F$ and $F'$ are two $\R$-modules of finite rank, then the natural map
$$
\Hom_\R(F,F')\to\Hom_{\Gamma(X,\R)}(\Gamma(X,F),\Gamma(X,F'))
$$ 
is bijective. (Recall that $\Gamma(X,F)$ is defined just before Proposition 17.6.14 p.~429 of the book.)
\end{lem}
\begin{proof}
It suffices to prove the statement when $F=F'=\R$, which is easy.
\end{proof}

Let us turn to the proof of Lemma 18.5.3 p. 447. [As already pointed out, there are two typos in the proof: in (18.5.3) $M'|_U$ and $M|_U$ should be $M'(U)$ and $M(U)$, and, after the second display on p.~448, $s_1\in((\R^{\op})^{\oplus m}\otimes_\R P)(U)$ should be $s_1\in((\R^{\op})^{\oplus n}\otimes_\R P)(U)$.] 

For the reader's convenience we state (in a slightly different form) Lemma 18.5.3 (see Notation 17.6.13 p.~428 of the book): 

\begin{lem}[Lemma 18.5.3 p. 447]\lb{l1853}
Let $P$ be an $\R$-module. Assume that for all $U$ in $\C_X$, all free right $\R$-module $F',F''$ of finite rank, and all $\R|_U$-linear morphism $u:F'|_U\to F''|_U$, the sequence 
$$
0\to\Ker(u)\otimes_{\R|_U}P|_U\to F'|_U\otimes_{\R|_U}P|_U\to F''|_U\otimes_{\R|_U}P|_U
$$ 
is exact. Then $P$ is a flat $\R$-module.
\end{lem} 

(Recall that the notation $?|_U$ is defined in Notation 17.6.13 (ii) p.~428 of the book.)

\begin{proof}
Consider a monomorphism $M'\mono M$ of right $\R$-modules. It suffices to prove that the sheaf 
$$ 
K:=\Ker(M'\otimes_\R P\to M\otimes_\R P)
$$ 
of $k_X$-modules over $X$ vanishes. Let $K_0$ be the presheaf of $k_X$-modules over $X$ defined by 
$$
K_0(U):=\Ker\Big(M'(U)\otimes_{\R(U)}P(U)\to M(U)\otimes_{\R(U)}P(U)\Big),
$$ 
let $U$ be an object of $\C_X$, let $s$ be an element of $K_0(U)$, and let $\overline s$ be the image of $s$ in $K(U)$. We shall prove $\overline s=0$. By \S\ref{pshab} p.~\pr{pshab} above, Definition 18.2.2 p.~439 and Theorem 17.4.7 (iv) p.~421 of the book, $K$ is the sheaf associated to $K_0$. Hence, as $U$ and $s$ are arbitrary, Equality $\overline s=0$ will imply that the natural morphism $K_0\to K$ vanishes. By (17.4.12) p.~421 of the book, this vanishing will entail $K\iso0$, and thus, the lemma. Let us record this observation: 
%
\begin{equation}\lb{s=0il}
\text{Equality $\overline s=0$ implies the lemma.}
\end{equation}  
%  
By Lemma~\ref{techlem2} p.~\pr{techlem2} there exist 

\nn$\bu$ a commutative diagram 
$$
\begin{tikzcd}
{}&F'(U)\ar{d}{f}\ar[equal]{r}&F'(U)\ar{dd}{0}\\ 
F''(U)\ar{r}{\psi}\ar{d}[swap]{g}&F(U)\ar{d}{h}\\ 
M'(U)\ar{r}[swap]{\pp}&M(U)\ar[equal]{r}&M(U)
\end{tikzcd}
$$  
of right $\R(U)$-modules such that $F,F'$ and $F''$ are free right $\R$-modules of finite rank,  

\nn$\bu$ elements $t\in F'(U)\otimes_{\R(U)}P(U)$, $u\in F''(U)\otimes_{\R(U)}P(U)$ such that the commutative diagram 
\begin{equation}\lb{ff'f''}
\begin{tikzcd}
{}&F'(U)\otimes_{\R(U)}P(U)\ni t\ar{d}{f_1}\\ 
u\in F''(U)\otimes_{\R(U)}P(U)\ar{r}{\psi_1}\ar{d}[swap]{g_1}&F(U)\otimes_{\R(U)}P(U)\ar{d}{h_1}\\ 
s\in M'(U)\otimes_{\R(U)}P(U)\ar{r}[swap]{\pp_1}&M(U)\otimes_{\R(U)}P(U)
\end{tikzcd}
\end{equation} 
satisfies $g_1(u)=s$ and $\psi_1(u)=f_1(t)$. 

By Lemma~\ref{rff'} p.~\pr{rff'}, the commutative diagram \qr{d447} also induces the commutative diagram 
$$
\DT:=\left\{
\begin{tikzcd}
N\ar{r}\ar{d}[swap]{k_2}&F'|_U\ar{d}{f_2}\\ 
F''|_U\ar{r}{\psi_2}\ar{d}[swap]{g_2}&F|_U\ar{d}{h_2}\\ 
M'|_U\ar{r}[swap]{\pp_2}&M|_U,
\end{tikzcd}
\right.
$$
the top square being cartesian. Then $\pp_2$ is a monomorphism by Proposition 17.6.6 p.~425 and Notation 17.6.13 p.~428 of the book (recall that $M'\to M$ is a monomorphism by assumption). This implies $g_2\ci k_2=0$. Hence $\DT$ is a commutative diagram of \emph{complexes}. The condition that the top square is cartesian is equivalent to the exactness of 
$$ 
\Sigma:=\Big(0\to N\to F'|_U\oplus F''|_U\to F|_U\Big).
$$  
The sequence $\Sigma\otimes_{\R|_U}P|_U$ being exact thanks to the assumption in Lemma~\ref{l1853} p.~\pr{l1853}, we see that the commutative diagram of complexes $\DT\otimes_{\R|_U}P|_U$ has a cartesian top square, and that, by left exactness of $\Gamma(U;-)$ (see \S\ref{a438} p.~\pr{a438}), the commutative diagram of complexes $\Gamma(U;\DT\otimes_{\R|_U}P|_U)$, that is (see Notation 17.6.13 p.~428 of the book), 
$$
\begin{tikzcd}
(N\otimes_{\R|_U}P|_U)(U)\ar{r}\ar{d}[swap]{k_1}&(F'\otimes_\R P)(U)\ar{d}{f_1}\ni t\\ 
u\in(F''\otimes_\R P)(U)\ar{r}{\psi_1}\ar{d}[swap]{g_3}&(F\otimes_\R P)(U)\ar{d}{h_3}\\ 
\overline s\in\left(M'\otimes_\R P\right)(U)\ar{r}[swap]{\pp_1}&\left(M\otimes_\R P\right)(U) 
\end{tikzcd}
$$ 
(see \qr{ff'f''}) has also a cartesian top square, and satisfies $g_3(u)=\overline s$ and 
\begin{equation}\lb{p1uf1t}
\psi_1(u)=f_1(t).
\end{equation} 
We have used the isomorphisms 
\begin{equation}\lb{mupu}
\Gamma\left(U;M|_U\otimes_{\R|_U}P|_U\right)\iso\Gamma\big(U;\left(M\otimes_\R P\right)|_U\big)\iso\left(M\otimes_\R P\right)(U),
\end{equation} 
and similarly with $M'$ instead of $M$. Indeed, the first isomorphism in \qr{mupu} is a particular case of (18.2.5) p.~441 of the book, and the second isomorphism in \qr{mupu} results from the last two displays on p.~428 of the book. In other words, we have 
\begin{equation}\lb{nrpu}
(N\otimes_{\R|_U}P|_U)(U)\iso(F'\otimes_\R P)(U)\tm_{(F\otimes_\R P)(U)}(F''\otimes_\R P)(U).
\end{equation} 
Note that \qr{p1uf1t} implies 
$$
x:=(t,u)\in(F'\otimes_\R P)(U)\tm_{(F\otimes_\R P)(U)}(F''\otimes_\R P)(U).
$$ 
If $y$ is the element of $(N\otimes_{\R|_U}P|_U)(U)$ corresponding to $x$ under Isomorphism~\qr{nrpu}, then we get $k_1(y)=u$, and thus $\overline s=g_3(u)=g_3(k_1(y))=0$. By \qr{s=0il}, this completes the proof. 
\end{proof}

%%

\sbs{Brief comments} 

\begin{s}
P.~452, Part (i) (a) of the proof of Lemma 18.6.7. As already mentioned, $\OO_U$ and $\OO_V$ stand presumably for $\OO_X|_U$ and $\OO_Y|_V$ (and it would be better, in the penultimate display of the page, to write $\OO_V$ instead of $\OO_Y|_V$), and, a few lines before the penultimate display of the page, $f_W^{-1}:\OO_U^{\oplus n}\xr u\OO_U^{\oplus m}$ should be (I think) $f_W^{-1}:\OO_W^{\oplus n}\to\OO_W^{\oplus m}$. 

Also, one may refer to \qr{ttau} p.~\pr{ttau} and \S\ref{fdagger} p.~\pr{fdagger} to describe the morphism of sites $f_W:W\to V$. More precisely, we define, in the notation \qr{ttau}, the functor $(f_W)^\tau:((\C_Y)_V)^{\op}\to((\C_X)_W)^{\op}$ by
$$
(f_W)^\tau(V'\to V):=\big(f^\tau(V')\to f^\tau(V)\to W\big).
$$
Finally, let us rewrite explicitly one of the key equalities (see \S\ref{fdagger} p.~\pr{fdagger}): 
$$
f^\dg(\OO_Y^{\oplus nm})(W)=\col_{(f^\tau(V)\to W)\in((\C_Y)^{\op})_W}\OO_Y^{\oplus nm}(V),
$$ 
where $f^\tau(V)\to W$ is a morphism in $(\C_X)^{\op}$ (corresponding to a morphism $W\to f^t(V)$ in $\C_X$).
\end{s}
\printindex
\end{document}
