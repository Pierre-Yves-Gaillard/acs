% about "categories and sheaves" (version-0363)
% old version:
% https://docs.google.com/document/d/1n47WZ7Ka9hSBzIXCBkboFPYzhe6lHJdQZ3QZliwOs_4/edit
\documentclass[12pt]{article}
% !TEX encoding = UTF-8 Unicode
\usepackage[T1]{fontenc}
\usepackage[utf8]{inputenc}
\usepackage{amssymb,amsmath}
\usepackage[a4paper,headheight=14.5pt]{geometry}
\usepackage{makeidx}
\usepackage{tikz-cd}
\usepackage[pdfusetitle]{hyperref}
\usepackage{datetime}
\usepackage{amsthm}
%\usepackage{comment}
\usepackage{fancyhdr}
%\usepackage{showkeys}%%\usepackage{showlabels}
\addtolength{\parskip}{.5\baselineskip}
\makeindex
\pagestyle{fancy}

\newtheorem{thm}{Theorem}
\newtheorem{lem}[thm]{Lemma}
\newtheorem{prop}[thm]{Proposition}
\newtheorem{cor}[thm]{Corollary}
\newtheorem{df}[thm]{Definition}
\newtheorem{nota}[thm]{Notation}

\theoremstyle{remark}
\newtheorem{claim}[thm]{Claim}
\newtheorem{cond}[thm]{Condition}
\newtheorem{conv}[thm]{Convention}
\newtheorem{rk}[thm]{Remark}

\theoremstyle{definition}
\newtheorem{s}[thm]{\S}

\hyphenation{Grothen-dieck mono-mor-phism mono-mor-phisms car-di-nal car-di-nals}

\newcommand{\dis}{\displaystyle}\newcommand{\ds}{\displaystyle}
\newcommand{\bu}{\bullet}
\newcommand{\nn}{\noindent}
\newcommand{\cc}{\mathcal}
\newcommand{\mc}{\mathcal}
\newcommand{\bb}{\mathbb}
\newcommand{\oo}{\operatorname}

\newcommand{\A}{\mathcal A}
\newcommand{\B}{\mathcal B}
\newcommand{\C}{\mathcal C}
\newcommand{\F}{\mathcal F}
\newcommand{\G}{\mathcal G}
\newcommand{\J}{\mathcal J}
\newcommand{\M}{\mathcal M}
\newcommand{\SSS}{\mathcal S}
\newcommand{\U}{\mathcal U}
\newcommand{\V}{\mathcal V}
\newcommand{\Cat}{\mathbf{Cat}}
\newcommand{\Set}{\mathbf{Set}}
\newcommand{\pt}{\{\text{pt}\}}

\newcommand{\ee}{\varepsilon}
\newcommand{\pp}{\varphi}
\newcommand{\epi}{\twoheadrightarrow}
\newcommand{\fthat}{(f^t)\ \widehat{}\ }
\newcommand{\HOM}{\cc H\!\mathit{om}}
\newcommand{\mono}{\rightarrowtail}
\newcommand{\incl}{\hookrightarrow}
\newcommand{\parar}{\rightrightarrows}
\newcommand{\si}{\Leftarrow}
\newcommand{\ssi}{\Leftrightarrow}
\newcommand{\then}{\Rightarrow}
\newcommand{\xl}{\xleftarrow}
\newcommand{\xr}{\xrightarrow}

% LIMITS
% old
\newcommand{\ilim}{\operatornamewithlimits{\underset{\longrightarrow}{lim}}}
\newcommand{\plim}{\operatornamewithlimits{\underset{\longleftarrow}{lim}}}
% new
\DeclareMathOperator*{\colim}{colim}
\DeclareMathOperator*{\col}{colim}
\DeclareMathOperator*{\icolim}{``\!\colim\!"}
\DeclareMathOperator*{\ic}{``\!\colim\!"}

\DeclareMathOperator{\Ad}{Add}
\DeclareMathOperator{\card}{card}
\DeclareMathOperator{\Coim}{Coim}
\DeclareMathOperator{\Coker}{Coker}
\DeclareMathOperator{\D}{D}
\DeclareMathOperator{\Ext}{Ext}
\DeclareMathOperator{\Ima}{Im}
\DeclareMathOperator{\IM}{IM}
\DeclareMathOperator{\hy}{h}
\DeclareMathOperator{\ky}{k}
\DeclareMathOperator{\id}{id}
\DeclareMathOperator{\jj}{j}
\DeclareMathOperator{\Fct}{Fct}
\DeclareMathOperator{\Hom}{Hom}
\DeclareMathOperator{\RHom}{RHom}
\DeclareMathOperator{\Ind}{Ind}
\DeclareMathOperator{\Ker}{Ker}
\DeclareMathOperator{\Mc}{Mc}
\DeclareMathOperator{\Mod}{Mod}
\DeclareMathOperator{\Mor}{Mor}
\DeclareMathOperator{\Ob}{Ob}
\DeclareMathOperator{\op}{op}
\DeclareMathOperator{\PSh}{PSh}
\DeclareMathOperator{\Qis}{Qis}
%\ar[yshift=0.7ex]{r}\ar[yshift=-0.7ex]{r}

%%%%%%%%%%%%%%%%%%%%%%%%%%%%%%%%%%%%%%%%%%%%%%%%%%%%%%%%%%%

\title{About \em{Categories and Sheaves}}
\author{Pierre-Yves Gaillard}
\date{\today, \currenttime}

\begin{document}

\maketitle

\nn The last version of this text is available at

\nn\href{http://iecl.univ-lorraine.fr/~Pierre-Yves.Gaillard/DIVERS/KS/}{http://iecl.univ-lorraine.fr/$\sim$Pierre-Yves.Gaillard/DIVERS/KS/}

\tableofcontents\newpage

\nn The purpose of this text is to make a few comments about the book 

\textbf{Categories and Sheaves} by Kashiwara and Schapira, Springer 2006, 

\nn referred to as ``the book'' henceforth. 

An important reference is

\nn[GV] Grothendieck, A. and Verdier, J.-L. (1972). Préfaisceaux. In Artin, M., Grothendieck, A., and Verdier, J.-L., editors, Théorie des Topos et Cohomologie Etale des Schémas, volume 1 of Séminaire de géométrie algébrique du Bois-Marie, 4, pages 1-218. Springer. 

Links to the above text of Grothendieck and Verdier are available at \smallskip

\centerline{\href{http://goo.gl/df2Xw}{http://goo.gl/df2Xw}}

Here are two useful links:

\nn Schapira's Errata: \href{http://people.math.jussieu.fr/~schapira/books/Errata.pdf}{http://people.math.jussieu.fr/$\sim$schapira/books/Errata.pdf},

\nn nLab entry: \href{http://ncatlab.org/nlab/show/Categories+and+Sheaves}{http://ncatlab.org/nlab/show/Categories+and+Sheaves}. 

The tex and pdf files for this text are available at
 
\nn\href{http://iecl.univ-lorraine.fr/~Pierre-Yves.Gaillard/DIVERS/KS/}{http://iecl.univ-lorraine.fr/$\sim$Pierre-Yves.Gaillard/DIVERS/KS/}

\nn\href{https://github.com/Pierre-Yves-Gaillard/acs}{https://github.com/Pierre-Yves-Gaillard/acs}

\nn\href{http://goo.gl/mE37bM}{http://goo.gl/mE37bM}

\nn\href{https://app.box.com/s/ktfju6mts4bq3loknnrt}{https://app.box.com/s/ktfju6mts4bq3loknnrt}

\nn\href{http://goo.gl/klKgiW}{http://goo.gl/klKgiW}

\nn\href{https://www.mediafire.com/folder/am7yqw1whitdg/}{https://www.mediafire.com/folder/am7yqw1whitdg/}

\nn\href{https://mega.co.nz/#F!udI0CahS!YSL4YDDiougL0svrOHlyRA}{https://mega.co.nz/\#F!udI0CahS!YSL4YDDiougL0svrOHlyRA}

More links are available at \href{http://goo.gl/df2Xw}{http://goo.gl/df2Xw}. 

I have rewritten some of the proofs in the book. Of course, I'm not suggesting that my wording is better than that of Kashiwara and Schapira! I just tried to make explicit a few points which are implicit in the book. 

I adhere to Bourbaki's\index{Bourbaki} set theory as expounded in the book \textbf{Théorie des ensembles}, N. Bourbaki, Hermann, Paris, 1970. (I'm ignoring the ``Fascicule de résultats'' in the above book because I don't understand it.) 

The notation of the book will be freely used. We will sometimes write $\B^\A$ for $\Fct(\A,\B)$, $\alpha_i$ for $\alpha(i)$, $fg$ for $f\circ g$, and some parenthesis might be omitted. We write $\bigsqcup$\index{$\bigsqcup$} instead of $\coprod$\index{$\coprod$} for the coproduct\index{coproduct}. 

Following a suggestion of Pierre Schapira's, we shall denote projective limits\index{projective limit} by $\lim$\index{$\lim$} instead of $\plim$\index{$\plim$}, and inductive limits\index{inductive limit} by $\colim$\index{$\colim$} instead of $\ilim$\index{$\ilim$}. 

Thank you to Pierre Schapira for his interest!

%%%

\section{U-categories and U-small Categories\label{ucat}\index{$\U$-category}\index{$\U$-small category}}

Here are a few comments about the definition of a $\U$-category on p.~11 of the book. Let $\U$ be a universe. Recall that an element of $\U$ is called a $\U$-set\index{$\U$-set}. The following definitions are used in the book: 

\begin{df}[$\U$-category]\label{ucatg} 
A $\U$-{\em category} is a category $\C$ such that, for all objects $X,Y$, the set $\Hom_\C(X,Y)$ of morphisms from $X$ to $Y$ is equipotent to some $\U$-set. 
\end{df} 

\begin{df}[$\U$-small category] 
The category $\C$ is $\U$-{\em small} if in addition the set of objects of $\C$ is equipotent to some $\U$-set. 
\end{df} 

One could also consider the following variant: 

\begin{df}[$\U$-category]\label{ducat} 
A $\U$-{\em category} is a category $\C$ such that, for all objects $X,Y$, the set $\Hom_\C(X,Y)$ is a $\U$-set. 
\end{df} 

\begin{df}[$\U$-small category]\label{small}
The category $\C$ is $\U$-{\em small}\index{$\U$-small category} if in addition the set of objects of $\C$ is a $\U$-set. 
\end{df} 

Note that a category $\C$ is a $\U$-category in the sense of Definition~\ref{ucatg} if and only if there is a $\U$-category in the sense of Definition~\ref{ducat} which is isomorphic to $\C$, and similarly for $\U$-small categories. 
%
\begin{center}\fbox{In this text we shall always use Definitions \ref{ducat} and \ref{small}.}
\end{center}

We often assume implicitly that a universe $\U$ has been chosen, and we say ``category'' and ``small category'' instead of ``$\U$-category'' and ``$\U$-small category''.

%%%

\section{Typos and Details}

$*$ P.~11, Definition 1.2.1, Condition (b): $\Hom(X,X)$ should be $\Hom_{\C}(X,X)$. 

\nn$*$ P.~14, definition of $\Mor(\C)$. As the hom-sets of $\C$ are not assumed to be disjoint, it seems better to define $\Mor(\C)$ as a category of functors. See \S\ref{d125} p.~\pageref{d125}. 

\nn$*$ P.~25, Corollary 1.4.6. Due to the definition of $\U$-small category used in this text (see Section~\ref{ucat} p.~\pageref{ucat}), the category $\C_A$ of the corollary is no longer $\U$-small, but only canonically isomorphic to some $\U$-small category. 

\nn$*$ P.~25, Proof of Corollary 1.4.6 (second line): $\hy_{\C}$ should be $\hy_{\C'}$. 

\nn$*$ P.~33, Exercise 1.19: the arrow from $L_1\circ R_1\circ L_2$ to $L_2$ should be $\eta_1\circ L_2$ instead of $\varepsilon_1\circ L_2$. 

\nn$*$ P.~37, Remark 2.1.5: ``Let $I$ be a small set'' should be ``Let $I$ be a small category''.

\nn$*$ P.~41, sixth line: (i) should be (a). 

\nn$*$ P.~52, fourth line: $\Mor(I,\C)$ should be $\oo{Fct}(I,\C)$.

\nn$*$ P.~53, Part (i) (c) of the proof of Theorem 2.3.3 (Line 2): ``$\beta\in\oo{Fct}(J,\A)$'' should be ``$\beta\in\oo{Fct}(J,\C)$''.

\nn$*$ P.~54, second display: we should have $i\to\pp(j)$ instead of $\pp(j)\to i$.

\nn$*$ P.~58, Corollary 2.5.3: The assumption that $I$ and $J$ are small is not necessary. (The statement does not depend on the Axiom of Universes.) 

\nn$*$ P.~58, Proposition 2.5.4: Parts (i) and (ii) could be replaced with the statement: ``If two of the functors $\pp,\psi$ and $\pp\circ\psi$ are cofinal, so is the third one''.

\nn$*$ Pp.~63-64, statement and proof of Corollary 2.7.4: all the $h$ are slanted, but they should be straight.

\begin{s}\label{27i}
P.~65, Exercise 2.7 (i): ``$\cdot\times_ZY:\Set_Z\to\Set_Z$'' should be ``$\cdot\times_ZY:\Set_Z\to\Set_Y$''.
\end{s}

\nn$*$ P.~74, last four lines: $\alpha$ should be replaced with $\pp$.

\nn$*$ P.~80, last display: a ``$\ds\ilim$'' is missing.

\nn$*$ P.~83, Statement of Proposition 3.3.7 (iv) and (v): $k$ might be replaced with $R$. 

\nn$*$ Pp 83 and 85, Proof of Proposition 3.3.7 (iv): ``Proposition 3.1.6'' should be ``Theorem 3.1.6''. Same typo on p.~85, Line 6.

\nn$*$ P.~84, Proposition 3.3.13. It is clear from the proof (I think) that the intended statement was the following one: If $\C$ is a category admitting finite inductive limits and if $A:\C^{\op}\to\Set$ is a functor, then we have 
$$
\C\text{ small and }\C_A\text{ filtrant }\then A\text{ left exact }\then\C_A\text{ filtrant}.
$$

\nn$*$ P.~85, proof of Proposition 3.3.13, proof of implication ``$\C_A$ filtrant $\then$ $A$ commutes with finite projective limits''. One can either use Corollary~\ref{316} p.~\pageref{316}, or notice that $\C$ can be assumed to be small. (The argument is the same in both cases.)

\nn$*$ P.~88, Proposition 3.4.3 (i). It would be better to assume that $\C$ admits small inductive limits.

\nn$*$ P.~89, last sentence of the proof of Proposition 3.4.4. The argument is slightly easier to follow if $\psi'$ is factorized as 
$$
(J_1)^{j_2}\xr a(J_1)^{\psi_2(j_2)}\xr b(K_1)^{\psi_2(j_2)}\xr c(K_1)^{\pp_2(i_2)}.
$$ 
Then $a,b$ and $c$ are respectively cofinal by Parts (ii), (iii), and (iv) of Proposition 3.2.5 p.~79 of the book.

\nn$*$ P.~90, Exercise 3.2: ``Proposition 3.1.6'' should be ``Theorem 3.1.6''.

\nn$*$ P.~115, line 4: ``two morphisms $i_1,i_2:Y\to Y\sqcup_XY$'' should be ``two morphisms $i_1,i_2:Y\parar Y\sqcup_XY$''. 

\nn$*$ P.~115, Line 8: $i_1\circ g=i_2\circ g$ should be $g\circ i_1=g\circ i_2$.

\nn$*$ P.~120, proof of Theorem 5.2.6. We define $u':X'\to F$ as the element of $F(X')$ corresponding to the element $(u,u_0)$ of $F(X)\times_{F(X_1)}F(Z_0)$ under the natural bijection. (Recall $X':=X\sqcup_{X_1}Z_0$.)

\nn$*$ P.~121, proof of Proposition 5.2.9. The fact that, in Proposition 5.2.3 p.~118 of the book, only Part~(iv) needs the assumption that $\C$ admits small coproducts is implicitly used in the sequel of the book.

\nn$*$ P.~128, proof of Theorem 5.3.9. Last display: $\sqcup$ should be $\cup$. It would be simpler in fact to put 
$$
\Ob(\F_n):=\{Y_1\sqcup_XY_2\ |\ X\to Y_1\text{ and }X\to Y_2\text{ are morphisms in }\F_{n-1}\}.
$$ 

\nn$*$ P.~128, proof of Theorem 5.3.9,, just before the ``q.e.d.'': Corollary 5.3.5 should be Proposition 5.3.5.

\nn$*$ P.~132, Line 2: It would be slightly better to replace ``for small and filtrant categories $I$ and $J$'' with ``for small and filtrant categories $I$ and $J$, and functors $\alpha:I\to\C,\beta:J\to\C$''.

\nn$*$ P.~132, Line 3: $\Hom_\C(A,B)$ should be $\Hom_{\Ind(\C)}(A,B)$.

\nn$*$ P.~132, Lines 4 and 5: \guillemotleft We may replace ``filtrant and small'' by ``filtrant and cofinally small'' in the above definition\guillemotright: see Proposition~\ref{355} p.~\pageref{355}.

\nn$*$ P.~132, Corollary 6.1.6: The following fact is implicit. Let $\C\xrightarrow{F}\C'\xrightarrow{G}\C''$ be functors, let $X'$ be in $\C'$, and assume that $G$ is fully faithful. Then the functor $\C_{X'}\to\C_{G(X')}$ induced by $G$ is an isomorphism.

\nn$*$ P.~133, proof of Proposition 6.1.8, Line 2: ``It is enough to show that $A$ belongs to $\Ind(\C)$''. More generally: Let $I\xrightarrow{\alpha}\C\xrightarrow{F}\C'$ be functors. Assume that $F$ is fully faithful, and that there is an $X$ in $\C$ such that $F(X)\simeq\colim F(\alpha)$. Then $X\simeq\colim\alpha$. The proof is obvious.

\nn$*$ P.~133, Proposition 6.1.9. ``There exists a unique functor ...'' should be ``There exists a functor ... Moreover, this functor is unique up to unique isomorphism.''

\begin{s}\label{s133ii} 
P.~133. In Part (ii) of Proposition 6.1.9 the authors, I think, intended to write 
$$
``\ilim"(IF\circ\alpha)\xrightarrow\sim IF(``\ilim"\alpha)
$$
instead of 
$$
IF(``\ilim"\alpha)\xrightarrow\sim``\ilim"(IF\circ\alpha). 
$$ 
\end{s}

\nn$*$ P.~134, proof of Proposition 6.1.12: ``$\C_A\times\C_{A'}$'' should be ``$\C_A\times\C'_{A'}$'' (twice).

\nn$*$ P.~136, proof of Proposition 6.1.16: see \S\ref{cipc} p.~\pageref{cipc}.

\nn$*$ P.~136, proof of Proposition 6.1.18. Second line of the proof: ``Corollary 6.1.14'' should be ``Corollary 6.1.15''. 

\nn$*$ P.~136, last line: ``the cokernel of $(\alpha(i),\beta(i))$'' should be ``the cokernel of\linebreak $(\pp(i),\psi(i))$''. 

\nn$*$ P.~138, second line of Section 6.2: ``the functor $``\ds\lim_{\longrightarrow}"$ is representable in $\C$'' should be ``the functor $``\ds\lim_{\longrightarrow}"\alpha$ is representable in $\C$''. Next line: ``natural functor'' should be ``natural morphism''.

\nn$*$ P.~141, Corollary 6.3.7 (ii): $\id$ should be $\id_\C$.

\nn$*$ P.~143, third line of the proof of Proposition 6.4.2: $\{Y_i\}_{I\in I}$ should be $\{Y_i\}_{i\in I}$.

\nn$*$ P.~144, proof of Proposition 6.4.2, Step (ii), second sentence: It might be better to state explicitly the assumption that $X_\nu^i$ is in $\C_\nu$ for $\nu=1,2$. 

\nn$*$ P.~146, Exercise 6.3. ``Let $\C$ be a small category'' should be ``Let $\C$ be a category''.

\nn$*$ P.~150, before Proposition 7.1.2. One could add after ``This implies that $F_{\SSS}$ is unique up to unique isomorphism'': Moreover we have $Q^\dagger F\simeq F_{\SSS}\simeq Q^\ddagger F$.

\nn$*$ P.~153, statement of Lemma 7.1.12. The readability might be slightly improved by changing $s:X\to X'\in\mathcal S$ to $(s:X\to X')\in\mathcal S$. Same for Line 4 of the proof of Lemma 7.1.21 p.~157.

\nn$*$ P.~156, first line of the first display and first line after the first display: $\C_{\cc S}$ should be $\C_{\cc S}^r$.

\nn$*$ P.~160, second line after the diagram: ``commutative'' should be ``commutative up to isomorphism''. Line 7 when counting from the bottom to the top: $F(s)$ should be $Q_{\mathcal S}(s)$.

\nn$*$ P.~163, last sentence of Remark 7.4.5: ``right localizable'' should be ``universally right localizable''.

\nn$*$ P.~168, Line 9: ``$f:X\to Y$'' should be ``$f:Y\to X$''.

\nn$*$ P.~170, Corollary 8.2.4. The period at the end of the last display should be moved to the end of the sentence.

\nn$*$ P.~172, proof of Lemma 8.2.10, first line: ``composition morphism'' should be ``addition morphism''.

\nn$*$ P.~179, about one third of the page: ``a complex 
\begin{tikzcd}X\ar{r}{u}&Y\ar[yshift=0.7ex]{r}{v}\ar[yshift=-0.7ex]{r}[swap]{w}&Z\end{tikzcd}'' 
should be ``a sequence 
\begin{tikzcd}X\ar{r}{u}&Y\ar[yshift=0.7ex]{r}{v}\ar[yshift=-0.7ex]{r}[swap]{w}&Z\end{tikzcd}''.

\nn$*$ P.~180, Lemma 8.3.11 (b) (i): $\Coker f\xr\sim\Coker f'$ should be $\Coker f'\xr\sim\Coker f$. Proof of Lemma 8.3.11: The notation $\Hom$ for $\Hom_\C$ occurs eight times. Lemma 8.3.11 is stated below as Lemma~\ref{8311} p.~\pageref{8311}. 

\nn$*$ P.~181, Lemma 8.3.13, second line of the proof: $h\circ f^2$ should be $f^2\circ h$.

\nn$*$ P.~186, Corollary 8.3.26. The proof reads: ``Apply Proposition 5.2.9''. One could add: ``and Proposition 5.2.3 (v)''.

\nn$*$ P.~187, proof of Proposition 8.4.3. More generally, if $F$ is a left exact additive functor between abelian categories, then, in view of the observations made on p.~183 of the book (and especially Exercise 8.17), $F$ is exact if and only if it sends epimorphisms to epimorphisms. (A solution to the important Exercise 8.17 is given in Section~\ref{817} p.~\pageref{817}.)

\nn$*$ P.~188. In the second diagram $Y'\overset{l'}{\rightarrowtail}Z$ should be $Y'\overset{l'}{\rightarrowtail}X$. After the second diagram: ``the set of isomorphism classes of $\Delta$'' should be ``the set of isomorphism classes of objects of $\Delta$''.

\nn$*$ P.~190, proof of Proposition 8.5.5 (a) (i): all the $R$ should be $R^{\op}$, except for the last one.

\nn$*$ P.~191: The equality $\psi(M)=G\otimes_RM$ is used in the second display, whereas $\psi(M)=M\otimes_RG$ is used in the third display. It might be better to use $\psi(M)=M\otimes_{R^{\op}}G$ both times. 

\nn$*$ P.~191, Proof of Theorem 8.5.8 (iii): ``the product of finite copies of $R$'' should be ``the product of finitely many copies of $R$''.

\nn$*$ P.~196, Proposition 8.6.9, last sentence of the proof of (i)$\then$(ii): ``Proposition 8.3.12'' should be ``Lemma 8.3.12''.

\nn$*$ P.~201, proof of Lemma 8.7.7, first line: ``we can construct a commutative diagram''. I think the authors meant ``we can construct an exact commutative diagram''.

\nn$*$ P.~218, middle of the page: ``$b:=\inf(J\setminus A)$'' should be ``$b:=\inf(J\setminus A')$'' (the prime is missing). 

\nn$*$ P.~218, Proof of Lemma 9.2.5, first sentence: ``Proposition 3.2.4'' should be ``Proposition 3.2.2''.

\nn$*$ P.~220, part (ii) of the proof of Proposition 9.2.9, last sentence of the first paragraph: $s(j)$ should be $\tilde s(j)$.

\nn$*$ P.~221, Lemma 9.2.15. ``Let $A\in\C$'' should be ``Let $A\in\C^\wedge$''.

\nn$*$ P.~224, proof of Proposition 9.3.2, line 2: ``there exist maps $S\to A(G)\to S$ whose composition is the identity'' should be ``there exist maps $A(G)\to S$ such that the composition $S\to A(G)\to S$ is the identity of $S$''.

\nn$*$ Pp 224-228, from Proposition 9.3.2 to the end of the section. The notation $G^{\sqcup S}$, where $S$ is a set, is used twice (each time on p.~224), and the notation $G^{\coprod S}$ is used many times in the sequel of the section. I think the two pieces of notation have the same meaning. If so, it might be slightly better to uniformize the notation.

\begin{s}\label{225}
P.~225, line 3: ``Since $N_s$ is a subobject of $A$ and $\card(A(G))<\pi$'' should be ``Since $\card(A(G))<\pi$''.

\nn$*$ P.~225, line~4: ``there exists $i_0\to i_1$ such that $N_{i_1}\to A$ is an epimorphism'' should be ``there exists $s:i_0\to i$ such that $N_s\to A$ is an epimorphism''.
\end{s}

\nn$*$ P.~226, four lines before the end: ``By 9.3.4 (c)'' should be ``By (9.3.4) (c)'' (the parenthesis are missing).

\nn$*$ P.~227. The second sentence uses Proposition~\ref{34i} p.~\pageref{34i}.

\nn$*$ P.~228, line~3: $\C$ should be $\C_\pi$.

\nn$*$ P.~228, Corollary 9.3.6: $\ilim$ should be $\sigma_\pi$.

\begin{s}\label{228}
P.~228: It might be better to state Part~(iv) of Corollary 9.3.8 as ``$G$ is in $\cc S$'', instead of ``there exists an object $G\in\cc S$ which is a generator of $\C$''. (Indeed, $G$ is already mentioned in Condition (9.3.1), which is one of the assumptions of Corollary 9.3.8.)
\end{s}

\nn$*$ P.~229, proof of 9.4.3 (i): it might be better to write ``containing $\mathcal S$ strictly'' (or ``properly''), instead of just ``containing $\mathcal S$''. 

\nn$*$ P.~229, proof of 9.4.4: ``The category $\C^X$ is nonempty, essentially small ...'': the adverb ``essentially'' is not necessary since $\C$ is supposed to be small.

\nn$*$ P.~237: ``Proposition 9.6.3'' should be ``Theorem 9.6.3'' (twice). 

\nn$*$ P.~237, proof of Corollary 9.6.6, first display: ``$\psi:\C\to\C$'' should be ``$\psi:\C\to\mathcal I_{inj}$''. 

\nn$*$ P.~237, end of proof of Corollary 9.6.6: it might be slightly more precise to write ``$X\to\iota(\psi(X))=K^{\Hom_\C(X,K)}$'' instead of ``$X\to\psi(X)=K^{\Hom_\C(X,K)}$''.

\nn$*$ P.~244, second diagram: the arrow from $X'$ to $Z'$ should be dotted. (For a nice picture of the octahedral diagram see p.~49 of Mili\v{c}i\'c's text

\href{http://www.math.utah.edu/~milicic/Eprints/dercat.pdf}{http://www.math.utah.edu/$\sim$milicic/Eprints/dercat.pdf}.)

\nn$*$ P.~245, beginning of the proof of Proposition 10.1.13: The letters $f$ and $g$ being used in the sequel, it would be better to write $X\xr fY\xr gZ\to TX$ instead of $X\to Y\to Z\to TX$. 

\nn$*$ P.~245, first display in the proof of Proposition 10.1.13: The subscript $\cc D$ is missing (three times) in $\Hom_{\cc D}$.

\nn$*$ P.~250, Line 1: ``TR3'' should be ``TR2''. After the second diagram, $s\circ f$ should be $f\circ s$.

\nn$*$ P.~251, right after Remark 10.2.5: ``Lemma 7.1.10'' should be ``Proposition 7.1.10''.

\nn$*$ P.~252, last five lines:

$\bu$ ``$u$ is represented by morphisms $u':\oplus_i\ X_i\xr{u'}Y'\xleftarrow sY$'' should be ``$u$ is represented by morphisms $\oplus_i\ X_i\xr{u'}Y'\xleftarrow sY$'',

$\bu$ $v'_i$ should (I believe) be $u'_i$,

$\bu$ $Q(u)$ should be $Q(u')$.

\nn$*$ P.~254. The functor $RF$ of Notation 10.3.4 coincides with the functor $R_{\cc NQ}F$ of Definition 7.3.1 p.~159 of the book.

\nn$*$ P.~266, Exercise 10.6. I think the authors forgot to assume that the top left square commutes.

\nn$*$ P.~278: The first display should start with $T''(s'')$ instead of $T''(s)$.

\nn$*$ P.~287, first display after Proposition~11.5.4: $v(X^{n,m})$ should be $v(X)^{n,m}$.

\begin{s}\label{290}
\nn$*$ P.~290, Line 17: as indicated in Pierre Schapira's Errata, one should read 
$$
d^{''n,m}=\Hom_\C((-1)^{m+1}d_X^{-m-1},Y^n).
$$
\end{s}

\nn$*$ P.~290, Line -3: ``We define the functor'' should be ``We define the isomorphisms of functors''.

\nn$*$ P.~303, just after the diagram: ``the exact sequence (12.2.2) give rise'' should be ``the exact sequence (12.2.2) gives rise''.

\nn$*$ P.~320, Display (13.1.2): we have $\oo{Qis}=N^{\oo{ub}}(\C)$.

\nn$*$ P.~321, Line 8: $\widetilde\tau\,{}^{\ge n}(X)\to\widetilde\tau\,{}^{\ge n}(X)$ should be $\widetilde\tau\,{}^{\ge n}(X)\to\tau^{\ge n}(X)$.

\nn$*$ P.~328, Line 8: I think the authors meant ``$X^i\to Z^i$ is an isomorphism for $i>n+d\,$'' instead of ``$i\ge n+d\,$''.

\begin{s}\label{1341}
P.~337, Theorem 13.4.1. ``Let $\C$ be an abelian category'' should be ``Let $\C$ be an abelian category admitting countable products'', and ``right localizable at $(Y,X)$'' should be ``universally right localizable at $(Y,X)$, and let $\oo{RHom}_\C$ denote its right localization''.
\end{s}

\nn$*$ P.~359, Line 3: $\sigma$ should be sh.

\nn$*$ P.~360, Line 5 of Step (ii) of the proof of Theorem 14.4.5: ``Then $X''$ is an exact complex in $\oo K^-(\cc P)$'' should be (I think) ``Then $X''$ is an exact complex in $\oo K^-(\cc C)$''.

\nn$*$ P.~364, Step (g) of the proof of Theorem 14.4.8: $\mc P_1=\oo K^-(\C_1)$ should be $\mc P_1=\C_1$.

\nn$*$ P.~365, line between the last two displays: ``adjoint'' should be ``derived''.

\nn$*$ P.~392, Lemma 16.1.6 (ii). It would be better to write $v:C\to U$ instead of $u:C\to U$ and $t\circ v$ instead of $t\circ u$.

\nn$*$ P.~396, proof of Lemma 16.2.4 (ii), last sentence of the proof: It would be better (I think) write ``by LE2 and LE3'' instead of ``by Proposition 16.1.11 (ii)''.

\nn$*$ P.~401, Line 6: $B''\to B$ should be $B''\to B'$.

\nn$*$ P.~406, first line of the second display: $(\C_Y)^\wedge$ should be $\C_Y$ (twice). (See \S\ref{fhat} p.~\pageref{fhat}.)

\nn$*$ P.~409, line 2: $\lambda\circ(\oo h_X^t)_A\simeq\oo h_A$ should be $\lambda\circ(\oo h_X^t)_A\simeq\oo h_A^t$. 

\begin{s}\label{1722}
P.~412, proof of Lemma 17.2.2 (ii), (b)$\then$(a), Step~(3). ``Since $\fthat(u_V)$ is an epimorphism by (2), $\fthat(u_V)$ is a local isomorphism'' should be ``Since $\fthat(u_V)$ is a local epimorphism by (2), $\fthat(u_V)$ is a local isomorphism''.
\end{s}

\nn$*$ P.~414, line before the last display: $h_X^\ddagger F$ should be $\oo h_X^\ddagger F$, \emph{i.e.} the h should be straight, not slanted. 

\nn$*$ P.~417, first sentence of the paragraph containing Display (17.4.2): $A,A'\in\C^\wedge$ should be $A,A'\in\C_X^\wedge$. 

\nn$*$ P.~418, last display: 
$$
\ilim:\ilim_{(B\to A)\in\cc{LI}_A}F(B)\to\ilim_{(B\to A)\in\cc{LI}_A}F^b(B)
$$ 
should be 
$$
\ilim_{(B\to A)\in\cc{LI}_A}:\ilim_{(B\to A)\in\cc{LI}_A}F(B)\to\ilim_{(B\to A)\in\cc{LI}_A}F^b(B).
$$

\nn$*$ P.~419, second line: ``applying Corollary 2.3.4 to $\theta=\id_{\cc{LI}_A}$'' should be ``applying Corollary 2.3.4 to $\pp=\id_{\cc{LI}_A}$''.

\nn$*$ P.~421, Theorem 17.4.7 (i): $(h_X^\ddagger F)^b\simeq(h_X^\ddagger F^a)$ should be $(\oo h_X^\ddagger F)^b\simeq(\oo h_X^\ddagger F^a)$, \emph{i.e.} the h's should be straight, not slanted.

\nn$*$ P.~424, proof of Theorem 17.5.2 (iv). ``The functor $f^\dagger$ is left exact'' should be ``The functor $f^\dagger$ is exact''. (See \S\ref{fdagger} p.~\pageref{fdagger}.) 

\nn$*$ P.~426, Line 5: ``morphism of sites by'' should be ``morphism of sites''.

\nn$*$ P.~428, Notation 17.6.13 (i). ``For $M\in\A$, let us denote by $M_A$ the sheaf associated with the constant presheaf $\C_X\ni U\mapsto M$'' should be  

``For $M\in\A$, let us denote by $M_A$ the sheaf over $\C_A$ associated with the constant presheaf $\C_A\ni(U\to A)\mapsto M$''. 

It might also be worth mentioning that $M_A$ is called the \emph{constant sheaf over $A$ with stalk} $M$\index{constant sheaf}. 

\nn$*$ P.~437, Line 3 of Step (ii) of the proof of Lemma 18.1.5: It might be better to write $``\bigoplus_{s\in A(U)}G(U\xr sA)"$ instead of $``\coprod_{s\in A(U)}G(U\xr sA)"$; indeed $\bigoplus$ is more usual that $\coprod$ to denote the coproduct of $k$-modules. 

\nn$*$ P.~438, right after ``q.e.d.'': ``Notations (17.6.13)'' should be ``Notations 17.6.13'' (no parenthesis). 

\nn$*$ P.~438, bottom: One can add that we have $\HOM_{\cc R}(\cc R,F)\simeq F$ for all $F$ in $\oo{PSh}(\cc R)$. 

\nn$*$ P.~439, after Definition 18.2.2: One can add that we have $F\overset{\text{\tiny psh}}{\otimes}_{\cc R}\cc R\simeq F$ for $F$ in $\oo{PSh}(\cc R)$ and $F\otimes_{\cc R}\cc R\simeq F$ for $F$ in $\Mod(\cc R)$. 

\nn$*$ P.~439, Proposition 18.2.3 (ii). Here is a slightly stronger statement: If $\cc{R,S,T}$ are $k_X$-algebras, if $F$ is a $(\cc T\otimes_{k_X}\cc R^{\op})$-module, if $G$ is an $(\cc R\otimes_{k_X}\cc S)$-module, and if $H$ is an $(\cc S\otimes_{k_X}\cc T)$-module, then there are isomorphisms 
$$
\Hom_{\cc S\otimes_{k_X}\cc T}(F\otimes_{\cc R}G,H)\simeq
\Hom_{\cc R\otimes_{k_X}\cc S}(G,\HOM_{\cc T}(F,H)), 
$$ 
$$
\HOM_{\cc S\otimes_{k_X}\cc T}(F\otimes_{\cc R}G,H)\simeq
\HOM_{\cc R\otimes_{k_X}\cc S}(G,\HOM_{\cc T}(F,H)), 
$$ 
functorial with respect to $F,G$, and $H$.

\nn$*$ P.~440, last line of second display: $\Hom_{\cc R(U)}(G(U)\otimes_kF(U),H(U))$ should be $\Hom_k(F(U)\otimes_{\cc R(U)}G(U),H(U))$. 

\nn$*$ P.~440, first line of the fourth display, $\overset{\text{\tiny psh}}{\otimes}_{\cc R(V)}$ should be $\otimes_{\cc R(V)}$. 

\nn$*$ P.~441. The proof of Proposition 18.2.5 uses Display (17.1.11) p.~409 of the book and Exercise 17.5 (i) p.~431 of the book (see \S\ref{175i} p.~\pageref{175i}).  

\nn$*$ P.~442, first line of Step (ii) of the proof of Proposition 18.2.7: $\HOM_{\cc R}(\cc R\otimes k_{XA},F)$ should be $\HOM_{\cc R}(\cc R\otimes_{k_X}k_{XA},F)$. 

\nn$*$ P.~442, Line 3 of last display of Section 18.2: $\jj_{A\to X!}\jj_{A\to X}^{-1}$ should be $\jj_{A\to X}^\ddagger\jj_{A\to X*}$. 

\nn$*$ P.~442. Lemma 18.3.1 (i) follows from Proposition 17.5.1 p.~432 of the book. 

\nn$*$ P.~443, first display: On the third and fourth lines, $\HOM_{k_X}$ should be $\HOM_{k_Z}$. 

\nn$*$ P.~443, sentence preceding Lemma 18.3.2: $j_{A\to X}$ should be $\oo j_{A\to X}$ (the slanted j should be straight). 

\nn$*$ Pp 447-8, proof of Lemma 18.5.3: in (18.5.3) $M'|_U$ and $M|_U$ should be $M'(U)$ and $M(U)$, and, after the second display on p.~448, $s_1\in((\cc R^{\op})^{\oplus m}\otimes_{\cc R}P)(U)$ should be $s_1\in((\cc R^{\op})^{\oplus n}\otimes_{\cc R}P)(U)$.

\nn$*$ P.~448, Proposition 18.5.4, Line 3 of the proof: $G^{\oplus I}\epi M$ should be $\cc G^{\oplus I}\epi M$.

\nn$*$ P.~452, Part (i) (a) of the proof of Lemma 18.6.7. I think that $\cc O_U$ and $\cc O_V$ stand for $\cc O_X|_U$ and $\cc O_Y|_V$. (If this is so, it would be better, in the penultimate display of the page, to write $\cc O_V$ instead of $\cc O_Y|_V$.) 

\nn$*$ P.~452, a few lines before the penultimate display of the page, $f_W^{-1}:\cc O_U^{\oplus n}\xr u\cc O_U^{\oplus m}$ should be (I think) $f_W^{-1}:\cc O_W^{\oplus n}\to\cc O_W^{\oplus m}$.

\nn$*$ P.~494, Index. I found useful to add the following subentries to the entry ``injective'': $\cc F$-injective, 231; $F$-injective, 253, 255, 330.

%%%

\section{About Chapter 1}

\subsection{Universes (p.~9)}

The book starts with a few statements which are not proved, a reference being given instead. Here are the proofs.

A \textbf{universe}\index{universe} is a set $\mathcal U$ satisfying 

(i) $\varnothing\in\mathcal U$,

(ii) $u\in U\in\mathcal U\then u\in \mathcal U$,

(iii) $U\in\mathcal U\then\{U\}\in\mathcal U$,

(iv) $U\in\mathcal U\then\mathcal P(U)\in\mathcal U$,

(v) $I\in\mathcal U$ and $U_i\in\mathcal U$ for all $i$ $\then$ $\bigcup_{i\in I}U_i\in\mathcal U$,

(vi) $\mathbb N\in\mathcal U$.

\nn We want to prove:

(vii) $U\in\mathcal U\then\bigcup_{u\in U}u\in\mathcal U$,

(viii) $U,V\in\mathcal U\then U\times V\in\mathcal U$,

(ix) $U\subset V\in\mathcal U\then U\in\mathcal U$,

(x) $I\in \mathcal U$ and $U_i\in\mathcal U$ for all $i$ $\then$ $\prod_{i\in I}U_i\in\mathcal U$.

\nn(We have kept Kashiwara and Schapira's numbering of Conditions (i) to (x).) 

\nn Obviously, (ii) and (v) imply (vii), whereas (iv) and (ii) imply (ix). Axioms (iii), (vi), and (v) imply

(a) $U,V\in\mathcal U\then\{U,V\}\in\mathcal U$,

\nn and thus

(b) $U,V\in\mathcal U\then(U,V):=\{\{U\},\{U,V\}\}\in\mathcal U$.

\nn\textbf{Proof of (viii).} If $u\in U$ and $v\in V$, then $\{(u,v)\}\in\mathcal U$ by (ii), (b), and (iii). Now (v) yields 
$$
U\times V=\bigcup_{u\in U}\ \bigcup_{v\in V}\ \{(u,v)\}\in\mathcal U.\text{ q.e.d.} 
$$ 

Assume $U,V\in\mathcal U$, and let $V^U$ be the set of all maps from $U$ to $V$. As $V^U\in\mathcal P(U\times V)$, Statements (viii), (iv), and (ii) give

(c) $U,V\in\mathcal U\then V^U\in\mathcal U$.

\nn\textbf{Proof of (x).} As 
$$
\prod_{i\in I}\ U_i\in\mathcal P\left(\left(\bigcup_{i\in I}U_i\right)^I\right),
$$
(x) follows from (v), (c), and (iv). q.e.d.

%%

\subsection{Brief comments}

\begin{s}\label{d125}
P.~14, category of morphisms\index{category of morphisms}. Here are some comments about Definition 1.2.5 p.~14:

\begin{nota}\label{c*}
For any category $\C$ define the category $\C^*$\index{$\C^*$} as follows. The objects of $\C^*$ are the objects of $\C$, the set $\Hom_{\C^*}(X,Y)$ is defined by 
$$
\Hom_{\C^*}(X,Y):=\{Y\}\times\Hom_{\C}(X,Y)\times\{X\},
$$
and the composition is defined by 
$$
(Z,g,Y)\circ(Y,f,X):=(Z,g\circ f,X).
$$ 
\end{nota}

Note that there are natural mutually inverse isomorphisms $\C\rightleftarrows\C^*$. 

\begin{nota}\label{mor}
%
Let $\C$ be a category. Define the category $\Mor(\C)$ \index{$\Mor$} by 
$$
\Ob(\Mor(\C)):=\bigcup_{X,Y\in\Ob(\C)}\Hom_{\C^*}(X,Y),
$$
$\ds\Hom_{\Mor(\C)}((Y,f,X),(V,g,U)):=$\bigskip 

$\hfill\ds\{(a,b)\in\Hom_\C(X,U)\times\Hom_\C(Y,V)\ | \ g\circ a=b\circ f\},$\bigskip

\nn{\em i.e.} 
$$
\begin{tikzcd}
X\ar{d}[swap]{f}\ar{r}{a}&U\ar{d}{g}\\ 
Y\ar{r}[swap]{b}&V,
\end{tikzcd}
$$ 
and the composition is defined in the obvious way.
\end{nota}

Observe that a functor $\A\to\B$ is given by two maps 
$$
\Ob(\A)\to\Ob(\B),\quad\Ob(\Mor(\A))\to\Ob(\Mor(\B))
$$ 
satisfying certain conditions.

When $\C$ is a small category (see Section~\ref{ucat} p. \pageref{ucat}), we assume that the hom-sets of $\C$ are disjoint.
\end{s}

%

\begin{s}\label{ffc}
P.~16, Definition 1.2.11 (iii). Note that fully faithful functors are conservative\index{conservative}. 
\end{s}

%

\begin{s}
P.~18, Definition 1.2.16. If $F:\C\to\C'$ is a functor and $X'$ an object of $\C'$, then we have natural isomorphisms 
\begin{equation}\label{opslice}
(\C_{X'})^{\op}\simeq(\C^{\op})^{X'},\quad(\C^{X'})^{\op}\simeq(\C^{\op})_{X'}.
\end{equation}
\end{s}

%%

\subsection{The Yoneda Lemma (p. 24)\index{Yoneda Lemma}}

We state the Yoneda Lemma for the sake of completeness:

\begin{thm}[Yoneda's Lemma]\label{yol}
Let $\C$ be a category, let $h:\C\to\C^\wedge$ be the Yoneda embedding, let $F$ be in $\C^\wedge$, let $A$ be in $\C$, and define 
$$
\begin{tikzcd} 
F(A)\ar[yshift=0.7ex]{r}{\pp}&\Hom_{\C^\wedge}(h(A),F)\ar[yshift=-0.7ex]{l}{\psi}
\end{tikzcd}
$$
by 
\begin{equation}\label{yo}
\pp(a)_X(f):=F(f)(a),\quad\psi(\theta):=\theta_A(\id_A)
\end{equation}
for 
$$
a\in F(A),\quad X\in\C,\quad f\in\Hom_\C(X,A),\quad\theta\in\Hom_{\C^\wedge}(h(A),F):
$$ 
$$
f\in\Hom_\C(X,A)\xr{\pp(a)_X}F(X)\xleftarrow{F(f)}F(A)\ni a.
$$
Then $\pp$ and $\psi$ are mutually inverse bijections. In the particular case where $F$ is equal to $h(B)$ for some $B$ in $\C$, we get 
$$
\pp(a)=h(a)\in\Hom_{\C^\wedge}(h(A),h(B)).
$$
This shows that $h$ is fully faithful.

Let $k:\C\to\C^\vee$ be the Yoneda embedding, let $F$ be in $\C^\vee$, let $A$ be in $\C$, and define 
$$
\begin{tikzcd} 
F(A)\ar[yshift=0.7ex]{r}{\pp}&\Hom_{\C^\vee}(F,k(A))=\Hom_{\Set^\C}(k(A),F)\ar[yshift=-0.7ex]{l}{\psi}
\end{tikzcd}
$$
by \eqref{yo} for 
$$
a\in F(A),\quad X\in\C,\quad f\in\Hom_\C(A,X),\quad\theta\in\Hom_{\Set^\C}(k(A),F):
$$ 
$$
f\in\Hom_\C(A,X)\xr{\pp(a)_X}F(X)\xleftarrow{F(f)}F(A)\ni a.
$$
Then $\pp$ and $\psi$ are mutually inverse bijections. In the particular case where $F$ is equal to $k(B)$ for some $B$ in $\C$, we get 
$$
\pp(a)=k(a)\in\Hom_{\C^\vee}(k(B),k(A)).
$$
This shows that $k$ is fully faithful.
\end{thm}
%
\begin{proof}
It suffices to prove the first statement: We have 
$$
\psi(\pp(a))=\pp(a)_A(\id_A)=F(\id_A)(a)=a
$$ 
and
$$
\pp(\psi(\theta))_X(f)=F(f)(\psi(\theta))=F(f)(\theta_A(\id_A))=\theta_X(f),
$$ 
the last equality following from the commutativity of the square 
$$
\begin{tikzcd}
h(A)(X)\ar{r}{\theta_X}&F(X)\\ 
h(A)(A)\ar{u}{h(A)(f)}\ar{r}[swap]{\theta_A}&F(A),\ar{u}[swap]{F(f)}
\end{tikzcd}
$$ 
which is equal to the square 
$$
\begin{tikzcd}
\Hom_\C(X,A)\ar{r}{\theta_X}&F(X)\\ 
\Hom_\C(A,A)\ar{u}{\circ f}\ar{r}[swap]{\theta_A}&F(A).\ar{u}[swap]{F(f)}
\end{tikzcd}
$$
\end{proof}
%
\begin{df}[universal element]\label{ue} 
Let $F:\C^{\op}\to\Set$ be a functor and $X$ an object of $\C$. An $(F,X)$\--{\em universal element}\index{universal element} is an element $u$ of $F(X)$ such that, for all $Y$ in $\C$, the map $\Hom_\C(Y,X)\to F(Y),\ f\mapsto F(f)(u)$ is bijective. 
\end{df}

The Yoneda Lemma says that $(F,X)$\--universal elements are in functorial bijection with isomorphisms $\hy_\C(X)\xr\sim F$, such an isomorphism being called a {\em representation of} $F$ {\em by} $X$.

\begin{df}[universal element]\label{ue2} 
Let $F:\C\to\Set$ be a functor and $X$ an object of $\C$. An $(F,X)$\--{\em universal element} is an element $u$ of $F(X)$ such that, for all $Y$ in $\C$, the map $\Hom_\C(X,Y)\to F(Y),\ f\mapsto F(f)(u)$ is bijective. 
\end{df}

The Yoneda Lemma says that $(F,X)$\--universal elements are in functorial bijection with isomorphisms $F\xr\sim\ky_\C(X)$, such an isomorphism being called a {\em representation of} $F$ {\em by} $X$.

%%

\subsection{Brief Comments}

\begin{s} 
P.~25, Corollary 1.4.7. A statement slightly stronger than Corollary 1.4.7 of the book can be proved more naively:

\begin{prop}\label{yp}
A morphism $f:A\to B$ in a category $\C$ is an isomorphism if and only if 
$$
\Hom_\C(X,f):\Hom_\C(X,A)\to\Hom_\C(X,B)
$$
is (i) surjective for $X=B$ and (ii) injective for $X=A$.
\end{prop}

\begin{proof} By (i) there is a $g:B\to A$ satisfying $f\circ g=\id_B$, yielding $f\circ g\circ f=f$, and (ii) implies $g\circ f=\id_A$.
\end{proof}
\end{s}

%%

\subsection{Horizontal\index{horizontal composition} and Vertical Compositions\index{vertical composition} (p. 19)}\label{hove1}

For each object $X$ of $\C_3$ the diagram 

$$
\begin{tikzcd}
%
\C_1&{}&\C_2\ar{ll}{}[near start,swap]{F_{11}}&{}&\C_3\ar{ll}{}[near start,swap]{F_{12}}\\ 
%
\C_1&\ar{u}{\theta_{11}}&\C_2\ar{ll}{}[near start,swap]{F_{21}}&\ar{u}{\theta_{12}}&\C_3\ar{ll}{}[near start,swap]{F_{22}}\\ 
%
\C_1&\ar{u}{\theta_{21}}&\C_2\ar{ll}{}[near start]{F_{31}}&\ar{u}{\theta_{22}}&\C_3\ar{ll}{}[near start]{F_{32}}
%
\end{tikzcd}
$$ 

\nn of categories, functors, and morphisms of functors yields the commutative diagram 

$$
\begin{tikzcd}
%
F_{11}F_{12}X&&F_{21}F_{12}X\ar{ll}[swap]{\theta_{11}F_{12}X}&&F_{31}F_{12}X\ar{ll}[swap]{\theta_{21}F_{12}X}\\ 
%
F_{11}F_{22}X\ar{u}{F_{11}\theta_{12}X}&&F_{21}F_{22}X\ar{ll}[swap]{\theta_{11}F_{22}X}\ar{u}{F_{21}\theta_{12}X}&&F_{31}F_{22}X\ar{ll}[swap]{\theta_{21}F_{22}X}\ar{u}[swap]{F_{31}\theta_{12}X}\\ 
%
F_{11}F_{32}X\ar{u}{F_{11}\theta_{22}X}&&F_{21}F_{32}X\ar{ll}{\theta_{11}F_{32}X}\ar{u}{F_{21}\theta_{22}X}&&F_{31}F_{32}X\ar{ll}{\theta_{21}F_{32}X}\ar{u}[swap]{F_{31}\theta_{22}X}
%
\end{tikzcd}
$$ 

\nn in $\C_1$. So, we get a well-defined morphism in $\C_1$ from $F_{31}F_{32}X$ to $F_{11}F_{12}X$, which is easily seen to define a morphism of functors from $F_{31}F_{32}$ to $F_{11}F_{12}$. 

\begin{nota}\label{nhove}
We denote this morphism of functors by
$$
\begin{pmatrix}
\theta_{11}&\theta_{12}\\ 
\theta_{21}&\theta_{22}
\end{pmatrix}:F_{31}F_{32}\to F_{11}F_{12}.
$$ 
If $\theta_{21}$ and $\theta_{22}$ are identity morphisms, we put 
$$
\theta_{11}*\theta_{12}:=
\begin{pmatrix}
\theta_{11}&\theta_{12}\\ 
\theta_{21}&\theta_{22}
\end{pmatrix}.
$$ 
If $\theta_{12}$ and $\theta_{22}$ are identity morphisms, we put 
$$
\theta_{11}\circ\theta_{21}:=
\begin{pmatrix}
\theta_{11}&\theta_{12}\\ 
\theta_{21}&\theta_{22}
\end{pmatrix}.
$$ 
\end{nota}

Let $m,n\ge1$ be integers, let $\C_1,\dots,\C_{n+1}$ be categories, let 
$$
F_{i,j}:\C_{j+1}\to\C_j,\quad1\le i\le m+1,\ 1\le j\le n
$$
be functors, let 
$$
\theta_{i,j}:F_{i+1,j}\to F_{i,j},\quad1\le i\le m,\ 1\le j\le n
$$
be morphisms of functors. For instance, if $m=2,n=4$, then we have 
$$
\begin{tikzcd}
%
\C_1&{}&\C_2\ar{ll}{}[near start,swap]{F_{11}}&{}&\C_3\ar{ll}{}[near start,swap]{F_{12}}&{}&\C_4\ar{ll}{}[near start,swap]{F_{13}}&{}&\C_5\ar{ll}{}[near start,swap]{F_{14}}\\ 
%
\C_1&\ar{u}{\theta_{11}}&\C_2\ar{ll}{}[near start,swap]{F_{21}}&\ar{u}{\theta_{12}}&\C_3\ar{ll}{}[near start,swap]{F_{22}}&\ar{u}{\theta_{13}}&\C_4\ar{ll}{}[near start,swap]{F_{23}}&\ar{u}{\theta_{14}}&\C_5\ar{ll}{}[near start,swap]{F_{24}}\\ 
%
\C_1&\ar{u}{\theta_{21}}&\C_2\ar{ll}{}[near start,swap]{F_{31}}&\ar{u}{\theta_{22}}&\C_3\ar{ll}{}[near start,swap]{F_{32}}&\ar{u}{\theta_{23}}&\C_4\ar{ll}{}[near start,swap]{F_{33}}&\ar{u}{\theta_{24}}&\C_5\ar{ll}{}[near start,swap]{F_{34}}.
\end{tikzcd}
$$ 

The following proposition is clear 

\begin{prop}
The operations $*$ and $\circ$ are associative, and, in the above setting, we have the equality 

$$
(\theta_{1,1}*\cdots*\theta_{1,n})\circ\cdots\circ(\theta_{m,1}*\cdots*\theta_{m,n})
$$ 

$$
=(\theta_{1,1}\circ\cdots\circ\theta_{m,1})*\cdots*(\theta_{1,n}\circ\cdots\circ\theta_{m,n}).
$$

\nn between functors from $F_{m+1,1}\cdots F_{m+1,n}$ to $F_{1,1}\cdots F_{1,n}$.
\end{prop}

\begin{nota}\label{nmat} 
We denote this morphism of functors by
$$
\begin{pmatrix}
\theta_{1,1}&\cdots&\theta_{1,n}\\
\vdots&&\vdots\\ 
\theta_{m,1}&\cdots&\theta_{m,n}
\end{pmatrix}:F_{m+1,1}\cdots F_{m+1,n}\to F_{1,1}\cdots F_{1,n}.
$$ 
\end{nota}

\begin{prop}\label{pil1}
We have, in the above setting,

$$
(\theta_{1,1}*\cdots*\theta_{1,n})\circ\cdots\circ(\theta_{m,1}*\cdots*\theta_{m,n})
=\begin{pmatrix}
\theta_{1,1}*\cdots*\theta_{1,n}\\
\vdots\\ 
\theta_{m,1}*\cdots*\theta_{m,n}
\end{pmatrix}
$$ 

$$
=\begin{pmatrix}
\theta_{1,1}&\cdots&\theta_{1,n}\\
\vdots&&\vdots\\ 
\theta_{m,1}&\cdots&\theta_{m,n}
\end{pmatrix}
$$

$$
=\begin{pmatrix}\theta_{1,1}\\ \vdots\\ \theta_{m,1}\end{pmatrix}*\cdots*
\begin{pmatrix}\theta_{1,n}\\ \vdots\\ \theta_{m,n}\end{pmatrix}
=(\theta_{1,1}\circ\cdots\circ\theta_{m,1})*\cdots*(\theta_{1,n}\circ\cdots\circ\theta_{m,n}).
$$
\end{prop}

\begin{df}[horizontal and vertical composition, Interchange Law]\label{dil1} 
We call $*$ the {\em horizontal composition}.\index{horizontal composition} We call $\circ$ the {\em vertical composition}.\index{vertical composition} We call the equalities in Proposition~\ref{pil1} the {\em Interchange Law}.\index{Interchange Law}
\end{df}

%%

\subsection{Equalities (1.5.8) and (1.5.9) (p.~29)}

Warning: many authors designate $\varepsilon$ by $\eta$ and $\eta$ by $\varepsilon$. 

\subsubsection{Statements}

We have a pair $(L,R)$ of adjoint functors: 
$$
\begin{tikzcd}
\C\ar[xshift=-.7ex]{d}[swap]{L}\\ 
\C'.\ar[xshift=.7ex]{u}[swap]{R}
\end{tikzcd}
$$

Using Notation~\ref{nhove} p.~\pageref{nhove}, Equalities (1.5.8) and (1.5.9) become respectively 

\begin{equation}\label{158}
(\eta*L)\circ(L*\ee)=L
\end{equation}

\nn and 

\begin{equation}\label{159}
(R*\eta)\circ(\eta*R)=R.
\end{equation}

\subsubsection{Pictures}

Let us try to illustrate these two equalities by diagrams:

Picture of $L\xleftarrow{\eta*L}LRL$:
 
$$
\begin{tikzcd}
%
\C'&{}&\C'\ar{ll}{}[swap]{1}&{}&\C\ar{ll}{}[swap]{L}&{}&\C\ar{ll}{}[swap]{1}\\ 
%
\C'&\ar{u}{\eta}&\C'\ar{ll}{}{LR}&\ar{u}{L}&\C\ar{ll}{}{L}&\ar{u}{1}&\C\ar{ll}{}{1}\\ 
%
&&&=\\ 
%
\C'&&&{}&&&\C\ar{llllll}[swap]{L}\\
%
\C'&&&\ar{u}{\eta*L}&&&\C\ar{llllll}{LRL}.
%
\end{tikzcd}
$$ 

Picture of $LRL\xleftarrow{L*\ee}L$:
 
$$
\begin{tikzcd}
%
\C'&{}&\C'\ar{ll}{}[swap]{1}&{}&\C\ar{ll}{}[swap]{L}&{}&\C\C\ar{ll}{}[swap]{RL}\\ 
%
\C'&\ar{u}{1}&\C'\ar{ll}{}{1}&\ar{u}{L}&\C\ar{ll}{}{L}&\ar{u}{\ee}&\C\ar{ll}{}{1}\\ 
%
&&&=\\ 
%
\C'&&&{}&&&\C\ar{llllll}[swap]{LRL}\\
%
\C'&&&\ar{u}{L*\ee}&&&\C\ar{llllll}{L}.
%
\end{tikzcd}
$$ 

Picture of \eqref{158}, that is, $(\eta*L)\circ(L*\ee)=L$:

$$
\begin{tikzcd}
%
\C'&&&{}&&&\C\ar{llllll}[near start,swap]{L}\\
%
\C'&&&\ar{u}{\eta*L}&&&\C\ar{llllll}[near start,swap]{LRL}\\
%
\C'&&&\ar{u}{L*\ee}&&&\C\ar{llllll}[near start]{L}\\ 
%
&&&=\\ 
%
\C'&&&{}&&&\C\ar{llllll}[swap]{L}\\ 
%
\C'&&&\ar{u}{L}&&&\C.\ar{llllll}{L}
%
\end{tikzcd}
$$ 

Picture of $R\xleftarrow{R*\eta}RLR$:
 
$$
\begin{tikzcd}
%
\C&{}&\C\ar{ll}{}[swap]{1}&{}&\C'\ar{ll}{}[swap]{R}&{}&\C'\ar{ll}{}[swap]{1}\\ 
%
\C&\ar{u}{1}&\C\ar{ll}{}{1}&\ar{u}{R}&\C'\ar{ll}{}{R}&\ar{u}{\eta}&\C'\ar{ll}{}{LR}\\ 
%
&&&=\\ 
%
\C&&&{}&&&\C'\ar{llllll}[swap]{R}\\
%
\C&&&\ar{u}{R*\eta}&&&\C'\ar{llllll}{RLR}.
%
\end{tikzcd}
$$ 

Picture of $RLR\xleftarrow{\ee*R}R$:
$$
\begin{tikzcd}
%
\C&{}&\C\ar{ll}{}[swap]{RL}&{}&\C'\ar{ll}{}[swap]{R}&{}&\C'\ar{ll}{}[swap]{1}\\ 
%
\C&\ar{u}{\ee}&\C\ar{ll}{}{1}&\ar{u}{R}&\C'\ar{ll}{}{R}&\ar{u}{1}&\C'\ar{ll}{}{1}\\ 
%
&&&=\\ 
%
\C&&&{}&&&\C'\ar{llllll}[swap]{RLR}\\
%
\C&&&\ar{u}{\ee*R}&&&\C'\ar{llllll}{R}.
%
\end{tikzcd}
$$ 

Picture of \eqref{159}, that is, $(R*\eta)\circ(\ee*R)=R$:

$$
\begin{tikzcd}
%
\C&&&{}&&&\C'\ar{llllll}[near start,swap]{R}\\
%
\C&&&\ar{u}{R*\eta}&&&\C'\ar{llllll}[near start,swap]{RLR}\\
%
\C&&&\ar{u}{\ee*R}&&&\C'\ar{llllll}[near start]{R}\\ 
%
&&&=\\ 
%
\C&&&{}&&&\C'\ar{llllll}[swap]{R}\\ 
%
\C&&&\ar{u}{R}&&&\C'.\ar{llllll}{R}
%
\end{tikzcd}
$$

\subsubsection{Proofs}

For the reader's convenience we prove \eqref{158} p.~\pageref{158} and \eqref{159} p.~\pageref{159}. It clearly suffices to prove \eqref{158}. 

Let us denote the functorial mutually inverse bijections defining the adjunction by 
%
\begin{equation}\label{bij}
\begin{tikzcd}
\Hom_\C(X,RX')\ar[yshift=0.7ex]{rr}{\lambda_{X,X'}}&&\Hom_{\C'}(LX,X'),\ar[yshift=-0.7ex]{ll}{\mu_{X,X'}}
\end{tikzcd}
\end{equation} 
%
and recall that $\ee_X$ and $\eta_{X'}$ are defined by
%
\begin{equation}\label{epseta}
\ee_X:=\mu_{X,LX}(\id_{LX}),\quad\eta_{X'}:=\lambda_{RX',X'}(\id_{RX'}).
\end{equation}

Equality \eqref{158} p.~\pageref{158} can be written 
$$
\lambda_{RLX,LX}(\id_{RLX})\circ L(\ee_X)=\id_{LX},
$$ 
and we have 
$$
\id_{LX}\overset{\text{(a)}}{=}\lambda_{RLX,LX}\big(\mu_{X,LX}(\id_{LX})\big)\overset{\text{(b)}}{=}\lambda_{RLX,LX}(\ee_X)\overset{\text{(c)}}{=}\big(\lambda_{RLX,LX}\circ(\circ\ee_X)\big)(\id_{RLX})
$$
$$
\overset{\text{(d)}}{=}\Big(\big(\circ L(\ee_X)\big)\circ\lambda_{RLX,LX}\Big)(\id_{RLX})\overset{\text{(e)}}{=}\lambda_{RLX,LX}(\id_{RLX})\circ L(\ee_X),
$$ 
the successive equalities being justified as follows:

(a) follows from \eqref{bij},

(b) follows from \eqref{epseta},

(c) is obvious,

(d) follows from the commutative square 
$$
\begin{tikzcd}
\Hom_\C(RLX,RLX)\ar{d}[swap]{\circ\ee}\ar{rr}{\lambda_{RLX,LX}}&&\Hom_{\C'}(LRLX,LX)\ar{d}{\circ L(\ee)}\\
\Hom_\C(RLX,RLX)\ar{rr}[swap]{\lambda_{X,LX}}&&\Hom_{\C'}(LRLX,LX),
\end{tikzcd}
$$ 

(e) is obvious.

%%%

\section{About Chapter 2}

\subsection{Brief Comments}

\begin{s}\label{212}
P.~36, Definition 2.1.2. 

\begin{df}[projective limit]\label{p}
If $\alpha:I^{\op}\to\C$ is a functor and $\Delta:\C\to\C^{I^{\op}}$ is the diagonal functor, then a {\em projective limit of}\index{projective limit} $\alpha$ {\em in} $\C$ is a pair 
$$
(X,p)\in\Ob(\C)\times\Hom_{\C^{I^{\op}}}(\Delta(X),\alpha)
$$
such that $p$ is a $(\Hom_{\C^{I^{\op}}}(\Delta(\ ),\alpha),X)$\--universal element\index{universal element} (see Definition~\ref{ue} p.~\pageref{ue}). For each $i$ in $I$ the morphism $p_i:X\to\alpha(i)$ is called the $i$-{\em projection}\index{projection} of $X$. (We almost always write $X$ for $(X,p)$, the mental picture being that $p$ is a structure $X$ is equipped with.)
\end{df}

Recall that the condition that $p$ is a $(\Hom_{\C^{I^{\op}}}(\Delta(\ ),\alpha),X)$\--universal element means that for each $Y$ in $\C$ the map 
$$
\Hom_\C(Y,X)\to\Hom_{\C^{I^{\op}}}(\Delta(Y),\alpha),\quad f\mapsto p\circ\Delta(f)
$$ 
is bijective. Here is a picture:
$$
Y\xr fX,\quad\Delta(Y)\xr{\Delta(f)}\Delta(X)\xr{p}\alpha.
$$

\begin{df}[inductive limit] 
If $\alpha:I\to\C$ is a functor, then an {\em inductive limit}\index{inductive limit} of $\alpha$ {\em in} $\C$ is a pair 
$$
(X,p)\in\Ob(\C)\times\Hom_{\C^I}(\alpha,\Delta(X))
$$
such that $p$ is a $(\Hom_{\C^I}(\alpha,\Delta(\ )),X)$\--universal element. For each $i$ in $I$ the morphism $p_i:\alpha(i)\to X$ is called the $i$-{\em coprojection}\index{coprojection} of $X$. (We almost always write $X$ for $(X,p)$, the mental picture being that $p$ is a structure $X$ is equipped with.)
\end{df}

Recall that the condition that $p$ is a $(\Hom_{\C^I}(\alpha,\Delta(\ )),X)$\--universal element means that for each $Y$ in $\C$ the map 
\begin{equation}\label{cue}
\Hom_\C(X,Y)\to\Hom_{\C^I}(\alpha,\Delta(Y)),\quad f\mapsto\Delta(f)\circ p
\end{equation}
is bijective. Here is a picture:
$$
X\xr fY,\quad\alpha\xr{p}\Delta(X)\xr{\Delta(f)}\Delta(Y).
$$
\end{s}

%

\begin{s}\label{c38}
P.~38, Proposition 2.1.6. Here is an example of a functor $\alpha:I\to\C^J$ such that $\colim\alpha$ exists in $\C^J$ but there is a $j$ in $J$ such that $\colim\ (\rho_j\circ\alpha)$ does not exist in $\C$. (Recall that $\rho_j:\C^J\to\C$ is the evaluation at $j\in J$.) This example is taken from Section 3.3 of the book \textbf{Basic Concepts of Enriched Category Theory} of G.M. Kelly\index{Kelly}:%\medskip 
%
\begin{center}\href{http://www.tac.mta.ca/tac/reprints/articles/10/tr10abs.html}{http://www.tac.mta.ca/tac/reprints/articles/10/tr10abs.html}
\end{center}

The category $J$ has two objects, 1, 2; it has exactly one nontrivial morphism; and this morphism goes from 1 to 2. The category $\C$ has exactly three objects, 1, 2, 3, and exactly four nontrivial morphisms, $f,g,h,g\circ f=h\circ f$, with 
$$
\begin{tikzcd}
1\ar{r}{f}&2\ar[yshift=.7ex]{r}{g}\ar[yshift=-.7ex]{r}[swap]{h}&3.
\end{tikzcd}
$$ 
Then $\C^J$ is the category of morphisms in $\C$. It is easy to see that the morphism 
%
\begin{equation}\label{38}
f\xrightarrow{(f,h)}g 
\end{equation}
%
in $\C^J$ is an epimorphism, and that this implies that the coproduct 
$$
g\sqcup_fg,
$$ 
taken with respect to (\ref{38}), exists and is isomorphic to $g$ (the coprojections being given by the identity of $g$). It is also easy to see that the coproduct $2\sqcup_12$ does not exist in $\C$.
\end{s}

%

\begin{s} 
P.~39, Proposition 2.1.7. The following slightly stronger statement holds, and is independent of the Axiom of Universes. 

Let $I, J, \C$ be categories and let 
$$
(X_{ij})_{(i,j)\in I\times J}
$$ 
be an inductive system in $\C$. Assume that $\colim_jX_{ij}$ exists in $\C$ for all $i$, and that 
\begin{equation}\label{limlim}
\colim_i\colim_jX_{ij}
\end{equation}
exists in $\C$. Then $\colim_{i,j}X_{ij}$ exists in $\C$ and is isomorphic to (\ref{limlim}).
\end{s}

%

\begin{s} 
P.~40, Proposition 2.1.10 (stated on p.~\pageref{2.1.10} below as Proposition~\ref{2.1.10}). Here is a slightly more general statement. 
%
\begin{prop}\label{2.1.10b}
Let 
$$
\begin{tikzcd}
I\ar{r}{\alpha}&\A\ar{d}[swap]{G}\ar{r}{F}&\B\\
&\C
\end{tikzcd}
$$
be functors. Assume that $\A$ admits inductive limits indexed by $I$, that $G$ commutes with such limits, and that for each $Y$ in $\B$ there is a $Z$ in $\C$ and an isomorphism 
$$
\Hom_\B(F(\ ),Y)\simeq\Hom_\C(G(\ ),Z)
$$
in $\A^\wedge$. Then $F$ commutes with inductive limits indexed by $I$.
\end{prop}
%
\begin{proof}
We have for any $Y$ in $\B$ 
$$ 
\Hom_\B\left(F\left(\colim\alpha\right),Y\right)\simeq
\Hom_\C\left(G\left(\colim\alpha\right),Z\right)
\xr\sim
\Hom_\C\left(\colim G(\alpha),Z\right)
$$
$$
\xr\sim\lim \ \Hom_\C(G(\alpha),Z)\simeq\lim \ \Hom_\B(F(\alpha),Y)\simeq\Hom_\B(\colim F(\alpha),Y),
$$ 
and the conclusion follows from Proposition~\ref{yp} p.~\pageref{yp}.
\end{proof}
\end{s}

%

\begin{s} 
P.~40, proof of Lemma 2.1.11 (minor variant).

\begin{lem}\label{l2111}
If $T$ is an object of a category $\C$, then 
$$
T\text{ is terminal }\ssi T\simeq\colim\id_\C.
$$
\end{lem}

\begin{proof} $\then$: Straightforward.

\nn$\si$: Let $\Delta:\C\to\C^\C$ be the diagonal functor, let $p\in\Hom_{\C^\C}(\id_\C,\Delta(T))$ be a $\Hom_{\C^\C}(\id_\C,\Delta(\ ))$\--universal element  (see Definition~\ref{ue} p.~\pageref{ue}), and let $X$ be an object of $\C$. It suffices to show 
%
\begin{equation}\label{2111ets}
\Hom_\C(X,T)=\{p_X\}.
\end{equation}
% 
By definition of $T$ and $p$, the map 
$$
\Hom_\C(T,T)\to\Hom_{\C^\C}(\id_\C,\Delta(T)),\quad f\mapsto\Delta(f)\circ p 
$$ 
is bijective. 

Claim: the inverse bijection is $\theta\mapsto\theta_T$. 

For any $\theta$ in $\Hom_{\C^\C}(\id_\C,\Delta(T))$ and any $Y$ in $\C$ we have $\theta_T\circ p_Y=\theta_Y$, and thus $\Delta(\theta_T)\circ p=\theta$, which implies the claim. 

The claim implies in particular 
%
\begin{equation}\label{it=pt}
\id_T=p_T.
\end{equation}

To prove \eqref{2111ets}, note that we have for $f$ in $\Hom_\C(X,T)$
$$
f=p_T\circ f=p_X,
$$
the first equality following from \eqref{it=pt}, and the second one from the definition of $\Hom_{\C^\C}(\id_\C,\Delta(T))$.
\end{proof}

\begin{cor}\label{c2111}
If $\C$ is a category and $F$ an object of $\C^\wedge$, then the following conditions are equivalent:

\nn{\em(a)} $F$ is representable,

\nn{\em(b)} $\C_F$ has a terminal object,

\nn{\em(c)} the identity of $\C_F$ has an inductive limit in $\C_F$.
\end{cor}

\begin{proof}
This follows from Lemma~\ref{l2111} above and Lemma 1.4.10 p.~26 of the book.
\end{proof}
\end{s}

%

\begin{s}
P.~43, end of Section~2.1. One could add the following observation, which will be used in \S\ref{17115b} p.~\pageref{17115b}: 

If $\A$ and $\C$ are categories, and, if, for each $X$ in $\C$, we denote by $\oo j^X$ the forgetful functor from $\C^X$ to $\C$, then we have
%
\begin{equation}\label{17115}
\Hom_{\A^\C}(F,G)\simeq\lim_{X\in\C}\Hom_{\A^{\C^X}}(F\circ\oo j^X,G\circ\oo j^X).
\end{equation}

Let us spell out the proof of the above statement: 

Let $L$ and $R$ be the left and right-hand sides of \eqref{17115}. 

The maps $\lambda\mapsto\lambda*\oo j^X$, where $*$ denotes horizontal composition (see Definition~\ref{dil1} p.~\pageref{dil1}), induce a map $f:L\to R$. Conversely, to an element  
$$
r=\Big(\big(\rho_{u:Y\to X}:F(Y)\to G(Y)\big)_{u:Y\to X}\Big)_X\in R
$$ 
we attach the family
$$
\ell=\big(\rho_{\id_X:X\to X}:F(X)\to G(X)\big)_X.
$$ 
We easily check that the assignment $r\mapsto\ell$ defines a map $g:R\to L$, and that $f$ and $g$ are mutually inverse bijections.
\end{s}

%

\begin{s}\label{fpl}
P.~50, Corollary 2.2.11. We also have:

\emph{A category admits finite projective limits if and only if it admits a terminal object and binary fibered products.}

Indeed, if $f,g:X\parar Y$ is a pair of parallel arrows, and if the square 
$$
\begin{tikzcd}
K\ar{d}\ar{r}&Y\ar{d}{\Delta}\\ 
X\ar{r}[swap]{(f,g)}&Y\times Y
\end{tikzcd}
$$
is cartesian, then $K\simeq\Ker(f,g)$. (As usual, $\Delta$ is the diagonal morphism.)
\end{s}

%

\begin{s}
P.~50, Definition 2.3.1. The three pieces of notation $\pp_*,\pp^\dagger$, and $\pp^\ddagger$ are justified by Notation 17.1.5 p.~407 (see also \eqref{ttau} p.~\pageref{ttau}). 
\end{s} 

% 

\begin{s}\label{phistar}
P.~50, Definition 2.3.1. Let $\pp:J\to I$ be a functor of small categories, let $\C$ be a category, and consider the functor $\pp_*:=\circ\pp:\C^I\to\C^J$. The following fact results from Proposition 2.1.6 p.~38 of the book: 

If $\C$ admits small inductive (resp. projective) limits, then so do $\C^I$ and $\C^J$, and $\pp_*$ commutes with such limits. 
\end{s} 

%

\begin{s} 
P.~51, Definition 2.3.2 (minor variant). We assume that no underlying universe has been given. Let $I\xleftarrow\pp J\xr\beta\C$ be functors, let $\beta$ be in $\C^J$, and let $\pp^\dagger\beta$ be in $\C^I$. The following conditions are equivalent:

\nn(a) $\pp^\dagger\beta$ represents $\Hom_{\C^J}(\beta,\pp_*(\ ))\in(\C^I)^\vee_\U$ for \emph{some} universe $\U$ such that $\C^J$ is a $\U$-category,

\nn(b) $\pp^\dagger\beta$ represents $\Hom_{\C^J}(\beta,\pp_*(\ ))\in(\C^I)^\vee_\U$ for \emph{any} universe $\U$ such that $\C^J$ is a $\U$-category. 

\begin{df}[Definition 2.3.2 p. 51]\label{232} 
If the above equivalent conditions hold, we say that $\pp^\dagger\beta$ {\em exists}. If $\pp^\dagger(F\circ\beta)$ exists and is isomorphic to $F\circ\pp^\dagger(\beta)$ for all functor $F:\C\to\C'$, we say that $\pp^\dagger\beta$ exists {\em universally}.
\end{df}
\end{s}

%%

\subsection{Theorem 2.3.3 (i) (p.~52)}

Recall the statement: 

\begin{thm}[Theorem 2.3.3 (i) p.~52]\label{233i}
Let $I\xleftarrow\pp J\xr\beta\C$ be functors. Assume that 
$$
\colim_{(\pp(j)\to i)\in J_i}\beta(j)
$$ 
exists in $\C$ for all $i$ in $I$. Then $\pp^\dagger(\beta)$ exists and we have 
%
\begin{equation}\label{236}
\pp^\dagger(\beta)(i)\simeq\colim_{(\pp(j)\to i)\in J_i}\beta(j)
\end{equation} 
%
for all $i$ in $I$. In particular, if $\C$ admits small inductive limits and $J$ is small, then 
$\pp^\dagger$ exists. If moreover $\pp$ is fully faithful, then $\pp^\dagger$ is fully faithful and there is an isomorphism $\id_{\C^J}\xr\sim\pp_*\circ\pp^\dagger$. 
\end{thm}
The proof in the book is divided into three Steps, called (a), (b), and (c). 

%

\subsubsection{Step (a)}\label{scji}

We define $\pp^\dagger(\beta)$ by \eqref{236}. The purpose of Step (a) is to show that $\pp^\dagger(\beta)$ is indeed a functor. Here is a variant of the argument of the book. The proof of the following lemma is obvious: 

\begin{lem}\label{r52}

Let $I$ and $J$ be two objects of the category $\Cat$ of small categories (see Definition~\ref{small} p.~\pageref{small}), let $\Phi:I\to\Cat$ be a functor, view $J$ as a constant functor from $I$ to $\Cat$, and let $\theta:\Phi\to J$ be a morphism of functors. Assume 

\begin{equation}\label{52} 
(\colim\theta)(i):=\colim(\theta_i)\in J\quad\forall\ i\in I. 
\end{equation} 
%
For any morphism $s:i\to i'$ in $I$, let $(\colim\theta)(s)$ be the natural morphism 
$$
(\colim\theta)(i)=\colim(\theta_{i'}\circ\Phi(s))\to
\colim\theta_{i'}=(\colim\theta)(i'). 
$$ 
Then $\colim\theta$ is a functor from $I$ to $J$. 
%
\end{lem}
%
Here is a picture:
$$
\begin{tikzcd}
I\ar{rr}{\Phi}&\ar{d}[swap]{\theta}&\Cat&\Phi(i)\ar{d}{\theta_i}\\ 
I\ar{rr}[swap]{J}&{}&\Cat&J.
\end{tikzcd}
$$

We want to prove that $\pp^\dagger$ defined by \eqref{236} p.~\pageref{236} is a functor. In the setting of Lemma~\ref{r52} we define $\Phi:I\to\Cat$ by $\Phi(i):=J_i$ and we consider the morphism of functors $\theta:\Phi\to\C$ such that $\theta_i:\Phi(i)=J_i\to\C$ is the composition of $\beta$ with the forgetful functor from $J_i$ to $J$. We assume \eqref{52}. Then $\colim\theta$ is nothing but $\pp^\dagger(\beta)$. In particular $\pp^\dagger(\beta)$ is a functor by Lemma~\ref{r52}. q.e.d.

%

\subsubsection{Step (b)}

The purpose of Step (b) is to prove 
\begin{equation}\label{stepb}
\Hom_{\C^I}(\pp^\dagger(\beta),\alpha)\simeq\Hom_{\C^J}(\beta,\pp_*(\alpha)) 
\end{equation} 
for all $\alpha:I\to\C$. As pointed out in the book, this can also be achieved by using Lemma 2.1.15 p.~42. Here is a sketch of the argument. We start with a reminder of Lemma 2.1.15. 

To any category $\A$ we attach the category $\Mor_0(\A)$ defined as follows. The objects of $\Mor_0(\A)$ are the triples $(X,f,Y)$ such that $f$ is a morphism in $\C$ from $X$ to $Y$. The morphisms in $\Mor_0(\A)$ from $(X,f,Y)$ to $(X',f',Y')$ are the pairs $(u,v)$ with $u:X\to X'$, $v:Y'\to Y$, and $f=v\circ f'\circ u$:
$$
\begin{tikzcd}
X\ar{d}{u}\ar{r}{f}&Y\\ 
X'\ar{r}[swap]{g}&Y'\ar{u}[swap]{v}.
\end{tikzcd}
$$ 
The composition of morphisms is the obvious one. Lemma 2.1.15 can be stated as follows: 

If $I$ and $\A$ are categories, and $a,b:I\parar\A$ are functors, then 
$$
(i,i\to j,j)\mapsto\Hom_\A(a(i),b(j))
$$ 
is a functor from $\Mor_0(I)^{\op}$ to $\Set$, and there is a natural isomorphism 
%
\begin{equation}\label{2115} 
\Hom_{\A^I}(a,b)\xr\sim\lim_{(i\to j)\in\Mor_0(I)}\Hom_\A(a(i),b(j)).
\end{equation}
%

Returning to \eqref{stepb}, we have functors 
$$
\begin{tikzcd}
J\ar{rr}{\pp}\ar{dr}[swap]{\beta}&&I\ar{dl}{\alpha}\\ 
&\C.
\end{tikzcd}
$$ 
Let us define the categories $M$ and $N$ as follows: an object of $M$ is a pair 
$$
(j,\pp(j)\to i\to i')
$$ 
with $j$ in $J$ and $i,i'$ in $I$. A morphism 
$$
\Big(j_1,\pp(j_1)\to i_1\to i'_1\Big)\to\Big(j_2,\pp(j_2)\to i_2\to i'_2\Big)
$$ 
is given by a triple of morphisms $j_1\to j_2,i_1\to i_2,i'_1\leftarrow i'_2$ such that the obvious diagram commutes. The category $N$ is the category $\Mor_0(J)$ defined in Definition 2.1.14 p.~42 of the book. Consider the functors 
$$
\gamma:M^{\op}\to\Set,\quad\Big(j,\pp(j)\to i\to i'\Big)\mapsto\Hom_\C\big(\beta(j),\alpha(i')\big), 
$$ 
$$
\delta:N^{\op}\to\Set,\quad(j\to j')\mapsto\Hom_\C\big(\beta(j),\alpha(\pp(j'))\big). 
$$ 
As we have 
$$
\Hom_{\C^I}\big(\pp^\dagger(\beta),\alpha\big)\xr\sim\lim\gamma,\quad
\Hom_{\C^J}\big(\beta,\pp_*(\alpha)\big)\xr\sim\lim\delta. 
$$ 
by \eqref{2115}, it suffices to show 
%
\begin{lem}
%
There is a natural bijection $\lim\gamma\simeq\lim\delta$. 
%
\end{lem} 
%
\begin{proof}
To define a map $\lim\gamma\to\lim\delta$, we attach, to a family 
$$
\big(\beta(j)\to\alpha(i')\big)_{\pp(j)\to i\to i'}
$$ 
and to a morphism $j\to j'$, a morphism $\beta(j)\to\alpha(\pp(j'))$ by setting 
$$
i=i'=\pp(j'),\quad(i\to i')=\id_{\pp(j')},
$$ 
and by taking as $\beta(j)\to\alpha(\pp(j'))$ the corresponding member of our family. We leave it to the reader to check that this defines indeed a map $\lim\gamma\to\lim\delta$. To define a map $\lim\delta\to\lim\gamma$, we attach, to a family 
$$
\big(\beta(j)\to\alpha(\pp(j'))\big)_{j\to j'}
$$ 
and to a chain of morphisms $\pp(j)\to i\to i'$, a morphism $\beta(j)\to\alpha(i')$ by setting 
$$
j':=j,\quad(j\to j'):=\id_{j},
$$ 
and by taking as $\beta(j)\to\alpha(i')$ the composition 
$$
\beta(j)\to\alpha(\pp(j))\to\alpha(i)\to\alpha(i'). 
$$ 
We leave it to the reader to check that this defines indeed a map $\lim\delta\to\lim\gamma$, and that this map is inverse to the map constructed above.
\end{proof}
%
\subsubsection{Step (c)}

In part (i) of their proof of Theorem 2.3.3 (i) p.~52 of the book, the authors define a map 
%
\begin{equation}\label{e233i} 
%
\Psi_{\alpha,\beta}:
\Hom_{\C^I}\big(\pp^\dagger(\alpha),\beta\big)\to
\Hom_{\C^J}\big(\alpha,\pp_*(\beta)\big),
%
\end{equation} 
%
and show that it is bijective. In particular, we have a bijection 
$$
f:=\Psi_{\alpha,\pp^\dagger(\alpha)}:
\Hom_{\C^I}\big(\pp^\dagger(\alpha),\pp^\dagger(\alpha)\big)\to
\Hom_{\C^J}\big(\alpha,\pp_*(\pp^\dagger(\alpha))\big),
$$
and we must check that $f(\id_{\pp^\dagger(\alpha)})$ is an isomorphism. To this end, we will define 
$$
u:\pp_*(\pp^\dagger(\alpha))\to\alpha,
$$
and leave it to the reader to verify that $f(\id_{\pp_*(\pp^\dagger(\alpha))})$ and $u$ are mutually inverse isomorphisms. As 
$$ 
\Big(\pp_*\big(\pp^\dagger(\alpha)\big)\Big)(j):=\pp^\dagger(\alpha)(\pp(j)):=\colim_{\pp(j')\to\pp(j)}\alpha(j'),
$$
we must define 
$$
u\Big(\pp(j')\to\pp(j)\Big):\alpha(j')\to\alpha(j),
$$
that is, we must attach, to each morphism $\pp(j')\to\pp(j)$, a morphism $\alpha(j')\to\alpha(j)$. As $\pp$ is fully faithful by assumption, there is an obvious way to do it.

%%

\subsubsection{A Corollary}

Here is a corollary to Theorem \ref{233i} p.~\pageref{233i} (which is Theorem 2.3.3 (i) p.~52 of the book):

\begin{cor}\label{c233i}
If, in the setting of Theorem \ref{233i}, we have $\C=\Set$ (and $I$ and $J$ are small), then $\pp^\dagger(\beta)(i)$ is (in natural bijection with) the quotient of 
$$
\bigsqcup_{j\in J}\ \beta(j)\times\Hom_I(\pp(j),i) 
$$ 
by the smallest equivalence relation $\sim$ satisfying the following condition: If $s:j\to j'$ is a morphism in $J$, if $x$ is in $\beta(j)$, and if $u'$ is in $\Hom_I(\pp(j'),i)$, then 
$$
(x,u'\circ\pp(s))\sim(\beta(s)(x),u'). 
$$
\end{cor}

\begin{proof}
Recall that Theorem~\ref{233i} states the existence of an isomorphism 
$$
\pp^\dagger(\beta)(i)\simeq\colim_{(\pp(j)\to i)\in J_i}\beta(j).
$$
By Proposition 2.4.1 p.~54 of the book, the right-hand side is, in a natural way, the quotient of 
$$
\bigsqcup_{(\pp(j)\to i)\in J_i}\ \beta(j)
$$ 
by a certain equivalence relation. We have 
$$
\bigsqcup_{(\pp(j)\to i)\in J_i}\beta(j)=\bigsqcup_{j\in J}\ \bigsqcup_{u\in\Hom_I(\pp(j),i)}\ \beta(j)\simeq\bigsqcup_{j\in J}\beta(j)\times\Hom_I(\pp(j),i),
$$ 
and it easy to see that the three data of the above bijection, of the equivalence relation in Proposition 2.4.1 of the book, and of the equivalence relation in Corollary~\ref{c233i} above are compatible.
\end{proof}

%%

\subsection{Brief Comments}

\begin{s} P.~53, Corollary 2.3.4. (Another proof will be given in Section~\ref{2111} p.~\pageref{2111}.) Recall that we have functors $\C\xleftarrow\beta J\xrightarrow\pp I$, where $I$ and $J$ are small. One can prove $\colim\beta\simeq\colim\pp^\dagger\beta$, that is 
%
\begin{equation}\label{coco}
\colim_j\beta(j)\simeq\colim_i\ \colim_{j,u}\beta(j),
\end{equation} 
%
where $(j,u)$ runs over $J_i$, with $u:\pp(j)\to i$, as follows: Let $L$ and $R$ be the left and right-hand sides of \eqref{coco}, let $f:R\to L$ be the obvious map, and let
$$ 
\beta(j)\xrightarrow{p_{i,j,u}}\colim_{j,u}\beta(j)\xrightarrow{q_i}\colim_i\ \colim_{j,u}\beta(j)
$$ 
be the coprojections. We easily check that the compositions 
$$
\beta(j)\xrightarrow{p_{\pp(j),j,\id_{\pp(j)}}}\colim_{j,u}\beta(j)\xrightarrow{q_{\pp(j)}}\colim_i\ \colim_{j,u}\beta(j)
$$ 
induce a map $g:L\to R$, and that $f$ and $g$ are mutually inverse bijections. q.e.d.
\end{s} 

% 

\begin{s}\label{spreptom}
P.~54, end of Section 2.3. Let 
\begin{equation}\label{epreptom}
\C\xleftarrow\beta K\xr\psi J\xr\pp I
\end{equation} 
be a diagram of functors. Assume that $I,J$ and $K$ are small, and that $\C$ admits small projective limits. Then Theorem 2.3.3 (ii) p.~52 of the book implies that the functors $\pp^\ddagger(\psi^\ddagger(\beta))$ and $(\pp\circ\psi)^\ddagger(\beta)$ from $I$ to $\C$ exist and are naturally isomorphic. 
\end{s}

%

\subsection{Kan Extensions of Modules}

Let $R$ be a ring, let $\U$ and $\V$ be universes such that $R\in\U\in\V$, put, with self-explanatory notation, 
$$
I:=\Mod^\U(R),\quad\C:=\Mod^\V(R),
$$ 
let $J$ be the full subcategory of $I$ whose single object is $R$, and let $\C\xleftarrow\beta J\xr\pp I$ be the inclusion functors. We identify $\Hom_R(R,M)$ to $M$ whenever convenient. 

We claim that the functor $\pp^\dagger(\beta):I\to\C$ satisfies 
%
\begin{equation}\label{pdb1}
\pp^\dagger(\beta)(M)\simeq M.
\end{equation} 

To prove \eqref{pdb1}, set 
$$
M':=\colim_{(x:R\to M)\in J_M}R\in\C, 
$$ 
and let $p_x:R\to M'$ be the coprojections. As Theorem 2.3.3 (i) p.~52 of the book implies $M'\simeq\pp^\dagger(\beta)(M)$, it suffices to prove $M'\simeq M$. We define a family of linear maps $\Phi_x:R\to M$ by setting $\Phi_x:=x$, and leave it to the reader to check that the $\Phi_x$ induce a linear map $\Phi:M'\to M$. We define the set theoretic map $\Psi:M\to M'$ by putting $\Psi(x):=p_x(1)$, and leave it to the reader to verify that $\Phi$ and $\Psi$ are mutually inverse bijections. This proves \eqref{pdb1}. 

We claim that the functor $\pp^\ddagger(\beta):I\to\C$ satisfies 
%
\begin{equation}\label{pddb1}
\pp^\ddagger(\beta)(M)\simeq M^{**}, 
\end{equation} 
%
where $M^{**}$ is the double dual of $M$. 

To prove \eqref{pddb1}, set 
$$
M':=\lim_{(f:M\to R)\in J^M}R\in\C, 
$$ 
and let $p_f:M'\to R$ be the projections. As Theorem 2.3.3 (ii) p.~52 of the book implies $M'\simeq\pp^\dagger(\beta)(M)$, it suffices to prove $M'\simeq M^{**}$. We define a family of linear maps $\Phi_f:M^{**}\to R$ by setting $\Phi_f(F):=F(f)$, and leave it to the reader to check that the $\Phi_f$ induce a linear map $\Phi:M^{**}\to M'$. We define the linear map $\Psi:M'\to M^{**}$ by putting $\Psi((\lambda_f))(g):=\lambda_g$, and leave it to the reader to verify that $\Phi$ and $\Psi$ are mutually inverse linear bijections. This proves \eqref{pddb1}. 

Let $R$, $\U$ and $\V$ be as above, put, with self-explanatory notation, 
$$
I:=\Mod^\U(R)^{\op},\quad\C:=\Mod^\V(R^{\op}),
$$ 
let $J$ be the full subcategory of $I$ whose single object is $R$, let $\pp:J\to I$ be the inclusion functor, and let $\beta:J\to\C$ be the obvious functor satisfying $\beta(R)=R^{\op}$. 

We claim that the functor $\pp^\dagger(\beta):I\to\C$ satisfies 
%
\begin{equation}\label{pdb2}
\pp^\dagger(\beta)(M)\simeq M^*,
\end{equation} 
%
where $M^*$ is the dual of $M$. 

To prove \eqref{pdb2}, set 
$$
M':=\colim_{(R\to M)\in J_M}R^{\op}=\colim_{(f:M\to R)\in(J^{\op})^M}R^{\op}\in\C, 
$$ 
and let $p_f:R^{\op}\to M'$ be the coprojections. As Theorem 2.3.3 (i) p.~52 of the book implies $M'\simeq\pp^\dagger(\beta)(M)$, it suffices to prove $M'\simeq M$. We define a family of linear maps $\Phi_f:R^{\op}\to M^*$ by setting $\Phi_f(1):=f$, and leave it to the reader to check that the $\Phi_f$ induce a linear map $\Phi:M'\to M$. We define the set theoretic map $\Psi:M^*\to M'$ by putting $\Psi(f):=p_f(1)$, and leave it to the reader to verify that $\Phi$ and $\Psi$ are mutually inverse bijections. This proves \eqref{pdb2}. 

We claim that the functor $\pp^\ddagger(\beta):I\to\C$ satisfies 
%
\begin{equation}\label{pddb2}
\pp^\ddagger(\beta)(M)\simeq M^*, 
\end{equation} 
%
where $M^*$ is the dual of $M$. 

To prove \eqref{pddb2}, set 
$$
M':=\lim_{(M\to R)\in J^M}R^{\op}=\lim_{(x:R\to M)\in(J^{\op})_M}R^{\op}\in\C, 
$$ 
and let $p_x:M'\to R^{\op}$ be the projections. As Theorem 2.3.3 (ii) p.~52 of the book implies $M'\simeq\pp^\dagger(\beta)(M)$, it suffices to prove $M'\simeq M^*$. We define a family of linear maps $\Phi_x:M^*\to R^{\op}$ by setting $\Phi_x(f):=f(x)$, and leave it to the reader to check that the $\Phi_x$ induce a linear map $\Phi:M^*\to M'$. We define the linear map $\Psi:M'\to M^*$ by putting $\Psi((\lambda_x))(y):=\lambda_y$, and leave it to the reader to verify that $\Phi$ and $\Psi$ are mutually inverse linear bijections. This proves \eqref{pddb2}. 

%% 

\subsection{Brief Comments}

\begin{s} 
P.~55, proof of Corollary 2.4.4 (iii) (minor variant).
%
\begin{prop}
If $\Delta:\Set\to\Set^I$ is the diagonal functor, then there is a canonical bijection
$$
\colim\Delta(S)\simeq\pi_0(I)\times S.
$$
\end{prop} 
%
\begin{proof}
On the one hand we have 
$$
\pi_0(I):=\Ob(I)/\!\!\sim\ , 
$$
where $\sim$ is the equivalence relation defined on p.~18 of the book. On the other hand we have by Proposition 2.4.1 p.~54 of the book 
$$
\colim\Delta(S)\simeq(\Ob(I)\times S)/\!\!\approx\ ,
$$
where $\approx$ is the equivalence relation described in the proposition. In view of the definition of $\approx$ and $\sim$, we get 
$$
(i,s)\approx(j,t)\ \ssi\ [i\sim j\text{ and }s=t].
$$ 
\end{proof}
\end{s}

%%

\subsection{Corollary 2.4.6 (p. 56)}

We shall give two other proofs. Recall the statement: 

\begin{prop}\label{246}
In the setting 
%
\begin{equation}\label{241s}
A\in\C'\xleftarrow{F}\C\xrightarrow{G}\C''\ni B, 
\end{equation} 
%
we have 
%
\begin{equation}\label{241} 
\colim_{(X,b)\in\C_B}\Hom_{\C'}(A,F(X))\simeq\colim_{(X,a)\in\C^A}\Hom_{\C''}(G(X),B). 
\end{equation} 
%
Here 
$$
a\in\Hom_{\C'}(A,F(X)),\quad b\in\Hom_{\C''}(G(X),B). 
$$ 
\end{prop}

\begin{proof}[First Proof]
Denote respectively by $L$ and $R$ the left and right-hand side of (\ref{241}), let 
$$
p_{X,b}:\Hom_{\C'}(A,F(X))\to L,\quad q_{X,a}:\Hom_{\C''}(G(X),B)\to R
$$
be the coprojections, and define 
$$
f_{X,b}:\Hom_{\C'}(A,F(X))\to R,\quad g_{X,a}:\Hom_{\C''}(G(X),B)\to L
$$
by
$$
f_{X,b}(a):=q_{X,a}(b),\quad g_{X,a}(b):=p_{X,b}(a).
$$
One easily checks that these maps define mutually inverse bijections between $L$ and $R$. 
\end{proof}

\begin{proof}[Second Proof]
Put 
$$
Q_1:=\colim_{(G(X)\to B)\in\C_B}\Hom_{\C'}(A,F(X)),\quad Q_2:=\colim_{(A\to F(X))\in\C^A}\Hom_{\C''}(G(X),B).
$$ 
We must prove that there is a natural bijection $Q_1\simeq Q_2$. Proposition 2.4.1 p.~54 of the book furnishes sets $S_1$ and $S_2$, and equivalence relations $R_1$ and $R_2$ on $S_1$ and $S_2$ respectively, such that $Q_i\simeq S_i/R_i$ for $i=1,2$. We shall prove that there is a natural bijection $S_1\simeq S_2$, and leave it to the reader to check that it induces a bijection $Q_1\simeq Q_2$. We have 
$$
S_1:=\bigsqcup_{(G(X)\to B)\in\C_B}\Hom_{\C'}(A,F(X))
$$

$$
\simeq\bigsqcup_{X\in\C}\quad\bigsqcup_{(G(X)\to B)\in\C_B}\Hom_{\C'}(A,F(X))
$$

$$
\simeq\bigsqcup_{X\in\C}\Hom_{\C'}(A,F(X))\times\Hom_{\C''}(G(X),B)
$$

$$
\simeq\bigsqcup_{X\in\C}\quad\bigsqcup_{(A\to F(X))\in\C^A}\Hom_{\C''}(G(X),B)
$$

$$
\bigsqcup_{(A\to F(X))\in\C^A}\Hom_{\C''}(G(X),B)=:S_2.
$$
\end{proof}

%%

\subsection{Brief Comments}

\begin{s} 
P.~56, proof of Lemma 2.4.7 (minor variant).

\begin{lem} 
If $I$ is a small category, $i_0$ is in $I$, and $k(i_0)\in\Set^I$ is the Yoneda functor $\Hom_I(i_0,\ )$, then $\colim k(i_0)$ is a terminal object of $\Set$. 
\end{lem}

\begin{proof}
For $X$ in $\Set$ we have (with self-explanatory notation)
$$
\Hom_{\Set}\left(\colim k(i_0),X\right)\simeq\Hom_{{\Set}^I}(k(i_0),\Delta(X))\simeq X,
$$
the first bijection being a particular case of \eqref{cue} p.~\pageref{cue}, and the second one following from the Yoneda Lemma (Theorem~\ref{yol} p.~\pageref{yol}).
\end{proof}
\end{s}

%

\begin{s} 
P.~58, implication (vi)$\then$(i) of Proposition 2.5.2. Here is a slightly stronger statement:

\begin{prop} 
If $\pp:J\to I$ is a functor, then the obvious map
%
\begin{equation}\label{om}
\colim\Hom_I(i,\pp)\to\pi_0(J^i)
\end{equation}
%
is bijective. 
\end{prop}

\begin{proof} 
Let $L_i$ be the left-hand side of \eqref{om}, and, for $j$ in $J$, let 
$$
p_j:\Hom_I(i,\pp(j))\to L_i
$$
be the coprojection. It is easy to check that the map 
$$
\Ob(J^i)\to L_i,\quad\Big(j,s:i\to\pp(j)\Big)\mapsto p_j(s)
$$
factors through $\pi_0(J^i)$, and that the induced map $\pi_0(J^i)\to L_i$ is inverse to (\ref{om}).
\end{proof}
\end{s}

%

\subsection{Proposition 2.6.3 (i) (p.~61)}

Let $\C$ be a category and let $A$ be in $\C^\wedge$. Consider the statements
%
\begin{equation}\label{263a}
\ic_{(X\to A)\in\C_A}X\xrightarrow\sim A,
\end{equation} 

\begin{equation}\label{263b}
\colim_{(X\to A)\in\C_A}\Hom_\C(Y,X)\xrightarrow\sim A(Y)\text{ for all }Y\in\C, 
\end{equation}

\begin{equation}\label{263c}
\Hom_{\C^\wedge}(A,B)\xrightarrow\sim\lim_{(X\to A)\in\C_A}B(X)\text{ for all }B\in\C^\wedge. 
\end{equation}

Clearly, \eqref{263b} implies \eqref{263a} and \eqref{263c}, and the proof of \eqref{263b} is straightforward. (See \S\ref{c38} p.~\pageref{c38} for the relationship between \eqref{263a}, \eqref{263b}, and \eqref{263c}.) 

Isomorphism \eqref{263a} can be decoded as follows: Consider the functor 
$$
\alpha:\C_A\to\C^\wedge,\quad(X\to A)\mapsto X,
$$ 
and let $p:\alpha\to\Delta(A)$ be the tautological morphism in $(\C^\wedge)^{\C_A}$ defined by 
%
\begin{equation}\label{pxa}
p_{X\to A}:=(X\to A)
\end{equation}
% 
for all $X\to A$ in $\C_A$. The claim is that the map 
$$
\Hom_{\C^\wedge}(A,B)\to\Hom_{\oo{Fct}(\C_A,\C^\wedge)}(\alpha,\Delta(B)),\quad\theta\mapsto\Delta(\theta)\circ p
$$ 
is bijective. 

Isomorphism \eqref{263b} can be decoded as follows: Consider the functor 
$$
\beta:\C_A\to\Set,\quad(X\to A)\mapsto\Hom_\C(Y,X), 
$$ 
and let $q:\beta\to\Delta(A(Y))$ be the morphism in $(\Set)^{\C_A}$ defined by 
$$
q_{X\to A}(Y\to X):=(Y\to X\to A)
$$ 
for all $X\to A$ in $\C_A$. The claim is that the map 
$$
\Hom_{\Set}(A(Y),S)\to\Hom_{\oo{Fct}(\C_A,\Set)}(\beta,\Delta(S)),\quad f\mapsto\Delta(f)\circ q
$$ 
is bijective. 

Isomorphism \eqref{263c} can be decoded as follows: Consider the functor 
$$
\gamma:(\C_A)^{\op}\to\Set,\quad(X\to A):=B(X),
$$ 
and let $r:\Delta(A(Y))\to\gamma$ be the morphism in $(\Set)^{\C_A}$ defined by 
$$
r_{X\to A}(A\to B):=(X\to A\to B)
$$ 
for all $X\to A$ in $\C_A$. The claim is that the map 
$$
\Hom_{\Set}(S,B(X))\to\Hom_{\oo{Fct}((\C_A)^{\op},\Set)}(\gamma,\Delta(S)),\quad g\mapsto r\circ\Delta(g)
$$ 
is bijective. 

%%

\subsection{Brief Commments}

\begin{s} 
P.~61, Proposition 2.6.3 (ii). Here is a Lemma implicitly used in the proof of Proposition 2.6.3 (ii): 

\begin{lem} 
Let $\alpha:I\to\A$ be a functor, let $A$ in $\A$ be the inductive limit of $\alpha$, let $\alpha':I\to\A_A$ be the obvious functor, and let $a$ in $\A_A$ be the identity of $A$. Then the inductive limit of $\alpha'$ is $a$. 
\end{lem} 

\begin{proof}
If $I$ and $\C$ are categories, we write $\Delta$ for the diagonal functor from $\C$ to $\C^I$. 

Let $b=(b:B\to A)$ be an object of $\A_A$. We must check that there is a canonical bijection
%
\begin{equation}\label{1}
\Hom_{\A_A}(a,b)\simeq\Hom_{(\A_A)^I}(\alpha',\Delta(b)).
\end{equation}
%
We have a canonical bijection
%
\begin{equation}\label{2}
\Hom_\A(A,B)\simeq\Hom_{\A^I}(\alpha,\Delta(B))
\end{equation}
%
and inclusions 
$$
\Hom_{\A_A}(a,b)\subset\Hom_\A(A,B),\quad
%
\Hom_{(\A_A)^I}(\alpha',\Delta(b))\subset\Hom_{\A^I}(\alpha,\Delta(B)).
$$
It is straightforward to check that (\ref{2}) induces (\ref{1}).
\end{proof}
\end{s}

%

\begin{s} 
P.~61, Proposition 2.6.4 (minor variant). Usually, we assume implicitly that a universe is given. Here we make an exception to this rule. 

Let $\alpha:I\to\C$ be a functor, let $X$ be an object of $\C$, and let $p:\alpha\to\Delta(X)$ be a morphism in $\C^I$ (where $\Delta(X)$ is constant with value $X$). For all universe $\U$ such that $\C$ is a $\U$-category, we write $h_\U$ for the Yoneda embedding from $\C$ to $\C^\wedge_\U$.

\begin{df}[universal inductive limit]\label{uil}
If the natural map $\colim\Hom_\C(Y,\alpha)\to\Hom_\C(Y,X)$ is bijective for all $Y$ in $\C$, we say that $X$ is a {\em universal} inductive limit\index{universal inductive limit} of $\alpha$, and that $\colim\alpha$ exists {\em universally}\index{universal existence of a limit} in $\C$.
\end{df}

\begin{prop}[Proposition 2.6.4 p. 61]\label{puil}
The following conditions are equivalent (recall that no universe is given {\em a priori}):

\nn{\em(a)} $X$ is a universal inductive limit of $\alpha$,

\nn{\em(b)} $\colim h_\U\circ\alpha\simeq h_\U(X)$ for \emph{some} universe $\U$ such that $I$ is $\U$-small and $\C$ is a $\U$-category,

\nn{\em(c)} $\colim h_\U\circ\alpha\simeq h_\U(X)$ for \emph{any} universe $\U$ such that $I$ is $\U$-small and $\C$ is a $\U$-category, 

\nn{\em(d)} $\colim F\circ\alpha\simeq F(X)$ for \emph{any} functor $F:\C\to\C'$. 
\end{prop}

\begin{proof} The implications (a)$\ssi$(b)$\ssi$(c)$\si$(d) are clear. To prove (c)$\then$(d), we choose a universe $\U$ such that $I$ is $\U$-small, and $\C$ and $\C'$ are $\U$-categories, and we note
$$
\Hom_{\C'}(F(X),X')\simeq\Hom_{\C^\wedge_\U}\big(h_\U(X),\Hom_{\C'}(F(\ ),X')\big)
$$
$$
\simeq\lim\Hom_{\C^\wedge_\U}\big(h_\U\circ\alpha,\Hom_{\C'}(F(\ ),X')\big)\simeq\lim\Hom_{\C'}(F\circ\alpha,X').
$$
\end{proof}
\end{s}

%

\begin{s} 
Proposition~\ref{puil} yields the following minor variant of Theorem 2.3.3 p.~52 (i) of the book. Let $I\xleftarrow\pp J\xr\beta\C$ be functors and let $\beta$ be in $\C^J$. 

\begin{thm}[Theorem 2.3.3 (i) p. 52]\label{233}
If 
%
\begin{equation}\label{e233}
\colim_{(\pp(j)\to i))\in J_i}\beta(j)
\end{equation}
%
exists in $\C$ for all $i$ in $I$, then $\pp^\dagger\beta$ exists, and $(\pp^\dagger\beta)(i)$ is isomorphic to \eqref{e233}. If, in addition, \eqref{e233} exists universally in the sense of Definition~\ref{uil} p.~\pageref{uil}, then $\pp^\dagger\beta$ exists universally in the sense of Definition~\ref{232} p.~\pageref{232}. 
\end{thm}
\end{s}

%

\begin{s}\label{c271b}
P.~62, Proposition 2.7.1. Consider the diagram 
$$
\begin{tikzcd}
\C\ar{r}{\hy_\C}\ar{dr}[swap]{F}&\C^\wedge\ar{d}{\widetilde F}&I\ar{l}[swap]{\alpha}\\
&\A,
\end{tikzcd}
$$
where $I$ is a small category and $\widetilde F$ is defined by 
$$
\widetilde F(A):=\colim_{(X\to A)\in\C_A}F(X). 
$$
Let us rewrite the proof of the fact that the natural morphism $\colim\widetilde F(\alpha)\to\widetilde F\left(\colim\alpha\right)$ is an isomorphism. 

By Proposition 2.1.10 p. 40 of the book (stated on p.~\pageref{2.1.10} below as Proposition~\ref{2.1.10}), it suffices to check that the functor $G:\A\to\C^\wedge$ defined by 
$$
G(X')(X):=\Hom_{\A}(F(X),X').
$$ 
is right adjoint to $\widetilde F$. This results from the following computation: 
$$
\Hom_{\A}\left(\widetilde F(A),X'\right)=
\Hom_{\A}\left(\colim_{(X\to A)\in\C_A}F(X),X'\right)\simeq 
\lim_{(X\to A)\in\C_A}\Hom_{\A}(F(X),X')
$$
$$
=\lim_{(X\to A)\in\C_A}G(X')(X)\simeq\Hom_{\C^\wedge}(A,G(X')), 
$$ 
the last isomorphism following from \eqref{263c} p.~\pageref{263c}. q.e.d.
\end{s}

%

\begin{s}\label{bil}
P.~62. In the setting of Proposition 2.7.1, the functors
$$
\A^\C\to\A,\quad F\mapsto(\oo h_\C^\dagger F)(A)\quad\text{and}\quad
\C^\wedge\to\A,\quad A\mapsto(\oo h_\C^\dagger F)(A)
$$ 
commute with small inductive limits. 

Indeed, for the first functor the conclusion follows from the isomorphism 
$$
(\oo h_\C^\dagger F)(A)\simeq\colim_{(U\to A)\in\C_A}F(U),
$$ 
and, for the second functor it follows from Proposition 2.7.1 p.~62 of the book.
\end{s} 

%%

\subsection{Three Formulas}

Here is a complement to Section 2.3 pp~52-54 of the book, complement which will be used in \S\ref{prepa5} p.~\pageref{prepa5} to prove Proposition 17.1.9 p.~409 of the book. Let $\C$ be a small category and $A$ an object of $\C^\wedge$. In this section we shall use the following notation: The Yoneda embedding $\C\to\C^\wedge$ will be denoted by $h(\C)$, and the forgetful functor $\C_A\to\C$ by $j(\C_A)$: 
$$
h(\C):\C\to\C^\wedge,\qquad j(\C_A):\C_A\to\C.
$$
Let $\A$ be a category admitting small inductive and projective limits. 

\subsubsection{First Formula}

Let $F:\C\to\A$ be a functor. Recall that there is a unique functor 
$$
\lambda:(\C^\wedge)_A\to(\C_A)^\wedge
$$ 
such that 
%
\begin{equation}\label{lambda}
\lambda(B\to A)(U\to A)=\Hom_{(\C^\wedge)_A}(U\to A,B\to A)
\end{equation} 
%
for all $(B\to A)$ in $(\C^\wedge)_A$ and all $(U\to A)$ in $\C_A$. (See Lemma 1.4.12 p.~26 of the book.) We claim 
%
\begin{equation}\label{prepa1}
h(\C_A)^\ddagger(F\circ j(\C_A))\circ\lambda\simeq h(\C)^\ddagger(F)\circ j((\C^\wedge)_A)
\end{equation}
%
(see the diagram \eqref{qcd} below).

\begin{proof} 
Consider the diagram 
%
\begin{equation}\label{qcd}
\begin{tikzcd}
\C_A\ar[bend left]{rrrr}{j(\C_A)}\ar{rr}{F\circ j(\C_A)}\ar{dd}[swap]{h(\C_A)}&&\A&&\C\ar{ll}[swap]{F}\ar{dd}{h(\C)}\\ 
{}\\ 
(\C_A)^\wedge\ar{uurr}[swap]{h(\C_A)^\ddagger(F\circ j(\C_A))}&&(\C^\wedge)_A\ar{ll}{\lambda}\ar{rr}[swap]{j((\C^\wedge)_A)}&&\C^\wedge.\ar{uull}[swap]{h(\C)^\ddagger(F)}
\end{tikzcd}
\end{equation}
%
We have 
%
\begin{align*} 
%
h(\C_A)^\ddagger(F\circ j(\C_A))(\lambda(B\to A))&\simeq\lim_{((B\to A)\to(U\to A))\in((\C^\wedge)_A)^{B\to A}}F(U)\\ \\ 
%
&\simeq\lim_{(B\to U)\in\C^B}F(U)\\ \\ 
%
&\simeq h(\C)^\ddagger(F)(B)\\ \\ 
%
&\simeq h(\C)^\ddagger(F)\Big(j\big((\C^\wedge)_A\big)(B\to A)\Big). 
% 
\end{align*} 
\end{proof}

%

\subsubsection{Second Formula}

\begin{prop}\label{prepa2a}
Consider the quasi-commutative diagram 
$$
\begin{tikzcd}
\C_A\ar{rr}{j(\C_A)}\ar{dr}[swap]{G}&&\C\ar{dl}{j(\C_A)^\ddagger(G)}\\ 
{}&\A,
\end{tikzcd}
$$ 
and let $U$ be in $\C$. Then we have 
\begin{equation}\label{prepa2}
j(\C_A)^\ddagger(G)(U)\simeq\prod_{U\to A}G(U\to A).
\end{equation} 
\end{prop}

\begin{lem}\label{cicau}
The discrete category $A(U)$ is cocofinal in $(\C_A)^U$.
\end{lem}

\begin{proof}[Proof of Lemma \ref{cicau}]
We probably give too many details, and the reader may want to skip this proof. An object $a$ of $(\C_A)^U$ is given by a triple 
$$
a=(U_a,U_a\xr{u_a}A;U\xr{s_a}U_a),
$$ 
and a morphism from $a$ to 
$$
b=(U_b,U_b\xr{u_b}A;U\xr{s_b}U_b)\in(\C_A)^U
$$ 
is given by a commutative diagram 
$$
\begin{tikzcd}
{}&U\ar{dl}[swap]{s_a}\ar{dr}{s_b}\\ 
U_a\ar{dr}[swap]{u_a}\ar{rr}{c}&&U_b\ar{dl}{u_b}\\ 
{}&A.
\end{tikzcd}
$$ 
The embedding $\pp:A(U)\to(\C_A)^U$ implicit in the statement of Lemma~\ref{cicau} is given by 
$$
\pp(u)=(U,U\xr uA;U\xr{\id_U}U). 
$$ 
It is easy to see that, for any $b$ in $(\C_A)^U$, there is precisely one pair $(u,c)$ such that $u$ is in $A(U)$ and $c$ is a morphism from $U$ to $U_b$ making the diagram 
$$
\begin{tikzcd}
{}&U\ar{dl}[swap]{\id_U}\ar{dr}{s_b}\\ 
U\ar{dr}[swap]{u}\ar{rr}{c}&&U_b\ar{dl}{u_b}\\ 
{}&A
\end{tikzcd}
$$ 
commute. This implies the lemma. 
\end{proof}

\begin{proof}[Proof of Proposition \ref{prepa2a}]
We have 
$$
j(\C_A)^\ddagger(G)(U)\simeq\lim_{(U\to j(\C_A)(V\to A))\in(\C_A)^U}G(V\to A)
$$
$$
\simeq\lim_{U\to A}G(U\to A)\simeq\prod_{U\to A}G(U\to A),
$$
the penultimate isomorphism following from Lemma~\ref{cicau}. 
\end{proof} 

% 

\subsubsection{Third Formula}

Put $j:=j(\C_A),\ h:=h(\C),\ h_A:=h(\C_A)$, and consider the quasi-commutative diagram 
$$
\begin{tikzcd}
\C^\wedge\ar{d}[swap]{h^\dagger(j^\dagger(G))}&
\C\ar{l}[swap]{h}\ar{d}[swap]{j^\dagger(G)}&
\C_A\ar{l}[swap]{j}\ar{r}{h_A}\ar{d}{G}&
(\C_A)^\wedge\ar{d}{(h_A)^\dagger(G)}&
(\C^\wedge)_A\ar{l}[swap]{\lambda}\\ 
\A\ar[equal]{r}&\A\ar[equal]{r}&\A\ar[equal]{r}&\A.
\end{tikzcd}
$$ 
(See \eqref{lambda} p.~\pageref{lambda} for the definition of $\lambda$.) Let $B$ be in $\C^\wedge$. We claim 
\begin{equation}\label{prepa3}
h^\dagger(j^\dagger(G))(B)\simeq(h_A)^\dagger(G)(\lambda(B\times A\to A)).
\end{equation} 

\begin{proof}
We have, for $U$ in $\C$, 
$$
j^\dagger(G)(U)\simeq\colim_{(j(V\to A)\to U)\in(\C_A)_U}G(V\to A)\simeq\colim_{(A\leftarrow V\to U)\in(\C_A)_U}G(V\to A)
$$
$$
\simeq\colim_{((V\to A)\to(U\times A\to A))\in(\C_A)_{U\times A\to A}}G(V\to A)\simeq(h_A)^\dagger(G)(\lambda(U\times A\to A)),
$$
that is: 
%
\begin{equation}\label{prepa4}
j^\dagger(G)(U)\simeq(h_A)^\dagger(G)(\lambda(U\times A\to A)).
\end{equation} 
%
For $B$ in $\C^\wedge$ we get 
$$
h^\dagger(j^\dagger(G))(B)
$$
$$
\simeq\colim_{(U\to B)\in\C_B}j^\dagger(G)(U)
$$ 
%
\begin{equation}\label{prepa4a}
\simeq\colim_{(U\to B)\in\C_B}(h_A)^\dagger(G)(\lambda(U\times A\to A))
\end{equation} 
\begin{equation}\label{prepa4b}
\simeq(h_A)^\dagger(G)\left(\lambda\left(\colim_{(U\to B)\in\C_B}(U\times A\to A)\right)\right)
\end{equation} 
%
\begin{equation}\label{prepa4b'}
\simeq(h_A)^\dagger(G)\left(\lambda\left(\left(\icolim_{(U\to B)\in\C_B}(U\times A)\right)\to A\right)\right)
\end{equation}
\begin{equation}\label{prepa4c}
\simeq(h_A)^\dagger(G)\left(\lambda\left(\left(\icolim_{(U\to B)\in\C_B}U\right)\times A\to A\right)\right)
\end{equation} 
%
$$
\simeq(h_A)^\dagger(G)(\lambda(B\times A\to A)),
$$ 
where 
\eqref{prepa4a} follows from \eqref{prepa4}, 
\eqref{prepa4b} follows from the fact that $(h_A)^\dagger(G)$ commutes with small inductive limits by Proposition~2.7.1 p.~62 of the book (see also \S\ref{c271b} p.~\pageref{c271b}),  
\eqref{prepa4b'} follows from Lemma 2.1.13 (i) p.~41 of the book, and 
\eqref{prepa4c} follows from the fact that small inductive limits in $\Set$ are stable by base change (Definition 2.2.6 p.~47 of the book, stated below as Definition~\ref{dsbc} p.~\pageref{dsbc}). 
\end{proof}

%% 

\subsection{Notation 2.7.2 (p. 63)}

Recall that $F:\C\to\C'$ is a functor of small categories. The formula 
$$
\widehat F(A)(X')=\colim_{(X\to A)\in\C_A}\Hom_{\C'}(X',F(X))
$$
may also be written as 
$$
\widehat F(A)=\icolim_{(X\to A)\in\C_A}F(X).
$$
It might be worth stating explicitly the isomorphism 
$$
\widehat F\circ\hy_\C\xr\sim\hy_{\C'}\circ F, 
$$ 
that is, the diagram 
$$
\begin{tikzcd}
\C\ar{r}\ar{d}[swap]{\hy_\C}\ar{r}{F}&\C'\ar{d}{\hy_{\C'}}\\ 
\C^\wedge\ar{r}[swap]{\widehat F}&\C'^\wedge.
\end{tikzcd}
$$ 
quasi-commutes. 

\begin{prop}\label{yf} 
Let $\U\in\V$ be universes, let $\A$ be the category of $\U$-small categories (see Definition~\ref{small} p.~\pageref{small}), let $\B$ be the category whose objects are the $\V$-small $\U$-categories (see Definition~\ref{ducat} p.~\pageref{ducat}) and whose morphisms are the isomorphism classes of functors. Then there is a functor $\Phi:\A\to\B$ satisfying $\Phi(\C)=\C^\wedge_\U$ and $\Phi(F)=\widehat F$, and there is a morphism of functors $\theta$ from the ``inclusion'' $\A\to\B$ to $\Phi$ satisfying $\theta_\C=\hy_\C$ for all $\C$ in $\A$. 
\end{prop}

\begin{rk}\label{cof}
Let $A'$ be in $\cc C'^\wedge$, and let $\cc C_{A'\circ F}\xr\pp\cc C'_{A'}\xr\psi\C'^\wedge$ be the natural functors. By \eqref{263a} p.~\pageref{263a} the natural morphism $\colim(\psi\circ\pp)\to\colim\psi$ induces a morphism $f:\widehat F(A'\circ F)\to A'$ functorial in $A'$. Moreover, if $\pp$ is cofinal, then $f$ is an isomorphism. (This remark will be used to prove Proposition~\ref{myprop1} p.~\pageref{myprop1}.)
\end{rk}

The proof is obvious.

\begin{rk}
If $F$ is fully faithful, then there is an isomorphism $\widehat F(A)\circ F\xr\sim A$ functorial in $A\in\C^\wedge$.
\end{rk} 

\begin{proof}
We have 
$$
\widehat F(A)(F(X))=\colim_{(Y\to A)\in\C_A}\Hom_{\C'}(F(X),F(Y))
$$
$$
\simeq\colim_{(Y\to A)\in\C_A}\Hom_\C(X,Y)\xr\sim A(X),
$$
the last isomorphism following from \eqref{263b} p.~\pageref{263b}.
\end{proof}

As observed in the book (see also \S\ref{c271b} p.~\pageref{c271b}):

\begin{rk}\label{272}
The functor $\widehat F$ commutes with small inductive limits.
\end{rk} 

Let $X$ be in $\C$ and $A$ be a terminal object of $\C^\wedge$. We have 
$$
\widehat F(A)(F(X))\simeq\bigsqcup_{Y\in\C}\Hom_{\C'}(F(X),F(Y)).
$$ 
Let us identify these two sets. 

\begin{rk}\label{272b}
If $A$ is a terminal object of $\C^\wedge$, then there is a unique functor $G:\C\to\C'_{\widehat F(A)}$ such that we have, in the above notation, 
$$
G(X):=\id_{F(X)}\in\Hom_{\C'}(F(X),F(X)).
$$ 
Moreover, the composition of $G$ with the forgetful functor $\C'_{\widehat F(A)}\to\C'$ is $F$.
\end{rk} 

The proof is obvious. 

%%

\subsection{Brief Comments} 

\begin{s}\label{opddagg} 
P.~63, Corollary 2.7.4. Here is a variant: 

Let $\C$ be a category and $\A$ a category admitting small projective limits, let $h:\C\to\C^\wedge$ the Yoneda embedding, and let $\oo{Fct}^{p\ell}((\C^\wedge)^{\op},\A)$ be the category of functors from $(\C^\wedge)^{\op}$ to $\A$ commuting with small projective limits. Then the functors 
$$
\begin{tikzcd}
\oo{Fct}^{pl}((\C^\wedge)^{\op},\A)\ar[yshift=0.7ex]{r}{(h^{\op})_*}&\oo{Fct}(\C^{\op},\A)\ar[yshift=-0.7ex]{l}{(h^{\op})^\ddagger}
\end{tikzcd}
$$
are mutually quasi-inverse equivalences. 

Let $F$ be in $\oo{Fct}((\C^\wedge)^{\op},\A)$. Assume $(A_i)$ is a projective system in $(\C^\wedge)^{\op}$, or, equivalently, $(A_i)$ is an inductive system in $\C^\wedge$. In particular $(F(A_i))$ is a projective system in $\A$. 

Then $F$ is in $\oo{Fct}^{p\ell}((\C^\wedge)^{\op},\A)$ if and only if the following condition holds: 

For any system $(A_i)$ as above, the natural morphism 
$$
F\left(\col_iA_i\right)\to\lim_iF(A_i)
$$ 
is an isomorphism. 

The functor $(h^{\op})^\ddagger$ is given by 
$$ 
(h^{\op})^\ddagger(F)(A)=\lim_{(U\to A)\in\C_A}F(U). 
$$ 

The functors 
$$
\A^{\C^{\op}}\to\A,\quad F\mapsto(h^{\op})^\ddagger(F)(A)\quad\text{and}\quad
\C^\wedge\to\A,\quad A\mapsto(h^{\op})^\ddagger(F)(A)
$$ 
commute with small projective limits. 

For a justification, see \S~\ref{bil} p.~\pageref{bil}.
\end{s}

%%

\begin{s} 
P.~64. It might be worth displaying the formula 
%
\begin{equation}\label{275}
\widehat F(A)(X')\simeq\colim_{(X\to A)\in\C_A}\Hom_{\C'}(X',F(X))\simeq
\colim_{(X'\to F(X))\in\C^{X'}}A(X),
\end{equation} 
%
which is contained in the proof of Proposition 2.7.5 p.~64 of the book, and which follows from Corollary 2.4.6 p.~56 of the book (see Proposition~\ref{246} p.~\pageref{246}). Recall that $F:\C\to\C'$ is a functor of small categories, that $A$ is in $\C^\wedge$, and that $X'$ is in $\C'$. 

For the reader's convenience we reproduce the statement of Proposition 2.7.5: 

\begin{prop}
If $F:\C\to\C'$ is a functor of small categories, then the functors $\widehat F$ and $(F^{\op})^\dagger$ from $\C^\wedge$ to $\C'^\wedge$ are isomorphic. 
\end{prop} 

This follows from \eqref{275}.
\end{s}

%

\begin{s}
P.~64, end of Chapter 2. One could add the following observation: If $\C$ is a small category, if $A$ is in $\C^\wedge$, if $B$ is a terminal object of $(\C_A)^\wedge$, and if $F:\C_A\to\C$ is the forgetful functor, then we have 
$$
\widehat F(B)\simeq A.
$$ 
Indeed, we have 
$$
\widehat F(B)(X)\simeq\colim_{((Y\to A)\to B)\in(\C_A)_B}\Hom_\C(X,F(Y\to A))
$$
$$
\simeq\colim_{(Y\to A)\in\C_A}\Hom_\C(X,Y)\simeq A(X),
$$ 
the last isomorphism following from \eqref{263b} p.~\pageref{263b}. 
\end{s}

%%

\section{About Chapter 3}

\subsection{Brief Comments}

\begin{s} 
P.~72, proof of Lemma 3.1.2. Here is a minor variant of the proof of the following statement: 

If $\pp:J\to I$ is a functor with $I$ filtrant and $J$ finite, then $\lim\Hom_I(\pp,i)\neq\varnothing$ for some $i$ in $I$. 

Indeed, let $S$ be a set of morphisms in $J$. It is easy to prove 
$$
(\exists\ i\in I)\left(\exists\ a\in\prod_{j\in J}\Hom_I(\pp(j),i)\right)\ (\forall\ (s:j\to j')\in S)\ (a_{j'}\circ\pp(s)=a_j) 
$$ 
by induction on the cardinal of $S$, and to see that this implies the claim. q.e.d.
\end{s}

%

\begin{s}% old version:  
%https://docs.google.com/document/d/1SnpQds-AJDOA5hF38N8pMusBqYRe7atCJZkEHpBIgL0/edit

P.~74, Theorem 3.1.6, Part (i) of the proof of implication (a)$\then$(b) (minor variant). 

To prove that filtrant small inductive limits in $\Set$ commute with finite projective limits, one can argue as follows. 

Let $I$ be a filtrant small category, let $J$ be a finite category, let $\alpha:I\times J^{\op}\to\Set$ be a functor, and put 
$$
X_{ij}:=\alpha(i,j),\quad Y_i:=\lim_jX_{ij},\quad Y:=\colim_iY_i,
$$
$$
Z_j:=\colim_iX_{ij},\quad Z:=\lim_jZ_j.
$$ 
We claim that the natural map $f:Y\to Z$ is bijective. 

To prove this, we define $g:Z\to Y$ and leave it to the reader to check that $f$ and $g$ are mutually inverse bijections. 

We define $g:Z\to Y$ as follows: Let 
$$
\eta_i:Y_i\to Y,\quad\zeta_{ij}:X_{ij}\to Z_j
$$ 
be the coprojections. It is easy to see that, for each $z=(z_j)$ in $Z$, there is a pair 
$$
(i,(x_{ij})_j)\in\Ob(I)\times\prod_jX_{ij}
$$ 
such that 
$$
z_j=\eta_{ij}(x_{ij})\ \forall\ j,\quad(x_{ij})_j\in Y_i. 
$$ 
It is also easy to see  that the element $\eta_i((x_{ij})_j)$ in $Y$ does not depend on the choice of the above pair $(i,(x_{ij})_j)$, so that we can set $g(z):=\eta_i((x_{ij})_j)$. 
\end{s}

%

\begin{s} 
P.~74, Theorem 3.1.6. Here is an immediate corollary: 
%
\begin{cor}\label{316}
Let $I$ be a (not necessarily small) filtrant $\U$-category, $J$ a finite category, and $\alpha:I\times J^{\op}\to\Set$ a functor such that $\colim_i\alpha(i,j)$ exists in $\Set$ for all $j$. Then $\colim_i\lim_j\alpha(i,j)$ exists in $\Set$, and the natural map 
$$
\colim_i\lim_j\alpha(i,j)\to
\lim_j\colim_i\alpha(i,j)
$$ 
is bijective. 
\end{cor}
%
This corollary is implicitly used in the proof of Proposition 3.3.13 p.~84.
\end{s}

%

\begin{s} 
P.~75, Proposition 3.1.8 (i). In the proof of Proposition 3.3.15 p.~85 of the book, a slightly stronger result is needed (see \S\ref{3315} p.~\pageref{3315}). We state and prove this stronger result. 
%
\begin{prop}\label{318i} 
%
Let 
$$
\begin{tikzcd}
J\ar{r}{\pp}&I\ar{r}{\theta}&L&K\ar{l}[swap]{\psi}
\end{tikzcd}
$$
be a diagram of categories. Assume that $\psi$ is cofinal, and that the obvious functor $\pp_k:J_{\psi(k)}\to I_{\psi(k)}$ is cofinal for all $k$ in $K$. Then $\pp$ is cofinal. 
%
\end{prop} 
%
\begin{proof}
Pick a universe making $I,J,K$, and $L$ small, and let $\alpha:I\to\Set$ be a functor. We have the following five bijections:
$$
\colim\ \alpha\circ\pp\ \simeq\ 
%
\colim_{\ell\in L}\ \colim_{\theta(\pp(j))\to\ell}\ \alpha(\pp(j))\ \simeq\ 
%
\colim_{k\in K}\ \colim_{\theta(\pp(j))\to\psi(k)}\ \alpha(\pp(j))
$$
$$
\ \simeq\ \colim_{k\in K}\ \colim_{\theta(i)\to\psi(k)}\ \alpha(i)\ \simeq\ 
%
\colim_{\ell\in L}\ \colim_{\theta(i)\to\ell}\ \alpha(i)\ \simeq\ 
%
\colim\ \alpha.
$$
Indeed, the first and fifth bijections follow from \eqref{coco} p.~\pageref{coco}, the second and fourth bijections follow from the cofinality of $\psi$, the third bijection follows from the cofinality of $\pp_k$. In view of Proposition 2.5.2 p.~57 of the book, this proves the claim.
\end{proof}
\end{s}

%

\begin{s}\label{cipc}
P.~75. Throughout the section about the IPC Property, one can assume that $\A$ is a big category. This applies in particular to Corollary 3.1.12 p.~77, corollary used in this generalized form at the end of the proof of Proposition 6.1.16 p.~136 of the book.
\end{s}

%

\begin{s}\label{poc}
P.~77, bottom. The following fact, which results from Propositions 3.1.11 (ii) p.~77 and 2.5.2 (iv) p.~57 of the book, will be used on p.~419 of the book (see \S\ref{1744i} p.~\pageref{1744i}):

If $(I_\gamma\to J_\gamma)_{\gamma\in\Gamma}$ is a family of cofinal functors, then the natural functor 
$$
\prod I_\gamma\to\prod J_\gamma
$$ 
is cofinal. 
\end{s}

%

\begin{s} 
P.~78, Proposition 3.2.2. It is easy to see that Condition (iii) is equivalent to
%
\begin{equation}\label{78} 
\colim\ \Hom_I(i,\pp)\simeq\pt\quad\text{for all }i\in I, 
\end{equation} 
%
which is Condition (vi) in Proposition 2.5.2 p.~57 of the book. (Proposition 2.5.2 states, among other things, that \eqref{78} is equivalent to the cofinality of $\pp$.)
\end{s}

%

\begin{s} 
P.~80. Propositions 3.2.4 and 3.2.6 can be combined as follows. 

\begin{prop}\label{comb}
Let $\pp:J\to I$ be fully faithful. Assume that $I$ is filtrant and cofinally small, and that for each $i$ in $I$ there is a morphism $i\to\pp(j)$ for some $j$ in $J$. Then $\pp$ is cofinal and $J$ is filtrant and cofinally small. 
\end{prop} 

\begin{proof}
In view of Proposition 3.2.4 it suffices to show that $J$ is cofinally small. By Proposition 3.2.6, there is a small full subcategory $S\subset I$ cofinal to $I$. For each $s$ in $S$ pick a morphism $s\to\pp(j_s)$ with $j_s$ in $J$. Then, for each $j$ in $J$ there are morphisms $\pp(j)\to s\to\pp(j_s)$ with $s$ in $S$. As $\pp$ is full there is a morphism $j\to j_s$, and we conclude by using again Proposition 3.2.6.
\end{proof}
\end{s}

%

\begin{s} 
P.~80, proof of Lemma 3.2.8 (minor variant). As already pointed out, a ``$\ds\ilim$'' is missing in the last display. Recall the statement:
\begin{lem}
Let $I$ be a small ordered set, $\alpha:I\to\C$ a functor. Let $\cc J$ denote the set of finite subsets of $I$, ordered by inclusion. To each $J$ in $\cc J$, associate the restriction $\alpha_J:\cc J\to\C$ of $\alpha$ to $\cc J$. Then $\cc J$ is small and filtrant and moreover
$$
\colim\alpha\simeq\colim_{J\in\cc J}\colim\alpha_J.
$$
\end{lem}
\begin{proof}
Put
$$
A:=\colim\alpha,\quad
\beta_J:=\colim\alpha_J,\quad
B:=\colim\beta.
$$
Let 
$$
p_i:\alpha_i\to A,\quad 
p_{i,J}:\alpha_i\to\beta_J,\quad 
p_J:\beta_J\to B
$$
be the coprojections. Note that $p_{i,J}$ is defined only for $i$ in $J$. We easily check that 

\nn$\bu$ the morphisms$f_i:=p_{\{i\}}\circ p_{i,\{i\}}:\alpha_i\to B$ induce a morphism $f:A\to B$, 

\nn$\bu$ the morphisms $g_{i,J}:=p_i:\alpha_i\to A$ (with $i$ in $J$) induce a morphism $g_J:\beta_J\to A$, 

\nn$\bu$ the morphisms $g_J$ induce a morphism $g:B\to A$, 

\nn$\bu$ $f$ and $g$ are mutually inverse isomorphisms.
\end{proof}
\end{s}

% 

For the reader's convenience we reproduce Definition 3.3.1 p.~81. 

\begin{df}[Definition 3.3.1 p.~81, exactness] 
Let $F:\C\to\C'$ be a functor.

\nn\emph{(i)} We say that $F$ is \emph{right exact}\index{right exact} if the category $\C_{X'}$ is filtrant for all $X'$ in $\C'$. 

\nn\emph{(ii)} We say that $F$ is \emph{left exact}\index{left exact} if $F^{\op}:\C^{\op}\to\C'^{\op}$ is right exact, or equivalently if the category $\C^{X'}$ is cofiltrant for all $X'$ in $\C'$.

\nn\emph{(iii)} We say that $F$ is \emph{exact}\index{exact functor} if it is both right and left exact.
\end{df}

\begin{s} 
P.~81, proof of Proposition 3.3.2 (minor variant). Recall the statement:

\begin{prop}[Proposition 3.3.2 p.~81]\label{p332} 
Consider functors $I\xrightarrow\alpha\C\xrightarrow F\C'$, and assume that $I$ is finite, that $F$ is right exact, and that $\colim\alpha$ exists in $\C$. Then $\colim(F\circ\alpha)$ exists in $\C'$, and the natural morphism $\colim(F\circ\alpha)\to\colim\alpha$ is an isomorphism. 
\end{prop} 

\begin{proof}
Let $X'$ be in $\C'$. It suffices to show that the natural map 
$$
\Hom_{\C'}(F(\colim\alpha),X')\to\lim\Hom_{\C'}(F(\alpha),X')
$$ 
is bijective. Replacing Setting (\ref{241s}) p.~\pageref{241s} with 
$$
X\in\C\xleftarrow{\id_\C}\C\xrightarrow{F}\C'\ni X', 
$$ 
Isomorphism (\ref{241}) p.~\pageref{241} gives 
$$
\colim_{(F(Y)\to X')\in\C_{X'}}\Hom_\C(X,Y)\simeq\colim_{(X\to Y)\in\C^X}\Hom_{\C'}(F(Y),X').
$$ 
The identity of $X$ being an initial object of $\C^X$, we have 
$$
\colim_{(X\to Y)\in\C^X}\Hom_{\C'}(F(Y),X')\simeq\Hom_{\C'}(F(X),X'),
$$ 
and thus
%
\begin{equation}\label{332} 
\colim_{(F(Y)\to X')\in\C_{X'}}\Hom_\C(X,Y)\simeq\Hom_{\C'}(F(X),X').
\end{equation} 
%
We have bijections 
$$ 
\Hom_{\C'}(F(\colim\alpha),X')\simeq\colim_{(F(Y)\to X')\in\C_{X'}}\Hom_\C(\colim\alpha,Y)
$$
$$
\xr\sim\colim_{(F(Y)\to X')\in\C_{X'}}\lim\Hom_\C(\alpha,Y)\xr\sim\lim\colim_{(F(Y)\to X')\in\C_{X'}}\Hom_\C(\alpha,Y)
$$
$$
\simeq\lim\Hom_{\C'}(F(\alpha),X'). 
$$ 
The first and last bijections follow from \eqref{332}, the second one is clear, and the third one can be justified as follows: Inductive limits over the category $\C_{X'}$, which is filtrant because $F$ is right exact, commute with projective limits over the finite category $I$.
\end{proof}
\end{s}

%

\begin{s}
P.~84, Proposition 3.3.13. Recall the statement:

\begin{prop}[Proposition 3.3.13 p. 84]
Let $\C$ be a category admitting finite inductive limits, and let $A$ be in $\C^\wedge$. Then $A$ is left exact if and only if $\C_A$ is filtrant.
\end{prop}

We spell out the details of the proof of the implication $\C_A$ is filtrant $\then$ $A$ left exact.

By Proposition 3.3.3 of the book, stated as Proposition~\ref{333} p.~\ref{333} below, it suffices to show that $A$ commutes with finite projective limits. Let $(X_i)$ be a finite inductive system in $\C$. We must check that the natural map 
$$
A(\colim_i X_i)\to\lim_iA(X_i)
$$ 
is bijective. Consider the diagram below, in which we abbreviate 
$$
(Y\to A)\in\C_A
$$ 
by $Y$: 
$$
\begin{tikzcd}
\colim_Y\Hom_\C(\colim_iX_i,Y)\ar{r}{a}\ar{dd}[swap]{d}&\colim_Y\lim_i\Hom_\C(X_i,Y)\ar{d}{b}\\ 
{}&\lim_i\colim_Y\Hom_\C(X_i,Y)\ar{d}{c}\\ 
A(\colim_iX_i)\ar{r}[swap]{e}&\lim_iA(X_i).
\end{tikzcd}
$$ 
The maps $a,b,c,$ and $d$ are bijective for the following reasons: obviously for $a$, because of \eqref{263b} p.~\pageref{263b} for $c$ and $d$, and by Theorem 3.1.6 p.~74 of the book for $b$. Thus $e$ is also bijective.
\end{s}

%

\begin{s}\label{3315}
P.~85, proof of Proposition 3.3.15. To prove that $\A\to\C$ is cofinal, one can apply Proposition~\ref{318i} p.~\pageref{318i} with $J=\A,I=\C,L=\C',K=\cc S$. 
\end{s}

%%%

\subsection{Proposition 3.4.3 (i) (p.~88)}

This section is divided into two parts. In Part 1 we give a proof of Proposition 3.4.3 (i) which is slightly different from the one in the book. In Part 2 we derive from Proposition 3.4.3 (i) another proof of \eqref{coco} p.~\pageref{coco}. 

\subsubsection{Part 1} 

The statement is phrased as follows: 

``For any category $\C$ and any functor $\alpha:M[I\to K\rightarrow J]\to\C$ we have $\colim\alpha\simeq\colim_{j\in J}\colim_{i\in I_{\psi(j)}}\alpha(i,j,\pp(i)\to\psi(j))$.'' 

One needs some assumption ensuring the existence of the indicated inductive limits. Here we shall assume that $\C$ admits inductive limits indexed by $J$ and $I_{\psi(j)}$ for all $j$ in $J$. 

Recall the notation. We have functors $I\xrightarrow\pp K\xleftarrow\psi J$ between small categories, and 
$$
M:=M[I\xrightarrow\pp K\xleftarrow\psi J] 
$$ 
is the category defined in Definition 3.4.1 p.~87 of the book. We also have a functor $\alpha:M\to\C$. 

Choose a universe $\U$ making $\C$ a small category, let $\Phi$ be the functor from $J$ to $\Cat$ defined by $\Phi(j):=I_{\psi(j)}$, view $\C$ is a constant functor from $J$ to $\Cat$, and let $\theta:\Phi\to\C$ be the morphism of functors such that $\theta_j:I_{\psi(j)}\to\C$ is the composition of $\alpha$ with the natural functor from $I_{\psi(j)}$ to $M$. Then 
$$
j\mapsto\colim_{(i,u)\in I_{\psi(j)}}\alpha(i,j,u) 
$$ 
($u$ being a morphism in $K$ from $\pp(i)$ to $\psi(j)$) is a functor from $J$ to $\C$ by Lemma~\ref{r52} p.~\pageref{r52}. 

Isomorphism 
%
\begin{equation}\label{coco2}
\colim\alpha\simeq\colim_j\ \colim_{i,u}\alpha(i,j,u),
\end{equation} 
%
where $(i,u)$ runs over $I_{\psi(j)}$ (with $u:\pp(i)\to\psi(j)$), will result from 

\begin{prop}
Assume $I,J$, and $K$ are small categories, and $\beta:M^{\op}\to\Set$ is a functor. Then 
$$
j\mapsto\lim_{(i,u)\in I_{\psi(j)}}\beta(i,j,u)
$$ 
is a functor from $J^{\op}$ to $\Set$, and we have
%
\begin{equation}\label{lili} 
\lim\beta=\lim_j\ \lim_{(i,u)\in I_{\psi(j)}}\beta(i,j,u) 
\end{equation} 
%
as an equality between subsets of the product $P$ of the $\beta(i,j,u)$. 
\end{prop} 

\begin{proof}
The first claim follows from Lemma~\ref{r52} p.~\pageref{r52}. To prove the second claim, let $L$ and $R$ denote respectively the left and right-hand side of \eqref{lili}, let $x=(x(i,j,u))$ be in $P$, and let us denote generic morphisms in $I$ and $J$ by $v:i\to i'$ and $w:j\to j'$. Then $x$ is in $L$ if and only if 
%
\begin{equation}\label{fe} 
(v,w)\in\Hom_M((i,j,u),(i',j',u'))\then x(i,j,u)=\beta(v,w)(x(i',j',u',)), 
\end{equation} 
%
whereas $x$ is in $R$ if and only if \eqref{fe} holds when $v$ or $w$ is an identity morphism, so that the equality $L=R$ follows from the fact that any morphism 
$$
(v,w):(i,j,u)\to(i',j',u')
$$ 
in $M$ satisfies $(v,w)=(v,j')\circ(i,w)$.
\end{proof}

In view of Theorem 3.1.6 p.~74 of the book, Isomorphism~\eqref{coco2} implies 
%
\begin{prop}\label{cocop} 
If $J$ and $I_{\psi(j)}$ are filtrant for all $j$ in $J$, then $M$ is filtrant.
\end{prop}

\subsubsection{Part 2}\label{2111}

Here is another proof of \eqref{coco} p.~\pageref{coco}: In the above setting, let us assume 
$$
K=J,\quad\psi=\id_J,
$$ 
and let $\alpha:I\to\C$ be a functor. We must prove 
$$
\colim_i\alpha(i)\simeq\colim_j\ \colim_{i,u}\alpha(i). 
$$ 
(Recall: $u:\pp(i)\to j$.) In view of \eqref{coco2}, it suffices to prove that we have 
$$
\colim_{i,j,u}\alpha(i)\simeq\colim_i\alpha(i),
$$ 
or, in other words, that 
%
\begin{equation}\label{coco3} 
\text{the projection $M\to I$ is cofinal.} 
\end{equation} 
%
If $i_0$ is in $I$, then an object of $M^{i_0}$ is a pair of morphisms $(i_0\to i,\pp(i)\to j)$. It is easy to see that $(i_0\xr\id i_0,\pp(i_0)\xr\id\pp(i_0))$ is an initial object of $M^{i_0}$, and \eqref{coco3} follows.

%%

\subsection{Brief Comments}

\begin{s} 
P.~89, Proposition 3.4.5 (iii). The proof uses implicitly the following fact: 

\begin{prop}\label{355}
If $F$ is a cofinally small filtrant category, then there is a small {\em filtrant} full subcategory of $F$ cofinal to $F$. 
\end{prop}

This results immediately from Corollary 2.5.6 p.~59 and Proposition 3.2.4 p.~79 (see Proposition~\ref{comb} p.~\pageref{comb}). This fact also justifies the sentence ``We may replace ``filtrant and small'' by ``filtrant and cofinally small'' in the above definition'' p.~132, Lines 4 and 5 of the book.
\end{s}

%%

\subsection{Six Closely Related Statements}

For the reader's convenience we collect six statements closely related to Exercise 3.4 (i) p.~90 of the book. 

\subsubsection{Proposition 2.1.10 p.~40}

\begin{prop}[Proposition 2.1.10 p.~40]\label{2.1.10}
If $F:\C\to\C'$ is a functor admitting a left adjoint, if $I$ is a small category, and if $\C$ admits projective limits indexed by $I$, then $F$ commutes with such limits.
\end{prop}

(See also Proposition~\ref{2.1.10b} p.~\pageref{2.1.10b}.)

\subsubsection{Definition 2.2.6 p.~47}

\begin{df}[Definition 2.2.6 p.~47, stability by base change]\label{dsbc}
Let $\C$ be a category which admits fiber products and inductive limits indexed by a category $I$.

\nn\emph{(i)} We say that inductive limits in $\C$ indexed by $I$ are \emph{stable by base change}\index{base change}\index{stable by base change} if for any morphism $Y\to Z$ in $\C$, the base change functor $\C_Z\to\C_Y$ given by 
$$
\C_Z\ni(X\to Z)\mapsto(X\times_ZY\to Y)\in\C_Y
$$ 
commutes with inductive limits indexed by $I$.

\nn\emph{(ii)} If $\C$ admits small inductive limits and \emph{(i)} holds for any small category $I$, we say that \emph{small inductive limits in $\C$ are stable by base change}.
\end{df}

\subsubsection{Exercise 2.7 (ii) p.~65}

\begin{prop}[Exercise 2.7 (ii) p.~65]\label{sbcs}
The base change functors in $\Set$ commute with small inductive and projective limits limits. In particular, small inductive limits in $\Set$ are stable by base change.
\end{prop}

(See \S\ref{27i} p.~\pageref{27i}.) Note that Proposition~\ref{sbcs} generalizes the distributivity of multiplication over addition in $\bb N$.

\subsubsection{Proposition 3.3.3 p.~82}

\begin{prop}[Proposition 3.3.3 p.~82]\label{333}
Let $F:\C\to\C'$ be a functor and assume that $\C$ admits finite projective limits. Then $F$ is left exact if and only if it commutes with such limits.
\end{prop}

\begin{cor}\label{bre}
In the setting of Proposition 2.7.1 p.~62 of the book, the functors
$$
\A^\C\to\A,\quad F\mapsto(\oo h_\C^\dagger F)(A)\quad\text{and}\quad
\C^\wedge\to\A,\quad A\mapsto(\oo h_\C^\dagger F)(A)
$$ 
are right exact. 
\end{cor}

\begin{proof}
This follows from Proposition~\ref{333} and \S\ref{bil} p.~\pageref{bil}.
\end{proof} 

\begin{cor}\label{bre2}
In the setting of \S\ref{opddagg} p.~\pageref{opddagg}, the functors
$$
\A^{\C^{\op}}\to\A,\quad F\mapsto(h^{\op})^\ddagger(F)(A)\quad\text{and}\quad
\C^\wedge\to\A,\quad A\mapsto(h^{\op})^\ddagger(F)(A)
$$ 
are left exact. 
\end{cor} 

\subsubsection{Proposition 3.3.6 p.~83}

\begin{prop}[Proposition 3.3.6 p.~83]\label{336}
A functor admitting a left adjoint is left exact.
\end{prop}

\subsubsection{Exercise 3.4 (i) p.~90}

\begin{prop}[Exercise 3.4 (i) p.~90]\label{34i}
If $F:\C\to\C'$ is a right exact functor and $f:X\epi Y$ is an epimorphism in $\C$, then $F(f):F(X)\to F(Y)$ is an epimorphism in $\C'$.
\end{prop}

(This exercise is used in the second sentence of p.~227 of the book.)

\begin{proof}
Let $f'_1,f'_2:F(Y)\rightrightarrows X'$ be morphisms in $\C'$ satisfying 
$$
f'_1\circ F(f)=f'_2\circ F(f)=:f'.
$$
This is visualized by the diagram
$$
\begin{tikzcd}
\Big(F(X)\ar{r}{f}&F(Y)\ar[yshift=.7ex]{r}{f'_1}\ar[yshift=-.7ex]{r}[swap]{f'_2}&X'\Big)=\Big(F(X)\ar{r}{f'}&X'\Big).
\end{tikzcd}
$$ 
For $i=1,2$ let $f_i$ be the morphism $f$ viewed as a morphism from $(X,f')$ to $(Y,f'_i)$ in $\C_{X'}$: 
$$
\begin{tikzcd}
F(X)\ar{dr}[swap]{f'}\ar{rr}{F(f)}&&F(Y)\ar{dl}{f'_i}\\ 
{}&X'.
\end{tikzcd}
$$
As $\C_{X'}$ is filtrant, there are morphisms $\gamma_i:(Y,f'_i)\to(Z,g')$, defined by morphisms $g_i:Y\to Z$, such that $\gamma_1\circ f_1=\gamma_2\circ f_2$:
$$
\begin{tikzcd}
F(X)\ar{d}[swap]{f'}\ar{r}{F(f)}&F(Y)\ar{d}[swap]{f'_i}\ar{r}{F(g_i)}&F(Z)\ar{d}{g'}\\ 
X'\ar[equal]{r}&X'\ar[equal]{r}&X'.
\end{tikzcd}
$$
As $f$ is an epimorphism, the equality $g_1\circ f=g_2\circ f$ implies $g_1=g_2=:g$, and thus $f'_1=g'\circ F(g)=f'_2$.
\end{proof}

%%%

\section{About Chapter 4}

\begin{s}
P.~93, Lemma 4.1.2. Here is a slightly more general statement:

\begin{lem}\label{proj3}
Let $\C$ be a category, let $P:\C\to\C$ be a functor, let $\ee:\id_\C\to P$ be a morphism of functors, and let $X$ be an object of $\C$. Then the following conditions are equivalent:

\nn{\em(a)} $\ee_{P(X)}$ is an isomorphism and $P(\ee_X)$ is an epimorphism,

\nn{\em(b)} $P(\ee_X)$ is an isomorphism and $\ee_{P(X)}$ is a monomorphism,

\nn{\em(c)} $\ee_{P(X)}$ and $P(\ee_X)$ are equal isomorphisms.
\end{lem}

\begin{proof} It is enough to prove (a)$\then$(c)$\Leftarrow$(b). 

\nn(a)$\then$(c): Put $u:=(\ee_{P(X)})^{-1}\circ P(\ee_X)$. It suffices to show 
%
\begin{equation}\label{uidpx}
u=\id_{P(X)}.
\end{equation}
%
We have 
$$
u\circ\ee_X=(\ee_{P(X)})^{-1}\circ P(\ee_X)\circ\ee_X=(\ee_{P(X)})^{-1}\circ\ee_{P(X)}\circ\ee_X=\ee_X,
$$ 
and thus 
$$
P(u)\circ P(\ee_X)=P(\ee_X)=\id_{P^2(X)}\circ P(\ee_X).
$$
As $P(\ee_X)$ is an epimorphism, this implies $P(u)=\id_{P^2(X)}$, and thus 
$$
\ee_{P(X)}\circ u=P(u)\circ \ee_{P(X)}=\ee_{P(X)}.
$$ 
As $\ee_{P(X)}$ is an isomorphism, this implies \eqref{uidpx}, as required.

\nn(b)$\then$(c): We shall use several times the assumption that $P(\ee_X)$ is an isomorphism. Put $v:=P(\ee_X)^{-1}\circ\ee_{P(X)}$. It suffices to show 
%
\begin{equation}\label{vidpx}
v=\id_{P(X)}.
\end{equation}
%
We have 
$$
v\circ\ee_X=P(\ee_X)^{-1}\circ\ee_{P(X)}\circ\ee_X=P(\ee_X)^{-1}\circ P(\ee_X)\circ\ee_X=\ee_X,
$$ 
$$
P(v)\circ P(\ee_X)=P(\ee_X),
$$
$$
P(v)=\id_{P^2(X)},
$$
$$
\ee_{P(X)}\circ v=P(v)\circ\ee_{P(X)}=\ee_{P(X)}=\ee_{P(X)}\circ\id_{P(X)}.
$$ 
As $\ee_{P(X)}$ is a monomorphism, this implies \eqref{vidpx}, as required. 
\end{proof}

Definition 4.1.1 p.~93 of the book can be stated as follows:

\begin{df}[Definition 4.1.1 p.~93, projector] 
Let $\C$ be a category. A {\em projector}\index{projector} on $\C$ is the data of a functor $P:\C\to\C$ and a morphism $\ee:\id_\C\to P$ such that each object $X$ of $\C$ satisfies the equivalent conditions of Lemma~\ref{proj3}. 
\end{df}
\end{s}

%

\begin{s} 
P.~94, proof of (a)$\then$(b) in Proposition 4.1.3 (ii) (additional details): In the commutative diagram 
$$
\begin{tikzcd}
\Hom_\C(P(Y),X)\ar{d}{\sim}[swap]{\varepsilon_X\circ}\ar{r}{\circ\varepsilon_Y}&\Hom_\C(Y,X)\ar{d}{\varepsilon_X\circ}[swap]{\sim}\\ 
\Hom_\C(P(Y),P(X))\ar{r}[swap]{\circ\varepsilon_Y}{\sim}&\Hom_\C(Y,P(X)),
\end{tikzcd}
$$ 
the vertical arrows are bijective by (a), and the bottom arrow is bijective by (i).
\end{s}

%%

\begin{s} 
P.~95, proof of Proposition 4.1.4 (i) (additional details). The authors write: ``The two compositions 
$$
\begin{tikzcd}
P\ar[yshift=0.7ex]{r}{\ee\circ P}\ar[yshift=-0.7ex]{r}[swap]{P\circ\ee}&P^2\ar{r}{R\eta L}&P
\end{tikzcd}
$$ 

\nn are equal to $\id_P$''. If we translate this statement into the language of Notation~\ref{nhove} p.~\pageref{nhove} and Notation~\ref{nmat} p.~\pageref{nmat}, we get 

\begin{equation}\label{tra}
\begin{pmatrix}R*\eta*L\\ \ee*R*L\end{pmatrix}
=RL=\begin{pmatrix}R*\eta*L\\ R*L*\ee\end{pmatrix}.
\end{equation}

\nn To prove \eqref{tra}, write 

$$
\begin{pmatrix}R*\eta*L\\ \ee*R*L\end{pmatrix}
=\begin{pmatrix}R*\eta&L\\ \ee*R&L\end{pmatrix}
=\begin{pmatrix}R*\eta\\ \ee*R\end{pmatrix}\circ\begin{pmatrix}L\\ L\end{pmatrix}
\overset{\text{(a)}}{=}RL
$$

$$
\overset{\text{(b)}}{=}\begin{pmatrix}R\\ R\end{pmatrix}\circ\begin{pmatrix}\eta*L\\ L*\ee\end{pmatrix}
=\begin{pmatrix}R&\eta*L\\ R&L*\ee\end{pmatrix}
=\begin{pmatrix}R*\eta*L\\ R*L*\ee\end{pmatrix},
$$ 

\nn Equalities (a) and (b) resulting respectively from \eqref{159} p.~\pageref{159} and \eqref{158} p.~\pageref{158}, and the other equalities following from Proposition~\ref{pil1} p.~\pageref{pil1}. 
\end{s}

%%

\section{About Chapter 5}

\subsection{Beginning of Section 5.1 (p.~113)}

We want to define the notions of coimage (denoted by $\Coim$) and image (denoted by $\Ima$) in a slightly more general way than in the book. To this end we start by defining these notions in a particular context in which they coincide. To avoid confusions we (temporarily) use the notation $\IM$ for these particular cases. The proof of the following lemma is obvious. 

\begin{lem}\label{imset} 
For any set theoretical map $g:U\to V$ we have natural bijections 
$$ 
\Coker(U\times_VU\parar U)\simeq\IM g\simeq\Ker(V\parar V\sqcup_UV),
$$ 
where $\IM g$ denotes the image of $g$. 
\end{lem} 

Let $\C$ be a $\U$-small category, and let us denote by $\hy:\C\to\C^\wedge$ and $\ky:\C\to\C^\vee$ the Yoneda embeddings. For any morphism $f:X\to Y$ in $\C$ define $\IM\hy(f)$ in $\C^\wedge$ and $\IM\ky(f)$ in $\C^\vee$ by
$$
(\IM\hy(f))(Z):=\IM\,\hy(f)_Z,\quad(\IM\ky(f))(Z):=\IM\,\ky(f)_Z 
$$
for any $Z$ in $\C$. Lemma \ref{imset} implies

\begin{equation}\label{IM2}
\begin{split}
\IM\hy(f)\simeq\Coker(\hy(X)\times_{\hy(Y)}\hy(X)\parar\hy(X)),\\ \\ 
\IM\ky(f)\simeq\Ker(\ky(Y)\parar\ky(Y)\sqcup_{\ky(X)}\ky(Y)).
\end{split}
\end{equation}

\begin{df}[coimage, image]\label{dci}
In the above setting, the {\em coimage}\index{coimage} of $f$ is the object $\Coim f$ of $\C^\vee$ defined by 
$$ 
(\Coim f)(Z):=\Hom_{\C^\wedge}(\IM\hy(f),\hy(Z))
$$ 
for all $Z$ in $\C$, and the {\em image}\index{image} of $f$ is the object $\Ima f$ of $\C^\wedge$ defined by 
$$ 
(\Ima f)(Z):=\Hom_{\C^\vee}(\ky(Z),\IM\ky(f)) 
$$ 
for all $Z$ in $\C$. Moreover, we regard $(\Coim f)(Z)$ as a subset of $\Hom_\C(X,Z)$, and $(\Ima f)(Z)$ as a subset of $\Hom_\C(Z,Y)$. (These subsets will be spelled out by Proposition~\ref{epimono} below.)
\end{df} 

\begin{conv}\label{bra}
If $A\parar B\to C$ is a diagram in a given category, then the notation $[A\parar B\to C]$ shall mean that the two compositions coincide.
\end{conv}

The following proposition is obvious:

\begin{prop}\label{epimono}
If $f:X\to Y$ is a morphism in a category $\C$, and if $Z$ is an object of $\C$, then we have 
$$
(\Coim f)(Z)=\left\{x:X\to Z\ \bigg|\ \left[W\parar X\xr fY\right]\then\left[W\parar X\xr xZ\right]\ \forall\ W\in\C\right\},
$$
$$
(\Ima f)(Z)=\left\{y:Z\to Y\ \bigg|\ \left[X\xr fY\parar W\right]\then\left[Z\xr yY\parar W\right]\ \forall\ W\in\C\right\}.
$$ 
In particular, these two sets do not depend on the universe $\U$ making $\C$ a $\U$\--category. There are natural morphisms 
$$
\ky(X)\to\Coim f\to\ky(Y),\quad\hy(X)\to\Ima f\to\hy(Y)
$$ 
in $\C^\vee$ and $\C^\wedge$ respectively. Moreover, $\ky(X)\to\Coim f$ is an epimorphism, and $\Ima f\to\hy(Y)$ is a monomorphism. 
\end{prop} 

For the sake of emphasis we write
$$
\ky(X)\epi\Coim f\to\ky(Y),\quad\hy(X)\to\Ima f\mono\hy(Y).
$$ 

By \eqref{IM2} we have 
$$ 
(\Coim f)(Z)\simeq\Ker\Big(\Hom_\C(X,Z)\parar\Hom_{\C^\wedge}\big(\hy(X)\times_{\hy(Y)}\hy(X),\hy(Z)\big)\Big), 
$$ 
$$ 
(\Ima f)(Z)\simeq\Ker\Big(\Hom_\C(Z,Y)\parar\Hom_{\C^\vee}\big(\ky(Z),\ky(Y)\sqcup_{\ky(X)}\ky(Y)\big)\Big). 
$$
 
This implies 

\begin{prop}\label{coimim}
If $P:=X\times_YX$ exists in $\C$, then $\Coim f$ is naturally isomorphic to $\Coker(P\parar X)\in\C^\vee$. If $S:=Y\sqcup_XY$ exists in $\C$, then $\Ima f$ is naturally isomorphic to $\Ker(Y\parar S)\in\C^\wedge$. 
\end{prop} 

In view of Lemma~\ref{imset} and Proposition~\ref{coimim} we can replace the notation $\IM$ with $\Ima$ (or $\Coim$). The following proposition is obvious: 

\begin{prop}\label{fun}
We have: 

$f\mapsto\Ima\hy(f)$ and $\Ima$ are functors from $\Mor(\C)$ to $\C^\wedge$, 

$f\mapsto\Ima\ky(f)$ and $\Coim$ are functors from $\Mor(\C)$ to $\C^\vee$. 
\end{prop}

\begin{df}[strict epimorphism] 
A morphism $f:X\to Y$ in a category $\C$ is a {\em strict epimorphism}\index{strict epimorphism} if the morphism $\Coim f\to\ky(Y)$ is an isomorphism.
\end{df} 

The lemma below is obvious:

\begin{lem}\label{strepi}
A morphism $f:X\to Y$ in a category $\C$ is a strict epimorphism if and only if, for all $Z$ in $\C$, the map 
$$
\circ f:\Hom_\C(Y,Z)\to\Hom_\C(X,Z)
$$ 
induces a bijection 
$$
\Hom_\C(Y,Z)\xr\sim(\Coim f)(Z).
$$ 
By Proposition~\ref{epimono} p.~\pageref{epimono}, this condition does not depend on the universe $\U$ making $\C$ a $\U$-category. Moreover, a strict epimorphism is an epimorphism. 
\end{lem} 

%%

\subsection{Brief Comments}

\begin{s}\label{515i}
P.~115, Proposition 5.1.5 (i). For the sake of completeness we spell out some details, and, for the reader's convenience we reproduce Proposition 5.1.5 (i) p.~115 of the book. 

\begin{prop}[Proposition 5.1.5 (i) p.~115] 
If $\C$ is a category admitting finite inductive and projective limits, then the following five conditions on a morphism $f:X\to Y$ are equivalent:

\nn\emph{(a)} $f$ is an epimorphism and $\Coim f\to\Ima f$ is an isomorphism,

\nn\emph{(b)} $\Coim f\xr\sim Y$,

\nn\emph{(c)} the sequence $X\times_YX\parar X\to Y$ is exact,

\nn\emph{(d)} there exists a pair of parallel arrows $g,h:Z\parar X$ such that $f\circ g=f\circ h$ and $\Coker(g,h)\to Y$ is an isomorphism,

\nn\emph{(e)} for any $Z$ in $\C$, the set $\Hom_\C(Y,Z)$ is isomorphic to the set of morphisms $u:X\to Z$ satisfying $u\circ v_1=u\circ v_2$ for any pair of parallel morphisms $v_1,v_2:W\parar X$ such that $f\circ v_1=f\circ v_2$. 
\end{prop} 

Here are the additional details: 

\nn(b)$\then$(a): The composition $\Coim f\to\Ima f\to Y$ being an isomorphism by assumption, $\Ima f\to Y$ is an epimorphism. Then Proposition 5.1.2 (iv) of the book implies that $f$ is an epimorphism and that $\Ima f\to Y$ is an isomorphism, from which we conclude that $\Coim f\to\Ima f$ is an isomorphism.

\nn(c)$\ssi$(e): Write $P$ for $X\times_YX$. Recall that (c) says that $P\parar X\to Y$ is exact. By Proposition~\ref{epimono} p.~\pageref{epimono} Condition~(e) is equivalent to the condition in Lemma~\ref{strepi} p.~\pageref{strepi}. Thus, it suffices to show that, letting $Z$ be an object of $\C$, we have in the notation of Convention~\ref{bra} p.~\pageref{bra} 
$$
[P\parar X\xr xZ]\iff\Big((\forall\ W\in\C)\ [W\parar X\xr fY]\then[W\parar X\xr xZ]\Big).
$$ 
Implication $\si$ is clear. Let us prove $\then$. Assuming 
$$
[P\parar X\xr xZ]\quad\text{and}\quad[W\parar X\xr fY],
$$ 
we must check $[W\parar X\xr xZ]$. As the morphisms $W\parar X$ factor through $P\parar X$ by definitions of $P$, the statement is obvious. 

(Note that the equivalence (c)$\ssi$(e) also follows from Proposition~\ref{coimim} p.~\pageref{coimim} and Proposition~\ref{strepi} p.~\pageref{strepi}.)
\end{s}

%

\begin{s} 
P.~116, proof of Proposition 5.1.7 (i) (minor variant). Recall the statement: 

\begin{prop}[Proposition 5.1.7 (i) p. 116]
Let $\C$ be a category admitting finite inductive and projective limits in which epimorphisms are strict. Let us denote by $I'_g$ the coimage of any morphism $g$ in $\C$. Let $f:X\to Y$ be a morphism in $\C$ and $X\xr u I'_f\xr v Y$ its factorization through $I'_f$. Then $v$ is a monomorphism. 
\end{prop}

\begin{proof}
Consider the commutative diagram
$$
\begin{tikzcd}
X\ar[two heads]{d}[swap]{b}\ar[two heads]{r}{u}&I'_f\ar[two heads]{d}{a}\ar{r}{v}&Y\\
I'_{a\circ u}\ar{ur}{d}\ar[two heads]{r}[swap]{c}&I'_v.\ar{ru}
\end{tikzcd}
$$ 
(We first form $a$, then $b$ and $c$, and finally $d$; the existence of $d$ is a very particular case of Proposition~\ref{fun} p.~\pageref{fun}.) By (the dual of) Proposition 5.1.2 (iv) p.~114 of the book, it suffices to show that $a$ is an isomorphism. As $a\circ u$ is a strict epimorphism, Proposition 5.1.5 (i), (a)$\then$(b), p.~115 of the book, implies that $c$ is an isomorphism. We claim that $d\circ c^{-1}$ is inverse to $a$. We have 
$$
a\circ d\circ c^{-1}=c\circ c^{-1}=\id_{I'_v}
$$ 
and 
$$
d\circ c^{-1}\circ a\circ u=d\circ c^{-1}\circ c\circ b=d\circ b=u=\id_{I'_f}\circ u,
$$ 
and the conclusion follows from the fact that $u$ is an epimorphism.
\end{proof}
\end{s}

%

\begin{s}
P.~117, Definition 5.2.1 (definition of a system of generators). There is a comment about this in Pierre Schapira's Errata 

\href{http://people.math.jussieu.fr/~schapira/books/Errata.pdf}{http://people.math.jussieu.fr/$\sim$schapira/books/Errata.pdf}.

As observed at the bottom of p.~121 of the book, the definition can be stated as follows:

\begin{df}[generator, system of generators] 
Let $S$ be a set of objects of a category $\C$ and $\cc S$ the corresponding subcategory. We say that $S$ is a {\em system of generators}\index{generator}\index{system of generators} if the functor $\pp:\C\to\cc S^\wedge$, $X\mapsto\Hom_\C(\ ,X)$ is conservative.
\end{df}
\end{s}

%

\begin{s}
P.~118, second display: Isomorphism 
$$
\Hom_{\Set}\Big(\Hom_\C(G,X),\Hom_\C(G,X)\Big)\simeq\Hom_{\C^\vee}(G^{\sqcup\Hom_\C(G,X)},X)
$$ 
is a particular case of the following isomorphism, which holds for any $\U$-set $S$ and any objects $G$ and $X$ of $\C$: 
$$
\Hom_{\Set}(S,\Hom_\C(G,X))\simeq\Hom_{\C^\vee}(G^{\sqcup S},X).
$$ 
\end{s}

%

\begin{s}
P.~118, proof of Proposition 5.2.3: the proof of (ii) uses Proposition~\ref{34i} p.~\pageref{34i} and Proposition 3.3.7 (i) p.~83 of the book. 

Proof of (v): We add a few details. Recall the setting: the category $\C$ admits finite projective limits, small coproducts, and a generator $G$. We already know that $\pp_G:=\Hom_\C(G,\ )$ is conservative, left exact, and preserves monomorphisms. 

Put $\pp:=\pp_G:=\Hom_\C(G,\ )$. Let $f_i:Y_i\mono X$ $(i=1,2)$ be two monomorphisms such that $\pp(f_1)$ and $\pp(f_2)$ have same image. We want to prove:

Claim 1: $\pp(f_1)$ and $\pp(f_2)$ define the same subobject of $X$.

Form the cartesian square
%
\begin{equation}\label{523v1}
\begin{tikzcd}
Y_1\times_XY_2\ar{r}{p_1}\ar{d}[swap]{p_2}&Y_1\ar[tail]{d}{f_1}\\ 
Y_2\ar[tail]{r}[swap]{f_2}&X.
\end{tikzcd}
\end{equation}

Claim 2: $p_1$ and $p_2$ are isomorphisms. 

Clearly, Claim 2 implies Claim 1. Applying $\pp$ to \eqref{523v1}, we get the commutative diagram 
$$
\begin{tikzcd}
\pp(Y_1\times_XY_2)\ar{rd}{\sim}\ar[bend left]{rrd}{\pp(p_1)}\ar[bend right]{rdd}[swap]{\pp(p_2)}\\ 
{}&\pp(Y_1)\times_{\pp(X)}\pp(Y_2)\ar{r}{q_1}\ar{d}[swap]{q_2}&\pp(Y_1)\ar[tail]{d}{\pp(f_1)}\\ 
{}&\pp(Y_2)\ar[tail]{r}[swap]{\pp(f_2)}&\pp(X),
\end{tikzcd}
$$ 
the morphism 
$$
\pp(Y_1\times_XY_2)\to\pp(Y_1)\times_{\pp(X)}\pp(Y_2)
$$ 
being an isomorphism by left-exactness of $\pp$. As $\pp(f_1)$ and $\pp(f_2)$ are injective and have same image, $q_1$ and $q_2$ are bijective, and thus $\pp(p_1)$ and $\pp(p_2)$ are also bijective. As $\pp$ is conservative,  this implies Claim~2, and we saw that Claim~2 implies Claim~1.
\end{s}

%

\begin{s}
P.~119, Theorem 5.2.5: see Corollary~\ref{c2111} p.~\pageref{c2111}.
\end{s}

%

\begin{s}
Corollary 5.2.10 p. 121 follows from Proposition 5.2.9 p.~121 of the book and Theorems 5.2.5 and 5.2.6 p.~119 of the book.
\end{s}

%

\begin{s}
P.~122, sentence following Definition 5.3.1. This sentence is ``Note that if $\cc F$ is strictly generating,\index{strictly generating subcategory} then $\Ob(\cc F)$ is a system of generators''. See \S\ref{ffc} p.~\pageref{ffc}.
\end{s}

%%

\subsection{Lemma 5.3.2 (p.~122)} 

Here is a minor variant of the proof of Lemma 5.3.2. 

\begin{lem}[Lemma 5.3.2 p.~122]
If $\F\subset\G$ are full subcategories of a category $\C$, and if $\F$ is strictly generating, then $\G$ is strictly generating. 
\end{lem} 

\begin{proof}
Let 
$$
\begin{tikzcd}
\C\ar{r}{\gamma}\ar{dr}[swap]{\pp}&\G^\wedge\ar{d}{\rho}\\
&\F^\wedge
\end{tikzcd}
$$ 
be the natural functors ($\rho$ being the restriction), and let $X$ and $Y$ be in $\C$. We have 
$$
\begin{tikzcd}
\Hom_\C(X,Y)\ar{r}{\gamma'}\ar{dr}{\sim}[swap]{\pp'}&\Hom_{\G^\wedge}(\gamma(X),\gamma(Y))\ar{d}{\rho'}\\ 
{}&\Hom_{\F^\wedge}(\pp(X),\pp(Y)). 
\end{tikzcd}
$$ 
We want to prove that $\gamma'$ is bijective. As $\pp'$ is bijective, it suffices to show that $\gamma'$ is surjective. Let $\xi$ be in $\Hom_{\G^\wedge}(\gamma(X),\gamma(Y))$. There is a (unique) $f$ in $\Hom_\C(X,Y)$ such that  
\begin{equation}\label{rhoxi}
\rho(\xi)=\pp(f),
\end{equation}
and it suffices to prove $\xi=\gamma(f)$. Let $Z$ be in $\G$ and $z$ be in $\Hom_\C(Z,X)$. It suffices to show that the morphisms 
$$
\begin{tikzcd}
Z\arrow[yshift=0.7ex]{r}{\xi_Z(z)}\arrow[yshift=-0.7ex]{r}[swap]{f\circ z}&Y
\end{tikzcd}
$$ 
coincide. As $\F$ is strictly generating, it suffices to show that the morphisms 
$$
\begin{tikzcd}
\pp(Z)\arrow[yshift=0.7ex]{rr}{\pp(\xi_Z(z))}\arrow[yshift=-0.7ex]{rr}[swap]{\pp(f\circ z)}&&\pp(Y)
\end{tikzcd}
$$ 
coincide. Let $W$ be in $\F$. It suffices to show that the maps 
$$
\begin{tikzcd}
\pp(Z)(W)\arrow[yshift=0.7ex]{rr}{\pp(\xi_Z(z))_W}\arrow[yshift=-0.7ex]{rr}[swap]{\pp(f\circ z)_W}&&\pp(Y)(W)
\end{tikzcd}
$$ 
coincide, that is, it suffices to show that the maps 
$$
\begin{tikzcd}
\Hom_\C(W,Z)\arrow[yshift=0.7ex]{rr}{\xi_Z(z)\circ}\arrow[yshift=-0.7ex]{rr}[swap]{f\circ z\circ}&&\Hom_\C(W,Y)
\end{tikzcd}
$$ 
coincide. We have, for $w$ in $\Hom_\C(W,Z)$,
$$
\xi_Z(z)\circ w
\overset{\text{(a)}}{=}\xi_W(z\circ w)
\overset{\text{(b)}}{=}\rho(\xi)_W(z\circ w)
\overset{\text{(c)}}{=}\pp(f)_W(z\circ w)
\overset{\text{(d)}}{=}f\circ z\circ w, 
$$ 
Equality (a) following from the functoriality of $\xi$ (see diagram below), Equality (b) following from the definition of $\rho$, Equality (c) following from \eqref{rhoxi}, and Equality (d) following from the definition of $\pp$.
\end{proof}

For the reader's convenience, we add the diagram 
$$
\begin{tikzcd}
Z&z\in\Hom_\C(Z,X)\ar{r}{\xi_Z}\ar{d}[swap]{\circ w}&\Hom_\C(Z,Y)\ar{d}{\circ w}\\ 
W\ar{u}{w}&\Hom_\C(W,X)\ar{r}[swap]{\xi_W}&\Hom_\C(W,X).
\end{tikzcd}
$$

%%

\subsection{Brief Comments} 

\begin{s} 
P.~122, proof of Lemma 5.3.3 (minor variant). Recall the statement: 

\begin{lem}[Lemma 5.3.3 p.~122] 
If $\C$ is a category admitting small inductive limits and $\F$ is a small full subcategory of $\C$, then the functor $\pp:\C\to\F^\wedge$ admits a left adjoint $\psi:\F^\wedge\to\C$, and for $A$ in $\F^\wedge$ we have 
$$
\psi(A)\simeq\colim_{(Y\to A)\in\F_A}Y. 
$$ 
\end{lem} 

\begin{proof}
We have, for $X$ in $\C$ and $A$ in $\F^\wedge$, 
$$
\Hom_\C\left(\colim_{(Y\to A)\in\F_A}Y,X\right)\simeq\lim_{(Y\to A)\in\F_A}\Hom_\C(Y,X)
$$ 
$$
\simeq\lim_{(Y\to A)\in\F_A}\pp(X)(Y)\simeq\Hom_{\F^\wedge}(A,\pp(X)),
$$  
the last isomorphism following from \eqref{263c} p.~\pageref{263c}.
\end{proof}
\end{s}

%%

\subsection{Theorem 5.3.6 (p.~124)} 

As an exercise, I rewrite parts of the proof. The difference between the rewriting and the original proof is very slight. Here is the statement of the theorem. (Recall that $\C$ is a category, that $\F$ is an essentially small full subcategory of $\C$, that the functor $\pp:\C\to\F^\wedge$ is defined by $\pp(X)(Y):=\Hom_\C(Y,X)$, and that, by definition, $\F$ is strictly generating if and only if $\pp$ is fully faithful.)

\begin{thm}[Theorem 5.3.6 p.~124]\label{536} 
Let $\C$ be a category satisfying the conditions \emph{(i)-(iii)} below:

\nn\emph{(i)} $\C$ admits small inductive limits and finite projective limits, 

\nn\emph{(ii)} small filtrant inductive limits are stable by base change (Definition 2.2.6 p.~47 of the book, stated above as Definition~\ref{dsbc} p.~\pageref{dsbc}), 

\nn\emph{(iii)} any epimorphism is strict. 

\nn Let $\F$ be an essentially small full subcategory of $\C$ such that 

\nn\emph{(a)} $\Ob(\F)$ is a system of generators,

\nn\emph{(b)} $\F$ is closed by finite coproducts in $\C$. 

\nn Then $\F$ is strictly generating.
\end{thm}

\begin{proof} We may assume from the beginning that $\F$ is small.

\nn Step 1. By Proposition 5.2.3 (i) p.~118 of the book, the functor $\pp$ is conservative and faithful.

\nn Step 2. By Proposition 1.2.12 p.~16 of the book, a morphism $f$ in $\C$ is an epimorphism as soon as $\pp(f)$ is an epimorphism.

\nn Step 3. Let us fix $X$ in $\C$, and let $\zeta:\C_X\to\C$ be the forgetful functor, so that an object of $\C_X$ consists of a morphism $z:\zeta(z)\to X$ in $\C$. Let $(z_i)_{i\in I}$ be a small filtrant inductive system in $\C_X$. We claim that the natural morphism
$$
\colim_i\Coim z_i\to\Coim\left(\colim_i\zeta(z_i)\to X\right)
$$
is an isomorphism.
% old version: 
% https://docs.google.com/document/d/1-bgnW0CaMksEslkcQPfmcPCVRyQHWSC4JscWC9YaZzA/edit

\nn Step 3'. Let $A$ be in $\F^\wedge$, and let $(B_i\to A)_{i\in I}$ be a small filtrant inductive system in $(\F^\wedge)_A$. We claim 
$$
\colim_i\Coim(B_i\to A)\xrightarrow{\sim}
\Coim\left(\colim_iB_i\to A\right).
$$
Step 4. We claim that there is, for all $z:Z\to X$ in $\F_X$ and all $Y$ in $\C$, a natural isomorphism 
$$
\Hom_\C(\Coim z,Y)\simeq\Hom_{\F^\wedge}(\Coim\pp(z),\pp(Y)).
$$ 

\nn Step 5. Let us denote by $I$ the set of finite subsets of $\Ob(\F_X)$, ordered by inclusion. Regarding $I$ as a category, it is small and filtrant. Let $\xi:I\to\F_X$ be the functor defined by letting $\xi(A)$ be the natural morphism $\bigsqcup_{z\in A}\zeta(z)\to X$. We claim that the natural morphism 
$$
\colim_{A\in I}\pp\zeta\xi(A)\to\pp(X) 
$$ 
is an epimorphism.

\nn Step 6. We claim that there is a natural isomorphism 
$$
\colim_{A\in I}\Coim\xi(A)\simeq X. 
$$

These steps imply the theorem: Indeed, we have, in the above setting, 
%
\begin{align*} 
%
\Hom_\C(X,Y)&\simeq\Hom_\C\left(\colim_{A\in I}\Coim\xi(A),Y\right)&\text{by Step 6}\\ \\ 
%
&\simeq\lim_{A\in I}\Hom_\C(\Coim\xi(A),Y)\\ \\ 
%
&\simeq\lim_{A\in I}\Hom_{\F^\wedge}\Big(\Coim\pp\xi(A),\pp(Y)\Big)&\text{by Step 4}\\ \\ 
%
&\simeq\Hom_{\F^\wedge}\left(\colim_{A\in I}\Coim\pp\xi(A),\pp(Y)\right)\\ \\ 
%
&\simeq\Hom_{\F^\wedge}\left(\Coim\left(\colim_{A\in I}\pp\zeta\xi(A)\to\pp(X)\right),\pp(Y)\right)&\text{by Step 3'}\\ \\ 
%
&\simeq\Hom_{\F^\wedge}(\pp(X),\pp(Y))&\text{by Step 5.}
%
\end{align*} 

It remains to prove Steps 3, 3', 4, 5, and 6. We refer to the book for Step 3. The proof of Step 3' is the same. (It is easy to see that small inductive limits in $\F^\wedge$ are stable by base change --- Definition 2.2.6 p.~47 of the book, stated above as Definition~\ref{dsbc} p.~\pageref{dsbc}.) 

\begin{proof}[Proof of Step 4] 
We have 
$$
\Hom_\C(\Coim z,Y)\simeq\Hom_\C\big(\Coker(Z\times_XZ\rightrightarrows Z),Y\big)
$$
$$
\simeq\Ker\big(\Hom_\C(Z,Y)\rightrightarrows\Hom_\C(Z\times_XZ,Y)\big),
$$ 
and also 
$$
\Hom_\C(Z,Y)\simeq\Hom_{\F^\wedge}(\pp(Z),\pp(Y))
$$ 
by the Yoneda Lemma. As $\pp$ is faithful, the natural map 
$$
\Hom_\C(Z\times_XZ,Y)\to\Hom_{\F^\wedge}\big(\pp(Z\times_XZ),\pp(Y)\big)
$$
$$
\simeq\Hom_{\F^\wedge}\big(\pp(Z)\times_{\pp(X)}\pp(Z),\pp(Y)\big).
$$ 
is injective. This yields 
$$
\Hom_\C(\Coim z,Y)
$$
$$
\simeq\Ker\Big(\Hom_{\F^\wedge}\big(\pp(Z),\pp(Y)\big)\rightrightarrows\Hom_{\F^\wedge}\big(\pp(Z)\times_{\pp(X)}\pp(Z),\pp(Y)\big)\Big)
$$
$$
\simeq\Hom_{\F^\wedge}\Big(\Coker\big(\pp(Z)\times_{\pp(X)}\pp(Z)\rightrightarrows\pp(Z)\big),\pp(Y)\Big)
$$
$$
\simeq\Hom_{\F^\wedge}(\Coim\pp(z),\pp(Y)).
$$ 
\end{proof}

\begin{proof}[Proof of Step 5]
Let $Z$ be in $\F$. We must show that the natural map 
$$
\colim_{A\in I}\ (\pp\zeta\xi(A))(Z)\to\pp(X)(Z):=\Hom_\C(Z,X) 
$$
is surjective. Let $z$ be in $\Hom_\C(Z,X)$. It suffices to check that $z$ is in the image of the natural map 
$$
(\pp\zeta\xi(\{z\}))(Z)=\Hom_\C(Z,Z)\xrightarrow{z\circ}\Hom_\C(Z,X),
$$
which is obvious. 
\end{proof}

\begin{proof}[Proof of Step 6] 
As Step 3 implies 
$$
\colim_{A\in I}\Coim\xi(A)\simeq\Coim\left(\colim_{A\in I}\zeta\xi(A)\to X\right),
$$ 
it suffices to prove 
%
\begin{equation}\label{step6a}
\Coim\left(\colim_{A\in I}\zeta\xi(A)\to X\right)\simeq X.
\end{equation}
% 
Epimorphisms being strict, it is enough, in view of Proposition 5.1.5 (i), (a)$\then$(b), p.~115 of the book to check that 
%
\begin{equation}\label{step6b}
\colim_{A\in I}\zeta\xi(A)\to X
\end{equation}
% 
is an epimorphism. Let 
$$
\colim_{A\in I}\pp\zeta\xi(A)\xrightarrow{b}\pp\left(\colim_{A\in I}\zeta\xi(A)\right)\xrightarrow{a}\pp(X)
$$
be the natural morphisms. As $a\circ b$ is an epimorphism by Step~5, $a$ is an epimorphism, and Step~2 implies that \eqref{step6b} is also an epimorphism. 
\end{proof}
\end{proof}

%%

\subsection{Theorem 5.3.9 (p.~128)}

To prove the existence of $\F$, one can also argue as follows. 

\begin{lem} 
Let $\C$ be a category admitting finite inductive limits, and let $\A$ be a small full subcategory of $\C$. Then:

\nn{\em(a)} There is a small full subcategory $\B$ of $\C$ such that $\A\subset\B\subset \C$ and that $\B$ is closed by finite inductive limits in the following sense: if $(X_i)$ is a finite inductive system in $\B$ and $X$ is an inductive limit of $(X_i)$ in $\C$, then $X$ is isomorphic to some object of $\B$.

\nn{\em(b)} There is a small full subcategory $\A'$ of $\C$ such that $\A\subset\A'\subset \C$ and that each finite inductive system in $\A$ has a limit in $\A'$. 
\end{lem} 

\begin{proof}
Since there are only countably many finite categories up to isomorphism, (b) is clear. To prove (a), let $\A\subset\A'\subset\A''\subset\cdots$ be a tower of full subcategories obtained by iterating the argument used to prove (b), and let $\B$ be the union of the $\A^{(n)}$.
\end{proof}

%%%

\section{About Chapter 6}

\subsection{Theorem 6.1.8 (p. 132)} 

Recall the statement:

\begin{thm}[Theorem 6.1.8 p.~132]\label{618}
If $\C$ is a category, then the category $\Ind(\C)$ admits small filtrant inductive limits and the natural functor $\Ind(\C)\to\C^\wedge$ commutes with such limits.
\end{thm} 

%The proposition below follows from Theorem~\ref{618} above and Proposition~\ref{333} p.~\pageref{333}. 

%\begin{prop} If $\C$ is a category, then small filtrant inductive limits exist in $\Ind(\C)$ and are right exact. \end{prop}

Here is a minor variant of Step (i) of the proof of Theorem 6.1.8. We must show:  

\begin{lem} 
If $\alpha:I\to\Ind(\C)$ is a functor, if $I$ is small and filtrant, and if $A=\ic\alpha$ is in $\C^\wedge$, then $\C_A$ is filtrant. 
\end{lem} 

\begin{proof}
Let $M$ be the category attached by Definition 3.4.1 p. 87 of the book to the functors 
$$
\C\xrightarrow h\C^\wedge\xleftarrow{\iota\circ\alpha}I,
$$ 
where $h:\C\to\C^\wedge$ and $\iota:\Ind(\C)\to\C^\wedge$ are the natural embeddings. Proposition~\ref{cocop} p.~\pageref{cocop} implies that $M$ is filtrant, and that it suffices to check that Conditions (iii) (a) and (iii) (b) of Proposition 3.2.2 p.~78 of the book hold for the obvious functor $M\to\C_A$. Let us do it for Condition (iii) (b), the case of Condition (iii) (a) being similar and simpler. 

For all $i$ in $I$ and all $X$ in $\C$ let 
$$
p_i:\alpha(i)\to A\quad\text{and}\quad p_i(X):\Hom_\C(X,\alpha(i))\to A(X)
$$
be the coprojections.

Given an object $c$ of $\C_A$, and object $m$ of $M$, and a pair of parallel morphisms $\sigma,\sigma':c\parar\pp(m)$ in $\C_A$, we must find a morphism $\tau:m\to n$ in $M$ satisfying $\pp(\tau)\circ\sigma=\pp(\tau)\circ\sigma'$. 

Let $c$ be given by the morphism $X\to A$ in $\C^\wedge$, let $m$ be given by the morphism $Y\to\alpha(i)$ in $\Ind(\C)$, and let $\sigma$ and $\sigma'$ be given by the morphisms $s,s':X\parar Y$ making the diagram below commute:
$$
\begin{tikzcd}
X\ar{dd}\ar[yshift=0.7ex]{r}{s}\ar[yshift=-0.7ex]{r}[swap]{s'}&Y\ar{d}{y}\\ 
{}&\alpha(i)\ar{d}{p_i}\\ 
A\ar[equal]{r}&A.
\end{tikzcd}
$$ 

Then we are looking for and object $n$ of $M$ given by a morphism $Z\to\alpha(j)$, and for a morphism $t:Y\to Z$ defining the sought-for morphism $\tau$. 

As $p_i(X)(y\circ s)$ equals $p_i(X)(y\circ s')$ in $A(X)\simeq\colim\Hom_\C(X,\alpha)$ and $I$ is filtrant, there is a morphism $t:i\to j$ in $I$ such that $\alpha(t)\circ y\circ s=\alpha(t)\circ y\circ s'$, and we can set $Z:=\alpha(j)$ and $z:=\id_{\alpha(j)}$. The situation is depicted by the commutative diagram
$$
\begin{tikzcd}
X\ar{dd}\ar[yshift=0.7ex]{r}{s}\ar[yshift=-0.7ex]{r}[swap]{s'}&Y\ar{d}{y}\ar{r}&\alpha(j)\ar[equal]{d}\\ 
{}&\alpha(i)\ar{d}{p_i}\ar{r}{\alpha(t)}&\alpha(j)\ar{d}{p_j}\\ 
A\ar[equal]{r}&A\ar[equal]{r}&A.
\end{tikzcd}
$$
\end{proof}

%%

\subsection{Proposition 6.1.9 (p.~133)} 

\subsubsection{Proof of Proposition 6.1.9}

The following point is implicit in the book, and we give additional details for the reader's convenience. Proposition 6.1.9 results immediately from the statement below:

\begin{prop} 
Let $\A$ be a category which admits small filtrant inductive limits, let $F:\C\to\A$ be a functor, and let $\C\xr i\Ind(\C)\xr j\C^\wedge$ be the natural embeddings. Then the functor $i^\dagger(F):\Ind(\C)\to\A$ exists, commutes with small filtrant inductive limits, and satisfies $i^\dagger(F)\circ i\simeq F$. Conversely, any functor $\widetilde F:\Ind(\C)\to\A$ commuting with small filtrant inductive limits with values in $\C$, and satisfying $\widetilde F\circ i\simeq F$, is isomorphic to $i^\dagger(F)$. 
\end{prop} 

\begin{proof}
The proof is essentially the same as that of Proposition 2.7.1 on p.~62 of the book. (See also \S\ref{c271b} p.~\pageref{c271b}.) Again, we give some more details about the proof of the fact that $i^\dagger(F)$ commutes with small filtrant inductive limits. Put $\widetilde F:=i^\dagger(F)$. 

Let us attach the functor $B:=\Hom_\A(F(\ ),Y)\in\C^\wedge$ to the object $Y$ of $\A$. To apply Proposition~\ref{2.1.10b} p.~\pageref{2.1.10b} to the diagram 
$$
\begin{tikzcd}
I\ar{r}{\alpha}&\Ind(\C)\ar{d}[swap]{j}\ar{r}{\widetilde F}&\A\\
&\C^\wedge
\end{tikzcd}
$$
(where $I$ is a small filtrant category), it suffices to check that there is an isomorphism 
$$
\Hom_\A\left(\widetilde F(\ ),Y\right)\simeq
\Hom_{\C^\wedge}(\ \ ,B)
$$ 
in $\Ind(\C)^\wedge_\V$, where $\V$ is a universe containing $\U$ such that $\C^\wedge$ is a $\V$-category. We have 
$$
\widetilde F(A):=\colim_{(X\to A)\in\C_A}F(X),
$$ 
as well as the following isomorphisms functorial in $A\in\Ind(\C)$:
$$
\Hom_\A\left(\widetilde F(A),Y\right)=
\Hom_\A\left(\colim_{(X\to A)\in\C_A}F(X),Y\right)\simeq
\lim_{(X\to A)\in\C_A}B(X)
$$
$$
\simeq
\lim_{(X\to A)\in\C_A}\Hom_{\C^\wedge}((j\circ i)(X),B)
\simeq
\Hom_{\C^\wedge}\left(\icolim_{(X\to A)\in\C_A}X,B\right)\simeq
\Hom_{\C^\wedge}(j(A),B).
$$
\end{proof}

\subsubsection{Comments about Proposition 6.1.9} 

Let us record Part (i) of the Proposition as 
%
\begin{equation}\label{133i}
IF\circ\iota_\C\simeq\iota_{\C'}\circ F, 
\end{equation} 
%
and note that we have, in the setting of Corollary 6.3.2 p.~140, 
%
\begin{equation}\label{140}
\colim(F\circ\alpha)\xr\sim(JF)(\ic\alpha).
\end{equation} 
%
Let us also record Part (ii) of the Proposition as 
% 
\begin{equation}\label{133ii}
\ic(IF\circ\alpha)\xr\sim IF(\ic\alpha).
\end{equation} 
% 
(See \S\ref{s133ii} p.~\pageref{s133ii}.)

%%

\subsection{Proposition 6.1.12 (p.~134)}\label{6112}

We give some more details about the proof. Recall the setting: We have two categories $\C_1$ and $\C_2$, and we shall define functors
$$
\begin{tikzcd}
\Ind(\C_1\times\C_2)\ar[yshift=0.7ex]{r}{\theta}&\Ind(\C_1)\times\Ind(\C_2),\ar[yshift=-0.7ex]{l}{\mu}
\end{tikzcd}
$$ 
and prove that they are mutually quasi-inverse equivalences. (In fact, we shall only define the effect of $\theta$ and $\mu$ on objects, leaving also to the reader the definition of the effect of these functors on morphisms.) But first let us introduce some notation. We shall consider functors 
$$
A\in\Ind(\C_1\times\C_2);\quad A_i,B_i\in\Ind(\C_i);
$$ 
objects $X_i,Y_i,\dots$ in $\C_i$; and elements 
$$
x\in A(X_1,X_2),\ y\in A(Y_1,Y_2),\dots,\quad x_i\in A_i(X_i),\ y_i\in A_i(Y_i),\dots
$$ 
When we write 
$$
\colim_x\ \cdots,\quad\colim_{x_i}\ \cdots,\quad\colim_{x_1,x_2}\ \cdots,
$$ 
we mean, in the first case, not only that $x$ runs over the elements of $A(X_1,X_2)$, but also that $X_1$ and $X_2$ themselves run over the objects of $\C_1$ and $\C_2$, so that we are taking the inductive limit of some functor defined over $(\C_1\times\C_2)_A$. In the other cases, the interpretation is similar. 

Let us define $\theta$ and $\mu$: The functor $\theta$ is defined by $\theta(A)=(A_1,A_2)$ with
%
\begin{equation}\label{ai}
A_i:=\ic_x\ X_i, 
\end{equation}
% 
and let 
%
\begin{equation}\label{dot}
a_i(x):X_i\to A_i,\quad a_i(x,Y_i):\Hom_{\C_i}(Y_i,X_i)\to A_i(Y_i)
\end{equation}
% 
be the coprojections. The functor $\mu$ is defined by 
$$
\mu(A_1,A_2):=\ic_{x_1,x_2}\ (X_1,X_2). 
$$ 

\begin{prop}[Proposition 6.1.12 p.~134]\label{p6112}
The functors $\theta$ and $\mu$ are mutually quasi-inverse.
\end{prop}

\begin{proof}
We have 
$$
\mu(A_1,A_2)(X_1,X_2)\simeq\colim_{y_1,y_2}\Big(\Hom_{\C_1}(X_1,Y_1)\times\Hom_{\C_2}(X_2,Y_2)\Big)
$$
$$
\overset{\text{(a)}}{\simeq}\colim_{y_1}\Hom_{\C_1}(X_1,Y_1)\times\colim_{y_2}\Hom_{\C_2}(X_2,Y_2)
\overset{\text{(b)}}{\simeq}A_1(X_1)\times A_2(X_2),
$$ 
Isomorphism~(a) following from the IPC Property (see pp 75-77 of the book), and Isomorphism~(b) following from \eqref{263b} p.~\pageref{263b}. We record this as
$$
\mu(A_1,A_2)(X_1,X_2)\simeq A_1(X_1)\times A_2(X_2).
$$
This suggests the notation $A_1\times A_2$ for $\mu(A_1,A_2)$, notation which we shall use henceforth.

Let us prove
%
\begin{equation}\label{6112a}
\theta\circ\mu\simeq\id_{\Ind(\C_1)\times\Ind(\C_2)}.
\end{equation}
%
If $A_i$ is in $\Ind(\C_i)$ for $i=1,2$; if $A$ is $A_1\times A_2$; and if $(B_1,B_2)$ is $\theta(A)$, then we have 
$$ 
B_1
\overset{\text{(a)}}{\simeq}\ic_x\ X_1\simeq\ic_{x_1,x_2}\ X_1
\overset{\text{(b)}}{\simeq}\ic_{x_1}\ X_1
\overset{\text{(c)}}{\simeq}A_1.
$$ 
Indeed, Isomorphism~(a) follows from \eqref{ai}, Isomorphism~(b) from the definition of $A$, Isomorphism~(b) from the fact that the projection 
$$
(\C_1)_{A_1}\times(\C_2)_{A_2}\to(\C_1)_{A_1}
$$ 
is cofinal by Lemma~\ref{proj} below coupled with the fact that $(\C_2)_{A_2}$ is connected, and Isomorphism~(c) from our old friend \eqref{263a} p.~\pageref{263a}. (By the way, in this proof we are using \eqref{263a} a lot without explicit reference.)

\begin{lem}\label{proj}
If $I$ and $J$ are categories and if $J$ is connected, then the projection $I\times J\to I$ is cofinal.
\end{lem}

\begin{proof}
Let $i_0$ be in $I$. We must check that $(I\times J)^{i_0}$ is connected. We have $(I\times J)^{i_0}\simeq I^{i_0}\times J$, and it is easy to see that a product of connected categories is connected. 
\end{proof}

This ends the proof of \eqref{6112a}.

Let us prove
\begin{equation}\label{6112b}
\mu\circ\theta\simeq\id_{\Ind(\C_1\times\C_2)}.
\end{equation}

Let $A$ be in $\Ind(\C_1\times\C_2)$ and set $(A_1,A_2):=\theta(A)$. We shall define morphisms $A\to A_1\times A_2$ and $A_1\times A_2\to A$, and leave it to the reader to check that these morphisms are mutually inverse isomorphisms of functors. 

\nn$\bu$ Definition of the morphism $A\to A_1\times A_2$: Recall 
$$
A\simeq\ic_y\ (Y_1,Y_2), 
$$ 
and let $y$ be in $A(Y_1,Y_2)$. We shall define $y_i$ in $A_i(Y_i),i=1,2$. In Notation~\eqref{dot} p.~\pageref{dot} put 
$$
y_i:=a_i(y,Y_i)(\id_{Y_i}). 
$$ 
To define our morphism $A\to A_1\times A_2$ we consider the commutative diagram
$$
\begin{tikzcd}
A\simeq\ds\ic_y(Y_1,Y_2)\ar{r}&\ds\ic_{x_1,x_2}(X_1,X_2)\simeq A_1\times A_2\\ 
(Y_1,Y_2)\ar{u}{y}\ar[equal]{r}&(Y_1,Y_2),\ar{u}[swap]{(y_1,y_2)}
\end{tikzcd}
$$ 
(see \eqref{263a} p.~\pageref{263a} and \eqref{pxa} p.~\pageref{pxa}) and we leave it to the reader to check that this diagram does define the indicated morphism.

\nn$\bu$ Definition of the morphism $A_1\times A_2\to A$. We must define the morphism represented by the dashed arrow in the diagram  
$$
\begin{tikzcd}
A_1\times A_2\simeq\ds\ic_{x_1,x_2}(X_1,X_2)\ar[dashed]{r}&\ds\ic_x(X_1,X_2)\simeq A\\ 
(X_1,X_2)\ar{u}{(x_1,x_2)}\ar[equal]{r}&(X_1,X_2)\ar{u}[swap]{x}
\end{tikzcd}
$$ 
(see \eqref{263a} p.~\pageref{263a} and \eqref{pxa} p.~\pageref{pxa}). Let $x_i$ in $A_i(X_i)$ be given for $i=1,2$. The category $(\C_1\times\C_2)_A$ being filtrant, there is, by Proposition 3.1.3 p.~73 of the book, a 5-tuple 
$$
\zeta:=(Z_1,Z_2,z,f_1,f_2)
$$ 
with $Z_i$ in $\C_i$, $z$ in $A(Z_1,Z_2)$, and $f_i$ in $\Hom_{\C_i}(X_i,Z_i)$, such that $x_i=a_i(z,X_i)(f_i)$ for $i=1,2$ (see \eqref{dot} p.~\pageref{dot}). We choose such a 5-tuple $\zeta$ and put $x:=A(f_1,f_2)(z)$: 
$$
\begin{tikzcd}
A\ar[equal]{r}&A\\ 
(X_1,X_2)\ar{u}{x}\ar{r}[swap]{(f_1,f_2)}&(Z_1,Z_2)\ar{u}[swap]{z}
\end{tikzcd}
$$ 
One checks that $x$ does not depend on the choice of $\zeta$, and that this process defines a morphism from $A_1\times A_2$ to $A$. This ends the proofs of Isomorphism~\eqref{6112b} p.~\pageref{6112b} and Proposition~\ref{p6112} p.~\pageref{p6112}. 
\end{proof} 

%%

\subsection{Brief Comments}

\begin{s} 
P.~136, proof of Proposition 6.1.18 (i) (minor variant). Recall the statement: 

\begin{prop}[Proposition 6.1.18 (i) p.~136] 
If a category $\C$ admits cokernels, do does $\Ind(\C)$. 
\end{prop}

This follows from Proposition~\ref{p332} p.~\pageref{p332}. 

% old version 
% https://docs.google.com/document/d/1bt5g2ANcF0Fnngn7mMdYA8qgN-K0iVKjWbtTEG7PPZU/edit

\end{s}

%

\begin{s} 
P.~137, table. In view of Corollary 6.1.17 p.~136, one can add two lines to the table:\bigskip 

\begin{center}
\begin{tabular}{|c|c|c|c|}\hline
&&$\C\to\Ind(\C)$&$\Ind(\C)\to\C^\wedge$\\ \hline
1&finite inductive limits&$\circ$&$\times$\\ \hline
2&finite coproducts&$\circ$&$\times$\\ \hline
3&small filtrant inductive limits&$\times$&$\circ$\\ \hline
4&small coproducts&$\times$&$\times$\\ \hline
5&small inductive limits&$\times$&$\times$\\ \hline
6&finite projective limits&$\circ$&$\circ$\\ \hline
7&small projective limits&$\circ$&$\circ$\\ \hline
\end{tabular}
\end{center}%\bigskip 
\nn(In Line 6 we assume that $\C$ admits finite projective limits, whereas in Line 7 we assume that $\C$ admits small projective limits.)%\bigskip 
\end{s}

%

\begin{s} 
P.~138, proof of Proposition 6.1.21. One can also argue as follows. Assume $\C$ admits finite projective limits. By Remark 2.6.5 p.~62 and Corollary 6.1.17 p.~136, all the inclusions represented in the diagram 
\[
\begin{tikzcd}
{}&\C^\wedge_\V\ar[-]{ld}\ar[-]{rd}\\
\C^\wedge_\U\ar[-]{rd}&&\Ind^\V(\C)\ar[-]{ld}{i}\\
&\Ind^\U(\C)\ar[-]{d}\\
&\C,
\end{tikzcd}
\]
except perhaps inclusion $i$, commute with finite projective limits. Thus inclusion $i$ commutes with finite projective limits. The argument for $\U$-small projective limits is the same. q.e.d.
\end{s}

%

\begin{s} P.~142, proof of Corollary 6.3.7. Let us check the isomorphism 
$$
\kappa(X)\simeq\ic\rho\circ\xi. 
$$ 
Recall the setting:
$$
\begin{tikzcd}
I\ar{r}{\xi}&\C^{\text fp}\ar{d}[swap]{\iota_\C}\ar{r}{\rho}&\C\ar[yshift=-.9ex]{dl}{\kappa'}\ar{d}{\iota_\C}\\ 
{}&\Ind(\C^{\text fp})\ar{ur}{J\rho}\ar{r}[swap]{I\rho}&\Ind(\C),
\end{tikzcd}
$$ 
$\kappa'$ being quasi-inverse to $J\rho$ (for more details, see p.~141 of the book), and $\kappa$ is defined by $\kappa:=I\rho\circ\kappa'$. We have 
$$
\kappa(X)=I\rho(\kappa'(\colim(\rho\circ\xi)))\simeq I\rho(\ic\xi)
$$

$$
\simeq\ic(I\rho\circ\iota_\C\circ\xi)\simeq\ic(\rho\circ\xi), 
$$ 

\nn the three isomorphisms being respectively justified by \eqref{140} p.~\pageref{140}, \eqref{133ii} p.~\pageref{133ii}, and \eqref{133i} p.~\pageref{133i}. q.e.d.
\end{s}

%%

\subsection{Theorem 6.4.3 (p.~144)}

Notational convention for this section, {\em and for this section only!} Superscripts will {\em never} be used to designate a category of the form $\C^{X'}$ attached to a functor $\C\to\C'$ and to an object $X'$ of $\C'$. Only two categories of the form $\C_{X'}$ (again attached to a functor $\C\to\C'$ and to an object $X'$ of $\C'$) will be considered in this section. As a lot of subscripts will be used, we shall denote these categories by 
%
\begin{equation}\label{slice}
\C/G(a)\text{ and }L/a
\end{equation}
%
instead of $\C_{G(a)}$ and $L_a$, to avoid confusion. Superscripts will {\em always} be used to designate categories of functors, like the category $\B^\A$ of functors from $\A$ to $\B$. 

Recall the statement: 
%
\begin{thm}[Theorem 6.4.3 p.~144]\label{643} 
If $\C$ is a category and $K$ is a finite category such that $\Hom_K(k,k)$ $=\{\id_k\}$ for all $k$ in $K$, then the natural functor 
$$
\Phi:\Ind(\C^K)\to\Ind(\C)^K
$$ 
is an equivalence.
\end{thm}

The key point is to check that 
%
\begin{equation}\label{es} 
\Phi\text{ is essentially surjective.} 
\end{equation} 
%
(The fact that $\Phi$ is fully faithful is proved as Proposition 6.4.1 p.~142 of the book.) 

In the book \eqref{es} is proved by an inductive argument. The limited purpose of this section is to attach, in an ``explicit'' way (in the spirit of the proof of Proposition 6.1.13 p.~134 of the book), to an object $G$ of $\Ind(\C)^K$ a small filtrant category $N$ and a functor $F:N\to\C^K$ such that 
$$ 
\Phi(\ic F)\simeq G. 
$$ 

As in the book we assume, as we may, that any two isomorphic objects of $K$ are equal. 

Let $\C,K$ and $G$ be as above. We consider $\C$ as being given once and for all, so that, in the notation below, the dependence on $\C$ will be implicit. For each $k$ in $K$, let $I_k$ be a small filtrant category and let 
$$
\alpha_k:I_k\to\C
$$ 
be a functor such that 
$$
G(k)=\ic\alpha_k.
$$ 

We define the category 
$$
N:=N\{K,G,(\alpha_k)\}
$$ 
as follows:

\nn[Beginning of the definition of the category $N:=N\{K,G,(\alpha_k)\}$.] An \emph{object} of $N$ is a pair $((i_k),P)$, where each $i_k$ is in $I_k$ and $P$ is a functor from $K$ to $\C$, subject to the conditions 

\nn$\bu\ \alpha_k(i_k)=P(k)$ for all $k$, 

\nn$\bu$ the coprojections $u_k(i_k):\alpha_k(i_k)=P(k)\to G(k)$ induce a morphism of functors 
%
\begin{equation}\label{u':}
u':P\to G.
\end{equation}

\nn(We regard $\C$ as a subcategory of $\Ind(\C)$.) The picture is very similar to the second diagram of p.~135 of the book: For each morphism $f:k\to\ell$ in $K$ we have the commutative square 
$$ 
\begin{tikzcd} 
\alpha_k(i_k)\ar[equal]{r}&P(k)\ar{r}{P(f)}\ar{d}[swap]{u_k(i_k)}&P(\ell)\ar{d}{u_\ell(i_\ell)}\ar[equal]{r}&\alpha_\ell(i_\ell)\\ 
{}&G(k)\ar{r}[swap]{G(f)}&G(\ell) 
\end{tikzcd} 
$$ 
in $\Ind(\C)$. 

A \emph{morphism} from $((i_k),P)$ to $((j_k),Q)$ is a pair $((f_k),\theta)$, where each $f_k$ is a morphism $f_k:i_k\to j_k$ in $I_k$, and $\theta:P\to Q$ is a morphism of functors, subject to the condition $\theta_k=\alpha_k(f_k)$ for all $k$: 
$$ 
\begin{tikzcd} 
\alpha_k(i_k)\ar{r}{\alpha_k(f_k)}\ar[equal]{d}&\alpha_k(j_k)\ar[equal]{d}\\ 
P(k)\ar{r}[swap]{\theta_k}&Q(k).
\end{tikzcd} 
$$ 
[End of the definition of the category $N:=N\{K,G,(\alpha_k)\}$.] 

The functor $F':K\to\C^N$ corresponding to $F:N\to\C^K$ is given by 
$$
F'(k)=\alpha_k\circ p_k,
$$ 
where $p_k:N\to I_k$ is the natural projection: 
$$
N\xr{p_k}I_k\xr{\alpha_k}\C.
$$ 
In other words, we set
$$
F\big((i_k),P\big)(k_0):=\alpha_{k_0}(i_{k_0}).
$$ 

\begin{lem}\label{npk}
The category $N$ is small and filtrant, and the functor $p_k$ is cofinal.
\end{lem} 

Clearly, Lemma~\ref{npk} implies Theorem~\ref{643}.  

\begin{proof}[Proof of Lemma~\ref{npk}]
We start as in the proof of Theorem 6.4.3 p.~144 of the book: 

We order $\Ob(K)$ be decreeing $k\le\ell$ if and only if $\Hom_K(k,\ell)\neq\varnothing$, and argue by induction on the cardinal $n$ of $\Ob(K)$. 

If $n=0$ the result is clear.

Otherwise, let $a$ be a maximal object of $K$; let $L$ be the full subcategory of $K$ such that 
$$
\Ob(L)=\Ob(K)\setminus\{a\}; 
$$ 
let $G_L:L\to\Ind(\C)$ be the restriction of $G$ to $L$; let  
$$
\widetilde{\alpha_a}:I_a\to\C/G(a)
$$ 
(see \eqref{slice} p.~\pageref{slice} for the definition of $\C/G(a)$) be the functor defined by 
$$
\widetilde{\alpha_a}(i_a):=\Big(u(i_a):\alpha_a(i_a)\to G(a)\Big);
$$
and put 
$$
N':=N\{L,G_L,(\alpha_\ell)\}.
$$ 

We define the functor 
$$ 
\pp:N'\to\big(\C/G(a)\big)^{L/a} 
$$ 
(see \eqref{slice} p.~\pageref{slice} for the definition of $L/a$) as follows. Let $((i_\ell),Q)$ be in $N'$. In particular, $Q$ is a functor from $L$ to $\C$, and we have, for each $\ell$ in $L$, a morphism 
$$
Q(\ell)=\alpha_\ell(i_\ell)\xr{u'(\ell)}\icolim\alpha_\ell=G(\ell) 
$$ 
in $\C$ (see \eqref{u':} p.~\pageref{u':}). Letting $\ell\xr fa$ be a morphism in $K$ viewed as an object of $L/a$, we put  
$$
\pp\big((i_\ell),Q\big)\left(\ell\xr fa\right):=\left(Q(\ell)\xr{u'(\ell)}G(\ell)\xr{G(f)}G(a)\right)\in\C/G(a).
$$

Letting 
$$
\Delta:\C/G(a)\to\big(\C/G(a)\big)^{L/a}
$$ 
be the diagonal functor, we can form the category 
$$
M:=M\left[N'\xrightarrow{\pp}\big(\C/G(a)\big)^{L/a}\xleftarrow{\ \Delta\circ\widetilde{\alpha_a}}I_a\right].
$$ 
Concretely, an object of $M$ is a triple 
%
\begin{equation}\label{il}
\Big(\big((i_\ell),Q\big),i_a,\big(\xi_f:Q(\ell)\to\alpha_a(i_a)\big)_{f:\ell\to a}\Big),
\end{equation}
%  
where $((i_\ell),Q)$ is an object of $N'$, where $i_a$ is an object of $I_a$, where $f$ runs over the morphisms from $\ell$ to $a$ in $K$, and where $\xi_f$ is a morphism from $Q(\ell)$ to $\alpha_a(i_a)$ which makes the square  
$$
\begin{tikzcd}
Q(\ell)\ar{r}{\xi_f}\ar{d}[swap]{u'(\ell)}&\alpha_a(i_a)\ar{d}{u(i_a)}\\ 
G(\ell)\ar{r}[swap]{G(f)}&G(a) 
\end{tikzcd}
$$ 
in $\C$ commute, and a morphism from \eqref{il} to 
$$
\Big(\big((i'_\ell),Q'\big),i'_a,\big(\xi'_f:Q'(\ell)\to\alpha_a(i'_a)\big)_{f:\ell\to a}\Big)
$$ 
is given by a family $(f_k:i_k\to i'_k)_{k\in K}$ of morphisms in $I_k$ making the squares 
$$
\begin{tikzcd}
Q(\ell)\ar{r}{\alpha_\ell(f_\ell)}\ar{d}[swap]{\xi_f}&Q'(\ell)\ar{d}{\xi'_f}\\ 
\alpha_a(i_a)\ar{r}[swap]{\alpha_a(f_a)}&\alpha_a(i'_a) 
\end{tikzcd}
$$ 
in $\C$ commute. (Recall $Q(\ell)=\alpha_\ell(i_\ell)$, $Q'(\ell)=\alpha_\ell(i'_\ell)$.)

We shall define functors 
$$
\begin{tikzcd}
N\ar[yshift=0.7ex]{r}{\lambda}&M\ar[yshift=-0.7ex]{l}{\mu}
\end{tikzcd}
$$ 
and leave it to the reader to check that they are mutually inverse isomorphisms. (In fact, we shall only define the effect of $\lambda$ and $\mu$ on objects, leaving also to the reader the definition of the effect of these functors on morphisms.)

We shall define maps 
$$
\begin{tikzcd}
\Ob(N)\ar[yshift=0.7ex]{r}{\lambda}&\Ob(M).\ar[yshift=-0.7ex]{l}{\mu}
\end{tikzcd}
$$ 

To define $\lambda$ let $((i_k),P)$ be in $N$, and let $Q$ be the restriction of $P$ to $L$. Then $\lambda((i_k),P)$ will be of the form 
$$
\Big(\big((i_\ell),Q\big),i_a,\big(\xi_f:Q(\ell)\to\alpha_a(i_a)\big)_{f:\ell\to a}\Big).
$$ 
As $Q(\ell)=P(\ell)$ and $\alpha_a(i_a)=P(a)$, we can (and do) put $\xi_f:=P(f)$. 

To define $\mu$ let 
$$
\Xi:=\Big(\big((i_\ell),Q\big)\ ,\ i_a\ ,\ \pp((i_\ell),Q)\to\Delta(\widetilde{\alpha_a}(i_a))\Big)
$$ 
be in $M$. The object $\mu(\Xi)$ of $N$ will be of the form $((i_k),P)$, so that we must define a functor $P:K\to\C$. 

We define $P(k)$ by putting $P(\ell):=Q(\ell)$ for $\ell$ in $L$, and $P(a):=\alpha_a(i_a)$. 

If $f:\ell\to m$ is a morphism in $L$, then we set $P(f):=Q(f):P(\ell)\to P(m)$. Let $\ell$ be in $L$. There is at most one morphism $f:\ell\to a$. If this morphism does exist, then we put $P(f):=\xi_f$. 

We leave it to the reader to check that $\lambda$ and $\mu$ are mutually inverse bijections. 

We also leave it to the reader to check that the set of morphisms in $M$ from $\lambda((i_k),P)$ to $\lambda((i'_k),P')$ is \emph{equal} (in the strictest sense of the word) to the set of morphisms in $N$ from $((i_k),P)$ to $((i'_k),P')$, so that we get an isomorphism 
$$ 
N\simeq M\left[N'\xrightarrow{\pp}(\C/G(a))^{L/a}\xleftarrow{\ \Delta\circ\widetilde{\alpha_a}}I_a\right]. 
$$

By induction hypothesis, 
%
\begin{equation}\label{n'saf}
N'\text{ is small and filtrant}
\end{equation} 
%
and the projection $N'\to I_\ell$ is cofinal for all $\ell$ in $L$. It follows from Proposition 2.6.3 (ii) p.~61 of the book that $\widetilde{\alpha_a}$ is cofinal. By assumption $\C/G(a)$ is filtrant, and Lemma~\ref{delta} below will imply that $\Delta$ is cofinal. Thus, 
%
\begin{equation}\label{daa}
\Delta\circ\widetilde{\alpha_a}\text{ is cofinal.}
\end{equation} 
% 
Taking Lemma~\ref{delta} below for granted, Lemma~\ref{npk} p.~\pageref{npk} now follows from \eqref{n'saf}, \eqref{daa}, and Proposition 3.4.5 p.~89 of the book. 
\end{proof} 

As already observed, Lemma~\ref{npk} implies Theorem~\ref{643} p.~\pageref{643}. The only remaining task is to prove 

\begin{lem}\label{delta}
If $I$ is a finite category and $\C$ a filtrant category, then the diagonal functor $\Delta:\C\to\C^I$ is cofinal.
\end{lem}

\begin{proof}
It suffices to verify Conditions (a) and (b) of Proposition 3.2.2 (iii) p.~78 of the book. Condition (b) is clear. To check Condition (a), let $\alpha$ be in $\C^I$. We must show that there is pair $(X,\lambda)$, where $X$ is in $\C$ and $\lambda$ is a morphism of functors from $\alpha$ to $\Delta(X)$. Let $S$ be a set of morphisms in $I$. It is easy to prove 
$$
(\exists\ Y\in\C)\left(\exists\ \mu\in\prod_{i\in I}\Hom_\C(\alpha(i),Y)\right)\Big(\forall\ (s:i\to j)\in S\Big)\Big(\mu_{j}\circ\alpha(s)=\mu_i\Big) 
$$ 
by induction on the cardinal of $S$, and to see that this implies the existence of $(X,\lambda)$.
\end{proof} 

%%

\subsection{Exercise 6.11 (p. 147)} 

We prove the following slightly more precise statement:

\begin{prop}\label{myprop1}
Let $F:\cc C\to\cc C'$ be a fully faithful functor, let $A'$ be in $\Ind(\cc C')$, and let $S$ be the set of objects $A$ of $\Ind(\cc C)$ such that $IF(A)\simeq A'$. Then the following conditions are equivalent: 

\nn{\em(a)} $S\neq\varnothing$, 

\nn{\em(b)} all morphism $X'\to A'$ in $\Ind(\cc C')$ with $X'$ in $\cc C'$ factorizes through $F(X)$ for some $X$ in $\cc C$, 

\nn{\em(c)} the natural functor $\cc C_{A'\circ F}\to\cc C'_{A'}$ is cofinal, 

\nn{\em(d)} $A'\circ F$ is in $S$.
\end{prop}

\begin{proof}\ 

\nn(a)$\then$(b). Let $f:X'\to IF(A)$ be a morphism in $\Ind(\cc C')$ with $X'$ in $\cc C'$ and $A$ in $\Ind(\cc C)$, let $\beta_0:I\to\cc C$ be a functor with $I$ small and filtrant and $\ic\beta_0\simeq A$; in particular $\ic(F\circ\beta_0)\simeq IF(A)$. By Proposition 6.1.13 p.~134 of the book there are functors $\alpha:J\to\cc C'$ and $\beta:J\to\cc C$, and there is a morphism of functors $\pp:\alpha\to F\circ\beta$ such that 

$J$ is small and filtrant, 

$\alpha$ is constant equal to $X'$, 

$\ic(F\circ\beta)\simeq IF(A)$, 

$\ic\pp\simeq f$. 

\nn Then $f$ factorizes as $X'=\alpha(j)\xr{\pp_j}F(\beta(j))\xr{p_j}IF(A)$, where $p_j$ is the coprojection.

\nn(b)$\then$(c). This follows from Proposition~\ref{comb} p.~\pageref{comb}. 

\nn(c)$\then$(d). This follows from Remark~\ref{cof} p.~\pageref{cof}. 

\nn(d)$\then$(a). This is obvious.
\end{proof}

%%%

\section{About Chapter 7}

\subsection{Brief Comments}

\begin{s} 
P.~149, Definition 7.1.1. The set $\cc S$ is a subset of $\Ob(\Mor(\C))$ (see Notation~\ref{mor} p.~\pageref{mor}). The proof of the following lemma (which will be used to prove \eqref{l} p.~\pageref{l}) is obvious. Neither Definition 7.1.1 nor the lemma below requires the Axiom of Universes.

\begin{lem}\label{711}
Let 
$$
\begin{tikzcd}\C\ar[yshift=.5ex]{r}{Q}&\C'\ar[yshift=-.4ex]{l}{R}\end{tikzcd}
$$ 
be functors such that $Q\circ R\simeq\id_{\C'}$, let $\cc S$ be a subset of $\Ob(\Mor(\C))$ such that $Q(s)$ is an isomorphism for all $s$ in $\cc S$, let 
$$
\theta:\id_\C\to R\circ Q
$$ 
satisfy $\theta_X\in\cc S$ for all $X$ in $\C$, let $\A$ be a category, and let $\B$ be the full subcategory of $\A^\C$ whose objects are the functors turning the elements of $\cc S$ into isomorphisms. Then the functors
$$
\begin{tikzcd}\A^{\C'}\ar[yshift=.4ex]{r}{\circ Q}&\B\ar[yshift=-.5ex]{l}{\circ R}\end{tikzcd}
$$
are mutually quasi-inverse equivalences. In particular, $Q$ is a localization of $\C$ by $\cc S$. 
\end{lem}
\end{s}

%

\begin{s} 
P.~154. Below the statement of Lemma 7.1.13 it is written: ``The verification is left to the reader. // Hence, we get a big category ...''. One might add between the two sentences something like: We also leave it to the reader to define the identity $\id_X$ of $X$ viewed as an object of $\C^r_{\mathcal S}$, and to check the equalities $f\circ\id_X=f$, $\id_X\circ g=g$ for $f$ in $\Hom_{\C^r_{\mathcal S}}(X,Y)$ and $g$ in $\Hom_{\C^r_{\mathcal S}}(Y,X)$.
\end{s}

%

\begin{s} 
P.~155. In the text between Lemma 7.1.15 and Theorem 7.1.16, one might add the following observation. The inverse of $(s:X\to X')\in\mathcal S$ is given by 
$$
X'\xr gY'\overset{t}{\leftarrow}X,
$$
where $g$ and $t$ are obtained by applying S3 with $f=\id_X$:
$$
\begin{tikzcd}
X\ar{r}{\id_X}\ar{d}[swap]{s}&X\ar[dashed]{d}{t}\\ X'\ar[dashed]{r}[swap]{g}&Y'.
\end{tikzcd}
$$
\end{s}

%

\begin{s}\label{778}
Let us spell out the proof of Remark 7.7.8 (ii) p.~156. Recall that $\cc S$ is a left and right multiplicative system in $\C$. For $X,Y\in\C$ we have 

$$
\Hom_{\C_{\cc S}^r}(X,Y)\simeq\colim_{(X'\to X)\in\cc S_X}\Hom_{\C_{\cc S}^r}(X,Y)\simeq\colim_{(X'\to X)\in\cc S_X}\Hom_{\C_{\cc S}^r}(X',Y)
$$

$$
\simeq
\colim_{(X'\to X)\in\cc S_X}\ \colim_{(Y\to Y')\in\cc S^Y}\Hom_{\C_{\cc S}^r}(X',Y')\simeq
\colim_{(Y\to Y')\in\cc S^Y}\ \colim_{(X'\to X)\in\cc S_X}\Hom_{\C_{\cc S}^r}(X',Y')
$$

$$
\simeq
\colim_{(Y\to Y')\in\cc S^Y}\Hom_{\C_{\cc S}^\ell}(X,Y')\simeq
\colim_{(Y\to Y')\in\cc S^Y}\Hom_{\C_{\cc S}^\ell}(X,Y)\simeq
\Hom_{\C_{\cc S}^\ell}(X,Y),
$$ 
the first and last isomorphisms being justified by the fact that $\cc S_X$ and $\cc S^Y$ are connected by Proposition 7.1.10 p.~153 of the book. This shows that $\C_{\cc S}^r$ and $\C_{\cc S}^\ell$ are isomorphic.
\end{s}

%

\begin{s}\label{durl} 
P.~159, Definition 7.3.1. Recall the definition: 

Let $\C$ be a $\U$-small category, let $\cc S$ be a right multiplicative system, and let $Q:\C\to\C_{\cc S}$ be the right the localization of $\C$ by $\cc S$. A functor $F:\C\to\A$ is said to be {\em right localizable} if $Q^\dagger F$ exists, in which case we say that $Q^\dagger F$ a {\em right localization} of $F$, and denote this functor by $R_{\cc S}F$. It might be worth displaying the formula 
%
\begin{equation}\label{rsfq}
(R_{\cc S}F)(Q(X))\simeq\colim_{(Q(Y)\to Q(X))\in\C_{Q(X)}}F(Y).
%(R_{\cc S}F)(Q(X))\simeq\colim_{(X\to Y)\in\cc S^X}F(Y).
\end{equation}
%
If this inductive limit is universal in the sense of Definition~\ref{uil} p.~\pageref{uil}, we say that $F$ is {\em universally} right localizable. \index{universally right localizable} By Theorem~\ref{233} p.~\pageref{233}, this is equivalent to saying that $Q^\dagger F$ exists universally in the sense of Definition~\ref{232} p.~\pageref{232}. 

Also, the following fact is implicit:

If $F(s)$ is an isomorphism for all $s$ in $\cc S$, then $F$ is universally right localizable and the functors $R_{\cc S}F$ and $F_{\cc S}$ are canonically isomorphic. (This is the case $\cc I=\C$ of Proposition 7.3.2 p.~160 of the book.)

The following conditions on the right localization $(\C_{\cc S},Q)$ of $\C$ are equivalent: 

\nn(a) $\id_\C$ is universally right localizable, 

\nn(b) any functor $F:\C\to\A$ is universally right localizable, 

\nn(c) any functor $F:\C\to\A$ is universally right localizable and satisfies 
$$
R_{\cc S}F\simeq F\circ R_{\cc S}\id_\C.
$$

\begin{df}[universal localization]\label{url2} 
Say that the right localization $(\C_{\cc S},Q)$ of $\C$ is {\em universal}\index{universal localization} if the above conditions are satisfied.
\end{df}
\end{s}

%

\begin{s}
P.~159, Definition of universally right localizable functor (Definition 7.3.1 (ii)): see \S\ref{durl} p.~\pageref{durl}. 
\end{s}

%

\begin{s}\label{732} 
P.~160, Proposition 7.3.2. If, in the setting of Proposition 7.3.2, any $t$ in $\cc T$ is an isomorphism, then the right localization $(\C_{\cc S},Q)$ of $\C$ is universal in the sense of Definition~\ref{url2}.

The following statement is easy to prove and implicit in the proof of Proposition 7.3.2. 

Let $\C$ be a category, let $\SSS$ a right multiplicative system, and let $F:\C\to\A$ be a functor such that $F(s)$ is an isomorphism for all $s$ in $\SSS$. Then $F$ is universally right localizable (see \S\ref{durl} p.~\pageref{durl}), $R_{\SSS}F\simeq F_{\SSS}$, and for any functor $K:\A\to\A'$ the diagram below commutes up to isomorphism
$$
\begin{tikzcd}
\C\ar{rr}{F}\ar{d}[swap]{Q}&&\A\ar{d}{K}\\
\C_{\SSS}\ar{rr}[swap]{(K\circ F)_{\SSS}}\ar{rru}[swap]{F_{\SSS}}&&\A'.
\end{tikzcd}
$$
\end{s}

%

\begin{s}
We paste Display (7.3.7) p.~161 of the book (see Formula~\eqref{rsfq} p.~\pageref{rsfq}):
%
\begin{equation}\label{rsfq2}
(R_{\cc S}F)(Q(X))\simeq\colim_{(X\to Y)\in\cc S^X}F(Y).
\end{equation}
%
In view of \S\ref{778} p.~\pageref{778}, this implies the following:

Let $X$ and $Y$ be two objects of $\C$. If $\cc S$ is a right multiplicative system in $\C$, then 
%
\begin{equation}\label{hcsr}%\label{hcsr}\label{hcsl}???
\Hom_{\C_{\cc S}^r}(X,Y)\simeq R_{\cc S}\big(\Hom_\C(\ \ ,Y)\big)(X).
\end{equation}
% 
Similarly, if $\cc S$ is a left multiplicative system in $\C$, then 
%
\begin{equation}\label{hcsl}
\Hom_{\C_{\cc S}^\ell}(X,Y)\simeq R_{\cc S}\big(\Hom_\C(X,\ \ )\big)(Y).
\end{equation}
% 
\end{s}

%%

\subsection{Proof of (7.4.3) (p.~162)}

Recall that $\SSS$ is a right multiplicative system in $\C$. We have the (non-commutative) diagram
$$
\begin{tikzcd}
\C\ar{rr}{F}\ar{d}[swap]{Q}\ar{dr}{\iota_\C}&&\A\ar{d}{\iota_\A}\\ 
\C_\SSS\ar{r}[swap]{\alpha_\SSS}&\Ind(\C)\ar{r}[swap]{IF}&\Ind(\A).
\end{tikzcd}
$$
Let $X$ be in $\C$. We must prove that there is an isomorphism 
$$
R_\SSS(\iota_\A\circ F)(Q(X))\simeq IF(\alpha_\SSS(Q(X)))
$$ 
in $\Ind(\C)$. Recall the following facts: 

\nn$\bu$ Proposition 7.4.1 p.~162 of the book implies
$$
A:=\alpha_\SSS(Q(X))=\colim_{(X',x')\in\SSS^X}\iota_\C(X')\in\Ind(\C).
$$ 
$\bu$ Display (7.3.7) p.~161 of the book implies
$$
B:=R_\SSS(\iota_\A\circ F)(Q(X))=\colim_{(X',x')\in\SSS^X}\iota_\A(F(X'))\in\Ind(\A).
$$
$\bu$ The definition of $IF$ p.~133 of the book implies
$$
C:=IF(A)=\colim_{(U,u)\in\C_A}\iota_\A(F(U))\in\Ind(\A).
$$

We want to prove $B\simeq C$.

\nn{\em Notation.} If $\alpha:I\to\B$ is a functor whose inductive limit is $X\in\B$, then we write $p(X,i):\alpha(i)\to X$ for the coprojection. (Of course this morphism depends on $\alpha$.) 

We shall define morphisms of functors 
$$
\begin{tikzcd}
B\ar[yshift=.7ex]{r}{f}&C\ar[yshift=-.7ex]{l}{g}.
\end{tikzcd}
$$

For $(X',x')$ in $\SSS^X$ we define $f(X',x'):\iota_\A(F(X'))\to C$ by 
$$
f(X',x'):=p(C,X',p(A,X',x')),
$$ 
and we claim that the morphisms $f(X',x')$ induce a morphism of functors $f:B\to C$. 

Let $(U,u)$ be in $\C_A$, that is
$$
u\in A(U)=\colim_{(X',x')\in\SSS^X}\Hom_\C(U,X').
$$ 
Choose $(X',x')$ in $\SSS^X$ and $f:U\to X'$ such that $u=p(A(U),X',x')(f)$, and put 
$$
g(U,u):=p(B,X',x')\circ\iota_\A(F(f)):\iota_\A(F(U))\to B.
$$ 
We claim that $g(U,u)$ does not depend on the choice of $X',x'$ and $f$; that the morphisms $g(U,u)$ induce a morphism of functors $g:C\to B$; and that $f$ and $g$ are mutually inverse. 

We leave the verification of these claims to the reader. 

%%%%

\section{About Chapter 8} 

\subsection{About Section 8.1}

Here is a comment about Lemma 8.1.2 (ii) p.~169. 

The fact that the notion of group object is independent of the choice of a universe $\U$ such that $\C$ is a $\U$-category is implicit in the proof. A way to make this point clear is to define the notion of a group object structure on an object $G$ of $\C$ without the Axiom of Universes. As in the book, we use the notation $G(X):=\Hom_\C(X,G)$. A group object structure on $G$ is given by a functorial family of maps 
$$
(\mu_X:G(X)^2\to G(X))_{X\in\C}
$$ 
such that 

\nn(a) $\mu_X$ is a group multiplication for all $X$ in $\C$, 

\nn(b) the map $G(Y)\to G(X)$ is a morphism of groups for all morphism $X\to Y$ in $\C$. 

%%%

\subsection{About Section 8.2}

\subsubsection{Definition 8.2.1 (p. 169)}

The proposition and lemma below are obvious. 
%
\begin{prop}\label{payp}
Let $\C$ be a pre-additive category, let $\A$ be the category of additive functors from $\C^{\op}$ to $\Mod(\bb Z)$, let $h:\C\to\A$ be the obvious functor satisfying $h(X)(Y)=\Hom_\C(Y,X)$ for all $X$ and $Y$ in $\C$, let $X$ be in $\C$ and $A$ in $\A$, and let 
$$
\begin{tikzcd}
\Hom_\A(h(X),A)\ar[yshift=.7ex]{r}{\Phi}&A(X)\ar[yshift=-.7ex]{l}{\Psi}
\end{tikzcd}
$$
be defined by 
$$
\Phi(\theta)=\theta_X(\id_X),\quad\Psi(x)(f)=A(f)(x).
$$
Then $\Phi$ and $\Psi$ are mutually inverse abelian group isomorphisms.
\end{prop}
%
\nn(See Theorem \ref{yol} p.~\pageref{yol}.)

\begin{conv}\label{payc}
In the above setting we denote $\A$ by $\C^\wedge$ and $h$ by $\hy_\C$. (This abuse is justified by Proposition~\ref{payp}.) We also use Definitions~\ref{ue} and \ref{ue2} p.~\pageref{ue} in this context. 
\end{conv} 

\begin{lem}\label{payl}
Let $\C$ and $\C'$ be pre-additive categories, let $\A$ be the category of additive functors from $\C$ to $\C'$, and let $\alpha:I\to\A$ be a functor such that $\colim(\alpha(X))$ exists in $\C'$ for all $X$ in $\C$. Then $\colim\alpha$ exists in $\A$ and satisfies 
$$
(\colim\alpha)(X)\simeq\colim(\alpha(X))
$$ 
for all $X$ in $\C$. (There is a similar statement for projective limits.)
\end{lem}

%%

\subsubsection{Lemma 8.2.3 (p. 169)}

Here is a statement contained in Lemma 8.2.3:

\begin{cor}\label{823}
Let $\C$ be a pre-additive category, let $X_1$ and $X_2$ be two objects of $\C$ such that the product $X=X_1\times X_2$ exists in $\C$, let $p_a:X\to X_a$ be the projection, and define $i_a:X_a\to X$ by 
$$
p_a\circ i_b=\begin{cases}\id_{X_a}&\text{if }a=b\\0&\text{if }a\not=b.\end{cases}
$$ 
Then $X$ is a coproduct of $X_1$ and $X_2$ by $i_1$ and $i_2$. Moreover we have 
$$
i_1\circ p_1+i_2\circ p_2=\id_{X_1\times X_2}.
$$
\end{cor}

Let us denote the object $X$ above by $X_1\oplus X_2$. The following lemma is implicit in the book. 

\begin{lem}
For $a=1,2$ let $f_a:X_a\to Y_a$ be a morphism in a pre-additive category $\C$. Assume that $X_1\oplus X_2$ and $Y_1\oplus Y_2$ exist in $\C$. Then we have 
$$
f_1\times f_2=f_1\sqcup f_2
$$ 
(equality in $\Hom_\C(X_1\oplus X_2,Y_1\oplus Y_2)$). 
\end{lem} 

We denote this morphism by $f_1\oplus f_2$.\medskip 

\begin{proof}
Put $X:=X_1\oplus X_2,\ Y:=Y_1\oplus Y_2$ and write 
$$
X_a\xr{i_a}X\xr{p_a}X_a,\quad Y_a\xr{j_a}Y\xr{q_a}Y_a
$$ 
for the projections and coprojections. We have $q_a\circ(f_1\times f_2)=f_a\circ p_a$ for all $a$, and we must show $q_b\circ (f_1\times f_2)\circ i_a=q_b\circ j_a\circ f_a$ for all $a,b$. This follows immediately from Corollary~\ref{823}.
\end{proof}

For the reader's convenience we state and prove Lemma 8.2.3 (ii) p.~169 of the book:

\begin{lem}[Lemma 8.2.3 (ii) p. 169]\label{823ii}
Let $\C$ be a pre-additive category; let $X,X_1,$ and $X_2$ be objects of $\C$; and, for $a=1,2$, let $X_a\xr{i_a}X\xr{p_a}X_a$ be morphisms satisfying 
$$
p_a\circ i_b=\delta_{ab}\ \id_{X_a},\quad i_1\circ p_1+i_2\circ p_2=\id_X.
$$
Then $X$ is a product of $X_1$ and $X_2$ by $p_1$ and $p_2$ and a coproduct of $X_1$ and $X_2$ by $i_1$ and $i_2$. 
\end{lem}

\begin{proof}
For any $Y$ in $\C$ we have 
$$
\Hom_\C(Y,p_a)\circ\Hom_\C(Y,i_b)=\delta_{ab}\ \id_{\Hom_\C(Y,X_a)},
$$ 
$$
\Hom_\C(Y,i_1)\circ\Hom_\C(Y,p_1)+\Hom_\C(Y,i_2)\circ\Hom_\C(Y,p_2)=\id_{\Hom_\C(Y,X)}.
$$ 
This implies that $\Hom_\C(Y,X)$ is a product of $\Hom_\C(Y,X_1)$ and $\Hom_\C(Y,X_2)$ by $\Hom_\C(Y,p_1)$ and $\Hom_\C(Y,p_2)$, and thus, $Y$ being arbitrary, that $X$ is a product of $X_1$ and $X_2$ by $p_1$ and $p_2$, and we conclude by applying this observation to the opposite category.
\end{proof}

Note also the following corollary to Lemma 8.2.3 (ii) (stated above as Lemma \ref{823ii}). 

\begin{cor}\label{823b}
Let $F:\C\to\C'$ be an additive functor of pre-additive categories; let $X,X_1,$ and $X_2$ be objects of $\C$; and, for $a=1,2$, let $X_a\xr{i_a}X\xr{p_a}X_a$ be morphisms such that $X$ is a product of $X_1$ and $X_2$ by $p_1,p_2$ and a coproduct of $X_1$ and $X_2$ by $i_1,i_2$. Then $F(X)$ is a product of $F(X_1)$ and $F(X_2)$ by $F(p_1),F(p_2)$ and a coproduct of $F(X_1)$ and $F(X_2)$ by $F(i_1),F(i_2)$. 
\end{cor}

%%

\subsubsection{Brief Comments}

\begin{s} 
P.~172, Lemma 8.2.9. Recall the statement:

\begin{lem}[Lemma 8.2.9 p. 172] 
Let $\C$ be a pre-additive category which admits finite products. Then $\C$ is additive.
\end{lem}

Let us check that $\C$ has a zero object. (This part of the proof is left to the reader by the authors.) 

Let $X$ and $Y$ be in $\C$. By Lemma 8.2.3 p.~169 of the book, the product $X\times Y$ is also a coproduct of $X$ and $Y$. Let us denote this object by $X\oplus Y$. Let $T$ be a terminal object of $\C$. For any $X$ in $\C$ we have a natural isomorphism $X\oplus T\simeq X$. In particular $T$ can be viewed as $T\sqcup T$ via the morphisms $T\xr0T\xl0T$. This implies $\Hom_\C(T,X)\simeq0$ for any $X$, and $T$ is a zero object. q.e.d.
\end{s}

%

\begin{s} 
P.~172, proof of Lemma 8.2.10. Recall the statement: 

If $\C$ is an additive category and $X$ is an object of $\C$, then $X$ is an abelian group object. 

The addition is given by the codiagonal morphism $\sigma:X\oplus X\to X$. This comment is only about the associativity of the addition. This associativity can also be proved as follows:

Put $X^n:=X\oplus\cdots\oplus X$ ($n$ factors), and let $X\xr{i_a}X^n\xr{\sigma_n}X$ be respectively the $a$-th coprojection and the codiagonal morphism. It clearly suffices to show that the composition 
$$
X^3\xr{\sigma_2\oplus X}X^2\xr{\sigma_2}X
$$ 
is equal to $\sigma_3$. This follows from the fact that the composition 
$$
X\xr{i_a}X^3\xr{\sigma_2\oplus X}X^2
$$ 
is equal to $i_b$ with 
$$
b=\begin{cases}1&\text{if }a=1,2\\2&\text{if }a=3.\end{cases}
$$ 
q.e.d.
\end{s}

%

\begin{s}
P.~173, Propositions 8.2.12, 8.2.13, and Theorem 8.2.14  (minor variant). 
\begin{nota}
If $\C$ and $\C'$ are categories admitting finite products, we denote by $\oo P(\C,\C')$ the category of those functors from $\C$ to $\C'$ which commute with finite products.
\end{nota}

\begin{prop}\label{8.2.12}
If $\C$ is an additive category, then the obvious functor 
$$
\Phi:\oo P(\C,\Mod(\mathbb Z))\to\oo P(\C,\Set)
$$ 
is an isomorphism, $\Phi^{-1}$ being given by Lemma 8.2.11 p.~172 of the book.
\end{prop}

\begin{proof}
The functor $\Phi$ being fully faithful by Proposition 8.2.12, it suffices to prove that the map 
$$
\oo{Ob}(\Phi):\oo{Ob}(\oo P(\C,\Mod(\mathbb Z)))\to\oo{Ob}(\oo P(\C,\Set))
$$ 
is bijective. The injectivity is obvious and the surjectivity follows from the proof of Proposition 8.2.13.
\end{proof}

Recall the statement of Theorem 8.2.14 p.~173 of the book:

\begin{thm}[Theorem 8.2.14 p. 173] 
Any additive category has a unique structure of a pre-additive category.
\end{thm}

\begin{proof}
Let $\C$ be our additive category. Thanks to Proposition~\ref{8.2.12} we identify $\oo P(\C,\Set)$ and $\oo P(\C,\Mod(\mathbb Z))$. We define the addition of $\Hom_\C(X,Y)$ for $X$ and $Y$ in $\C$ by evaluating the functor $\Hom_\C(X,\ )$ in $\oo P(\C,\Mod(\mathbb Z))$ on $Y$. The uniqueness is clear. If $f:Y\to Z$ is morphism in $\C$, then 
$$
\Hom_\C(X,f)=f\circ:\Hom_\C(X,Y)\to\Hom_\C(X,Z)
$$ 
is a morphism in $\Mod(\mathbb Z)$. If $g:W\to X$ is morphism in $\C$, then 
$$
\circ g:\Hom_\C(X,\ )\to\Hom_\C(W,\ )
$$ 
is a morphism in $\oo P(\C,\Mod(\mathbb Z))$, and 
$$
\circ g:\Hom_\C(X,Y)\to\Hom_\C(W,Y)
$$ 
is a morphism in $\Mod(\mathbb Z)$.
\end{proof}
\end{s}

%

\begin{s} 
P.~173, Proposition 8.2.15. Recall the setting: $F:\C\to\C'$ is a functor between additive categories, and the claim is: 
$$
F\text{ is additive }\ssi\ F\text{ commutes with finite products}.
$$ 

I think the authors forgot to prove Implication $\then$. Let us do it. It suffices to show that $F$ commutes with $n$-fold products for $n=0$ or $n=2$. 

Case $n=0$: Put $X:=F(0)$. We must prove $X\simeq 0$. The equality $0=1$ holds in the ring $\Hom_\C(X,X)$ because it holds in the ring $\Hom_\C(0,0)$. As a result, the morphisms $0\to X$ and $X\to 0$ are mutually inverse isomorphisms. 

Case $n=2$: Let $X_1,X_2$ be in $\C$. To check that the natural morphisms 
%
\begin{equation}\label{173} 
F(X_1\oplus X_2)\rightleftarrows F(X_1)\oplus F(X_2)
\end{equation} 
%
are mutually inverse isomorphisms, let $p_j:X_1\oplus X_2\to X_j$ and $i_j:X_j\to X_1\oplus X_2$ be the projections and coprojections, and apply Lemma 8.2.3 p.~169 of the book to the morphisms $p_j,i_j,F(p_j),F(i_j)$. q.e.d.
\end{s}

%%

\subsection{About Section 8.3}

\subsubsection{Proposition 8.3.4 (p. 176)}

Here are a few more details about the proof of Proposition 8.3.4. Recall the setting: We have a morphism $f:X\to Y$ in an abelian category $\C$. Let $P$ be the fiber product $X\times_YX$, let $p_a:P\to X$ be the projection, let $p$ be the morphism $p_1-p_2$ from $P$ to $X$, and consider the diagram 
$$
\begin{tikzcd}
\Ker f\ar{r}{h} &X\ar[equal]{d}\ar{r}{a}&\Coker h\\ 
P\ar{r}[swap]{p}&X\ar{r}[swap]{b}       &\Coker p\ar[equal]{r}&\Coim f,
\end{tikzcd}
$$ 
where $h,a,$ and $b$ are the natural morphisms. 

We claim $b\circ h=0$. Indeed, we define $c:\Ker f\to P$ by the condition $p_1\circ c=h,p_2\circ c=0$: 
$$
\begin{tikzcd}
{}&X\ar{dr}{f}\\ 
\Ker f\ar{ur}{h}\ar{dr}[swap]{0}\ar[dashed]{r}{c}&P\ar{u}{p_1}\ar{d}{p_2}&Y\\ 
{}&X\ar{ur}[swap]{f},
\end{tikzcd}
$$
and we get $b\circ h=b\circ p\circ c=0\circ c=0$. This proves the claim. We get a natural morphism $d:\Coker h\to\Coim f$ making the diagram 
$$
\begin{tikzcd}
\Ker f\ar{r}{h}&X\ar[equal]{d}\ar{r}{a}&\Coker h\ar{d}{d}\\ 
P\ar{r}[swap]{p}&X\ar{r}[swap]{b}&\Coim f
\end{tikzcd}
$$ 
commute. 

As $p$ factors through $h$, we have $a\circ p=0$, and we get a natural morphism $e:\Coim f\to\Coker h$ making the diagram 
$$
\begin{tikzcd}
\Ker f\ar{r}{h}&X\ar[equal]{d}\ar{r}{a}&\Coker h\\ 
P\ar{r}[swap]{p}&X\ar{r}[swap]{b}&\Coim f\ar{u}[swap]{e}
\end{tikzcd}
$$ 
commute. 

It is easy to see that $d$ and $e$ are mutually inverse isomorphisms. In short, there is a natural isomorphism $\Coker h\simeq\Coim f$ which makes the diagram
%
\begin{equation}\label{834a}
\begin{tikzcd}
\Ker f\ar{r}{h}&X\ar[equal]{d}\ar{r}{a}&\Coker h\ar[leftrightarrow]{d}{\sim}\\ 
P\ar{r}[swap]{p}&X\ar{r}[swap]{b}&\Coim f
\end{tikzcd}
\end{equation}
%
commute. 

Dually, let $S$ (for ``sum'') be the fiber coproduct $Y\oplus_XY$, let $i_a:Y\to S$ be the coprojection, let $i$ be the morphism $i_1-i_2$ from $Y$ to $S$, and consider the diagram 
%
$$
\begin{tikzcd}
\Ima f\ar{r}{a}&Y\ar[equal]{d}\ar{r}{i}&S\\ 
\Ker k\ar{r}{b}&Y\ar{r}[swap]{k}&\Coker f
\end{tikzcd}
$$ 
where $a,b,$ and $k$ are the natural morphisms. Then there is a natural isomorphism $\Ima f\simeq\Ker k$ which makes the diagram 
%
\begin{equation}\label{834b}
\begin{tikzcd}
\Ima f\ar[leftrightarrow]{d}[swap]{\sim}\ar{r}&Y\ar[equal]{d}\ar{r}{i}&S\\ 
\Ker k\ar{r}&Y\ar{r}[swap]{k}&\Coker f
\end{tikzcd}
\end{equation}
%
commute. Let us record these observations:
\begin{prop}\label{p834}
In the above setting there are natural isomorphisms 
$$
\Coker h\simeq\Coim f,\quad\Ima f\simeq\Ker k
$$ 
which make Diagrams \eqref{834a} and \eqref{834b} commute.
\end{prop}

Note that we can splice Diagrams \eqref{834a} and \eqref{834b}:
$$
\begin{tikzcd}
\Ker f\ar{r}{h} &X\ar[equal]{d}\ar{r}&\Coker h\ar[leftrightarrow]{d}{\sim}\\ 
P\ar{r}[swap]{p}&X\ar{r}             &\Coim f\ar{d}{\sim}\\ 
{}&{}&\Ima f\ar[leftrightarrow]{d}[swap]{\sim}\ar{r}&Y\ar[equal]{d}\ar{r}{i}&S\\ 
{}&{}&\Ker k\ar{r}&Y\ar{r}[swap]{k}&\Coker f.
\end{tikzcd}
$$ 

%%

\subsubsection{Definition 8.3.5 (p.~177)}

The following definitions and observations are implicit in the book. Let $\cc A$ be a subcategory of a pre-additive category $\cc B$, and let $\iota:\cc A\to \cc B$ be the inclusion. If $\cc A$ is pre-additive and $\iota$ is additive, we say that $\cc A$ is a {\em pre-additive subcategory} of $\cc B$. If in addition $\cc A$ and $\cc B$ are additive (resp. abelian), we say that $\cc A$ is {\em an additive (resp. abelian) subcategory} of $\cc B$. Now let $\cc A$ and $\cc B$ be categories. If $\cc B$ is pre-additive (resp. additive, abelian), then so is the category $\cc C:=\cc B^\cc A$ of functors from $\cc A$ to $\cc B$. Assume in addition that $\cc A$ is pre-additive. If $\cc B$ is pre-additive (resp. additive, abelian), then the full subcategory $\cc D:=\Ad(\cc A,\cc B)$ of $\cc C$ whose objects are the additive functors from $\cc A$ to $\cc B$ is a pre-additive (resp. additive, abelian) subcategory of $\cc C$.

%%

\subsubsection{The Complex (8.3.3) (p.~178)}

Let us just add a few more details about the proof of the isomorphisms
\begin{equation}\label{834}
\begin{split}
\Ima u\simeq\Coker(\pp:\Ima f\to\Ker g)\simeq\Coker(X'\to\Ker g)\\ 
\simeq\Ker(\psi:\Coker f\to\Ima g)\simeq\Ker(\Coker f\to X''),
\end{split}
\end{equation}
labeled (8.3.4) in the book. Recall that the underlying category $\C$ is abelian, and that the complex in question is denoted 
%
\begin{equation}\label{833}
X'\xrightarrow{f}X\xrightarrow{g}X''.
\end{equation}
%  
We shall freely use the isomorphism between image and coimage, as well as the abbreviations 
$$
K_v:=\Ker v,\quad K'_v:=\Coker v,\quad I_v:=\Ima v.
$$ 
Let us also write ``$A\overset{\sim}{\to}B$'' for ``the natural morphism $A\to B$ is an isomorphism''. 

Proposition \ref{p834} p.~\pageref{p834} can be stated as follows. 
%
\begin{prop}\label{p834b}
Let $f:X\to Y$ be a morphism, and consider the commutative diagram 
$$
\begin{tikzcd}
K_f\ar[tail]{rr}{h}&&X\ar{rr}{f}\ar[two heads]{dl}\ar[two heads]{dr}&&Y\ar[two heads]{rr}{k}&&K'_f\\ 
&K'_h\ar{rr}&&I_f\ar[tail]{ur}\ar{rr}&&K_k.\ar[tail]{ul}
\end{tikzcd}
$$ 
Then the bottom arrows are isomorphisms.
\end{prop}
%
Going back to our complex \eqref{833} p.~\pageref{833}, let us introduce the notation 
$$
\begin{tikzcd}
X'\ar{rrr}{f}\ar[equal]{d}&&&X\ar[equal]{d}\ar{rrr}{g}&&&X''\ar[equal]{d}\\ 
X'\ar[two heads]{r}{a}&I_f\ar[tail]{r}{\pp}&K_g\ar[equal]{d}\ar[tail]{r}{b}&X\ar[two heads]{r}{c}&K'_f\ar[equal]{d}\ar[two heads]{r}{\psi}&I_g\ar[tail]{r}{d}&X''\\ 
K_u\ar[tail]{rr}{e}&&K_g\ar[two heads]{dl}\ar[two heads]{dr}\ar{rr}{u}&&K'_f\ar[two heads]{rr}{h}&&K'_u\\ 
&K'_e\ar{rr}{\sim}[swap]{i}&&I_u\ar[tail]{ur}\ar{rr}{\sim}[swap]{j}&&K_h.\ar[tail]{ul}
\end{tikzcd}
$$ 
By Proposition~\ref{p834} p.~\pageref{p834} 
\begin{equation}\label{ijisos}
i\text{ and }j\text{ are isomorphisms.}
\end{equation}

We shall prove 
$$
\begin{tikzcd}
K'_{\pp\circ a}\ar{r}{k}[swap]{\sim}&K'_\pp\ar{r}{\ell}[swap]{\sim}&K'_e\ar{r}{i}[swap]{\sim}&I_u\ar{r}{j}[swap]{\sim}&K_h\ar{r}{m}[swap]{\sim}&K_\psi\ar{r}{n}[swap]{\sim}&K_{d\circ\psi}.
\end{tikzcd}
$$
This will imply \eqref{834} p.~\pageref{834}. 

The morphisms $k$ and $n$ are isomorphisms because $a$ is an epimorphism and $d$ a monomorphism. Thus, in view of \eqref{ijisos}, it only remains to prove that 
\begin{equation}\label{lmisos}
\ell\text{ and }m\text{ are isomorphisms.}
\end{equation}

There is a natural monomorphism from $I_f$ to $K_u$. Indeed, we have 
$$
u\circ\pp\circ a=c\circ f=0.
$$ 
As $a$ is an epimorphism, this implies $u\circ\pp=0$. 

It is easy to see that there is a natural monomorphisms from $K_u$ to $K_c$. By Proposition~\ref{p834} p.~\pageref{p834}, the composition $I_f\to K_c$ is an isomorphism. This implies $I_f\xr\sim K_u$. Similarly we prove $K'_u\xr\sim I_g$. 

We can thus complete our previous diagram as follows: 
$$
\begin{tikzcd}
X'\ar{rrr}{f}\ar[equal]{d}&&&X\ar[equal]{d}\ar{rrr}{g}&&&X''\ar[equal]{d}\\ 
X'\ar[two heads]{r}{a}&I_f\ar[dashed]{dl}[swap]{\sim}\ar[tail]{r}{\pp}&K_g\ar[equal]{d}\ar[tail]{r}{b}&X\ar[two heads]{r}{c}&K'_f\ar[equal]{d}\ar[two heads]{r}{\psi}&I_g\ar[tail]{r}{d}&X''\\ 
K_u\ar[tail]{rr}{e}&&K_g\ar[two heads]{dl}\ar[two heads]{dr}\ar{rr}{u}&&K'_f\ar[two heads]{rr}{h}&&K'_u\ar[dashed]{ul}[swap]{\sim}\\ 
&K'_e\ar{rr}{\sim}[swap]{i}&&I_u\ar[tail]{ur}\ar{rr}{\sim}[swap]{j}&&K_h.\ar[tail]{ul}
\end{tikzcd}
$$ 
(The two dashed arrows have been added.) Now \eqref{lmisos} is clear.

%%

\subsubsection{Brief Comments}

\begin{s} 
For the reader's convenience we state Lemma 8.3.11 p.~180. Consider the commutative square 
%
\begin{equation}\label{837}
\begin{tikzcd}
X'\ar{d}[swap]{g'}\ar{r}{f'}&Y'\ar{d}{g}\\ 
X\ar{r}[swap]{f}&Y
\end{tikzcd}
\end{equation} 
%
in the abelian category $\C$. 

\begin{lem}[Lemma 8.3.11 p.~180]\label{8311}
We have:

\nn\emph{(a)} Assume that \eqref{837} is cartesian. 

\emph{(i)} We have $\Ker f'\xr\sim\Ker f$. 

\emph{(ii)} If $f$ is an epimorphism, then \eqref{837} is cocartesian and $f'$ is an epimorphism.

\nn\emph{(b)} Assume that \eqref{837} is cocartesian.

\emph{(i)} We have $\Coker f'\xr\sim\Coker f$.

\emph{(ii)} If $f'$ is a monomorphism, then \eqref{837} is cartesian and $f$ is a
monomorphism.
\end{lem} 
\end{s} 

%

\begin{s}
P.~180, Lemma 8.3.12. Here is a minor variant:

\begin{lem}\label{8312}
For a complex $Z\to Y\to X$ in some abelian category, the following conditions are equivalent:

\nn{\em(a)} the complex is exact,

\nn{\em(b)} any commutative diagram of solid arrows
$$
\begin{tikzcd}
V\ar[dashed]{d}\ar[dashed, two heads]{r}&W\ar{dr}{0}\ar{d}\\ 
Z\ar{r}&Y\ar{r}&X
\end{tikzcd}
$$ 
can be completed as indicated ($V\to W$ being an epimorphism),

\nn{\em(c)} any commutative diagram of solid arrows
$$
\begin{tikzcd}
Z\ar{dr}[swap]{0}\ar{r}&Y\ar{d}\ar{r}&X\ar[dashed]{d}\\ 
{}&W\ar[tail,dashed]{r}&V
\end{tikzcd}
$$ 
can be completed as indicated ($W\to V$ being a monomorphism).
\end{lem}

\begin{proof}
Equivalence (a)$\ssi$(b) is proved in the book, and Equivalence (a)$\ssi$(c) follows by reversing arrows.
\end{proof}
\end{s}

%

\begin{s} 
Page 181, the Five Lemma (minor variant). 

\begin{thm}[Lemma 8.3.13 p.~181, Five Lemma] 
Consider the commutative diagram of complexes 
$$
\begin{tikzcd}
X^0\arrow[two heads]{d}[swap]{f^0}\arrow{r}{a^0}&
X^1\arrow[tail]{d}[swap]{f^1}\arrow{r}{a^1}&
X^2\arrow{d}{f^2}\arrow{r}{a^2}&
X^3\arrow[tail]{d}{f^3}\\ 
Y^0\arrow{r}[swap]{b^0}&
Y^1\arrow{r}[swap]{b^1}&
Y^2\arrow{r}[swap]{b^2}&
Y^3,
\end{tikzcd}
$$
where $f^0$ is an epimorphism, $f^1$ and $f^3$ are monomorphisms, and $X^1\to X^2\to X^3$ and $Y^0\to Y^1\to Y^2$ are exact. Then $f^2$ is a monomorphism. 
\end{thm} 

\begin{proof}
Note that Equivalence (a)$\ssi$(b) in Lemma~\ref{8312} p.~\pageref{8312} can be stated as follows: 

\nn$(*)\ f:X\to Y$ is an epimorphism if and only if any subobject of $Y$ is the image of some subobject of $X$. 

We write $fx$ for the image of a subobject $x$ of $X$, and $fg$ for $f\circ g$.

Put $x^2:=\Ker f^2$. Using $(*)$ we see that there is: 

\nn$\bu$ a subobject $x^1$ of $X^1$ such that $x^2=a^1x^1$ (because $f^3$ is a monomorphism, $f^3a^2x^2=0$, and $X^1\xr{a^1}X^2\xr{a^2}X^3$ is exact), 

\nn$\bu$ a subobject $y^0$ of $Y^0$ such that $f^1x^1=b^0y^0$ (because $b^1f^1x^1=0$ and $Y^0\xr{b^0}Y^1\xr{b^1}Y$ is exact), and 

\nn$\bu$ a subobject $x^0$ of $X^0$ such that $y^0=f^0x^0$ (because $f^0$ is an epimorphism). 

This yields 
$$
f^1a^0x^0=b^0f^0x^0=b^0y^0=f^1x^1,
$$
implying $a^0x^0=x^1$ (because $f^1$ is a monomorphism), and thus 
$$
0=a^1a^0x^0=a^1x^1=x^2.
$$ 
\end{proof}
\end{s}

%

\begin{s} 
P.~182, proof of the equivalence (iii)$\ssi$(iv) in Proposition 8.3.14. Here is the statement of the proposition:

\begin{prop}[Proposition 8.3.14 p. 182] 
Let $0\to X'\xr fX\xr gX''\to0$ be a short exact sequence in an abelian category $\C$. Then the conditions below are equivalent:
\begin{itemize}
\item[\em(i)] there exits $h:X''\to X$ such that $g\circ h=\id_{X''}$,
\item[\em(ii)] there exits $k:X\to X'$ such that $k\circ f=\id_{X'}$,
\item[\em(iii)] there exits $h:X''\to X$ and $k:X\to X'$ such that $\id_X=f\circ k+h\circ g$,
\item[\em(iv)] there exits $\pp=(k,g)$ and $\psi=(f,h)$ such that $X\xr\pp X'\oplus X''$ and $X'\oplus X''\xr\psi X$ are mutually inverse isomorphisms,
\item[\em(v)] for any $Y$ in $\C$, the map $\Hom_\C(Y,X)\xr{g\circ}\Hom_\C(Y,X'')$ is surjective,
\item[\em(vi)] for any $Y$ in $\C$, the map $\Hom_\C(X,Y)\xr{\circ f}\Hom_\C(X',Y)$ is surjective.
\end{itemize}
\end{prop}

The authors say that the equivalence (iii)$\ssi$(iv) is obvious. I agree, but here are a few more details. Implication (iv)$\then$(iii) is indeed obvious in the strongest sense of the word. Implication (iii)$\then$(iv) can be proved as follows. 

Assume (iii), that is, we have morphisms $h:X''\to X$ and $k:X\to X'$ such that 
%
\begin{equation}\label{fk+hg} 
f\circ k+h\circ g=\id_X.
\end{equation} 
% 
As $g\circ f=0$, this implies 
$$
g\circ h\circ g=g\circ f\circ k+g\circ h\circ g=g\circ\id_X=g. 
$$ 
Since $g$ is an epimorphism, this entails $g\circ h=\id_{X''}$. We prove similarly $k\circ f=\id_{X'}$. Let us record the two above equalities: 
% 
\begin{equation}\label{gh,kf} 
g\circ h=\id_{X''},\quad k\circ f=\id_{X'}.
\end{equation} 
% 
Now \eqref{fk+hg} and \eqref{gh,kf} imply 
$$
k\circ h=k\circ(f\circ k+h\circ g)\circ h=k\circ f\circ k\circ h+k\circ h\circ g\circ h=k\circ h+k\circ h,
$$ 
and thus 
%
\begin{equation}\label{kh} 
k\circ h=0, 
\end{equation} 
% 
and (iv) follows from \eqref{fk+hg},\eqref{gh,kf}, and \eqref{kh}. q.e.d.
\end{s}

%

\begin{s} 
P.~183. Here is an example showing that filtrant and cofiltrant small projective limits of $R$-modules are not exact in general: 
$$
\lim_{n\in\bb N}\big(\bb Z\to\bb Z/2^n\bb Z\to0\big)=\big(\bb Z\to\bb Z_2\to0\big).
$$
\end{s}

%

\begin{s}\label{gcsbc}
P.~186, Definition 8.3.24 (definition of a Grothendieck category\index{Grothendieck category}). By Lemma 8.3.9 p.~83 of the book, in a Grothendieck category $\U$-small filtrant inductive limits are stable by base change (Definition 2.2.6 p.~47 of the book, stated above as Definition~\ref{dsbc} p.~\pageref{dsbc}).
\end{s}

%

\begin{s} 
P.~186, Definition 8.3.24 (definition of a Grothendieck category). The condition that small filtrant inductive limits are exact is not automatic. I know no entirely elementary proof of this important fact. Here is a proof using a little bit of sheaf theory. To show that there is an abelian category where small filtrant inductive limits exist but are not exact, it suffices to prove that there is an abelian category $\C$ where small filtrant {\em projective} limits exist but are not exact. It is even enough to show that small products are not exact in $\C$. Let $X$ be a topological space, and let $U_0\supset U_1\supset\cdots$ be a decreasing sequence of open subsets whose intersection is a non-open closed singleton $\{a\}$. We can take for $\C$ the category of small abelian sheaves on $X$. To see this, let $G$ be the abelian presheaf over $X$ such that $G(U)$ is $\mathbb Z$ if $a$ is in $U$ and 0 otherwise, and, for each $n$ in $\mathbb N$, let $F_n$ be the abelian presheaf over $X$ such that $F_n(U)$ is $\mathbb Z$ if $U\subset U_n$ and 0 otherwise. These presheaves are easily seen to be sheaves. For each $n$ in $\mathbb N$ and each open set $U$ let $F_n(U)\to G(U)$ be the identity if $a$ is in $U\subset U_n$ and 0 otherwise. This family of morphisms defines, when $U$ varies, an epimorphism $\pp_n:F_n\epi G$. Put 
$$
F:=\prod_{n\in\mathbb N}F_n,\quad H:=\prod_{n\in\mathbb N}G,\quad\pp:=\prod_{n\in\mathbb N}\pp_n:F\to H.
$$ 
It suffices to show that the morphism $\pp(a):F(a)\to H(a)$ between the stalks at $a$ induced by $\pp$ is not an epimorphism. This is clear because $\pp(a)$ is the natural morphism 
$$
\bigoplus_{n\in\mathbb N}\mathbb Z\to\prod_{n\in\mathbb N}\mathbb Z.
$$
q.e.d.
\end{s}

%% 

\subsection{About Section 8.4} 

Here is a comment about Proposition 8.4.7 p.~187. 

Let us just rewrite in a slightly less concise way the part of the proof on p.~188 which starts with the sentence ``Define $Y:=Y_0\times_XG_i$'' at the fifth line of the last paragraph of the proof, and goes to the end of the proof. 

It suffices to show that there is a morphism $a_0:G_i\to Y_0$ satisfying $l_0\circ a_0=\pp$:
$$
\begin{tikzcd}
X'\ar{d}[swap]{h}\ar[tail]{r}{k_0}&Y_0\ar{dl}{g_0}\ar[tail]{r}{l_0}&X\\ 
Z&&G_i.\ar[dashed]{ul}{a_0}\ar{u}[swap]{\pp}
\end{tikzcd}
$$ 
Form the cartesian square 
$$
\begin{tikzcd}
Y\ar{r}{b}\ar[swap]{d}{c}&Y_0\ar[tail]{d}{l_0}\\
G_i\ar[swap]{r}{\pp}&X,
\end{tikzcd}
$$
and the cocartesian square 
$$
\begin{tikzcd}
Y\ar{r}{b}\ar[swap]{d}{c}&Y_0\ar{d}{\lambda}\\
G_i\ar[swap]{r}{a_1}&Y_1.
\end{tikzcd}
$$ 
Let $l_1:Y_1\to X$ be the morphism which makes the diagram 
$$
\begin{tikzcd}
{}&Y_0\ar{d}[swap]{\lambda}\ar{dr}{l_0}\\ 
Y\ar{ur}{b}\ar{dr}[swap]{c}&Y_1\ar[dashed]{r}{l_1}&X\\ 
{}&G_i\ar{u}{a_1}\ar{ru}[swap]{\pp}
\end{tikzcd}
$$ 
commutative. By Lemma \ref{8311} (a) (i) p.~\pageref{8311}, $c$ is a monomorphism, and, by Part (b) (ii) of the same lemma, $\lambda$ is also a monomorphism. As $Z$ is injective, there is a morphism $d:G_i\to Z$ satisfying $d\circ c=g_0\circ b$: 
$$
\begin{tikzcd}
Y\ar{r}{b}\ar[tail]{d}[swap]{c}&Y_0\ar{d}{g_0}\\ 
G_i\ar[dashed]{r}[swap]{d}&Z.
\end{tikzcd}
$$ 
By the definition of $Y_1$ there is a morphism $g_1:Y_1\to Z$ such that 
$$
\begin{tikzcd}
Y\ar{r}{b}\ar[swap]{d}{c}&Y_0\ar{d}[swap]{\lambda}\ar[bend left]{ddr}{g_0}\\
G_i\ar{r}{a_1}\ar[bend right]{rrd}[swap]{d}&Y_1\ar{dr}{g_1}\\ 
{}&{}&Z
\end{tikzcd}
$$ 
commutes. We get the commutative diagram
$$
\begin{tikzcd}
{}&Y_0\ar[equal]{d}\ar{rr}{l_0}&&X\ar[equal]{d}\\ 
X'\ar{d}[swap]{h}\ar[tail]{r}{k_0}&Y_0\ar{dl}[swap]{g_0}\ar[tail]{r}{\lambda}&Y_1\ar{dll}{g_1}\ar[tail]{r}{l_1}&X\\ 
Z&&&G_i.\ar{u}[swap]{\pp}\ar{lll}{d}\ar{ul}{a_1}
\end{tikzcd}
$$ 
As $\lambda$ is an isomorphism by maximality of $(Y_0,g_0,l_0)$, we can set $a_0:=\lambda^{-1}\circ a_1$, and we get 
$$
l_0\circ a_0=l_0\circ\lambda^{-1}\circ a_1=l_1\circ\lambda\circ\lambda^{-1}\circ a_1=l_1\circ a_1=\pp.
$$ 
q.e.d.
 
% https://docs.google.com/document/d/1r1eBAIdNxQNO4uJ7N1Ki3ewyPQBoYIxLLq5I54ciVa8/edit

%% 

\subsection{About Section 8.5}

\subsubsection{Brief Comments}

\begin{s} P.~190, Proposition 8.5.5. It might be worth writing explicitly the formulas (for $X\in\Mod(R,\C)$):
$$
\Hom_{R^{\op}}(N,\Hom_\C(X,Y))\simeq
\Hom_\C\left(N\otimes_RX,Y\right),
$$
$$
\Hom_R(M,\Hom_\C(Y,X))\simeq
\Hom_\C\left(Y,\Hom_R(M,X)\right),
$$
$$
R^{\op}\otimes_RX\simeq X,
$$
$$
\Hom_R(R,X)\simeq X.
$$
One could also mention explicitly the adjunctions
$$
\begin{tikzcd}
\Mod(R^{\op})\ar[xshift=-0.7ex]{d}[swap]{-\otimes_RX}&&&
\Mod(R)^{\op}\ar[xshift=-0.7ex]{d}[swap]{\Hom_\C(-,X)}\\
\C\ar[xshift=0.7ex]{u}[swap]{\Hom_\C(X,-)}&&&\C,\ar[xshift=0.7ex]{u}[swap]{\Hom_R(-,X)}
\end{tikzcd}
$$
where, we hope, the notation is self-explanatory.
\end{s}

%

\begin{s} P.~191, proof of Theorem 8.5.8 (iii) (minor variant). Recall the statement: 

\begin{lem}\label{858iii}
Let $\C$ be a Grothendieck category, let $G$ be a generator, let $R$ be the ring $\operatorname{End}_\C(G)^{\op}$, put $\M:=\Mod(R)$, let $\pp:\C\to\M$ be the functor defined by $\pp(X):=\Hom_\C(G,X)$. Then $\pp$ is fully faithful. 
\end{lem}

\begin{proof}
Let $\psi:\M\to\C$ be the functor defined by $\psi(M):=G\otimes_RM$, let $\C_0$ be the full subcategory of $\C$ whose objects are the direct sums of finitely many copies of $G$, and let $\M_0$ be the full subcategory of $\M$ whose objects are the direct sums of finitely many copies of $R$. Then $\pp$ and $\psi$ induce mutually quasi-inverse equivalences 
$$
\begin{tikzcd}
\C_0\ar[yshift=.7ex]{r}{\pp_{{}_0}}&\M_0.\ar[yshift=-.7ex]{l}{\psi_{{}_0}}
\end{tikzcd}
$$ 
We can assume that $\C_0$ and $\M_0$ are small (in the sense of Definition~\ref{small} p.~\pageref{small}). If $\lambda:\C\to(\C_0)^\wedge$ and $\lambda':\M\to(\M_0)^\wedge$ are the obvious functors, then the diagram 
$$
\begin{tikzcd}
\C\ar{r}{\pp}\ar{d}[swap]{\lambda}&\M\ar{d}{\lambda'}\\
(\C_0)^\wedge\ar{r}[swap]{\widehat\pp_{{}_0}}&(\M_0)^\wedge
\end{tikzcd}
$$ 
quasi-commutes. The functors $\lambda$ and $\lambda'$ are fully faithful by \S\ref{gcsbc} p.~\pageref{gcsbc} and Theorem 5.3.6 p.~124 (stated above as Theorem~\ref{536} p.~\pageref{536}). As $\widehat\pp_{{}_0}$ is an equivalence (a quasi-inverse being $\widehat\psi_{{}_0}$), the proof is complete.
\end{proof}
\end{s}

%%

\subsubsection{Theorem 8.5.8 (iv) (p.~191)} 

Here is a minor variant of Step (a) of the proof of Theorem 8.5.8 (iv). Recall the statement: 

\begin{lem}
In the setting of Lemma~\ref{858iii}, assume that there is a finite set $F$, an epimorphism $R^F\epi M$ in $\M$, a small set $S$, and a monomorphism $M\rightarrowtail R^{\oplus S}$. Let $\psi:\M\to\C$ be the functor defined by $\psi(M):=G\otimes_RM$. Then $\psi(M)\to\psi(R^{\oplus S})$ is a monomorphism. 
\end{lem}

\begin{proof}
There is a finite subset $F'$ of $S$ such that $M\rightarrowtail R^{\oplus S}$ factors as 
$$
M\rightarrowtail R^{F'}\rightarrowtail R^{\oplus S}.
$$ 
As $R^{F'}$ is a direct summand of $R^{\oplus S}$, the morphism $\psi(R^{F'})\to\psi(R^{\oplus S})$ is a monomorphism. In other words, we may assume $S=F'$, and it suffices to check that $\psi(M)\to\psi(R^{F'})$ is a monomorphism, or, more explicitly, that 
%
\begin{equation}\label{fpsi}
f:\psi(M)\to G^{F'}\text{ is a monomorphism.}
\end{equation}

Applying the right exact functor $\psi$ to 
$$
R^F\epi M\rightarrowtail R^{F'},
$$
we get 
$$
\begin{tikzcd}
K\ar[tail]{r}{i}\ar[bend right]{rrr}{0}&G^F\ar[two heads]{r}{p}&\psi(M)\ar{r}{f}&G^{F'},
\end{tikzcd}
$$
where $K:=\Ker(f\circ p)$. Applying $\pp$ we obtain
$$
\begin{tikzcd}
\pp(K)\ar{r}{\pp(i)}\ar[bend right]{rrr}{0}&R^F\ar{r}{\pp(p)}&\pp(\psi(M))\ar{r}{\pp(f)}&R^{F'}.
\end{tikzcd}
$$
The commutative diagram
$$
\begin{tikzcd}
\pp(K)\ar[equal]{d}\ar{rrr}{0}&&&R^{F'}\ar[equal]{d}\\
\pp(K)\ar{r}{\pp(i)}&R^F\ar[equal]{d}\ar{r}{\pp(p)}&\pp(\psi(M))\ar{r}{\pp(f)}&R^{F'}\ar[equal]{d}\\
&R^F\ar{r}[swap]{a}&M\ar[tail]{r}[swap]{b}\ar{u}&R^{F'}.
\end{tikzcd}
$$ 
yields $b\circ a\circ\pp(i)=0$. As $b$ is a monomorphism, we get $a\circ\pp(i)=0$, and thus $\pp(p)\circ\pp(i)=0$. Since $\pp$ is faithful by Lemma~\ref{858iii} p.~\pageref{858iii}, this implies 
%
\begin{equation}\label{pi=0}
p\circ i=0.
\end{equation} 

Let us prove \eqref{fpsi}. Let $x:X\to\psi(M)$ be a morphism in $\C$ satisfying $f\circ x=0$. It suffices to prove 
%
\begin{equation}\label{x=0}
x=0.
\end{equation}
% 
As $p$ is an epimorphism, the diagram of solid arrows 
$$
\begin{tikzcd}
Y\ar[dashed]{d}[swap]{y}\ar[dashed, two heads]{r}{c}&X\ar{d}{x}\\ 
G^F\ar[two heads]{r}[swap]{p}&\psi(M)
\end{tikzcd}
$$ 
can be completed, by Lemma \ref{8311} (b) (i) p.~\pageref{8311}, to a commutative square as indicated, $c$ being an epimorphism. The commutative diagram of solid arrows 
$$
\begin{tikzcd}
{}&Y\ar[dashed]{dl}[swap]{z}\ar{d}{y}\ar[two heads]{r}{c}&X\ar{d}[swap]{x}\ar{dr}{0}\\ 
K\ar{r}[swap]{i}&G^F\ar[two heads]{r}[swap]{p}&\psi(M)\ar{r}[swap]{f}&G^F
\end{tikzcd}
$$ 
can in turn be completed to a commutative diagram as indicated, and we get 
$$
x\circ c=p\circ i\circ z=0 
$$  
by \eqref{pi=0}. As $c$ is an epimorphism, this implies successively \eqref{x=0}, \eqref{fpsi}, and the lemma. 
\end{proof}

%% 

\subsection{About Section 8.7}


P.~199, Lemma 8.7.4 (ii). This comment is about the claim that the natural functor $E:\cc D'_{\cc S}\to\C$ is an equivalence. I don't understand the proof of the faithfulness of $E$ given in the book. I think that it suffices, in view of Proposition 7.1.2 (i) p.~150 and Theorem 7.1.16 p.~155 of the book, to check that
%
\begin{equation}\label{l}
Q:\cc D'\to\C\text{ is a localization of }\cc D'\text{ by }\cc S.
\end{equation}
%
To prove \eqref{l}, one can apply Lemma~\ref{711} p.~\pageref{711} with $R:\C\to\cc D'$ defined by $R(X):=(0\to X)$. 

%% 

\subsection{About the Exercises} 

\subsubsection{Exercise 8.4 (p.~202)}

Recall the statement: 

Let $\C$ be an additive category and $\cc S$ a right multiplicative system. Prove that the localization $\C_{\cc S}$ is an additive category and $Q:\C\to\C_{\cc S}$ is an additive functor. 

It is easy to equip $\C_{\cc S}$ with a pre-additive structure making $Q$ additive. Then the result follows from Corollary~\ref{823b} p.~\pageref{823b}. 

The pre-additive structure on $\C_{\cc S}$ is described in a very detailed way at the beginning of the following text of Dragan Mili\v{c}i\'c:  
%
\begin{center}\href{http://www.math.utah.edu/~milicic/Eprints/dercat.pdf}{http://www.math.utah.edu/$\sim$milicic/Eprints/dercat.pdf}
\end{center} 

%%%

\subsubsection{Exercise 8.17 (p.~204)}\label{817} 

\paragraph{Preliminaries} 

\begin{lem} 
If  
%
\begin{equation}\label{e817}
X\xr fY\xr gZ 
\end{equation}
% 
are morphisms in an abelian category $\C$ (we do not assume $g\circ f=0$), then the commutative diagram 
$$
\begin{tikzcd}
\Ker(g\circ f)\ar{r}&X\ar{d}\ar{r}&\Ima(g\circ f)\ar[dashed]{dl}\ar{d}\ar{r}&0\\ 
0\ar{r}&\Ima g\ar{r}&Z\ar{r}&\Coker g
\end{tikzcd}
$$ 
of solid arrows, whose rows are exact sequences, can be completed as indicated. The situation can also be represented as follows: 
$$
\begin{tikzcd}
X\ar[two heads]{dd}\ar{rr}{f}&&Y\ar[two heads]{dl}\ar{dd}{g}\\ 
{}&\Ima g\ar[tail]{dr}\\ 
\Ima(g\circ f)\ar[dashed,tail]{ur}\ar[tail]{rr}&&Z.
\end{tikzcd}
$$ 
In particular $\Ima(g\circ f)\to\Ima g$ is a monomorphism. 
\end{lem} 

\begin{proof} 
We claim that the diagram of solid arrows 
$$
\begin{tikzcd}
{}&X\times_ZX\ar{r}{a}&X\ar{d}[swap]{f}\ar{r}{d}&\Coim(g\circ f)\ar{r}\ar[dashed]{ddl}{b}&0\\ 
{}&{}&Y\ar{d}[swap]{g}\\
0\ar{r}&\Ima g\ar{r}&Z\ar{r}[swap]{c}&Z\oplus_YZ,
\end{tikzcd}
$$ 
whose rows are exact sequences, can be completed as indicated. Indeed, the existence of $b$ follows from the equalities $g\circ f\circ a=g\circ0=0$. To prove the lemma, it is enough to check that $b$ factors through $\Ima g$, or, equivalently, that $c\circ b=0$. As $d$ is an epimorphism, the vanishing of $c\circ b$ is equivalent to the vanishing of $c\circ b\circ d$. But we have $c\circ b\circ d=c\circ g\circ f=0\circ f=0$. 
\end{proof}

\begin{lem}\label{817b1} 
If, in the setting of Lemma~\ref{817a}, $f$ is an epimorphism, then 
$$
\Ima(g\circ f)\to\Ima g
$$ 
is an isomorphism. 
\end{lem} 

\begin{proof} 
Consider the commutative square  
$$
\begin{tikzcd}
X\ar{d}\ar{r}{f}&Y\ar{d}{a}\\ 
\Ima(g\circ f)\ar{r}[swap]{b}&\Ima g,
\end{tikzcd}
$$ 
where $a$ and $b$ are the natural morphisms. As $f$ and $a$ are epimorphisms, so is $b$. 
\end{proof} 

%%

\paragraph{Exercise 8.17}

The exercise follows easily from Lemmas \ref{817a} and \ref{817b} below.

Let us denote the cokernel of any morphism $h:Y\to Z$ in any abelian category by $Z/\Ima h$. 

Recall that, by Proposition 8.3.18 p.~183 of the book, an additive functor between abelian categories $F:\C\to\C'$ is left exact if and only if
%
\begin{equation}\label{sex1}
\left.
\begin{matrix}
0\to X'\xr fX\xr gX''\text{ exact }\\ 
\then\\ 
0\to F(X')\overset{F(f)\ }{\longrightarrow}F(X)\overset{F(g)\ }{\longrightarrow}F(X'')\text{ exact}
\end{matrix}
\right\}
\end{equation}

Consider the condition
%
\begin{equation}\label{sex2}
\left.
\begin{matrix}
0\to X'\xr fX\xr gX''\to0\text{ exact }\\ 
\then\\ 
0\to F(X')\overset{F(f)\ }{\longrightarrow}F(X)\overset{F(g)\ }{\longrightarrow}F(X'')\text{ exact}
\end{matrix}
\right\}
\end{equation}

\begin{lem}\label{817a}
We have (\ref{sex1})$\ssi$(\ref{sex2}). 
\end{lem}

\begin{proof}
Implication $\then$ is clear. To prove $\si$, let 
$$
0\to X'\xr fX\xr gX''
$$
be exact. We must check that 
%
\begin{equation}\label{0fx'fxfx''}
0\to F(X')\to F(X)\to F(X'')
\end{equation} 
% 
is exact. Let $I$ be the image of $g$. The sequence 
$$ 
0\to X'\to X\to I\to0
$$ 
being exact, so is 
%
\begin{equation}\label{0fx'fxfi}
0\to F(X')\to F(X)\to F(I).
\end{equation} 
% 
This implies that \eqref{0fx'fxfx''} is exact at $F(X')$. The sequence 
$$ 
0\to I\to X''\to X''/I\to0
$$  
being exact, so is 
$$ 
0\to F(I)\to F(X''),
$$  
and we have 
%
\begin{equation}\label{kfxfx'}
\Ker\big(F(X)\to F(X'')\big)\simeq\Ker\big(F(X)\to F(I)\big).
\end{equation} 
% 
The exactness of \eqref{0fx'fxfi} implies 
%
\begin{equation}\label{kfxfi}
\Ker\big(F(X)\to F(I)\big)\simeq\Ima\big(F(X')\to F(X)\big), 
\end{equation} 
% 
and the exactness of \eqref{0fx'fxfx''} at $F(X)$ follows from \eqref{kfxfx'} and \eqref{kfxfi}. 
\end{proof} 

Consider the conditions below on our additive functor $F:\C\to\C'$:
%
\begin{equation}\label{ex1}
\left.
\begin{matrix}
0\to X'\xr fX\xr gX''\to0\text{ exact }\\ 
\then\\ 
0\to F(X')\overset{F(f)\ }{\longrightarrow}F(X)\overset{F(g)\ }{\longrightarrow}F(X'')\to0\text{ exact}
\end{matrix}
\right\}
\end{equation}
%
\begin{equation}\label{ex2}
\left.
\begin{matrix}
X'\xr fX\xr gX''\text{ exact }\\ 
\then\\ 
F(X')\overset{F(f)\ }{\longrightarrow}F(X)\overset{F(g)\ }{\longrightarrow}F(X'')\text{ exact}
\end{matrix}
\right\}
\end{equation}

\begin{lem}\label{817b}
We have (\ref{ex1})$\ssi$(\ref{ex2}).
\end{lem}

\begin{proof}
Implication $\si$ is clear. To prove $\then$, let 
$$
X'\xr fX\xr gX''
$$
be exact. We must show that 
%
\begin{equation}\label{fx'fxfx''}
F(X')\to F(X)\to F(X'')
\end{equation} 
% 
is exact. Let $K_g,K_f$ and $I_g$ denote the indicated kernels and image. The sequence 
$$ 
0\to I_g\to X''\to X''/I_g\to 0
$$ 
being exact, so is 
$$ 
0\to F(I_g)\to F(X''), 
$$ 
and we get 
%
\begin{equation}\label{kfxfx''}
\Ker\big(F(X)\to F(X'')\big)\simeq\Ker\big(F(X)\to F(I_g)\big). 
\end{equation} 
% 
The sequence 
$$ 
0\to K_g\to X\to I_g\to 0
$$ 
being exact, so is 
$$ 
F(K_g)\to F(X)\to F(I_g), 
$$ 
and we get 
%
\begin{equation}\label{kfxfig}
\Ker\big(F(X)\to F(I_g)\big)\simeq\Ima\big(F(K_g)\to F(X)\big). 
\end{equation} 
% 
The sequence 
$$ 
0\to K_f\to X'\to K_g\to 0 
$$ 
being exact, so is 
$$ 
F(X')\to F(K_g)\to0,  
$$ 
and the isomorphism 
%
\begin{equation}\label{ifkg}
\Ima\big(F(K_g)\to F(X)\big)\simeq\Ima\big(F(X')\to F(X)\big)  
\end{equation} 
% 
results from Lemma~\ref{817b1} p.~\pageref{817b1} with $F(X')\to F(K_g)\to F(X)$ instead of \eqref{e817} p.~\pageref{e817}. The exactness of \eqref{fx'fxfx''} follows from \eqref{kfxfx''}, \eqref{kfxfig}, and \eqref{ifkg}.
\end{proof} 

%%%%

\section{About Chapter 9}

I find Chapter 9 especially beautiful!

\subsection{Brief Comments}

\begin{s}
P.~217, beginning of Section 9.2. 

\begin{prop}\label{ppifil}
Let $\pi$ be an infinite cardinal. The following conditions on a small category $I$ are equivalent:

\nn{\em(a)} For any small category $J$ with $\card(\Mor(J))<\pi$ and any functor 
$$
\alpha:I\times J^{\op}\to\Set
$$ 
the natural map 
$$
\colim_{i\in I}\lim_{j\in J}\alpha(i,j)\to\lim_{j\in J}\colim_{i\in I}\alpha(i,j)
$$ 
is bijective.

\nn{\em(b)} For any small category $J$ with $\card(\Mor(J))<\pi$ and any functor 
$$
\alpha:I\times J^{\op}\to\Set
$$ 
the natural map 
$$
\colim_{i\in I}\lim_{j\in J}\alpha(i,j)\to\lim_{j\in J}\colim_{i\in I}\alpha(i,j)
$$ 
is surjective.

\nn{\em(c)} The following conditions hold:

{\em(c1)} for any $A\subset\Ob(I)$ such that $\card(A)<\pi$ there is a $j$ in $J$ such that for any $a$ in $A$ there is a morphism $a\to j$ in $I$,

{\em(c2)} for any $i$ and $j$ in $I$ and for any $B\subset\Hom_I(i,j)$ such that $\card(B)<\pi$ there is a morphism $j\to k$ in $I$ such that the composition $i\xr sj\to k$ does not depend on $s\in B$.

\nn{\em(d)} For any small category $J$ such that $\card(\Mor(J))<\pi$ and any functor $\pp:J\to I$ there is an $i$ in $I$ such that $\lim\Hom_I(\pp,i)\neq\varnothing$. 
\end{prop}

\begin{proof}
Implications (c) $\ssi$ (d) $\then$ (a) are proved in Proposition 9.2.1 p.~217 and Proposition 9.2.9 p.~219 of the book. Implication (a)$\then$(b) is obvious. The proof of Implication (b)$\then$(d) is the same as the proof of Implication (b)$\then$(a) in Theorem 3.1.6 p.~74 of the book.
\end{proof}

\begin{df}[$\pi$-filtrant category] 
Let $\pi$ be an infinite cardinal and $I$ a small category. Then $I$ is $\pi$-{\em filtrant}\index{$\pi$-filtrant} if (and only if) the equivalent conditions of Proposition~\ref{ppifil} are satisfied.
\end{df}
\end{s}

%

\begin{s}\label{922}
P.~218. One can make the following observation after Definition 9.2.2: If $I$ admits inductive limits indexed by categories $J$ such that $\card(\Mor(J))<\pi$, then $I$ is $\pi$-filtrant.

\begin{proof}
For $\pp:J\to I$ we have
$$
\lim\Hom_\C(\pp,\colim\pp)\xleftarrow\sim\Hom_\C(\colim\pp,\colim\pp)\neq\varnothing.
$$
\end{proof}
\end{s}

%

\begin{s}
P.~218, Lemma 9.2.5. 

\begin{lem}[Lemma 9.2.5 p.~218] 
Let $\pp:J\to I$ be a cofinal functor. If $J$ is $\pi$-filtrant, so is $I$.
\end{lem}

Clearly, $I$ satisfies conditions (a) and (b) in Proposition~\ref{ppifil} p.~\pageref{ppifil}. 
\end{s}

%

\begin{s}
For the reader's convenience we state and prove Proposition 9.2.10 p. 220. 

\begin{prop}[Proposition 9.2.10 p. 220]
If $\C$ admits small $\pi$-filtrant inductive limits, if $J$ is a category satisfying $\card(\Mor(J))<\pi$, if $\beta:J\to\C_\pi$ is a functor, and if $\colim\beta$ exists in $\C$, then it belongs to $\C_\pi$. 
\end{prop}

\begin{proof}
Let $\alpha:I\to\C$ be a functor with $I$ small and $\pi$-filtrant, and consider the commutative diagram
$$
\begin{tikzcd}
\colim_i\Hom_\C(\colim_j\beta(j),\alpha(i))\ar{r}{a}\ar{d}{b}[swap]{\sim}&\Hom_\C(\colim_j\beta(j),\colim_i\alpha(i))\ar{dd}{e}[swap]{\sim}\\ 
\colim_i\lim_j\Hom_\C(\beta(j),\alpha(i))\ar{d}{c}[swap]{\sim}\\ 
\lim_j\colim_i\Hom_\C(\beta(j),\alpha(i))\ar{r}{d}[swap]{\sim}&\lim_j\Hom_\C(\beta(j),\colim_i\alpha(i)).
\end{tikzcd}
$$ 
The morphisms $b$ and $e$ are isomorphisms for obvious reasons. The morphism $c$ is an isomorphism because of our assumptions about $I$ and $J$. The morphism $d$ is an isomorphism because $\beta(j)$ is in $\C_\pi$ for all $j$. Thus, the morphism $a$ is an isomorphism. 
\end{proof}
\end{s}

%

\begin{s} 
P.~220, proof of Corollary 9.2.11. 

\begin{cor}[Corollary 9.2.11 p.~220] 
If $\C$ admits small inductive limits and if $X$ is an object of $\C$, then $\C_\pi$ and $(\C_\pi)_X$ are $\pi$-filtrant. 
\end{cor}

This follows from \S\ref{922}.
\end{s}

%

\begin{s} 
P.~222, Proposition 9.2.17, proof of implication (ii)$\then$(i). I suspect that the argument of the book is better than the one given here, but, unfortunately, I don't understand it. Here is a less concise wording:

Recall the setting: $\C$ is a category admitting inductive limits indexed by any category $J$ such that $\card(\Mor(J))<\pi$, and $A$ is in $\Ind(\C)$. Conditions (i) and (ii) are as follows: 

\nn(i) $\C_A$ is $\pi$-filtrant, 

\nn(ii) for any category $J$ such that $\card(\Mor(J))<\pi$ and any functor $\pp:J\to\C$, the natural map $A(\colim\pp)\to\lim A(\pp)$ is surjective. 

To prove (ii)$\then$(i), let $J$ be a category satisfying 
$$
\card(\Mor(J))<\pi,
$$ 
and let $\psi:J\to\C_A$ be a functor. We must find a $\xi$ in $\C_A$ satisfying 
$$
\lim\Hom_{\C_A}(\psi,\xi)\neq\varnothing.
$$ 
Let $\pp:J\to\C$ be the composition of $\psi$ with the forgetful functor $\C_A\to\C$, and write 
$$
\psi(j)=\left(\pp(j),\pp(j)\xr{y_j}A\right)\in\C_A.
$$ 
In particular the family $(y_j)$ belongs to $\lim A(\pp)$. Our assumption about $\C$ implies that $\colim\pp$ exists in $\C$. Let $p_j:\pp(j)\to\colim\pp$ be the coprojection. By surjectivity of the map $A(\colim\pp)\to\lim A(\pp)$ in (ii), there is an $x:\colim\pp\to A$ such that $x\circ p_j=y_j$ for all $j$. Setting 
$$
\xi:=\left(\colim\pp,\colim\pp\xr x A\right)\in\C_A,
$$ 
and letting $f_j:\psi(j)\to\xi$ be the obvious morphism, we get $(f_j)\in\lim\Hom_{\C_A}(\psi,\xi)$. q.e.d. 
\end{s}

%%

\subsection{Section 9.3 (pp 223--228)}\label{s934}

Here is a slightly different wording. 

\subsubsection{Conditions (9.3.1) (p. 223)}\label{931}

Recall Conditions (9.3.1) of the book: $\C$ is a category satisfying  

(i) $\C$ admits small inductive limits,

(ii) $\C$ admits finite projective limits,

(iii) small filtrant inductive limits are exact, 

(iv) there exists a generator $G$,

(v) epimorphisms are strict.

\subsubsection{Summary of Section 9.3}

The main purpose of Section 9.3 of the book is to prove Corollaries 9.3.7 and 9.3.8 p.~228 of the book, and these corollaries could be stated immediately after Conditions (9.3.1) above. For the reader's convenience we recall the definition of a regular cardinal and state Corollary 9.3.7:

\begin{df}[regular cardinal]\label{rc} 
A cardinal $\pi$ is \emph{regular}\index{regular cardinal} if for any family of sets $(B_i)_{i\in I}$ we have 
$$
\card(I)<\pi,\quad\card(B_i)<\pi\ \forall\ i\quad\then\quad\card\left(\bigsqcup_iB_i\right)<\pi.
$$
\end{df}

\begin{cor}[Corollary 9.3.7 p. 228]
Assume (9.3.1). Then for any small subset $S$ of $\Ob(\C)$ there exists an infinite cardinal $\pi$ such that $S\subset\Ob(\C_\pi)$.
\end{cor}

We make a few comments about Corollary 9.3.8. Firstly, it would be simpler (I think) to replace $\cc S$ with $\C_\pi$ in the statement, since in the first sentence of the proof one sets $\cc S:=\C_\pi$. Secondly, in view of the way Theorem 9.6.1 p.~235 of the book is phrased, it would be better, even if it is a repetition, to incorporate Part~(iv) of Corollary 9.3.5 (which says that $\C_\pi$ is closed by finite projective limits) into Corollary 9.3.8. Then, Corollary 9.3.8 would read as follows:

\begin{cor}[Corollary 9.3.8 p.~227]\label{938}
Assume (9.3.1) and let $\kappa$ be a cardinal. Then there exists an infinite regular cardinal $\pi>\kappa$ such that 

\nn{\em(i)} $\C_\pi$ is essentially small,

\nn{\em(ii)} if $X\epi Y$ is an epimorphism and $X$ is in $\C_\pi$, then $Y$ is in $\C_\pi$,

\nn{\em(iii)} if $X\mono Y$ is a monomorphism and $Y$ in $\C_\pi$, then $X$ is in $\C_\pi$,

\nn{\em(iv)} $G$ is in $\C_\pi$,

\nn{\em(v)} for any epimorphism $f:X\epi Y$ in $\C$ with $Y$ in $\C_\pi$, there exists $Z$ in $\C_\pi$ and a monomorphism $g:Z\mono X$ such that $f\circ g:Z\to Y$ is an epimorphism,

\nn{\em(vi)} $\C_\pi$ is closed by inductive limits indexed by categories $J$ which satisfy $$\card(\Mor(J))<\pi,$$

\nn{\em(vii)} $\C_\pi$ is closed by finite projective limits.
\end{cor}

See also Theorem \ref{961} p. \pageref{961} below.

%

\subsubsection{Lemma 9.3.1 (p. 224)}

For the reader's convenience we state the lemma:

\begin{lem}[Lemma 9.3.1 p. 224]\label{l331}
Assume that Conditions (9.3.1) p.~223 of the book (see \S\ref{931} p.~\pageref{931}) hold, let $\pi$ be an infinite regular cardinal, let $I$ be a $\pi$-filtrant small category, let $\alpha:I\to\C$ be a functor, and let $\colim\alpha\to Y$ be an epimorphism in $\C$. Assume either $\card(Y(G))<\pi$ or $Y\in\C_\pi$. Then there is an $i_0$ in $I$ such that the obvious morphism $\alpha(i_0)\to Y$ is an epimorphism.
\end{lem}

The proof of Lemma~\ref{l331} uses twice the following lemma:

\begin{lem}\label{ppi} 
Let $\C$ be a category, let $\pi$ be an infinite cardinal, and let $\alpha:I\to\C$ be a functor admitting an inductive limit $X$ in $\C$. Assume that the coprojections $p_i:\alpha(i)\to X$ are monomorphisms, and consider the conditions below:

\nn{\em(a)} $I$ is $\pi$-filtrant and $X$ is $\pi$-accessible,

\nn{\em(b)} the identity of $X$ factors through the coprojection $p_i$ for some $i$,

\nn{\em(c)} the coprojection $p_i$ is an isomorphism for some $i$,

\nn{\em(d)} there is an $i$ in $I$ such that $\alpha(s):\alpha(i)\to\alpha(j)$ is an isomorphism for all morphism $s:i\to j$ in $I$.

\nn Then we have {\em(a)} $\then$ {\em(b)} $\ssi$ {\em(c)} $\then$ {\em(d)}. 
\end{lem}

\begin{proof}
This follows immediately from Exercise 1.7 p. 31 of the book.
\end{proof}

We give a slightly more detailed writing of the second sentence in Step (a) of the proof of Lemma 9.3.1 p.~224 of the book. This second sentence is 

``Set $S:=\colim_iY_i(G)\subset Y(G)$.'' 

Here is the rewriting: 

The coprojection $Y_i\to Y$ being a monomorphism, so is $Y_i(G)\to Y(G)$. As small filtrant inductive limits are exact in $\Set$ (Proposition 3.3.7 (iv) p. 83 of the book), $S:=\colim_iY_i(G)\to Y(G)$ is also a monomorphism.

%

\subsubsection{Proposition 9.3.2 (p. 224)}

\begin{prop}[Proposition 9.3.2 p.~224]\label{932} 
Let $\C$ be a category satisfying Conditions (9.3.1) of the book, conditions stated in Section~\ref{931} p.~\pageref{931} above. If $\pi$ is an infinite regular cardinal, if $A$ is in $\C$, and if 
$$
\card(A(G))<\pi,\quad\card\big(G^{\sqcup A(G)}(G)\big)<\pi,
$$ 
then $A$ is in $\C_\pi$.
\end{prop}

Here is a rewriting of the proof with a few more details:

\begin{proof}[Proof of Proposition~\ref{932}]${}$ 

\nn$\bu$ Step 1. Note that $\Set\ni S\mapsto G^{\sqcup S}\in\C$ is a well-defined covariant functor. Also note that $\card(G^{\sqcup S}(G))<\pi$ for any $S\subset A(G)$. Indeed, there are maps 
$$
S\to A(G)\to S
$$ 
whose composition is the identity. Hence, the composition 
$$
G^{\sqcup S}(G)\to G^{\sqcup A(G)}(G)\to G^{\sqcup S}(G)
$$ 
is the identity.

\nn$\bu$ Step 2. Let $I$ be a small $\pi$-filtrant category, let $(X_i)_{i\in I}$ be an inductive system in $\C$, and let $X$ be its inductive limit. Claim~\ref{lbij} below will imply Proposition~\ref{932}. 

\begin{claim}\label{lbij} 
The map 
$$
\lambda_A:\colim_{i\in I}\Hom_\C(A,X_i)\to\Hom_\C(A,X).
$$ 
is bijective. 
\end{claim}

\begin{claim}\label{linj} 
The map $\lambda_A$ is injective. 
\end{claim} 

\begin{proof}[Proof of Claim~\ref{linj}] 
(We shall only use $\card(A(G))<\pi$.) Suppose that $f,g:A\parar X_{i_0}$ have same image in $\Hom_\C(A,X)$. This just means that the two compositions 
$$
A\parar X_{i_0}\to X
$$ 
coincide. We must show that $f$ and $g$ have already same image in 
$$
\colim_{i\in I}\Hom_\C(A,X_i),
$$ 
that is, we must show that there is a morphism $s_1:i_0\to i_1$ in $I$ such that the two compositions $A\parar X_{i_0}\to X_{i_1}$ coincide. For each $s:i_0\to i$, set 
$$
N_s:=\Ker(A\parar X_i).
$$ 
By Corollary 3.2.3 (i) p.~79 of the book, $I^{i_0}$ is filtrant and the forgetful functor $I^{i_0}\to I$ is cofinal. One of our assumption, namely Condition (9.3.1) (iii) in Section~\ref{931} p.~\pageref{931}, says that small filtrant inductive limits are exact in $\C$. In particular, $\colim_{s\in I^{i_0}}$ is exact in $\C$, and we get  
$$
\colim_{s\in I^{i_0}}N_s\simeq\Ker\left(A\parar\colim_{s\in I^{i_0}}X_i\right)\simeq\Ker(A\parar X)\simeq A. 
$$ 
As $\card(A(G))<\pi$ by assumption, Lemma~\ref{l331} p.~\pageref{l331} implies that there is a morphism $s_1:i_0\to i_1$ in $I$ such that $N_s\to A$ is an epimorphism. Hence, the two compositions $A\parar X_{i_0}\to X_{i_1}$ coincide, as was to be shown. This proves Claim~\ref{linj}. 
\end{proof} 

It only remains, in order to prove Proposition~\ref{932}, to check that $\lambda_A$ is surjective. Let $f:A\to X$ be a morphism. Claim~\ref{pig=f} below will imply the surjectivity of $\lambda_A$, and thus the truth of Proposition~\ref{932}. 

\begin{claim}\label{pig=f} 
There is a morphism $g:A\to X_i$ such that $p_i\circ g=f$, where $p_i:X_i\to X$ is the coprojection.
\end{claim} 

\nn$\bu$ Step 3. Consider the following conditions: 

\nn(a) the diagram of solid arrows 
$$
\begin{tikzcd}
B\ar[dashed,two heads]{r}\ar[dashed]{d}&A\ar{d}{f}\\ 
X_{i_0}\ar{r}[swap]{p_{i_0}}&X
\end{tikzcd}
$$
can be completed to a commutative diagram as indicated (the morphism $B\to A$ being an epimorphism),

\nn(b) the diagram of solid arrows 
%
\begin{equation}\label{fa}
\begin{tikzcd}
C\ar[dashed,two heads]{r}{a}\ar[dashed]{d}[swap]{x}&A\ar{d}{f}\\ 
X_{i_0}\ar{r}[swap]{p_{i_0}}&X
\end{tikzcd}
\end{equation}
%
can be completed to a commutative diagram as indicated, with $\card(C(G))<\pi$ (the morphism $C\to A$ being an epimorphism).

We shall show that (a) holds, that (a) implies (b), and that (b) implies Claim \ref{pig=f}, and thus Proposition~\ref{932} p.~\pageref{932}. 

\nn$\bu$ Step 4: (a) holds. For each $i$ in $I$ define $Y_i:=A\times_XX_i$. As $\colim_i$ is exact in $\C$, we have $\colim_iY_i\simeq A$. As $\card(A(G))<\pi$, Lemma 9.3.1 p.~224 of the book (stated above as Lemma~\ref{l331} p.~\pageref{l331}) implies that there is an $i_0$ in $I$ such that $B:=Y_{i_0}\to A$ is an epimorphism.

\nn$\bu$ Step 5: (a) implies (b). Assuming (a), we build the commutative square 
%
\begin{equation}\label{s5a}
\begin{tikzcd}
B\ar[two heads]{r}\ar{d}&A\ar{d}{f}\\ 
X_{i_0}\ar{r}[swap]{p_{i_0}}&X,
\end{tikzcd}
\end{equation}
% 
and we put $S:=\Ima(B(G)\to A(G))\subset A(G)$ and $C:=G^{\sqcup S}$, so that we have maps $B(G)\to S\to A(G)$. By Step~1 this implies $\card(C(G))<\pi$. The vertical arrows of the commutative diagram 
%
\begin{equation}\label{s5b}
\begin{tikzcd}
G^{\sqcup B(G)}\ar{r}\ar{d}&C\ar{r}&G^{\sqcup A(G)}\ar{d}\\ 
B\ar[two heads]{rr}&&A
\end{tikzcd}
\end{equation} 
%
being epimorphisms by Proposition 5.2.3 (iv) p.~118 of the book, so is $C\to A$. From the commutative diagram 
$$
\begin{tikzcd}
S\ar[equal]{rr}\ar[equal]{d}&&S\ar[equal]{d}\\ 
S\ar{r}&B(G)\ar{r}&S,
\end{tikzcd}
$$ 
we get, by Step 1, the commutative diagram 
%
\begin{equation}\label{s5c}
\begin{tikzcd}
C\ar[equal]{rr}\ar[equal]{d}&&C\ar[equal]{d}\\ 
C\ar{r}&G^{\sqcup B(G)}\ar{r}&C.
\end{tikzcd}
\end{equation} 
%
Splicing \eqref{s5a}, \eqref{s5b}, and \eqref{s5c} gives 
$$
\begin{tikzcd}
C\ar[equal]{rr}\ar[equal]{d}&&C\ar[equal]{d}\\ 
C\ar{r}\ar{rdd}[swap]{x}&G^{\sqcup B(G)}\ar{r}\ar{d}&C\ar[two heads]{d}{a}\\ 
{}&B\ar{r}\ar{d}&A\ar{d}\\ 
{}&X_{i_0}\ar{r}[swap]{p_{i_0}}&X.
\end{tikzcd}
$$ 
This proves (b).

\nn$\bu$ Step 6: (b) implies Claim~\ref{pig=f} p.~\pageref{pig=f}, and thus Proposition~\ref{932} p.~\pageref{932}. Assuming (b), form the cartesian square 
$$
\begin{tikzcd}
P\ar{r}\ar{d}&C\ar[two heads]{d}{a}\\ 
C\ar[two heads]{r}[swap]{a}&A.
\end{tikzcd}
$$
Epimorphisms in $\C$ being strict, the sequence $P\parar C\xr aA$ is exact. As 
$$
P(G)\le\card(C(G))^2<\pi,
$$ 
Claim~\ref{linj} implies that the natural map 
$$
\lambda_P:\colim_{i\in I}\Hom_\C(P,X_i)\to\Hom_\C(P,X)
$$ 
is injective. Consider the commutative diagram 
$$
\begin{tikzcd}
P\ar[yshift=0.7ex]{r}\ar[yshift=-0.7ex]{r}&C\ar{r}{a}\ar{d}[swap]{x}&A\ar{d}{f}\\ 
{}&X_{i_0}\ar{r}[swap]{p_{i_0}}&X.
\end{tikzcd}
$$ 
As $\lambda_P$ is injective, and as the compositions $P\parar C\xr xX_{i_0}\xr{p_{i_0}}X$ are equal, there is a morphism $s:i_0\to i$ such that the compositions $P\parar C\xr xX_{i_0}\xr{X_s} X_i$ are equal. The exactness of $P\parar C\xr aA$ implies the existence of a morphism $g:A\to X_i$ such that 
%
\begin{equation}\label{ga}
X_s\circ x=g\circ a.
\end{equation} 
% 

\begin{proof}[Proof of Claim~\ref{pig=f}]
It suffices to show that the above morphism $g$ satisfies $f=p_i\circ g$. Consider the diagram 
$$
\begin{tikzcd}
P\ar[yshift=0.7ex]{r}\ar[yshift=-0.7ex]{r}&C\ar{d}[swap]{x}\ar[two heads]{rr}{a}&&A\ar{dl}[swap]{g}\ar{d}{f}\\ 
{}&X_{i_0}\ar{r}[swap]{X_s}\ar[equal]{d}&X_i\ar{r}[swap]{p_i}&X\ar[equal]{d}\\ 
{}&X_{i_0}\ar{rr}[swap]{p_{i_0}}&&X.
\end{tikzcd}
$$ 
We have 
%
\begin{align*}
f\circ a&=p_{i_0}\circ x&\text{by \eqref{fa} p.~\pageref{fa}}\\ 
&=p_i\circ X_s\circ x\\ 
&=p_i\circ g\circ a&\text{by \eqref{ga} p.~\pageref{ga}}.
\end{align*}
%
As $a$ is an epimorphism, this forces $f=p_i\circ g$, and the proof of Claim~\ref{pig=f} is complete. 
\end{proof} 

As already indicated, Claim~\ref{pig=f} implies Proposition~\ref{932} p.~\pageref{932}. 
\end{proof}  

%%

\subsubsection{Definition of two infinite regular cardinals}\label{tirg}

(See (9.3.4) p. 226 of the book. We shall modify slightly the definition of $\pi_1$.) Let $\C$ be a category satisfying Conditions (9.3.1) in Section~\ref{931} p.~\pageref{931} above. Let $\pi_0$ be an infinite regular cardinal such that 
$$
\card\big(G(G)\big)<\pi_0,\quad\card\big(G^{\sqcup G(G)}(G)\big)<\pi_0.
$$ 
Now choose a cardinal $\pi_1\ge\pi_0$ such that we have for all set $A$ with $\card(A)<\pi_0$: 

$\card\big(G^{\sqcup A}(G)\big)<\pi_1$, 

if $X$ is a quotient of $G^{\sqcup A}$, then $\card\big(X(G)\big)<\pi_1$. 

\nn(Since the set of quotients of $G^{\sqcup A}$ is small by Proposition 5.2.9 p.~121 of the book, such a cardinal $\pi_1$ exists.) In the sequel of Section~\ref{s934} we assume 

\begin{cond}\label{ass}
Conditions (i)--(v) of Section~\ref{931} p.~\pageref{931} hold; $\pi_0$ and $\pi_1$ are as above; and $\pi$ is the successor of $2^{\pi_1}$.
\end{cond}

\nn The cardinals $\pi$ and $\pi_0$ satisfy 

(a) $\pi$ and $\pi_0$ are infinite regular cardinals,

(b) $G$ is in $\C_{\pi_0}$,

(c) $\pi'^{\pi_0}<\pi$ for any $\pi'<\pi$, 

(d) if $X$ is a quotient of $G^{\sqcup A}$ with $\card(A)<\pi_0$, then $\card\big(X(G)\big)<\pi$, 

(e) if $A$ is a set with $\card(A)<\pi_0$, then $\card\big(G^{\sqcup A}(G)\big)<\pi$.

\nn Condition~(a) holds because $\pi_0$ is infinite regular by assumption, and $\pi_0$ is infinite regular by Statement~(iv) p.~217 of the book. Condition~(b) holds by Proposition \ref{932} p.~\pageref{932}. Condition~(c) is proved as follows: if $\pi'<\pi$, then $\pi'\le2^{\pi_1}$ and 
$$
\pi'^{\pi_0}\le(2^{\pi_1})^{\pi_0}=2^{\pi_0\pi_1}=2^{\pi_1}<\pi.
$$ 
Conditions (d) and (e) are clear. 

\subsubsection{Lemma 9.3.3 (p. 226)}

We state Lemma 9.3.3 for the reader's convenience:

\begin{lem}[Lemma 9.3.3 p.~226]\label{933}
If Condition \ref{ass} holds, if $A$ is a set of cardinal $<\pi$, and if $X$ is a quotient of $G^{\sqcup A}$, then $\card(X(G))<\pi$.
\end{lem}

The beginning of the proof of Lemma 9.3.3 in the book uses implicitly the following two lemmas, which we prove for the sake of completeness.

\begin{lem}\label{cardipi}
If $A$ and $\pi$ are as above, and if $I:=\{B\subset A\,|\,\card(B)<\pi\}$, then we have $\card(I)<\pi$. 
\end{lem}

\begin{lem}\label{ord}
If $\alpha$ is a cardinal, then the cardinal of the set of those cardinals $\beta$ such that $\beta<\alpha$ does not exceed $\alpha$.
\end{lem}

\begin{proof}[Proof of Lemma~\ref{ord}]
Recall that a subset $S$ of an ordered set $X$ is a {\em segment} if $x<s\in S$ with $x\in X$ implies $x\in S$. In particular $X_{<x}$ (obvious notation) is a segment of $X$ for any $x$ in $X$. We take for granted the following well-known facts:

\nn$\bu$ every set can be well-ordered,

\nn$\bu$ if $T$ is a set of two non-isomorphic well-ordered sets, then there is a unique triple $(W_1,W_2,S)$ such that $T=\{W_1,W_2\}$ and $S$ is a proper segment of $W_2$ isomorphic to $W_1$,

\nn$\bu$ if $W$ is a well-ordered set, then the assignment $w\mapsto W_{<w}$ is an isomorphism of well-ordered sets from $W$ onto the set of proper segments of $W$.

Let $A$ be a well-ordered set of cardinal $\alpha$, and, for each cardinal $\beta$ with $\beta<\alpha$, let $B$ be a well-ordered set of cardinal $\beta$. Then $B$ is isomorphic to $A_{<a}$ for a unique $a$ in $A$, and the map $\beta\mapsto a$ is injective.
\end{proof}

\begin{proof}[Proof of Lemma~\ref{cardipi}] 
Putting $\alpha:=\card(A)$ we have
$$
\card(I)=\sum_{\pi'<\pi_0}\ \binom{\alpha}{\pi'}\le\sum_{\pi'<\pi_0}\ \alpha^{\pi_0}<\pi,
$$ 
the last inequality following from Lemma~\ref{ord}, (c), and (a). 
\end{proof}

\subsubsection{Theorem 9.3.4 (p. 227)}

\begin{thm}[Theorem 9.3.4 p.~227]\label{934}
Assume Condition~\ref{ass} p.~\pageref{ass} holds and let $X$ be an object of $\C$. Then we have 
$$
X\in\C_\pi\ssi\card(X(G))<\pi.
$$ 
\end{thm}

\begin{proof}[Proof of Theorem \ref{934}]${}$

\nn$\then:$ We prove this implication as in the book. For the reader's convenience we reproduce the argument: Set $I:=\{A\subset X(G)\ |\ \card(A)<\pi\}$. By Example 9.2.4 p.~218 of the book, $I$ is $\pi$-filtrant. We get the morphisms 
$$
G^{\sqcup A}\to G^{\sqcup X(G)}\to X
$$ 
for $A$ in $I$, and 
$$
\colim_{A\in I}G^{\sqcup A}\xr\sim G^{\sqcup X(G)}\to X.
$$ 
Then we see that $G^{\sqcup X(G)}\to X$ is an epimorphism by Proposition 5.2.3 (iv) p.~118 of the book, that $G^{\sqcup A}\to X$ is an epimorphism for some $A$ in $I$ by Lemma~\ref{l331} p.~\pageref{l331}, and that $\card(X(G))<\pi$ by Lemma~\ref{933} p.~\pageref{933}.

\nn$\si:$ In view of Proposition~\ref{932} p.~\pageref{932}, it suffices to prove  

\begin{equation}\label{934b}
\card\big(G^{\sqcup X(G)}(G)\big)<\pi.
\end{equation} 

To verify this inequality, we argue as in the proof of Lemma 9.3.3 p.~226 of the book (stated on p.~\pageref{933} above as Lemma~\ref{933}). (Conditions (b), (c), and (e) referred to below are stated in Section~\ref{tirg} p.~\pageref{tirg}.)

Let $I$ be the ordered set of all the subsets of $X(G)$ whose cardinal is $<\pi_0$. Then $I$ is $\pi_0$-filtrant by Example 9.2.4 p.~218 of the book, and we have 
$$
G^{\sqcup X(G)}\simeq\colim_{B\in I}G^{\sqcup B}.
$$ 
As $G$ is $\pi_0$-accessible by (b), we get 
$$
G^{\sqcup X(G)}(G)\simeq\colim_{B\in I}\ G^{\sqcup B}(G).
$$ 
By Lemma~\ref{cardipi} p.~\pageref{cardipi} we have $\card(I)<\pi$. Since $\card(G^{\sqcup B}(G))<\pi$ for all $B$ in $I$ by (e), this implies \eqref{934b}.
\end{proof}

\subsubsection{Brief Comments}

\nn$*$ P.~227, Corollary 9.3.5. In the proof of (i) we use Propositions 5.2.3 (iv) p.~118 and 5.2.9 p.~121 of the book. As already pointed out, in the proof of (iv), $\C$ should be $\C_\pi$. 

\nn$*$ P.~228, Corollary 9.3.6. As already pointed out, $\ilim$ in the statement should be $\sigma_\pi$. As for the proof, Conditions (i), (ii), and (iii) of Proposition 9.2.19 p.~223 of the book follow respectively from (9.3.1) (i) (see (i) at the beginning of Section~\ref{s934} p.~\pageref{s934}), (9.3.4) (b) (see (b) right after Condition~\ref{ass} p.~\pageref{ass}), and Corollary 9.3.5 (i) p.~227 of the book. 

\nn$*$ P.~228, Corollary 9.3.7. As $\{\card(X(G))\,|\,X\in S\}$ is a small set of cardinals, we may assume in Condition~\ref{ass} p.~\pageref{ass} that we have $\pi>\card(X(G))$ for all $X$ in $S$, and apply Theorem~\ref{934} p.~\pageref{934}. 

\nn$*$ P.~228, Corollary 9.3.8. The proof uses implicitly Proposition 5.2.3 (iv) p.~118 of the book and Example 9.2.4 p.~218 of the book. 

%%%

\subsection{Quasi-Terminal Object Theorem \index{Quasi-Terminal Object Theorem}}

Recall the following result:

\begin{thm}[Zorn's Lemma]\label{zorn}
If $X$ is an ordered set such that each well-ordered subset of $X$ has an upper bound, then $X$ has a maximal element.
\end{thm}

The purpose of this section is to prove a common generalization of Theorem~\ref{zorn} above and of Theorem 9.4.2 p.~229 of the book, stated below as Theorem~\ref{942}. We start with a reminder:

\begin{df}[Definition 9.4.1 p.~228, quasi-terminal object] 
An object $X$ of a category $\C$ is {\em quasi-terminal}\index{quasi-terminal object} if any morphism $u:X\to Y$ admits a left inverse.
\end{df}

\begin{thm}[Theorem 9.4.2 p.~229]\label{942} 
Any essentially small nonempty category admitting small filtrant inductive limits has a quasi-terminal object.
\end{thm}

Here is a weakening of the notion of inductive limit:

\begin{df}[small well-ordered upper bounds] 
Let $I$ be a nonempty well-ordered small set and $\alpha:I\to\C$ a functor. An {\em upper bound} for $\alpha$ is a morphism of functors $a:\alpha\to\Delta(X)$. (As usual, $\Delta(X)$ is the functor with constant value $X$.) If $\C$ has the property that any such functor admits some upper bound, we say that $\C$ {\em admits small well-ordered upper bounds}\index{small well-ordered upper bounds}. 
\end{df}

\begin{df}[special well-ordered small set] 
Let $\C$ be a category. A nonempty well-ordered small set $I$ is $\C$-{\em special}\index{special well-ordered small set} if it has no largest element and if, for any functor $\alpha:I\to\C$, there is some upper bound $(a_i:\alpha(i)\to X)_{i\in I}$ and some element $i_0$ in $I$ such that $a_{i_0}$ is an epimorphism. 
\end{df}

Our goal is to prove:

\begin{thm}[Quasi-Terminal Object Theorem]\label{qtot}
If $\C$ is a nonempty essentially small category $\C$ admitting small well-ordered upper bounds and a $\C$-special well-ordered set, then $\C$ has a quasi-terminal object.
\end{thm}

Theorem~\ref{qtot} clearly implies Zorn's Lemma (Theorem~\ref{zorn}). Lemma~\ref{945} below will show that Theorem~\ref{942} follows also from Theorem~\ref{qtot}. Theorem~\ref{942} will be used in the book to prove Theorem 9.5.5 p.~233.

The proof of Theorem~\ref{qtot} is essentially the same as the proof of Theorem~\ref{942} given in the book. For the reader's convenience, we spell out the details. 

\begin{lem}\label{943}
If $\C$ is a nonempty small category admitting small well-ordered upper bounds, then there is an $X$ in $\C$ such that, for all morphism $X\to Y$, there is a morphism $Y\to X$.
\end{lem}

\begin{proof}
Let $\cc F$ be the set of well-ordered subcategories of $\C$. For $I$ and $J$ in $\cc F$ we decree that $I\le J$ if and only if $I$ is an initial segment of $J$. This order is clearly inductive. Let $\cc S$ be a maximal element of $\cc F$. As $\cc S$ is small, it admits an upper bound $(a_S:S\to X)_{S\in\cc S}$. 

We shall prove that $X$ satisfies the conditions in the statement. Let $u:X\to Y$ be a morphism in $\C$. 

\nn(i) The object $Y$ is in $\cc S$. Otherwise, we can form the well-ordered subcategory $\widetilde{\cc S}$ of $\C$ by appending the element $Y$ to $\cc S$ and making it the largest element of $\widetilde{\cc S}$, the morphism $S\to Y$ being $u\circ a_S$. We have $\widetilde{\cc S}\in\cc F$ and $\cc S<\widetilde{\cc S}$, contradicting the maximality of $\cc S$. 

\nn(ii) As $Y$ is in $\cc S$, there is a morphism $Y\to X$, namely $a_Y$.
\end{proof} 

\begin{df}[Property~$(P)$]\label{pofa} 
We say that a morphism $a:A\to B$ in a given category has \emph{Property}~$(P)$ if for any morphism $b:B\to C$ there is a morphism $c:C\to B$ satisfying $c\circ b\circ a=a$.  
\end{df}  

\begin{lem}[Sublemma 9.4.4 p. 229]\label{944} 
If $\C$ is a small nonempty category admitting small well-ordered upper bounds, and if $X$ is an object of $\C$, then there is a morphism $f:X\to Y$ having Property~$(P)$.
\end{lem}

\begin{proof}
The category $\C^X$ is again nonempty, small, and admits small well-ordered upper bounds, so that Lemma~\ref{943} applies to it. Let $f:X\to Y$ be to $\C^X$ what $X$ is to $\C$ in Lemma~\ref{943}. Then it is easy to see that $f$ has Property~$(P)$. 
\end{proof}

We recall the notion of {\em construction by transfinite induction\index{transfinite induction}}. 

\begin{thm}[Construction by Transfinite Induction]\label{meta} 
Let $\U$ be a universe, let $F:\U\to\U$ be a map, and let $I$ be a well-ordered $\U$-set. Then there is a unique pair $(S,f)$ such that $S$ is a set, $f:I\to S$ is a surjection, and we have 
$$
f(i)=F(f(j)_{j<i})
$$ 
for all $i$ in $I$, where $f(j)_{j<i}$ is viewed as a family of elements of $\{f(j)\,|\,j\in I,j<i\}$ indexed by $\{j\in I\,|\,j<i\}$. (In particular, $S$ is a $\U$-set.)
\end{thm}

\begin{proof}
Uniqueness: Assume that $(S,f)$ and $(T,g)$ have the indicated properties. It suffices to prove $f(i)=g(i)$ for all $i$ in $I$. Suppose this is false, and let $i$ be the least element of $I$ such that $f(i)\neq g(i)$. We have 
$$
f(i)=F(f(j)_{j<i})=F(g(j)_{j<i})=g(i),
$$ 
a contradiction. 

Existence: Recall that a subset $J$ of $I$ is called a {\em segment} if $I\ni i<j\in J$ implies $i\in J$. Let $Z$ be the set of all triples $(J,S_J,f_J)$, where $J$ is a segment of $I$, where $f:J\to S_J$ is a surjection, and where we have $f_J(j)=F(f_J(k)_{k<j})$ for all $j$ in $J$. Decree that 
$$
Z\ni(J,S_J,f_J)\le(K,S_K,f_K)\in Z
$$ 
if and only if $J$ is a segment of $K$. By the uniqueness part, $(Z,\le)$ is inductive. Let $(J,S_J,f_J)$ be a maximal element of $Z$. It suffices to assume that $J$ is a proper segment of $I$ and to derive a contradiction. Let $k$ be the minimum of $I\setminus J$, put 
$$
K:=J\cup\{k\},\quad f_K(j):=f_J(j)\ \forall\ j\in J,
$$
$$
f_K(k):=F(f_K(j)_{j<k}),\quad S_K:=S_J\cup\{f_K(k)\}.
$$ 
Then $(K,S_K,f_K)$ contradicts the maximality if $(J,S_J,f_J)$. 
\end{proof}

\begin{proof}[Proof of the Quasi-Terminal Object Theorem (Theorem \ref{qtot} p.~\pageref{qtot})] 
Let $\C$ be as in the statement. We assume (as we may) that $\C$ is small. Let us choose a $\C$-special well-ordered set $I$, and let us define an inductive system $(X_i)_{i\in I}$ by transfinite induction as follows: For the least element $0$ of $I$ we choose an arbitrary object $X_0$ of $\C$. Let $i>0$ and assume that $X_j$ and $u_{jk}:X_k\to X_j$ have been constructed for $k\le j<i$. 

\nn(a) If $i=j+1$ for some $j$, take $u_{ij}:X_j\to X_i$ with Property~$(P)$, and put $u_{ik}:=u_{ij}\circ u_{jk}$ for any $k\le j$. 

\nn(b) If $i=\sup\,\{j\,|\,j<i\}$, let $(a_j:X_j\to X_i)_{j<i}$ be some upper bound for $(X_j)_{j<i}$ and put $u_{ij}:=a_j$. 

(Recall that, by Definition~\ref{pofa} p.~\pageref{pofa}, the condition ``$u_{ij}:X_j\to X_i$ has Property~$(P)$'' means that for any morphism $b:X_i\to C$ there is a morphism $c:C\to X_i$ satisfying $c\circ b\circ u_{ij}=u_{ij}$. Recall also that such a $u_{ij}$ exists by Lemma~\ref{944} p.~\pageref{944}.) 

Then $(X_i)_{i\in I}$ is indeed an inductive system in $\C$. As $I$ is $\C$-special, there is an upper bound $(b_i:X_i\to X)_{i\in I}$ for $(X_i)_{i\in I}$, and there is an $i_0$ in $I$ such that $b_{i_0}:X_{i_0}\to X$ is an epimorphism. 

We claim that $X$ is quasi-terminal. Let $u:X\to Y$ be a morphism. It suffices to prove the claim below: 
%
\begin{claim}\label{cqt}
There is a morphism $v:Y\to X$ such that $v\circ u=\id_X$. 
\end{claim} 

Consider the morphisms 
$$
\begin{tikzcd}
X_{i_0}\ar{rr}{u_{i_0+1,i_0}}&&X_{i_0+1}\ar{rr}{u\circ b_{i_0+1}}&&Y.
\end{tikzcd}
$$
As $u_{i_0+1,i_0}$ has Property~$(P)$, there is a morphism $w:Y\to X_{i_0+1}$ satisfying  
%
\begin{equation}\label{wu}
w\circ u\circ b_{i_0+1}\circ u_{i_0+1,i_0}=u_{i_0+1,i_0}.
\end{equation}  
% 
Put
%
\begin{equation}\label{vb}
v:=b_{i_0+1}\circ w:Y\to X.
\end{equation}
%
It suffices to show that $v$ satisfies the equality $v\circ u=\id_X$ in Claim~\ref{cqt} p.~\pageref{cqt}. We have 
%
\begin{align*}
v\circ u\circ b_{i_0}&=b_{i_0+1}\circ w\circ u\circ b_{i_0}&\text{by \eqref{vb}}\\ 
&=b_{i_0+1}\circ w\circ u\circ b_{i_0+1}\circ u_{i_0+1,i_0}\\ 
&=b_{i_0+1}\circ u_{i_0+1,i_0}&\text{by \eqref{wu}}\\ 
&=b_{i_0}\\ 
&=\id_X\circ b_{i_0}. 
\end{align*}
%
As $b_{i_0}$ is an epimorphism, this implies $v\circ u=\id_X$, proving Claim~\ref{cqt} p.~\pageref{cqt}, and thus the Quasi-Terminal Object Theorem (Theorem \ref{qtot} p.~\pageref{qtot}). 
\end{proof}

Here is a diagrammatic illustration of the above computation:
$$
\begin{tikzcd}
X_{i_0}\ar[equal]{d}\ar{rr}{b_{i_0}}&&X\ar[equal]{d}\ar{r}{u}&Y\ar[equal]{d}\ar{rr}{v}&&X\ar[equal]{d}\\ 
X_{i_0}\ar[equal]{d}\ar{r}[swap]{u_{i_0+1,i_0}}&X_{i_0+1}\ar{r}[swap]{b_{i_0+1}}&X\ar{r}[swap]{u}&Y\ar{r}[swap]{w}&X_{i_0+1}\ar[equal]{d}\ar{r}[swap]{b_{i_0+1}}&X\ar[equal]{d}\\ 
X_{i_0}\ar[equal]{d}\ar{rrrr}[swap]{u_{i_0+1,i_0}}&&&&X_{i_0+1}\ar{r}[swap]{b_{i_0+1}}&X\ar[equal]{d}\\ 
X_{i_0}\ar{rrrrr}[swap]{b_{i_0}}&&&&&X.
\end{tikzcd}
$$ 

For the reader's convenience we state and prove Sublemma 9.4.5 p.~229 of the book.

\begin{lem}[Sublemma 9.4.5 p.~229]\label{945}
If $\C$ is a small nonempty category admitting small filtrant inductive limits, if $\pi$ is an infinite regular cardinal such that $\card(\Mor(\C))<\pi$, if $I$ is a $\pi$-filtrant small category, and if $(X_i)_{i\in I}$ is an inductive system in $\C$ indexed by $I$, then there is an $i_0$ in $I$ such that the coprojection $X_{i_0}\to\colim_iX_i$ is an epimorphism. 
\end{lem}

\begin{proof}
Set $X:=\colim_iX_i$ and let $a_i:X_i\to X$ be the coprojection. For any $Y$ in $\C$ let 
$$
b^Y_i:\Hom_\C(Y,X_i)\to\colim_j\Hom_\C(Y,X_j)
$$ 
be the coprojection, let $F(Y)$ be the image of the natural map 
$$
\colim_j\Hom_\C(Y,X_j)\to\Hom_\C(Y,X),
$$ 
and define $\pp^Y$ by the commutative diagram 
$$
\begin{tikzcd}
\colim_j\Hom_\C(Y,X_j)\ar[two heads]{r}{\pp^Y}&F(Y)\ar[hook]{r}&\Hom_\C(Y,X)\\ 
\Hom_\C(Y,X_i).\ar{u}{b^Y_i}\ar{ur}[swap]{\pp^Y_i:=a_i\circ}
\end{tikzcd}
$$ 

Claim: There is an $i_0$ in $I$ such that $\pp^Y_{i_0}:=a_{i_0}\circ:\Hom_\C(Y,X_{i_0})\to F(Y)$ is surjective for all $Y$ in $\C$.

As $\card(\Hom_\C(Y,X))<\pi$, we have $F(Y)\in\Set_\pi$ by Corollary 9.2.12 p.~221 of the book. By Lemma 9.3.1 p.~224 of the book (stated above as Lemma~\ref{l331} p.~\pageref{l331}), there is an $i_Y$ in $I$ such that 
$$
a_{i_Y}\circ:\Hom_\C(Y,X_{i_Y})\to F(Y)
$$ 
is surjective. As $\card(\{i_Y\,|\,Y\in\Ob(\C)\})<\pi$ and $I$ is $\pi$-filtrant, there is an $i_0$ in $I$ such that, for any $Y$ in $\C$, there is a morphism $i_Y\to i_0$. This implies the claim. 

Let $i$ be in $I$. In particular, $a_i=\pp^{X_i}_i(\id_{X_i})$ is in $F(X_i)$. As 
$$
\pp^{X_i}_{i_0}:=a_{i_0}\circ:\Hom_\C(X_i,X_{i_0})\to F(X_i)
$$ 
is surjective by the claim, there is a morphism $h_i:X_i\to X_{i_0}$ such that $a_{i_0}\circ h_i=a_i$. 

Let us show that $a_{i_0}:X_{i_0}\to X$ is an epimorphism. Let $f_1,f_2:X\parar Y$ be a pair of parallel arrows such that $f_1\circ a_{i_0}=f_2\circ a_{i_0}$. Then, for any $i$ in $I$, we have 
$$
f_1\circ a_i=f_1\circ a_{i_0}\circ h_i=f_2\circ a_{i_0}\circ h_i=f_2\circ a_i.
$$ 
This implies $f_1=f_2$.
\end{proof}

We give again a diagrammatic illustration of the above computation:
$$
\begin{tikzcd}
X_i\ar[equal]{d}\ar{rr}{a_i}&&X\ar[equal]{d}\ar{r}{f_1}&Y\ar[equal]{d}\\ 
X_i\ar[equal]{d}\ar{r}{h_i}&X_{i_0}\ar[equal]{d}\ar{r}{a_{i_0}}&X\ar{r}{f_1}&Y\ar[equal]{d}\\ 
X_i\ar[equal]{d}\ar{r}{h_i}&X_{i_0}\ar{r}{a_{i_0}}&X\ar[equal]{d}\ar{r}{f_2}&Y\ar[equal]{d}\\ 
X_i\ar{rr}[swap]{a_i}&&X\ar{r}[swap]{f_2}&Y.
\end{tikzcd}
$$ 

%%%

\subsection{Lemma 9.5.3 (p. 231)}

We give more details about the proof, but first let us recall the setting:

Let $\C$ be a $\U$-category, let $\C_0$ be a subcategory of $\C$, and assume 

\nn(9.5.2) (i) $\C_0$ admits small filtrant inductive limits and $\C_0\to\C$ commutes with them.

\nn(9.5.2) (ii) Any diagram of solid arrows
%
\begin{equation}\label{952ii}
\begin{tikzcd}
X\ar{d}[swap]{f}\ar{r}{u}&Y\ar[dashed]{d}{g}\\ 
X'\ar[dashed]{r}[swap]{u'}&Y',
\end{tikzcd}
\end{equation}
%
with $u$ in $\Mor(\C_0)$ and $f$ in $\Mor(\C)$, can be completed to a commutative diagram with dashed arrows $u'$ in $\Mor(\C_0)$ and $g$ in $\Mor(\C)$.

\begin{lem}[Lemma 9.5.3 p. 231]
If $X'$ is in $\C_0$, if $I$ is a small set, and if  
$$
(u_i:X_i\to Y_i)_{i\in I},\quad(f_i:X_i\to X')_{i\in I}
$$ 
are families of morphisms in $\C_0$ and $\C$ respectively, then there is a morphism $u'$ in $\C_0$ and a family $(g_i)_{i\in I}$ of morphisms in $\C$ such that $g_i\circ u_i=u'\circ f_i$ for all $i$:
$$
\begin{tikzcd}
X_i\ar{d}[swap]{f_i}\ar{r}{u_i}&Y_i\ar[dashed]{d}{g_i}\\ 
X'\ar[dashed]{r}[swap]{u'}&Y'.
\end{tikzcd}
$$ 
\end{lem}

\begin{proof}
We assume, as we may, that $I$ is nonempty, well-ordered, and admits a maximum $m$. Let $0$ be the least element of $I$. We shall complete the following Task $(T_i)$ by transfinite induction on $i\in I$ (see Theorem~\ref{meta} p.~\pageref{meta}): 

\nn[Beginning of the description of Task $(T_i)$.] Construct, for each $j\le i$ in $I$, a commutative diagram 
$$
\begin{tikzcd}
X_j\ar{rr}{u_j}\ar{d}[swap]{f_j}&&Y_j\ar[dashed]{d}{h_j}\\
X'\ar[dashed]{r}[swap]{v_j}&Y'_{<j}\ar[dashed]{r}[swap]{w_j}&Y'_j,
\end{tikzcd}
$$ 
with $v_j,w_j$ in $\Mor(\C_0)$, and construct, for each $(i,j,k)$ in $I^3$ with $i\ge j>k$, a commutative diagram 
$$
\begin{tikzcd}
X'\ar{r}{v_j}\ar[equal]{d}&Y'_{<j}\ar{r}{w_j}&Y'_j\\
X'\ar{r}[swap]{v_k}&Y'_{<k}\ar{r}[swap]{w_k}&Y'_k,\ar[dashed]{ul}[swap]{p_{jk}}
\end{tikzcd}
$$
with $p_{jk}$ in $\Mor(\C_0)$, in such a way that we have
%
\begin{equation}\label{pijk}
p_{ij}\circ w_j\circ p_{jk}=p_{ik}\quad\forall\quad i>j>k,
\end{equation}
%
\begin{equation}\label{w0}
w_0=\id_{Y'_0}.
\end{equation}
%
Here is a diagrammatic illustration of \eqref{pijk}: 
$$
\begin{tikzcd}
Y'_{<i}\ar[equal]{rr}&&Y'_{<i}\\
Y'_{<j}\ar{r}{w_j}&Y'_j\ar{ul}[swap]{p_{ij}}\\ 
{}&Y'_k.\ar{ul}{p_{jk}}\ar{uur}[swap]{p_{ik}}
\end{tikzcd}
$$

\nn[End of the description of Task $(T_i)$.] 

\nn[Beginning of the accomplishment of Task $(T_i)$ for all $i$.] To handle Task $(T_0)$, we define $Y'_0,v_0$, and $h_0$ by (9.5.2) (ii):
$$
\begin{tikzcd}
X_0\ar{d}[swap]{f_0}\ar{r}{u_0}&Y_0\ar[dashed]{d}{h_0}\\
X'\ar[dashed]{r}[swap]{v_0}&Y'_0,
\end{tikzcd}
$$ 
and we define $Y'_{<0}$ and $w_0$ by the commutative diagram
$$
\begin{tikzcd}
X_0\ar{rr}{u_0}\ar{d}[swap]{f_0}&&Y_0\ar{d}{h_0}\\
X'\ar{r}{v_0}\ar[equal]{d}&Y'_{<0}\ar{r}{w_0}\ar[equal]{d}&Y'_0\ar[equal]{d}\\
X'\ar{r}[swap]{v_0}&Y'_0\ar{r}[swap]{\id}&Y'_0.
\end{tikzcd}
$$

Let $i$ in $I$ satisfy $i>0$, and let us tackle Task $(T_i)$. 

We assume (as we may) that Task $(T_j)$ has already been achieved for $j<i$, {\em i.e.} that the $Y'_{<j},Y'_j,h_j,v_j,w_j$ have already been constructed for $j<i$, that the $p_{jk}$ have already been constructed for $k<j<i$, and that all these morphisms satisfy the required conditions. 

It suffices to define $Y'_{<i},Y'_i,h_i,v_i,w_i,$ and $p_{ij}$ for $j<i$, in such a way that the required conditions are still satisfied. 

For $k<j<i$ we define $u_{jk}:Y'_k\to Y'_j$ by
%
\begin{equation}\label{ujk}
u_{jk}:=w_j\circ p_{jk}.
\end{equation}
%
By \eqref{pijk} we have $u_{jk}\circ u_{k\ell}=u_{j\ell}$ for all $\ell<k<j<i$. In particular, 
%
\begin{equation}\label{y'j}
(Y'_j)_{j<i}\text{ is an inductive system in }\C_0.
\end{equation}
%
We denote its limit (which exists in $\C_0$ thanks to (9.5.2) (i)) by $Y'_{<i}$, and we write $p_{ij}$ for the coprojection $Y'_j\to Y'_{<i}$. We also put
%
\begin{equation}\label{vi}
v_i:=p_{i0}\circ v_0,
\end{equation}
%
and we define
$$
Y'_{<i}\xr{w_i}Y'_i\xleftarrow{h_i}Y_i
$$
by (9.5.2) (ii):
$$
\begin{tikzcd}
X_i\ar{d}[swap]{v_i\circ f_i}\ar{r}{u_i}&Y_i\ar[dashed]{d}{h_i}\\
Y'_{<i}\ar[dashed]{r}[swap]{w_i}&Y'_i,
\end{tikzcd}
$$
so that we have  
%
\begin{equation}\label{hiui2}
h_i\circ u_i=w_i\circ v_i\circ f_i. 
\end{equation} 
%
We must check 
%
\begin{equation}\label{pikwk}
p_{ik}\circ w_k\circ v_k=v_i\ \forall\ k<i,
\end{equation}
%
\begin{equation}\label{pijwj}
p_{ij}\circ w_j\circ p_{jk}=p_{ik}\ \forall\ k<j<i.
\end{equation} 
%
To prove \eqref{pikwk}, first note that we have 
$$v_k=p_{k0}\circ w_0\circ v_0
$$ 
by induction hypothesis, $w_0=\id_{Y'_0}$ by \eqref{w0}, and thus 
%
\begin{equation}\label{vk}
v_k=p_{k0}\circ v_0.
\end{equation}
%
We get 
% 
\begin{align*}
p_{ik}\circ w_k\circ v_k&=p_{ik}\circ w_k\circ p_{k0}\circ v_0&\text{by \eqref{vk}}\\ 
&=p_{i0}\circ v_0&\text{by \eqref{pijk}}\\ 
&=v_i&\text{by \eqref{vi}}.
\end{align*}
%
This proves \eqref{pikwk}. We have 
% 
\begin{align*}
p_{ij}\circ w_j\circ p_{jk}&=p_{ij}\circ u_{jk}&\text{by \eqref{ujk}}\\ 
&=p_{ik}&\text{by \eqref{y'j}}.
\end{align*}
%
This proves \eqref{pijwj}. 

Task $(T_i)$ has been performed for the specific $i$ we have been considering, and thus Task $(T_i)$ has been completed for all $i$ in $I$. [End of the accomplishment of Task $(T_i)$ for all $i$.]

In particular Task $(T_m)$, where, remember, $m$ is the maximum of $I$, has also been achieved. Putting $Y':=Y'_m$ and 
%
\begin{equation}\label{gi}
g_i:=u_{mi}\circ h_i\ \forall\ i<m,
\end{equation}
%
\begin{equation}\label{gm}
g_m:=h_m,
\end{equation}
%
\begin{equation}\label{u'}
u':=w_m\circ v_m,
\end{equation}
%
we get
% 
\begin{align*}
g_i\circ u_i&=u_{mi}\circ h_i\circ u_i&\text{by \eqref{gi}}\\ 
&=u_{mi}\circ w_i\circ v_i\circ f_i&\text{by \eqref{hiui2}}\\ 
&=w_m\circ p_{mi}\circ w_i\circ v_i\circ f_i&\text{by \eqref{ujk}}\\ 
&=w_m\circ v_m\circ f_i&\text{by \eqref{pikwk}}\\ 
&=u'\circ f_i&\text{by \eqref{u'}}
\end{align*}
%
for $i<m$, and 
% 
\begin{align*}
g_m\circ u_m&=h_m\circ u_m&\text{by \eqref{gm}}\\ 
&=w_m\circ v_m\circ f_m&\text{by \eqref{hiui2}}\\ 
&=u'\circ f_m&\text{by \eqref{u'}}.
\end{align*}
%
\end{proof}

%%

\subsection{Theorems 9.5.4 and 9.5.5 (pp 232-234)}\label{954955}

The purpose of this section is to give a combined statement of Theorems 9.5.4 and 9.5.5. 

Let $\C$ be a $\U$-category, let $\cc F\subset\C_0$ be subcategories of $\C$ such that $\cc F$ is essentially small (see Remark~\ref{954} below), let $\pi$ be an infinite cardinal such that $X$ is in $\C_\pi$ for any $X\to Z$ in $\cc F$, and assume 

\nn(9.5.2) (i) $\C_0$ admits small filtrant inductive limits and $\C_0\to\C$ commutes with them;

\nn(9.5.2) (ii) any diagram of solid arrows
$$
\begin{tikzcd}
X\ar{d}[swap]{f}\ar{r}{u}&Y\ar[dashed]{d}{g}\\ 
X'\ar[dashed]{r}[swap]{u'}&Y',
\end{tikzcd}
$$ 
with $u$ in $\Mor(\C_0)$ and $f$ in $\Mor(\C)$, can be completed as indicated to a commutative diagram with dashed arrows $u'$ in $\Mor(\C_0)$ and $g$ in $\Mor(\C)$; 

\nn(9.5.6) for any $X$ in $\C_0$, the category $(\C_0)_X$ is essentially small;

\nn(9.5.7) any cartesian square 
$$
\begin{tikzcd}
X'\ar{d}[swap]{u}\ar{r}{f'}&Y'\ar{d}{v}\\ 
X\ar{r}[swap]{f}&Y
\end{tikzcd}
$$ 
in $\C$ with $f,f'$ in $\Mor(\C_0)$ decomposes into a commutative diagram 
$$
\begin{tikzcd}
X'\ar{d}[swap]{u}\ar{r}{f'}&Y'\ar{d}\ar{rd}{v}\\ 
X\ar{r}[swap]{g}&Z\ar{r}[swap]{h}&Y
\end{tikzcd}
$$ 
such that the square $(X',Y',Z,X)$ is cocartesian, $g$ and $h$ are in $\Mor(\C_0)$, and $f=h\circ g$; 

\nn(9.5.8) if a morphism $f:X\to Y$ in $\C_0$ is such that any cartesian square of solid arrows
$$
\begin{tikzcd}
U\ar{d}[swap]{u}\ar{r}{s}&V\ar{d}{v}\ar[dashed]{ld}[swap]{\xi}\\ 
X\ar{r}[swap]{f}&Y
\end{tikzcd}
$$ 
can be completed as indicated to a commutative diagram in $\C$ with the dashed arrow $\xi$, then $f$ is an isomorphism. 
%
\begin{thm} 
If the above conditions hold, then, for any $X$ in $\C_0$, there is a $\Mor(\C_0)$-injective object $Y$ of $\C$, and morphism $f:X\to Y$ in $\C_0$. If (9.5.2) holds, but (9.5.6), (9.5.7), and (9.5.8) do not necessarily hold, then there is an $\F$-injective object $Y$ of $\C$, and morphism $f:X\to Y$ in $\C_0$.
\end{thm}
%
\begin{rk}\label{954}
In the book $\cc F$ is supposed to be small, but the proof clearly works if $\cc F$ is essentially small. (See \S\ref{962} below.)
\end{rk}

%%

\subsection{Brief Comments}

\begin{s} 
P.~235, Theorem 9.6.1. In view of the comments made before Corollary~\ref{938} p.~\pageref{938}, Theorem 9.6.1 could be stated as follows:

\begin{thm}[Theorem 9.6.1 p. 235]\label{961}
Let $\C$ be a Grothendieck category. Then, for any small subset $E$ of $\Ob(\C)$, there exists an infinite cardinal $\pi$ such that

\nn{\em(i)} $\Ob(\C_\pi)$ contains $E$,

\nn{\em(ii)} $\C_\pi$ is a fully abelian subcategory of $\C$,

\nn{\em(iii)} $\C_\pi$ is essentially small,

\nn{\em(iv)} $\C_\pi$ contains a generator of $\C$,

\nn{\em(v)} $\C_\pi$ is closed by subobjects and quotients in $\C$,

\nn{\em(vi)} for any epimorphism $f:X\epi Y$ in $\C$ with $Y$ in $\C_\pi$, there exists $Z$ in $\C_\pi$ and a monomorphism $g:Z\mono X$ such that $f\circ g:Z\to Y$ is an epimorphism,

\nn{\em(vii)} $\C_\pi$ is closed by countable direct sums.
\end{thm}
\end{s}

%

\begin{s}\label{962}
P.~236, proof of Theorem 9.6.2. 

Line 3: One could change ``Let $\cc F$ be the set of monomorphisms $N\incl G$. This is a small set by Corollary 8.3.26'' to ``Let $\cc F$ be the set of monomorphisms $N\incl G$. This is an essentially small subcategory by Corollary 8.3.26''. In view of Remark~\ref{954}, we can still apply Theorem 9.5.4.

Line 6: Condition (9.5.2) (i) (see Section~\ref{954955} p.~\pageref{954955}) follows from 
%
\begin{lem}\label{952i}
Let $\C$ be a category. Assume that small filtrant inductive limits exist in $\C$ and are exact. Let $\alpha:I\to\C$ be a functor such that $I$ is small and filtrant, and $\alpha(s):\alpha(i)\to\alpha(j)$ is a monomorphism for all morphism $s:i\to j$ in $I$. Then the coprojection $p_i:\alpha(i)\to\colim\alpha$ is a monomorphism.
\end{lem}
%
\begin{proof}
By Corollary 3.2.3 p. 79 of the book, $I^i$ is filtrant and the forgetful functor $\pp:I^i\to I$ is cofinal. Define the morphism of functors 
$$
\theta\in\Hom_{\C^{I^i}}(\Delta(\alpha(i)),\alpha\circ\pp)
$$ 
($\Delta$ being the diagonal functor) by 
$$
\theta_{(s:i\to j)}:=\big(\alpha(s):\alpha(i)\to\alpha(j)\big). 
$$ 
As $\theta$ is a monomorphism, Proposition~\ref{34i} p.~\pageref{34i} implies that $\colim\theta$ is also a monomorphism. Then the conclusion follows from the commutativity of the diagram 
$$
\begin{tikzcd}
\colim\Delta(\alpha(i))\ar{d}[swap]{\sim}\ar[tail]{rr}{\colim\theta}&&\colim(\alpha\circ\pp)\ar{d}{\sim}\\ 
\alpha(i)\ar{rr}[swap]{p_i}&&\colim\alpha.
\end{tikzcd}
$$
\end{proof}
\end{s}

%%

\begin{s}
Pp 237-239. For the reader's convenience we first reproduce (with minor changes) two corollaries with their proof. 

\begin{cor}[Corollary 9.6.5 p. 237]
If $\C$ is a small abelian category, then $\Ind(\C)$ admits an injective cogenerator.
\end{cor}

\begin{proof}
Apply Theorem 8.6.5 (vi) p.~194 and Theorem 9.6.3 p.~236 of the book.
\end{proof}

\begin{cor}[Corollary 9.6.6 p. 237]
Let $\C$ be a Grothendieck category. Denote by $\cc I$ the full additive subcategory of $\C$ consisting of injective objects, and by $\iota:\cc I\to\C$ the inclusion functor. Then there exist a (not necessarily additive) functor $\Psi:\C\to\cc I$ and a morphism of functors $\id_\C\to\iota\circ\Psi$ such that $X\to\Psi(X)$ is a monomorphism for any $X$ in $\C$.
\end{cor}

\begin{proof}
The category $\C$ admits an injective cogenerator $K$ by Theorem 9.6.3 p.~236 of the book, and admits small products by Proposition 8.3.27 p.~186 of the book. Consider the (non additive) functor 
$$
\Psi:\C\to\cc I,\quad X\mapsto K^{\Hom_\C(X,K)}.
$$ 
The identity of 
$$
\Hom_{\Set}(\Hom_\C(X,K),\Hom_\C(X,K))\simeq\Hom_\C(X,K^{\Hom_\C(X,K)})
$$ 
defines a morphism $X\to\iota(\Psi(X))=K^{\Hom_\C(X,K)}$, and this morphism is a monomorphism by Proposition 5.2.3 (iv) p.~118 of the book. 
\end{proof}

We now add three parenthetical points: 

The first sentence of the proof of Lemma 9.6.8 p.~238 of the book follows from Proposition 5.2.3 (iv) p.~118 of the book. 

The third sentence of the proof of Lemma 9.6.9 p.~238 of the book follows from Proposition 5.2.3 (i) p.~118 of the book. 

In the proof of Theorem 9.6.10 p.~238 of the book, the exactness of $\C\to\oo{Pro}(\C)$ follows from Theorem 8.6.5 (ii) p.~194 of the book.
\end{s}

%%

\section{About Chapter 10}

\subsection{Definition of a Triangulated Category\index{definition of a triangulated category}\index{May (Peter May)}}

The purpose of this Section is to spell out the observation made by J. P. May that, in the definition of a triangulated category, Axiom TR4 of the book (p.~243) follows from the other axioms. See Section~1 of {\em The axioms for triangulated categories} by J. P. May:  
%
\begin{center}\href{http://www.math.uchicago.edu/~may/MISC/Triangulate.pdf}{http://www.math.uchicago.edu/$\sim$may/MISC/Triangulate.pdf} 
\end{center}
%
Various related links are given in the document \href{http://goo.gl/df2Xw}{http://goo.gl/df2Xw}. 

To make things as clear as possible, we remove TR4 from the definition of a triangulated category, and prove that any triangulated category satisfies TR4:

\begin{df}[triangulated category]\label{tr} 
A {\em triangulated category}\index{triangulated category} is an additive category $(\cc D,T)$ with translation endowed with a set of triangles satisfying Axioms {\em TR0, TR1, TR2, TR3}, and {\em TR5} on p.~243 of the book.
\end{df}

Let $(\cc D,T)$ be a triangulated category. In the book the theorem below is stated as Exercise 10.6 p.~266 and is used at the top of p.~251 within the proof of Theorem 10.2.3 p.~249.

\begin{thm}\label{mayt}
Let 
$$
\begin{tikzcd}
X^0\ar{r}{u}\ar{d}[swap]{f}&X^1\ar{d}\ar{r}{v}&X^2\ar[dashed]{d}\ar{r}{w}&TX^0\ar{d}{Tf}\\ 
Y^0\ar{r}\ar{d}[swap]{g}&Y^1\ar{d}\ar{r}&Y^2\ar[dashed]{d}\ar{r}&TY^0\ar{d}{Tg}\\ 
Z^0\ar[dashed]{r}\ar{d}[swap]{h}&Z^1\ar{d}\ar[dashed]{r}&Z^2\ar[dashed]{d}\ar[dashed]{r}&TZ^0\ar{d}{-Th}\\ 
TX^0\ar{r}[swap]{Tu}&TX^1\ar{r}[swap]{Tv}&TX^2\ar{r}[swap]{-Tw}&T^2X^0 
\end{tikzcd}
$$ 
be a diagram of solid arrows in $\cc D$. Assume that the first two rows and columns are distinguished triangles, and the top left square commutes\footnote{I think the assumption that the top left square commutes is implicit in the book.}. Then the dotted arrows may be completed in order that the bottom right small square anti-commutes, the eight other small squares commute, and all rows and columns are distinguished triangles. 
\end{thm}

\begin{cor}\label{may}
Any category which is triangulated in the sense of Definition~\ref{tr} satisfies {\em TR4}.
\end{cor} 

Recall Axiom TR5: If the diagram 
$$
\begin{tikzcd}
U\ar[equal]{dd}\ar{r}&V\ar[equal]{d}\ar{r}&W'\ar{r}&TU\\
&V\ar{r}&W\ar[equal]{d}\ar{r}&U'\ar{r}&TV\\
U\ar{rr}&&W\ar{rr}&&V'\ar{rr}&&TU
\end{tikzcd}
$$
commutes, and if the rows are distinguished triangles, then there is a distinguished triangle $W'\to V'\to U'\to TW'$ such that the diagram 
$$
\begin{tikzcd}
U\ar{r}\ar[equal]{d}&V\ar{d}\ar{r}&W'\ar{d}\ar{r}&TU\ar[equal]{d}\\
U\ar{d}\ar{r}&W\ar{r}\ar[equal]{d}&V'\ar{d}\ar{r}&TU\ar{d}\\
V\ar{d}\ar{r}&W\ar{d}\ar{r}&U'\ar{r}\ar[equal]{d}&TV\ar{d}\\
W'\ar{r}&V'\ar{r}&U'\ar{r}&TW' 
\end{tikzcd}
$$ 
commutes. 

\begin{proof}[Proof of Theorem~\ref{mayt}]
From 
$$
\begin{tikzcd}
X^0\ar[equal]{dd}\ar{r}&X^1\ar[equal]{d}\ar{r}&X^2\ar{r}&TX^0\\
&X^1\ar{r}&Y^1\ar[equal]{d}\ar{r}&Z^1\ar{r}&TX^1\\
X^0\ar{rr}&&Y^1\ar{rr}&&W\ar{rr}&&TX^0,
\end{tikzcd}
$$
where the last row is obtained by TR2, we get by TR5
\begin{equation}\label{v1}
\begin{tikzcd}
X^0\ar{r}\ar[equal]{d}&X^1\ar{d}\ar{r}&X^2\ar{d}{a}\ar{r}[swap]{w}&TX^0\ar[equal]{d}\\
X^0\ar{d}\ar{r}&Y^1\ar{r}\ar[equal]{d}&W\ar{d}{b}\ar{r}{d}&TX^0\ar{d}\\
X^1\ar{d}\ar{r}&Y^1\ar{d}\ar{r}&Z^1\ar{r}\ar[equal]{d}&TX^1\ar{d}\\
X^2\ar{r}[swap]{a}&W\ar{r}[swap]{b}&Z^1\ar{r}[swap]{c}&TX^2.
\end{tikzcd}
\end{equation}
%
From 
$$
\begin{tikzcd}
X^0\ar[equal]{dd}\ar{r}&Y^0\ar[equal]{d}\ar{r}&Z^0\ar{r}&TX^0\\
&Y^0\ar{r}&Y^1\ar[equal]{d}\ar{r}&Y^2\ar{r}&TY^0\\
X^0\ar{rr}&&Y^1\ar{rr}&&W\ar{rr}&&TX^0,
\end{tikzcd}
$$
we get by TR5
\begin{equation}\label{v2}
\begin{tikzcd}
X^0\ar{r}\ar[equal]{d}&Y^0\ar{d}\ar{r}&Z^0\ar{d}{e}\ar{r}[swap]{h}&TX^0\ar[equal]{d}\\
X^0\ar{d}\ar{r}&Y^1\ar{r}\ar[equal]{d}&W\ar{d}\ar{r}{d}&TX^0\ar{d}\\
Y^0\ar{d}\ar{r}&Y^1\ar{d}\ar{r}&Y^2\ar{r}\ar[equal]{d}&TY^0\ar{d}\\
Z^0\ar{r}[swap]{e}&W\ar{r}&Y^2\ar{r}&TZ^0.
\end{tikzcd}
\end{equation}
%
%We define $Z^0\to Z^1$ as the composition of $Z^0\xr eW$ in \eqref{v2} with $W\xr bZ^1$ in \eqref{v1}. 
From 
$$
\begin{tikzcd}
Z^0\ar[equal]{dd}\ar{r}{e}&W\ar[equal]{d}\ar{r}&Y^2\ar{r}&TZ^0\\
&W\ar{r}{b}&Z^1\ar[equal]{d}\ar{r}{c}&TX^2\ar{r}{-Ta}&TW\\
Z^0\ar{rr}&&Z^1\ar{rr}&&Z^2\ar{rr}{\ell}&&TZ^0,
\end{tikzcd}
$$
where the second row is obtained from $X^2\xr aW\xr bZ^1\xr cTX^2$ in \eqref{v1} by TR3 and TR0, and 
%
\begin{equation}\label{r3}
\text{the last row is obtained by TR2,} 
\end{equation}
%
we get by TR5 
%
\begin{equation}\label{v3}
\begin{tikzcd}
Z^0\ar{r}\ar[equal]{d}&W\ar{d}{b}\ar{r}&Y^2\ar{d}{j}\ar{r}&TZ^0\ar[equal]{d}\\
Z^0\ar{d}\ar{r}&Z^1\ar{r}\ar[equal]{d}&Z^2\ar{d}{k}\ar{r}[swap]{\ell}&TZ^0\ar{d}[swap]{Te}\\
W\ar{d}\ar{r}{b}&Z^1\ar{d}\ar{r}{c}&TX^2\ar{r}{-Ta}\ar[equal]{d}&TW\ar{d}\\
Y^2\ar{r}[swap]{j}&Z^2\ar{r}[swap]{k}&TX^2\ar{r}[swap]{-Ti}&TY^2,
\end{tikzcd}
\end{equation}
%
where 
%
\begin{equation}\label{c3}
X^2\xr iY^2\xr jZ^2\xr kTX^2\text{ is a distinguished triangle.} 
\end{equation}
%
We want to prove that the bottom right small square of 
%
\begin{equation}\label{v4}
\begin{tikzcd}
X^0\ar{r}{u}\ar{d}[swap]{f}&X^1\ar{d}\ar{r}{v}&X^2\ar{d}{i}\ar{r}{w}&TX^0\ar{d}{Tf}\\ 
Y^0\ar{r}\ar{d}[swap]{g}&Y^1\ar{d}\ar{r}&Y^2\ar{d}{j}\ar{r}&TY^0\ar{d}{Tg}\\ 
Z^0\ar{r}\ar{d}[swap]{h}&Z^1\ar{d}\ar{r}&Z^2\ar{d}{k}\ar{r}{\ell}&TZ^0\ar{d}{-Th}\\ 
TX^0\ar{r}[swap]{Tu}&TX^1\ar{r}[swap]{Tv}&TX^2\ar{r}[swap]{-Tw}&T^2X^0
\end{tikzcd}
\end{equation}
%
anti-commutes, that the eight other small squares commute, and that all rows and columns are distinguished triangles.

We list the nine small squares of each of the diagrams (\ref{v1}), (\ref{v2}), (\ref{v3}), (\ref{v4}) as follows:
$$
\begin{matrix}1&2&3\\ 4&5&6\\ 7&8&9
\end{matrix}
$$ 
and we denote the $j$-th small square of Diagram $(i)$ by $(i)j$. 

The commutativity of (\ref{v1})2 and (\ref{v2})5 implies that of (\ref{v4})2. 

The commutativity of (\ref{v1})3 and (\ref{v2})6 implies that of (\ref{v4})3.

The commutativity of (\ref{v2})7 and (\ref{v3})1 implies that of (\ref{v4})4.

The commutativity of (\ref{v2})8 and (\ref{v3})2 implies that of (\ref{v4})5. 

The commutativity of (\ref{v2})9 and (\ref{v3})3 implies that of (\ref{v4})6. 

The commutativity of (\ref{v2})3 and (\ref{v1})6 implies that of (\ref{v4})7. 

The commutativity of (\ref{v1})9 and (\ref{v3})8 implies that of (\ref{v4})8. 

To prove the anti-commutativity of the bottom right small square of \eqref{v4}, note 
%
\begin{align*}
Th\circ\ell&=Td\circ Te\circ\ell&\text{by \eqref{v2}}\\ 
&=-Td\circ Ta\circ k&\text{by \eqref{v3}}\\  
&=-Tw\circ k&\text{by \eqref{v1}}.
\end{align*} 

The third row and column are distinguished triangles by \eqref{r3} and \eqref{c3} respectively. It is easy to check that the other rows and columns are distinguished triangles too.
\end{proof}

%%

\subsection{Brief Comments}

\begin{s} 
P.~250, proof of Theorem 10.2.3 (iii). In view of Corollary~\ref{may} p.~\pageref{may}, it is not necessary to prove TR4.
\end{s}

%

\begin{s} P.~263, last sentence of the proof of Lemma 10.5.8. Consider the commutative diagram  
$$
\begin{tikzcd}
\oplus_i\,\pp(Z_i)\ar[equal]{d}\ar{r}&\oplus_i\,\pp(Y_i)\ar[equal]{d}\ar{r}&\oplus_i\,\widetilde\pp(X_i)\ar{d}\ar{r}&0\\ 
\oplus_i\,\pp(Z_i)\ar{r}&\oplus_i\,\pp(Y_i)\ar{r}&\widetilde\pp(\oplus_i\,X_i)\ar{r}&0, 
\end{tikzcd}
$$ 
whose rows are complexes. We already know that the bottom row is exact. The exactness of the top row follows (as in the proof of Lemma 10.5.7 (ii) p.~261 of the book) from the isomorphisms 
$$
\Coker(\oplus_i\,\pp(Z_i)\to\oplus_i\,\pp(Y_i))\simeq\oplus_i\,\Coker(\pp(Z_i)\to\pp(Y_i))\simeq\oplus_i\,\widetilde\pp(X_i).
$$
\end{s}

%%

\begin{s} P.~263, proof of Lemma 10.5.9. Before the sentence ``Since $Z_n$ and $X_n$ belong to $\cc K$, $X_{n+1}$ also belongs to $\cc K$'', one could add ``We may, and do, assume that $\cc K$ is saturated''.

Recall the Yoneda isomorphisms 
$$
\Hom_{\cc S^{\wedge,\text{prod}}}(\pp(X),H_0)\simeq H(X)\simeq\Hom_{\cc D^\wedge}(X,H)
$$ 
for $X$ in $\cc S$.

Note that Convention~\ref{payc} p.~\pageref{payc} can be applied.
\end{s}

%%

\subsection{Exercise 10.11 (p.~266)} 

Recall the statement: 

\nn(i) Let $\cc N$ be a null system in a triangulated category $\cc D$, let $Q:\cc D\to\cc D/\cc N$ be the localization functor, and let $f:X\to Y$ be a morphism in $\cc D$ satisfying $Q(f)=0$. Then $f$ factors through some object of $\cc N$. 

\nn(ii) The following conditions on $X$ in $\cc D$ are equivalent: 

\nn(a) $Q(X)\simeq0$,\quad(b) $X\oplus Y\in\cc N$ for some $Y\in\cc D$,\quad(c) $X\oplus TX\in\cc N$.

\begin{proof}\ 

\nn(i) The definition of $\cc D/\cc N$ and the assumption $Q(f)=0$ imply the existence of a morphism $s:Y\to Z$ in $\cc NQ$ such that $s\circ f=0$ (see (7.1.5) p.~155 of the book), and thus, in view of the definition of $\cc NQ$ (see (10.2.1) p.~249 of the book), the existence of a d.t. $W\to Y\to Z\to TW$ with $W$ in $\cc N$, and the conclusion follows from the fact that $\Hom_{\cc D}(X,\ )$ is cohomological (see Proposition 10.1.13 p.~245 of the book). 

\nn(ii)

\nn(a)$\then$(b): As $Q(\id_X)=0$, the first part of the exercise implies that $\id_X$ factors as $X\xr fZ\xr g X$ with $Z$ in $\cc N$. By TR2 there is a d.t. 
$$
X\xr fZ\xr hY\xr kTX.
$$ 
Since $g\circ f=\id_X$, the morphism $f$ is a monomorphism, and so is $Tf$. As $Tf\circ k=0$ by Proposition 10.1.11 p.~245 of the book, this implies $k=0$. Hence we have a morphism of d.t. 
$$
\begin{tikzcd}
X\ar[equal]{d}\ar{r}{f}&Z\ar{d}{(g,h)}\ar{r}{h}&Y\ar[equal]{d}\ar{r}{0}&TX\ar[equal]{d}\\ 
X\ar{r}&X\oplus Y\ar{r}&Y\ar{r}&TX
\end{tikzcd}
$$
(the bottom is a d.t. by Corollary 10.1.20 (ii) p.~248 of the book) and Proposition 10.1.15 p.~246 of the book implies that $(g,h)$ is an isomorphism.\bigskip 

\nn(b)$\then$(c): Let $\Delta_1,\dots,\Delta_5$ be the triangles
$$
\begin{tikzcd}
X\ar{r}&0\ar{r}&TX\ar{r}{=}&TX\\ 
Y\ar{r}{=}&Y\ar{r}&0\ar{r}&TY\\ 
X\oplus Y\ar{r}&Y\ar{r}&TX\ar{r}&TX\oplus TY\\ 
0\ar{r}&X\ar{r}&X\ar{r}&0\\ 
X\oplus Y\ar{r}&X\oplus Y\ar{r}&X\oplus TX\ar{r}&TX\oplus TY,
\end{tikzcd}
$$ 
with $\Delta_3:=\Delta_1\oplus\Delta_2$ and $\Delta_5:=\Delta_3\oplus\Delta_4$. It is easy to see that $\Delta_1,\Delta_2$, and $\Delta_4$ are distinguished. Then $\Delta_3$ and $\Delta_5$ are distinguished by Proposition 10.1.19 p.~247 of the book, and, as $X\oplus Y$ is in $\cc N$, Condition N'3 of Lemma 10.2.1 (b) p.~249 of the book implies that $X\oplus TX$ is in $\cc N$.

\nn(c)$\then$(a): Follows from Theorem 10.2.3 (iv) p.~249 of the book.
\end{proof} 

%%%

\section{About Chapter 11}

\begin{s}
P.~270. Recall that $(\A,T)$ is an additive category with translation. Let 
\begin{equation}\label{iliad}
(d_{X,i}:X_i\to TX_i)_{i\in I}
\end{equation} 
be an inductive system in $\A_d$. Assume that $X:=\colim_iX_i$ exists in $\A$. Then the natural morphism $\colim_id_{X,i}:X\to TX$ is an inductive limit of \eqref{iliad} in $\A_d$. There are analogous statements with ``projective'' instead of ``inductive'' and $\A_c$ instead of $\A_d$.
\end{s}

%

\begin{s}
P.~270, Definition 11.1.3. Here is a ``picture'' of the mapping cone of $f:X\to Y$:
$$
\begin{tikzcd}
TX\ar{rr}{-T(d_X)}\ar{rrdd}{T(f)}&&T^2X\\ 
\oplus&&\oplus\\ 
Y\ar{rr}[swap]{d_Y}&&TY.
\end{tikzcd}
$$
\end{s}

%

\begin{s}
P.~271, Remark 11.1.5. We have:
$$
d_{\Mc(T(f))}=
\begin{pmatrix}
d_{T^2X}&0\\ \\ 
T^2(f)&d_{TY}
\end{pmatrix},\quad 
d_{T(\Mc(f))}=
\begin{pmatrix}
d_{T^2X}&0\\ \\ 
-T^2(f)&d_{TY}
\end{pmatrix},
$$ 
and 
$$
T(\Mc(f))=T^2X\oplus TY\xr{\begin{pmatrix}-1&0\\ 0&1\end{pmatrix}}T^2X\oplus TY=\Mc(T(f))
$$ 
is a differential isomorphism.
\end{s}

%%

\subsection{Theorem 11.2.6 (p~273)} 

Here is a minor comment about the verification of Axiom TR5 in the proof of Theorem 11.2.6. We stated Axiom~TR5 right after Corollary~\ref{may} p.~\pageref{may} above. For the reader's convenience we restate it. 

If the diagram 
$$
\begin{tikzcd}
U\ar[equal]{dd}\ar{r}&V\ar[equal]{d}\ar{r}&W'\ar{r}&TU\\
&V\ar{r}&W\ar[equal]{d}\ar{r}&U'\ar{r}&TV\\
U\ar{rr}&&W\ar{rr}&&V'\ar{rr}&&TU
\end{tikzcd}
$$
commutes, and if the rows are distinguished triangles, then there is a distinguished triangle $W'\to V'\to U'\to TW'$ such that the diagram below commutes:
$$
\begin{tikzcd}
U\ar{r}\ar[equal]{d}&V\ar{d}\ar{r}&W'\ar{d}\ar{r}&TU\ar[equal]{d}\\
U\ar{d}\ar{r}&W\ar{r}\ar[equal]{d}&V'\ar{d}\ar{r}&TU\ar{d}\\
V\ar{d}\ar{r}&W\ar{d}\ar{r}&U'\ar{r}\ar[equal]{d}&TV\ar{d}\\
W'\ar{r}&V'\ar{r}&U'\ar{r}&TW'.
\end{tikzcd}
$$ 

Going back to the proof of TR5 on p.~275, we consider the diagram
$$
\begin{tikzcd}
X\ar[equal]{dd}\ar{r}{f}&Y\ar[equal]{d}\ar{r}{\alpha(f)}&TX\oplus Y\ar{r}{\beta(f)}&TX\\
&Y\ar{r}{g}&Z\ar[equal]{d}\ar{r}{\alpha(g)}&TY\oplus Z\ar{r}{\beta(g)}&TY\\
X\ar[swap]{rr}{g\circ f}&&Z\ar[swap]{rr}{\alpha(g\circ f)}&&TX\oplus Z\ar[swap]{rr}{\beta(g\circ f)}&&TX.
\end{tikzcd}
$$ 
The goal of the proof is then to construct a commutative diagram
$$
\begin{tikzcd}
X\ar{r}{f}\ar[equal]{d}&Y\ar{r}{\alpha(f)}\ar{d}{g}&TX\oplus Y\ar{r}{\beta(f)}\ar{d}{u}&TX\ar[equal]{d}\\ 
X\ar{r}{g\circ f}\ar{d}[swap]{f}&Z\ar{r}{\alpha(g\circ f)}\ar[equal]{d}&TX\oplus Z\ar{r}{\beta(g\circ f)}\ar{d}{v}&TX\ar{d}{T(f)}\\ 
Y\ar{r}{g}\ar{d}[swap]{\alpha(f)}&Z\ar{r}{\alpha(g)}\ar{d}[swap]{\alpha(g\circ f)}&TY\oplus Z\ar{r}{\beta(g)}\ar[equal]{d}&TY\ar{d}{T(\alpha(f))}\\ 
TX\oplus Y\ar{r}[swap]{u}&TX\oplus Z\ar{r}[swap]{v}&TY\oplus Z\ar{r}[swap]{w}&T^2X\oplus TY.
\end{tikzcd}
$$ 

%%

\subsection{Example 11.6.2 (i) (p.~290)} 

(As already stated, there is a typo; see \S\ref{290} p.~\pageref{290}.) Let $\C,\C'$ and $\C''$ be additive categories with translation. If $F:\C\times\C'\to\C''$ is a bifunctor of additive categories with translation and if $\C''$ admits countable direct sums, then, as explained in the book, $F$ induces a bifunctor of additive categories with translation 
$$
F_\oplus:\operatorname{C}(\C)\times\operatorname{C}(\C')\to\operatorname{C}(\C'').
$$ 
If $\C''$ admits countable products instead of direct sums, then $F$ induces a bifunctor of additive categories with translation 
$$
F_\pi:\operatorname{C}(\C)\times\operatorname{C}(\C')\to\operatorname{C}(\C'').
$$ 
The precise formulas are given in the book. If $F:\C\times\C'^\op\to\C''$ is a bifunctor of additive categories with translation and if $\C''$ admits countable products, then $F$ induces again a bifunctor of additive categories with translation 
$$
F_\pi:\operatorname{C}(\C)\times\operatorname{C}(\C')^{\op}\to\operatorname{C}(\C'').
$$ 
The formulas defining $F_\pi$ in this setting are almost the same as in the previous setting, and we give them without further comments:
$$
F_\pi(Y,X)^{n,m}=F(Y^n,X^{-m}),
$$
$$
d'^{n,m}=F(d_Y^n,X^{-m}),
$$
$$
d''^{n,m}=(-1)^{m+1}F(Y^n,d_X^{-m-1}),
$$
$$
\theta_{Y,X}:F_\pi(TY,X)\to TF_\pi(Y,X),
$$
$$
\theta'_{Y,X}:F_\pi(Y,T^{-1}X)\to TF_\pi(Y,X).
$$
$$
\theta_{Y,X}^{i+j}:F_\pi(TY,X)^{i+j}\to(TF_\pi(Y,X))^{i+j},
$$
$$
\theta_{Y,X}^{i,j}:F_\pi((TY)^i,X^{-j})=F(Y^{i+1},X^{-j})\to F_\pi(Y,X)^{i+j+1}=(TF_\pi(Y,X))^{i+j},
$$
$$
{\theta'}_{Y,X}^{i+j}:F_\pi(Y,T^{-1}X)^{i+j}\to(TF_\pi(Y,X))^{i+j},
$$
$$
{\theta'}_{Y,X}^{i,j}:F_\pi(Y^i,(T^{-1}X)^{-j})=F(Y^i,X^{-j-1})\to F_\pi(Y,X)^{i+j+1}=(TF_\pi(Y,X))^{i+j},
$$
the morphism ${\theta'}_{Y,X}^{i,j}$ being $(-1)^i$ times the canonical embedding. 

%%%

\section{About Chapter 12}

\subsection{Avoiding the Snake Lemma (p. 297)} \index{Snake Lemma} \index{avoiding the Snake Lemma}

This is about Sections 12.1 and 12.2 of the book. I think the Snake Lemma can be avoided as follows: 

Let $\A$ be an abelian category. 

\begin{lem}\label{sl1}
If 
$$
\begin{tikzcd}
{}&X'\ar{d}{u}\ar{r}{f}&X\ar{d}{v}\ar{r}{g}&X''\ar{d}{w}\ar{r}&0\\ 
0\ar{r}&Y'\ar{r}[swap]{f'}&Y\ar{r}[swap]{g'}&Y''.
\end{tikzcd}
$$ 
is a commutative diagram in $\A$ with exact rows, then the sequence 
$$
\Ker u\to\Ker v\to\Ker w\xr0\Coker u\to\Coker v\to\Coker w
$$
is exact at $\Ker v$ and $\Coker v$. If in addition $w$ is a monomorphism or $u$ is an epimorphism, then the whole sequence is exact.
\end{lem}
%
The proof is straightforward (and much easier than that of the Snake Lemma). 

Let $(\A,T)$ be an abelian category with translation. 

\begin{lem}[see Theorem 12.2.4 p.~301]\label{sl2}
If $0\to X\xr fY\xr g Z\to0$ is an exact sequence in $\A_c$, then the sequence $H(X)\to H(Y)\to H(Z)$ is exact. If, in addition, $H(T^nX)\simeq0$ (respectively $H(T^nZ)\simeq0$) for all $n$, then $T^nY\to T^nZ$ (respectively $T^nX\to T^nY$) is a qis for all $n$. 
\end{lem}

\begin{proof}
Taking into account Display (12.2.1) p.~300 of the book, apply Lemma~\ref{sl1} to the commutative diagram 
$$
\begin{tikzcd}
{}&\Coker T^{-1}d_X\ar{d}{d_X}\ar{r}{f}&\Coker T^{-1}d_Y\ar{d}{d_Y}\ar{r}{g}&\Coker T^{-1}d_Z\ar{d}{d_Z}\ar{r}&0\\ 
0\ar{r}&\Ker Td_X\ar{r}[swap]{f}&\Ker Td_Y\ar{r}[swap]{g}&\Ker Td_Z.
\end{tikzcd}
$$ 
\end{proof}

\begin{prop}[Corollary 12.2.5 p.~301]\label{sl3}
The functor 
$$
H:\text K_c(\A)\to\A
$$ 
is cohomological.  
\end{prop}

\begin{proof}
Let $X\to Y\to Z\to TX$ be a d.t. in $K_c(\A)$. It is isomorphic to 
$$
V\xr{\alpha(u)}\Mc(u)\xr{\beta(u)}TU\to TV
$$ 
for some morphism $u:U\to V$. Since the sequence 
$$
0\to V\to\Mc(u)\to TU\to0
$$ 
in $\A_c$ is exact, it follows from Lemma~\ref{sl2} that the sequence 
$$ 
H(V)\to H(\Mc(u))\to H(TU)
$$ 
is exact.
\end{proof}
%
\begin{prop}[Corollary 12.2.6 p.~302]\label{sl4}
Let $0\to X\xr f Y\xr g Z\to0$ be an exact sequence in $\A_c$ and define $\pp:\Mc(f)\to Z$ by $\pp:=(0,g)$. Then $\pp$ is a morphism in $\A_c$, and this morphism is a qis. In particular, there are natural morphisms $H(T^nZ)\to H(T^{n+1}X)$ such that the sequence 
$$
\cdots\to H(X)\to H(Y)\to H(Z)\to H(TX)\to\cdots
$$
is exact. 
\end{prop}

\begin{proof}
The commutative diagram in $\A_c$ with exact rows 
$$
\begin{tikzcd}
0\ar{r}&X\ar{d}[swap]{\id_X}\ar{r}{\id_X}&X\ar{d}{f}\ar{r}&0\ar{d}\ar{r}&0\\ 
0\ar{r}&X\ar{r}[swap]{f}&Y\ar{r}[swap]{g}&Z\ar{r}&0
\end{tikzcd}
$$ 
yields the exact sequence 
$$
0\to\Mc(\id_X)\to\Mc(f)\xr\pp\Mc(0\to Z)\to0
$$
in $\A_c$. As $H(\Mc(\id_X))\simeq0$, $\pp$ is a qis by Lemma~\ref{sl2}.
\end{proof}

%%%

\section{About Chapter 13}

\subsection{Brief Comments}

\begin{s}\label{q337}
P.~337. Theorem 13.4.1 suggests the following question: 

Let $\C$ be an abelian category and let $X$ and $Y$ be in $\oo K(\C)$. Is the natural morphism 
%
\begin{equation}\label{spalt}
\colim_{(X'\to X),(Y\to Y')\in\oo{Qis}}\Hom_{\oo K(\C)}(X',Y')\to\Hom_{\oo D(\C)}(X,Y)
\end{equation}
%
an isomorphism?

Theorem 13.4.1 and Corollary~\ref{1432} p.~\pageref{1432} imply that the answer is yes if $\C$ is a Grothendieck category. 

(Observe that the Axiom of Universes is not necessary to define the morphism~\eqref{spalt}.)
\end{s}

%

\begin{s}
P.~337. (See \S\ref{1341} p. \pageref{1341} and \S\ref{q337} p.~\pageref{q337}.) In view of \eqref{hcsr} p.~\pageref{hcsr}, \eqref{hcsl} p.~\pageref{hcsl}, and Theorem 13.4.1 p.~337 of the book, the functors

$$
\Hom_\C^\bu:\oo K(\C)\times\oo K(\C)^{\op}\to\oo K(\Mod(\mathbb Z))
$$ 

\nn and 

$$
\Hom_{\oo K(\C)}:\oo K(\C)\times\oo K(\C)^{\op}\to\Mod(\mathbb Z)
$$ 

\nn give rise to the following functorial isomorphisms:

$$
R^0\Hom_\C^\bu(X,Y)\simeq R^0(\Hom_\C^\bu(X,\ \ ))(Y)\simeq R^0(\Hom_\C^\bu(\ \ ,Y))(X)
$$

$$
\simeq R\Hom_{\oo K(\C)}(X,Y)\simeq R(\Hom_{\oo K(\C)}(X,\ \ ))(Y)\simeq R(\Hom_{\oo K(\C)}(\ \ ,Y))(X)
$$ 

$$
\simeq\Hom_{\oo D(\C)}(X,Y).
$$

\nn(In penultimate last line, we use Notation 10.3.8 p.~255 of the book.) This implies 

$$
R\Hom_\C^\bu(X,Y)\simeq R(\Hom_\C^\bu(X,\ \ ))(Y)\simeq R(\Hom_\C^\bu(\ \ ,Y))(X).
$$
\end{s}

%%

\subsection{Exercise 13.15 (p. 342)}

Here is a partial solution. Let $\C$ be an abelian category. 

\begin{lem}\label{738}
Let $Z\to Y\to X\to W\to0$ be an exact sequence and $Y\to V$ a morphism in $\C$, let $U$ be the fiber coproduct $V\oplus_YX$, and let $U\to W$ be the morphism which makes 
$$
\begin{tikzcd}
Z\ar{d}\ar{r}&Y\ar{d}\ar{r}&X\ar{d}\ar{r}&W\ar[equal]{d}\ar{r}&0\\ 
0\ar{r}&V\ar{r}&U\ar{r}&W\ar{r}&0
\end{tikzcd}
$$ 
a commutative diagram of complexes. Then the bottom row is exact.
\end{lem}

\begin{proof}
We shall use Lemma~\ref{8312} p.~\pageref{8312}.

\nn Exactness at $V$: Let $V\to T$ be a morphism. By Lemma~\ref{8312}, (c)$\then$(a), it suffices to show that the diagram of solid arrows 
$$
\begin{tikzcd}
V\ar{r}\ar{d}&U\ar[dashed]{d}\\ 
T\ar[dashed,tail]{r}&S
\end{tikzcd}
$$ 
can be completed to a commutative diagram as indicated. To do this, we decree that the above completed square is cocartesian, and note that $T\to S$ is a monomorphism by Lemma \ref{8311} (b) (ii) p.~\pageref{8311}.

\nn Exactness at $U$: Let $U\to T$ be a morphism whose composition with $V\to U$ is zero. By Lemma~\ref{8312}, (c)$\then$(a), it suffices to show that the commutative diagram of solid arrows 
$$
\begin{tikzcd}
V\ar{r}\ar{dr}[swap]{0}&U\ar{r}\ar{d}&W\ar[dashed]{d}\\ 
{}&T\ar[dashed,tail]{r}&S
\end{tikzcd}
$$ 
can be completed as indicated. This follows from the fact that, by Lemma~\ref{8312}, (a)$\then$(c), the commutative diagram of solid arrows 
$$
\begin{tikzcd}
Y\ar{r}\ar{dr}[swap]{0}&X\ar{r}\ar{d}&W\ar[dashed]{d}\\ 
{}&T\ar[dashed,tail]{r}&S
\end{tikzcd}
$$ 
can itself be completed as indicated.

\nn Exactness at $W$: Let $T\to W$ be a morphism. By Lemma~\ref{8312}, (b)$\then$(a), it suffices to show that the diagram of solid arrows 
$$
\begin{tikzcd}
S\ar[dashed,two heads]{r}\ar[dashed]{d}&T\ar{d}\\ 
U\ar{r}&W
\end{tikzcd}
$$ 
can be completed to a commutative diagram as indicated. This follows from the fact that, by Lemma~\ref{8312}, (a)$\then$(b), the commutative diagram of solid arrows 
$$
\begin{tikzcd}
S\ar[dashed,two heads]{r}\ar[dashed]{d}&T\ar{d}\\ 
X\ar{r}&W
\end{tikzcd}
$$ 
can itself be completed to a commutative diagram as indicated.
\end{proof}

Let $X$ and $Y$ be in $\C$, let $E$ be the set of short exact sequences 
$$
0\to Y\to Z\to X\to0,
$$ 
and let $\sim$ be the following equivalence relation on $E$: the exact sequences 
$$
0\to Y\to Z\to X\to0
$$ 
and 
$$
0\to Y\to W\to X\to0
$$ 
are equivalent if and only if there is a commutative diagram 
$$
\begin{tikzcd}
0\ar{r}&Y\ar[equal]{d}\ar{r}&Z\ar{d}\ar{r}&X\ar[equal]{d}\ar{r}&0\\ 
0\ar{r}&Y\ar{r}&W\ar{r}&X\ar{r}&0.
\end{tikzcd}
$$ 
(This is easily seen to be indeed an equivalence relation.) To the element 
$$
0\to Y\to Z\to X\to0
$$ 
of $E$ we attach the morphism in 
$$
\Hom_{\oo D(\C)}(X,Y[1])=\Ext^1_\C(X,Y)
$$ 
suggested by the diagram 
$$
\begin{tikzcd}
{}&X\\ 
Y\ar[equal]{d}\ar{r}&Z\ar{u}\\ 
Y,
\end{tikzcd}
$$ 
where each row is a complex (viewed as an object of $\oo D(\C)$), with the convention that only the possibly nonzero terms are indicated (the top morphism being a qis). 

We claim: 

(a) this process induces a map from $E/\!\!\sim$ to $\Ext^1_\C(X,Y)$, 

(b) this map (a) is bijective. 

Claim (a) is left to the reader. To prove (b) we construct the inverse map. To this end, we start with a complex $Z^\bullet$, a qis $Z^\bullet\to X$, and a morphism $Z^\bullet\to Y[1]$ representing our given element of $\Ext^1_\C(X,Y)$. The natural morphism $\tau^{\le0}Z^\bullet\to Z^\bullet$ being a qis, we can replace $Z^\bullet$ with $\tau^{\le0}Z^\bullet$, or, in other words, we may, and will, assume $Z^n\simeq0$ for $n>0$. Letting $Z$ be the fiber coproduct $Y\oplus_{Z^{-1}}Z^0$, Lemma~\ref{738} p.~\pageref{738} yields an exact sequence $0\to Y\to Z\to X\to0$. It is easy to see that this process defines a map from $\Ext^1_\C(X,Y)$ to $E/\!\!\sim$, and that this map is inverse to the map constructed before. q.e.d.

%%%

\section{About Chapter 14}

\subsection{Proposition 14.1.6 (p.~349)}

Here are some additional details about Step~(ii) of the proof of Proposition 14.1.6. 

We refer the reader to the book for a precise description of the setting. The following facts can be easily verified: 

The morphisms 
$$
f:X\to Y,\quad\pp:X\to I,\quad g:Y\to Z
$$ 
in $\A$ satisfy 
%
\begin{equation}\label{minac}
f,\pp,g\text{ are in fact morphisms in }\A_c.
\end{equation}
%
We also have morphisms 
$$
\widetilde h:T^{-1}\to I,\quad\widetilde\psi:Y\to I,\quad\psi:Y\to I,\quad h:T^{-1}Y\to I,\quad\xi:Z\to I
$$ 
in $\A$, and we have the equalities 
%
\begin{equation}\label{ppf}
\pp=\psi\circ f,
\end{equation}
%
\begin{equation}\label{hpm1}
h=T^{-1}d_I\circ T^{-1}\psi-\psi\circ T^{-1}d_Y,
\end{equation}
%
\begin{equation}\label{hpm2}
h=T^{-1}d_I\circ T^{-1}\psi+\psi\circ d_{T^{-1}Y},
\end{equation}
%
\begin{equation}\label{hhg}
h=\widetilde h\circ T^{-1}g,
\end{equation}
%
\begin{equation}\label{htilde}
\widetilde h=T^{-1}d_I\circ T^{-1}\xi-\xi\circ T^{-1}d_Z,
\end{equation}
%
\begin{equation}\label{psitildef}
\widetilde\psi=\psi-\xi\circ g. 
\end{equation}

To prove $h\circ T^{-1}f=0$, we note: 
%
\begin{align*}
h\circ T^{-1}f&=T^{-1}d_I\circ T^{-1}\psi\circ T^{-1}f+\psi\circ d_{T^{-1}Y}\circ T^{-1}f&\text{by \eqref{hpm2}}\\ 
&=T^{-1}d_I\circ T^{-1}\pp+\psi\circ d_{T^{-1}Y}\circ T^{-1}f&\text{by \eqref{ppf}}\\ 
&=T^{-1}d_I\circ T^{-1}\pp+\psi\circ f\circ d_{T^{-1}X}&\text{by \eqref{minac}}\\ 
&=T^{-1}d_I\circ T^{-1}\pp+\pp\circ d_{T^{-1}X}&\text{by \eqref{ppf}}\\ 
&=0&\text{by \eqref{minac}}. 
\end{align*}

To prove that $\widetilde\psi$ is a morphism in $\A_c$, we note: 
%
\begin{align*}
d_I\circ\widetilde\psi-T\widetilde\psi\circ d_Y&=d_I\circ\psi-d_I\circ\xi\circ g-T\psi\circ d_Y+T\xi\circ Tg\circ d_Y&\text{by \eqref{psitildef}}\\ 
&=(d_I\circ\psi-T\psi\circ d_Y)-(d_I\circ\xi\circ g-T\xi\circ Tg\circ d_Y)\\ 
&=Th-(d_I\circ\xi\circ g-T\xi\circ Tg\circ d_Y)&\text{by \eqref{hpm1}}\\ 
&=Th-(d_I\circ\xi\circ g-T\xi\circ d_Z\circ g)&\text{by \eqref{minac}}\\ 
&=Th-T\widetilde h\circ g&\text{by \eqref{htilde}}\\ 
&=0&\text{by \eqref{hhg}}. 
\end{align*}

%%

\subsection{Brief Comments}

\begin{s} 
P.~350, last paragraph. In view of the comments made about Corollary~\ref{938} p.~\pageref{938} and Theorem~\ref{961} p.~\pageref{961}, one could replace ``there exists an essentially small full subcategory $\cc S$ of $\A_c$ such that \dots'' with ``there exists an infinite cardinal $\pi$ such that $(\A_c)_\pi$ is essentially small and satisfies \dots'', and replace $\cc S$ with $(\A_c)_\pi$ in (14.1.4) p.~351 of the book.
\end{s}

%

\begin{s}\label{14112}
P.~352, Corollary 14.1.12. We also have the following corollary: 

Let $\U_0\subset\U$ be universes, let $(\A,T)$ be a Grothendieck $\U$-category with translation, and let $\A_0\subset\A$ be a fully abelian subcategory with translation. Assume that $\A_0$ is a Grothendieck $\U_0$-category. Then the natural functor $\D_c(\A_0)\to\D_c(\A)$ is fully faithful.

This follows immediately from Corollary 14.1.12 (i). 
\end{s}

%

\begin{s} 
P.~352, Corollary 14.1.12 (iv). Here are slightly more precise statements:

\nn(iii) the functor $Q:\oo K_c(\A)\to\oo D_c(\A)$ admits a right adjoint $R_q:\oo D_c(\A)\to\oo K_c(\A)$, this right adjoint is triangulated, satisfies $Q\circ R_q\simeq\id_{\oo D_c(\A)}$, and is isomorphic to the composition of $\iota:\oo K_{c,\oo{hi}}(\A)\to\oo K_c(\A)$ and a quasi-inverse of $Q\circ\iota$,

\nn(iv) the right localization $(\oo D_c(\A),Q)$ of $\oo K_c(\A)$ is universal in the sense of Definition~\ref{url2} p.~\pageref{url2}. (See \S\ref{732} p.~\pageref{732}.)
\end{s}

%

\begin{s}
Corollary 14.3.2 p.~356. Let us add one sentence to the statement:
%
\begin{cor}\label{1432}
Let $k$ be a commutative ring and let $\C$ be a Grothendieck $k$-abelian category. Then $(\oo K_{\oo{hi}}(\C),\oo K(\C)^{\op})$ is $\Hom_\C$-injective, and the functor $\Hom_\C$ admits a right derived functor 
$$
\oo{RHom}_\C:\oo D(\C)\times\oo D(\C)^{\op}\to\oo D(k).
$$ 
If $X$ and $Y$ are in $\oo K(\C)$, then for any qis $Y\to I$ with $I$ in $\oo K_{\oo{hi}}(\C)$ (such exist) we have 
$$
\oo{RHom}_\C(X,Y)\xr\sim\Hom_{\oo K(\C)}(X,I)\xr\sim\Hom_{\oo D(\C)}(X,I).
$$ 
Moreover, $H^0(\oo{RHom}_\C(X,Y))\simeq\Hom_{\oo D(\C)}(X,Y)$ for $X,Y$ in $\oo D(\C)$.
\end{cor}
\end{s}

%

\begin{s}%\label{s144}
P.~357, Section 14.4. Having been unable to solve Part (iii) of Exercise 8.37 p.~211, I suggest the following changes to Section 14.4. (I might be missing something. If so, thank you for letting me know.)

\nn(a) Replace Assumption (14.4.1) p.~358 with: ``$\C$ admits inductive limits indexed by the ordered set $\bb N$, and such limits are exact''.

\nn(b) Say (only for the duration of this comment) that a full saturated subcategory $\A$ of a category $\B$ is {\em closed by coproducts} if the coproduct of any family of objects of $\A$ which exists in $\B$ belongs to $\A$.

\nn(c) In Lemma 14.4.2 p. 359, replace ``full triangulated'' with ``full saturated triangulated'', and ``closed by small direct sums'' with ``closed by direct sums (in the sense of the above definition)''. 

There are analog observations for the other statements of Section 14.4.
\end{s}

%

\begin{s}%\label{s1448}
Statement of Theorem 14.4.8 p.~361. I know that the statement is already very long, but I shall consider here a minor variant which would make it even longer! More precisely, (14.4.5) could be stated as follows:

Let $X_i$ be in $\oo K(\C_i)$ for $i=1,2,3$, and let $P_i\to X_i$ ($i=1,2$) and $X_3\to I$ be qis with $X_i$ in $\widetilde{\mc P}_i$ and $I$ in $\oo K_{\oo{hi}}(\C_3)$ (such exist). Consider the functorial morphisms of abelian groups
\begin{equation}\label{1448}
\begin{split}
\Hom_{\oo D(\C_3)}(LG(X_1,X_2),X_3)&\xr a\Hom_{\oo D(\C_3)}(G(P_1,P_2),I)\xleftarrow b\\ 
\Hom_{\oo K(\C_3)}(G(P_1,P_2),I)&\simeq\Hom_{\oo K(\C_1)}(P_1,F_1(P_2,I))\xr c\\ 
\Hom_{\oo D(\C_1)}(P_1,F_1(P_2,I))&\xleftarrow d\Hom_{\oo D(\C_1)}(X_1,RF_1(X_2,X_3)),
\end{split}
\end{equation}
where the middle isomorphism is the obvious one. Then $a,b,c,d$ are isomorphisms. There is an analogous statement for $F_2$.
\end{s}

%

\begin{s}
Step (f) of the proof of Theorem 14.4.8 p.~364. We already know that $a,b,c,d$ in \eqref{1448} are isomorphisms. As explained in the book, we have morphisms  
\begin{equation}\label{f1}
\begin{split}
\oo{RHom}_{\C_3}(LG(X_1,X_2),X_3)&\to\\ 
\oo{RHom}_{\C_1}(RF_1(X_2,LG(X_1,X_2),RF_1(X_2,X_3))&\to\\ 
\oo{RHom}_{\C_1}(X_1,RF_1(X_2,X_3)).
\end{split}
\end{equation}
Applying $H^0$ we get, in view of Theorem 13.4.1 p.~337 of the book, morphisms 
\begin{equation}\label{f2}
\begin{split}
\oo{Hom}_{\oo D(\C_3)}(LG(X_1,X_2),X_3)&\to\\ 
\oo{Hom}_{\oo D(\C_1)}(RF_1(X_2,LG(X_1,X_2),RF_1(X_2,X_3))&\to\\ 
\oo{Hom}_{\oo D(\C_1)}(X_1,RF_1(X_2,X_3)).
\end{split}
\end{equation}
By (1.5.7) p. 29 of the book, Composition~\eqref{f2} coincides with Composition~\eqref{1448}, and is, thus, an isomorphism. This implies that Composition~\eqref{f1} is also an isomorphism.
\end{s}

%%%

\section{About Chapter 16}

\subsection{Sieves and Local Epimorphisms}\index{sieve}\index{local epimorphism}

This section is about the beginning of Section 16.1 p.~389 of the book. Let $\C$ be a category whose hom-sets are disjoint, let $M$ be the set of morphisms of $\C$, and for each $U$ in $\C$ let $M_U\subset M$ be the set of morphisms whose target is $U$. A subset $S$ of $M_U$ is a \textbf{sieve} \index{sieve} over $U$ if it is a right ideal of $M$, in the sense that $S$ contains all the morphisms of the form $s\circ f$ with $s$ in $S$. If $S$ is a sieve over $U$ and $f:V\to U$ is a morphism, we put
$$
S\times_UV:=\{W\to V|(W\to V\to U)\in S\}.
$$
One easily checks that this is a sieve over $V$. 

To a sieve $S$ over $U$ we attach the subobject $A_S$ of $U$ in $\C^\wedge$ by the formula
$$
A_S(V):=S\cap\Hom_\C(V,U).
$$ 
Conversely, to an object $A\to U$ of $(\C^\wedge)_U$ we attach the sieve $S_A$ over $U$ by putting 
$$
S_{A\to U}:=\{V\to A\to U\}.
$$ 

Let $\Sigma_U$ be the set of sieves over $U$. Let $(\Gamma_U)_{U\in\C}$ be a subfamily of the family $(\Sigma_U)_{U\in\C}$ and consider the following conditions:

\begin{cond}\label{gt}
$\ $

\nn GT1: for all $U$ in $\C$ we have: $M_U\in\Gamma_U$,

\nn GT2: for all $U$ in $\C$ we have: $\Gamma_U\ni S\subset S'\in\Sigma_U\implies S'\in\Gamma_U$,

\nn GT3: for all $U$ in $\C$ we have: $S\in\Gamma_U,\ (V\to U)\in M\implies S\times_UV\in\Gamma_V$,

\nn GT4: for all $U$ in $\C$ we have: 
$$
S\in\Gamma_U,\ S'\in\Sigma_U,\ S'\times_UV\in\Gamma_V\ \forall\ (V\to U)\in S\implies S'\in\Gamma_U.
$$
\end{cond}

Consider the following conditions on a set $\cc E$ of morphisms in $\C^\wedge$:

\nn LE1: $\id_U$ is in $\cc E$ for all $U$ in $\C$,

\nn LE2: if the composition of two elements of $\cc E$ exists, it belongs to $\cc E$,

\nn LE3: if the composition $v\circ u$ of two morphisms of $\C^\wedge$ exists and is in $\cc E$, then $v$ is in $\cc E$,

\nn LE4: a morphism $A\to B$ in $\C^\wedge$ is in $\cc E$ if and only if, for any morphism $U\to B$ in $\C^\wedge$ with $U$ in $\C$, the projection $A\times_BU\to U$ is in $\cc E$.

As proved in the book, $\cc E$ contains all epimorphisms. 

The elements of $\cc E$ are called \textbf{local epimorphisms}.

Let $(\Gamma_U)_{U\in\C}$ be a subfamily of the family $(\Sigma_U)_{U\in\C}$ satisfying GT1\--GT4, let $\U$ be a universe such that $\C$ is $\U$-small, and let 
%
\begin{equation}\label{cce}
\cc E=\cc E(\Gamma,\U)
\end{equation}
%
be the set of those morphisms $A\to B$ in $\C^\wedge$ such that, for any morphism $U\to B$ in $\C^\wedge$ with $U$ in $\C$, the sieve $S_{A\times_BU\to U}$ is in $\Gamma_U$. 

\begin{rk}\label{1613i}
A morphism $A\to U$ in $\C^\wedge$ is in $\cc E$ if and only if $S_{A\to U}$ is in $\Gamma_U$. 
\end{rk}

\begin{proof}
This results easily from the following observation: In the setting 
$$
A\to U\leftarrow V
$$ 
(obvious notation), we have $S_{A\times_UV\to V}=S_{A\to U}\times_UV$. 
\end{proof}

Let us check that $\cc E$ satisfies LE1-LE4:

\nn LE1 follows immediately from GT1.

\nn LE2: Let $A\to B\to C$ be a diagram in $\C^\wedge$, and assume that the two arrows are in $\cc E$. Consider the diagram of solid arrows with cartesian squares 
$$
\begin{tikzcd}
F\ar[dashed]{r}\ar[dashed]{d}&V\ar[dashed]{d}\\ 
D\ar{r}\ar{d}&E\ar{r}\ar{d}&U\ar{d}\\ 
A\ar{r}&B\ar{r}&C
\end{tikzcd}
$$
in $\C^\wedge$ (with $U$ in $\C$). We have that $S_{E\to U}$ is in $\Gamma_U$ (because $B\to C$ is in $\cc E$) and we must prove that $S_{D\to U}$ is in $\Gamma_U$. Let $V\to U$ be in $S_{E\to U}$, and let us complete the diagram with cartesian squares as indicated. By GT4 it suffices to check that $S_{F\to V}$ is in $\Gamma_V$. But this follows from the assumption that $A\to B$ is in $\cc E$ (together with a transitivity property of cartesian squares which has already been tacitly used).

\nn LE3 follows immediately from GT2.

\nn LE4. We must check: 
\begin{center}
$A\to B$ is in $\cc E$\\ $\iff$\\ for any morphism $U\to B$ in $\C^\wedge$ with $U$ in $\C$, the projection $A\times_BU\to U$ is in $\cc E$.
\end{center}
Implication $\then$ is obvious, and Implication $\si$ follows from Remark~\ref{1613i}. 

Conversely, given an object $U$ of $\C$ and a set $\cc E$ of morphisms in $\C^\wedge$ satisfying LE1-LE4, put
$$
\Gamma_U:=\{S\in\Sigma_U|(A_S\to U)\in\cc E\}.
$$
Let us check that $\Gamma(\cc E):=(\Gamma_U)_{U\in\C}$ satisfies GT1-GT4.

GT1 follows from LE1 and the equality $(A_{M_U}\to U)=\id_U$. 

GT2 follows from LE3 and the fact that, in the setting 
$$
\Gamma_U\ni S\subset S'\in\Sigma_U\implies S'\in\Gamma_U,
$$ 
the morphism $A_S\to U$ factors as $A_S\to A_{S'}\to U$. 

To prove GT3, note that if $S$ is a sieve over $U$ and $V\to U$ is a morphism in $\C$, then we have 
%
\begin{equation}\label{asuv}
A_{S\times_UV}=A_S\times_UV.
\end{equation}
%
In view of LE4, this implies GT3.  

The lemma below will helps us verify GT4. 

\begin{lem}\label{prepagt4}
Let $s:V\to U$ be a morphism in $\C$ and $S$ a sieve over $U$. Then $s$ is in $S$ if and only if $s$ factors through the natural morphism $i:A_S\to U$.
\end{lem}

\begin{proof}
By the Yoneda Lemma (Lemma~\ref{yol} p.~~\pageref{yol}), there is a bijection 
$$
S\cap\Hom_\C(V,U)\xr\pp\Hom_{\C^\wedge}(V,A_S)
$$
such that $\pp(s)_W=s\circ$ for all $W$ in $\C$. 

Assume that $s$ is in $S$ and let us show that there is a morphism $v:V\to A_S$ satisfying $i\circ v=s$. It suffices to prove $i\circ\pp(s)=s$ and to put $v:=\pp(s)$. We have for all $W$ in $\C$
$$
(i\circ\pp(s))_W=i_W\circ\pp(s)_W=i_W\circ(s\circ)=s\circ=s_W.
$$ 

Conversely, assuming that $v$ is in $\Hom_{\C^\wedge}(V,A_S)$, it suffices to prove that $i\circ v$ is in $S$. We have 
$$
i\circ v=(i\circ v)\circ\id_V=(i\circ v)_V(\id_V)=i_V(v_V(\id_V))=v_V(\id_V)\in A_S(V)\subset S. 
$$ 
\end{proof} 

\begin{lem}
Condition GT4 holds.
\end{lem}

\begin{proof} 
Let us assume 
%
\begin{equation}\label{sgu}
S\in\Gamma_U,\ S'\in\Sigma_U,\ S'\times_UV\in\Gamma_V\ \forall\ (V\to U)\in S.
\end{equation}
%
It suffices to check $S'\in\Gamma_U$, or, equivalently, 
%
\begin{equation}\label{as'}
(A_{S'}\to U)\in\cc E.
\end{equation}

Form the cartesian square 
$$
\begin{tikzcd}
B\ar{r}\ar{d}&A_S\ar{d}\\ 
A_{S'}\ar{r}&U.
\end{tikzcd}
$$ 
As $A_S\to U$ is in $\cc E$ by assumption, it suffices, by LE2 and LE3, to check  
%
\begin{equation}\label{bas}
(B\to A_S)\in\cc E.
\end{equation}
%
Let $V\to A_S$ be a morphism in $\C^\wedge$ with $V$ in $\C$, and let 
$$
\begin{tikzcd}
C\ar{r}\ar{d}&V\ar{d}\\ 
B\ar{r}&A_S
\end{tikzcd}
$$ 
be a cartesian square. By LE4 it is enough to verify 
%
\begin{equation}\label{cv}
(C\to V)\in\cc E.
\end{equation} 
% 
The morphism $V\to U$ being in $S$ by Lemma~\ref{prepagt4}, the sieve $S'\times_UV$ is in $\Gamma_V$ by \eqref{sgu}, and $A_{S'\times_UV}\to V$ is in $\cc E$ by definition of $\Gamma_V$. We have 
$$
\cc E\ni(A_{S'\times_UV}\to V)\simeq(A_{S'}\times_UV\to V)\simeq(C\to V).
$$ 
Indeed, the first isomorphism holds by \eqref{asuv} p.~\pageref{asuv}, and the second one holds because the rectangle 
$$
\begin{tikzcd}
C\ar{r}\ar{d}&V\ar{d}\\ 
B\ar{r}\ar{d}&A_S\ar{d}\\ 
A_{S'}\ar{r}&U
\end{tikzcd}
$$ 
is cartesian. This proves successively \eqref{cv}, \eqref{bas}, \eqref{as'}, and the lemma.  
\end{proof}

We have proved that $(\Gamma_U)_{U\in\C}$ satisfies GT1-GT4. 

It is now easy to prove 

\begin{thm}
If $\C$ is a $\U$-small category, if $\Gamma$ is a subfamily of $(\Sigma_U)_{U\in\C}$ satisfying GT1-GT4, and if $\cc E$ is a set of morphisms in $\C^\wedge_\U$ satisfying LE1-LE4, then the equalities $\cc E=\cc E(\Gamma,\U)$ and $\Gamma=\Gamma(\cc E)$ are equivalent.
\end{thm}

\begin{cor}\label{leu}
Let $\U\subset\U'$ be universes, let $\C$ be a $\U$-small category, let $\Gamma$ be a subfamily of $(\Sigma_U)_{U\in\C}$ satisfying GT1-GT4 (see Conditions~\ref{gt} p.~\pageref{gt}), let $u:A\to B$ be a morphism in $\C^\wedge_\U$, and let $u':A'\to B'$ be the corresponding morphism in $\C^\wedge_{\U'}$. Then $u$ is in $\cc E(\Gamma,\U)$ (see \eqref{cce} p.~\pageref{cce}) if and only if $u'$ is in $\cc E(\Gamma,\U')$.
\end{cor}

%%

\subsection{Brief Comments}

\begin{s}\label{390}
P.~390, Axioms LE1-LE4. The set of local epimorphisms attached to the natural Grothendieck topology associated with a small topological space $X$ can be described as follows. 

Let $f:A\to B$ be a morphism in $\C^\wedge$, where $\C$ is the category of open subsets of $X$. For each pair $(U,b)$ with $U$ in $\C$ and $b$ in $B(U)$ let $\Sigma(U,b)$ be the set of those $V$ in $\C_U$ such that there is an $a$ in $A(V)$ satisfying $f_V(a)=b_V$, where $b_V$ is the restriction of $b$ to $V$. Then $f$ is a local epimorphism if and only if 
$$
U=\bigcup_{V\in\Sigma(U,b)}V
$$ 
for all $(U,b)$ as above.

Moreover, a morphism $u:A\to U$ in $(\oo{Op}_X)^\wedge$ with $U$ in $\oo{Op}_X$ is a local epimorphism if and only if for all $x$ in $U$ there is a $V$ in $\oo{Op}_X$ such that $x\in V$ and $A(V)\ne\varnothing$. 
\end{s}

% https://docs.google.com/document/d/1vpjtCfl-qZCuuI-wkR5rbsav31DfDsZx1mQteXzu-UM/edit

\begin{s}
For any universe $\U$, any $\U$-small category $\C$, and any subfamily $\Gamma$ of $(\Sigma_U)_{U\in\C}$ satisfying GT1-GT4 (see Conditions~\ref{gt} p.~\pageref{gt}), let $\cc M(\Gamma,\U)$ and $\cc I(\Gamma,\U)$ denote respectively the set of local monomorphisms and local isomorphisms attached to $\cc E(\Gamma,\U)$ (see \eqref{cce} p.~\pageref{cce}). Corollary \ref{leu} p. \pageref{leu} implies:

Let $\U\subset\U'$ be universes, let $\C$ be a $\U$-small category, let $\Gamma$ be a subfamily of $(\Sigma_U)_{U\in\C}$ satisfying GT1-GT4, let $u:A\to B$ be a morphism in $\C^\wedge_\U$, and let $u':A'\to B'$ be the corresponding morphism in $\C^\wedge_{\U'}$. Then $u$ is in $\cc M(\Gamma,\U)$ (resp. in $\cc I(\Gamma,\U)$) if and only if $u'$ is in $\cc M(\Gamma,\U')$ (resp. in $\cc I(\Gamma,\U')$).
\end{s}

%

\begin{s}\label{s1623}
P.~395, Lemma 16.2.3 (iii). Consider the conditions

\nn(b) for any diagram $C\parar A\to B$ such that $C$ is in $\C$ and the two compositions coincide, there exists a local epimorphism $D\to C$ such that the two compositions $D\to C\parar A$ coincide,

\nn(c) for any diagram $C\parar A\to B$ such that $C$ is in $\C^\wedge$ and the two compositions coincide, there exists a local epimorphism $D\to C$ such that the two compositions $D\to C\parar A$ coincide,

\nn(d) for any diagram $C\parar A\to B$ such that $C$ is in $\C$ and the two compositions coincide, there exists a local isomorphism $D\to C$ such that the two compositions $D\to C\parar A$ coincide,

\nn(e) for any diagram $C\parar A\to B$ such that $C$ is in $\C^\wedge$ and the two compositions coincide, there exists a local isomorphism $D\to C$ such that the two compositions $D\to C\parar A$ coincide.

Recall that (a) is the condition that $A\to B$ is a local monomorphism. Lemma 16.2.3 p.~395 of the book implies 
%
\begin{equation}\label{e1623}
\text{Conditions (a), (b), (c), (d), (e) are equivalent.}
\end{equation}
%
Indeed, Part (iii) of the lemma says that (a), (b) and (c) are equivalent. Clearly (e) implies (c) and (d), and (d) implies (b). It suffices to check that (c) implies (e). Let $C\parar A\to B$ be as in the assumption (c), let $D\to C$ be the local epimorphism furnished by (c), and let $I$ be its image. The two compositions $I\to C\parar A$ coincide because $D\to I$ is an epimorphism, and $I\to C$ is a local isomorphism by Part (ii) of the lemma. q.e.d.
\end{s}

%

\begin{s} 
P.~397, Notation 16.2.5 (ii). The fact that 
\begin{equation}\label{1625}
\text{such a $w$ is necessarily a local isomorphism}
\end{equation}
follows from Lemma 16.2.4 (vii) p. 396.
\end{s}

%

\begin{s} 
P.~398, proof of Lemma 16.2.7: see \S\ref{fpl} p.~\pageref{fpl}.
\end{s}

% https://docs.google.com/document/d/1WVqlvFLqBG4k-afo0HVq-VV57s9WLch15mOq5CyTTg4/edit

\begin{s}
Right after Display (16.3.1) p.~399 of the book, in view of the natural isomorphism 
$$
A^a(U)\simeq\Hom_{({\C^\wedge})_{\cc{LI}}}(Q(U),Q(A)),
$$ 
the map $A^a(U')\to A^a(U)$ induced by a morphism $U\to U'$ can also be described by the diagram 
$$
Q(U)\to Q(U')\to Q(A).
$$ 
Similarly, the map $A(U)\to A^a(U)$ at the top of p.~400 of the book can also be described by the diagram 
$$
A(U)\simeq\Hom_{\C^\wedge}(U,A)\to\Hom_{({\C^\wedge})_{\cc{LI}}}(Q(U),Q(A))\simeq A^a(U).
$$ 
Then Lemma 16.3.1 can be stated as follows. 

If 
$$
U\xleftarrow sB\xr uA
$$ 
is a diagram in $\C^\wedge$ with $U$ in $\C$ and $s$ a local isomorphism, and if 
$$
v=Q(u)\circ Q(s)^{-1}\in A^a(U)\simeq\Hom_{({\C^\wedge})_{\cc{LI}}}(Q(U),Q(A)), 
$$ 
then 
\begin{equation}\label{vseu}
v\circ s=\ee(A)\circ u.
\end{equation}

Indeed, \eqref{vseu} is equivalent to $v\circ Q(s)=Q(u)$.
\end{s}

%

\begin{s} 
P.~400, Step (ii) in the proof of Lemma 16.3.2 (additional details):

We want to prove that $A\to A^a$ is a local monomorphism. In view of \eqref{e1623} p.~\pageref{e1623} it suffices to check that Condition~(b) of \S\ref{s1623} p.~\pageref{s1623} holds. 

Recall that the functor 
$$
\alpha:(\cc{LI}_U)^{\op}\to\Set,\quad(B\xr sU)\mapsto\Hom_{\C^\wedge}(B,A)
$$ 
satisfies $\colim\alpha\simeq A^a(U)$. Let $i(s):\alpha(s)\to A^a(U)$ be the coprojection, and let $f_1,f_2:U\parar A$ be two morphisms such that the compositions $U\parar A\to A^a$ coincide. By definition of the natural morphism $A\to A^a$, we have 
$$
i(\id_U)(f_1)=i(\id_U)(f_2).
$$ 
By the fact that $\cc{LI}_U$ is cofiltrant, and by Proposition 3.1.3 p.~73 of the book, there is a morphism 
$$
\pp:(B\xr sU)\to(U\xr{\id_U}U)
$$ 
in $\cc{LI}_U$ such that $\alpha(\pp)(f_1)=\alpha(\pp)(f_2)$. This means that the compositions $B\to U\parar A$ coincide. q.e.d.
\end{s}

%

\begin{s} P.~401, Step (i) of the proof of Proposition 16.3.3. See \eqref{e1623} p.~\pageref{e1623} and \eqref{1625} p.~\pageref{1625}. (As already mentioned, $B''\to B$ should be $B''\to B'$.)
\end{s}

%%%

\section{About Chapter 17}

\subsection{Brief Comments}

\begin{s}
P.~405, Chapter 17. It seems to me it would be more convenient to denote by $f^t$ the functor from $(\C_Y)^{\op}$ to $(\C_X)^{\op}$ (and {\em not} the functor from $\C_Y$ to $\C_X$) which defines $f$. To avoid confusion, we shall adopt here the following convention:

If $f:X\to Y$ is a morphism of presites, \index{morphism of presites} then we keep the notation $f^t$ for the functor from $\C_Y$ to $\C_X$, and we designate by $f^\tau$ \index{$f^\tau$} the functor from $(\C_Y)^{\op}$ to $(\C_X)^{\op}$:
\begin{equation}\label{ttau}
f^t:\C_Y\to\C_X,\quad f^\tau:(\C_Y)^{\op}\to(\C_X)^{\op}.
\end{equation}
In other words, we set $f^\tau:=(f^t)^{\op}$. 

We keep the same definition of left exactness (based on $f^t$) of $f:X\to Y$ as in the book.

The motivation for introducing the functor $f^\tau$ can be described as follows: The diagram 
$$
\begin{tikzcd}
J\ar{dr}\ar{rr}{\pp}&&I\ar{dl}\\ 
{}&\C,
\end{tikzcd}
$$ 
representing the general setting of Section 2.3 p.~50 of the book, is now replaced by the diagram 
$$
\begin{tikzcd}
(\C_Y)^{\op}\ar{dr}\ar{rr}{f^\tau}&&(\C_X)^{\op}\ar{dl}\\ 
{}&\A.
\end{tikzcd}
$$ 
(See also \S\ref{fhat} p.~\pageref{fhat} and \S\ref{fdagger} p.~\pageref{fdagger}.)
\end{s}

%

\begin{s}\label{fhat}
P.~406. Recall that, in the first line of the second display, $(\C_Y)^\wedge$ should be $\C_Y$ (twice). In notation \eqref{ttau}, Formula \eqref{275} p.~\pageref{275} gives, for $B$ in $\C_Y^\wedge$ and $U$ in $\C_X$, 
%
\begin{equation}\label{efhat1}
\fthat(B)(U)\simeq\colim_{(V\to B)\in(\C_Y)_B}\Hom_{(\C_X)}(U,f^t(V))\simeq\colim_{(U\to f^t(V))\in(\C_Y)^U}B(V).
\end{equation}
%
For the sake of emphasis, we state: 

\begin{prop}\label{p406}
The functor $\fthat$ commutes with small inductive limits (Proposition 2.7.1 p.~62 of the book, Remark~\ref{272} p.~\pageref{272}). Moreover, if $f$ is left exact, then $\fthat$ is exact (Corollary 3.3.19 p.~87 of the book).
\end{prop}

If $f:X\to Y$ is a continuous map of small topological spaces, if $B$ is in $(\oo{Op}_Y)^\wedge$ and $U$ in $\oo{Op}_X$, then \eqref{efhat1} gives 
%
\begin{equation}\label{efhat2}
\fthat(B)(U)\simeq\colim_{f^{-1}(V)\supset U}B(V).
\end{equation}
%
\end{s}

%

\begin{s}\label{fdagger} 
P.~407. Let $f:X\to Y$ be a morphism of presites and let $\A$ be a category admitting small inductive and projective limits. In the notation of \eqref{ttau} p.~\pageref{ttau}, we set $f^\dagger:=(f^\tau)^\dagger$ and $f^\ddagger:=(f^\tau)^\ddagger$: 
$$
f^\dagger,f^\ddagger:\PSh(X\A)\to\PSh(Y,\A).
$$
Then (17.1.3) and (17.1.4) follow respectively from (2.3.6) and (2.3.7) p.~52 of the book. For the sake of completeness, let us rewrite (17.1.3) and (17.1.4) (in the notation of \eqref{ttau}):
%
\begin{equation}\label{1713}
f^\dagger(G)(U)=\colim_{(f^\tau(V)\to U)\in((\C_Y)^{\op})_U}G(V),
\end{equation}
%
with $G$ in $\PSh(Y,\A),U$ in $\C_X$, $f^\tau(V)\to U$ being a morphism in $(\C_X)^{\op}$ (corresponding to a morphism $U\to f^t(V)$ in $\C_X$), 
%
\begin{equation}\label{1714}
f^\ddagger(G)(U)=\lim_{(U\to f^\tau(V))\in((\C_Y)^{\op})^U}G(V),
\end{equation}
%
with $G$ in $\PSh(Y,\A),U$ in $\C_X$, $U\to f^\tau(V)$ being a morphism in $(\C_X)^{\op}$ (corresponding to a morphism $f^t(V)\to U$ in $\C_X$).
\end{s}

%

\begin{s}\label{s408}
P.~408, comment preceding Convention 17.1.6. Let us recall the comment: 

We extend presheaves over $X$ to presheaves over $\widehat X$ using the functor $\oo h_X^\ddagger$ associated with the Yoneda embedding $\oo h_X^t=\oo h_{\C_X}$. Hence, for $F$ in $\oo{PSh}(X,\A)$ and $A$ in $\C_X^\wedge$, we have 
$$
(\oo h_X^\ddagger F)(A)=\lim_{(U\to A)\in(\C_X)_A}F(U).
$$ 
By Corollary 2.7.4 p.~63 of the book, the functor 
$$
\oo h_X^\ddagger:\oo{PSh}(X,\A)\to\oo{PSh}(\widehat X,\A)
$$ 
induces an equivalence of categories between $\oo{PSh}(X,\A)$ and the full subcategory of $\oo{PSh}(\widehat X,\A)$ whose objects are the $\A$-valued presheaves over $\widehat X$ which commute with small projective limits. 

One can add that a quasi-inverse is given by 
$$
\oo h_{X*}:\oo{PSh}(\widehat X,\A)\to\oo{PSh}(X,\A). 
$$ 
\end{s}

%

\begin{s}
P.~408, Convention 17.1.6. Recall the convention: If $F$ is an $\A$-valued presheaf over $X$ and $A$ is a presheaf of sets over $X$, then we put 
%
\begin{equation}\label{408}
F(A):=(\oo h_X^\ddagger F)(A)=\lim_{(U\to A)\in(\C_X)_A}F(U).
\end{equation}
% 
(Note that the same comment is made at the beginning of Section 17.3 p.~414.) This convention of extending each presheaf $F$ over $X$ to a presheaf, still denoted by $F$, over $\widehat X$ which commutes with small projective limits implies that we have, for $A,B$ in $\C^\wedge$, 
$$
B(A)\simeq\Hom_{\C_X^\wedge}(A,B).
$$ 

In the notation of \S\ref{opddagg} p.~\pageref{opddagg}, Convention 17.1.6 can be described as follows:

If $X$ is a site, if $\C$ is the corresponding category, if $h:\C\to\C^\wedge$ is the Yoneda embedding, if $F$ is an $\A$ valued sheaf over $X$, and if $A$ is an object of $\C^\wedge$, then Convention 17.1.6 consists in putting 
$$
F(A):=(h^{\op})^\ddagger(F)(A).
$$ 
\end{s}

%

\begin{s}\label{prepa5}
P.~409, Proposition 17.1.9 follows immediately from \eqref{prepa1} p.~\pageref{prepa1}, \eqref{prepa2} p.~\pageref{prepa2}, and \eqref{prepa3} p.~\pageref{prepa3}.
\end{s}

%

\begin{s}\label{17115b}
P.~410, Formula (17.1.15) follows from \eqref{17115} p.~\pageref{17115}.
\end{s} 

%

\begin{s}
P.~411, Definition 17.2.1 of the notions of sites and of morphism of site. Proposition~\ref{yf} p.~\pageref{yf} implies:

If $\U$ is a universe, then there is a $\U$-category $\A$ (see Definition~\ref{ducat} p.~\pageref{ducat}) whose objects are the $\U$-small categories (see Definition~\ref{small} p.~\pageref{small}) and whose morphisms are are morphisms of sites.
\end{s}

%

\begin{s}
P.~412, proof of Lemma 17.2.2 (ii), (b)$\then$(a), Step~(1): $\fthat$ is right exact by Proposition~\ref{333} p.~\pageref{333} and Proposition~\ref{p406} p.~\pageref{p406}.
\end{s}

%

\begin{s}
P.~412, proof of Lemma 17.2.2 (ii), (b)$\then$(a), Step~(3). See \S\ref{1722} p.~\pageref{1722}. This is essentially a copy and paste of the book.

Claim: if a local isomorphism $u:A\to B$ in $\C_Y^\wedge$ is either a monomorphism or an epimorphism, then $\fthat(u)$ is a local isomorphism in $\C_X^\wedge$. 

Proof of the claim: Let $V\to B$ be a morphism in $\C_Y^\wedge$ with $V$ in $\C_Y$. Then $u_V: A\times_BV\to V$ is either a monomorphism or an epimorphism by Proposition~\ref{sbcs} p.~\pageref{sbcs} and Proposition~\ref{34i} p.~\pageref{34i}. Let us show that $\fthat(u_V)$ is a local isomorphism. 

If $u_V$ is a monomorphism, $\fthat(u_V)$ is a local isomorphism by assumption. 

If $u_V$ is an epimorphism, then $u_V$ has a section $s:V\to A\times_BV$. Since $u_V$ is a local isomorphism by Lemma 16.2.4 (i) p.~395 of the book, $s$ is a local isomorphism. Since 
$$
\fthat(u_V)\circ\fthat(s)\simeq\id_{f^t(V)}
$$ 
is a local monomorphism, and $\fthat(s)$ is a local epimorphism by Step~(2), Lemma 16.2.4 (vi) p.~396 of the book implies that $\fthat(u_V)$ is a local monomorphism. Since $\fthat(u_V)$ is an epimorphism by Step~(2), we see that $\fthat(u_V)$ is a local isomorphism. This proves the claim.

Taking the inductive limit with respect to $V\in(\C_Y)_B$, we conclude by Proposition 16.3.4 p.~401 of the book that $\fthat(u)$ is a local isomorphism.
\end{s}

%

\begin{s}\label{1724ii}
P.~413, Definition 17.2.4 (ii): see  Remark~\ref{272b} p.~\pageref{272b}.
\end{s}

%

\begin{s}
P.~413. Lemma 17.2.5 (ii) and Exercise 2.12 (ii) p.~66 of the book imply: If $f:X\to Y$ is weakly left exact, then $\fthat:\C_Y^\wedge\to\C_X^\wedge$ commutes with projective limits indexed by small connected categories.
\end{s} 

%

\begin{s}\label{1725ii} 
P.~413, Lemma 17.2.5 (ii). Here is a corollary: 

Let $f:X\to Y$ be a weekly left exact morphism of sites such that $\fthat(u)$ is a local epimorphism if and only if $u$ is a local epimorphism. Then $\fthat(u)$ is a local monomorphism if and only if $u$ is a local monomorphism, and $\fthat(u)$ is a local isomorphism if and only if $u$ is a local isomorphism. 
\end{s} 

%

\begin{s}
P.~413, Example 17.2.7 (i). Recall that $f:X\to Y$ is a continuous map of small topological spaces. As explained in the book, to see that $f$ is a morphism of sites, it suffices to check that, if $u:B\to V$ is a local epimorphism in $(\oo{Op}_Y)^\wedge$ with $V$ in $\oo{Op}_Y$, then $\fthat(B)\to f^{-1}(V)$ is a local epimorphism in $(\oo{Op}_X)^\wedge$. This follows immediately from \S\ref{390} p.~\pageref{390} and \eqref{efhat2} p.~\pageref{efhat2}. 
\end{s}

%

\begin{s} 
P.~414, Definition 17.2.8 (minor variant):

\begin{df}[Definition 17.2.8 p.~414, Grothendieck topology]\label{1778} 
Let $X$ be a small presite. We assume, as we may, that the hom-sets of $\C_X$ are disjoint. A {\em Grothendieck topology}\index{Grothendieck topology} on $X$ is a set $\cc T$ of morphisms of $\C_X$ which satisfies Axioms LE1-LE4 p.~390. Let $\cc T'$ and $\cc T$ be Grothendieck topologies. We say that $\cc T$ is {\em stronger than} $\cc T'$, or that $\cc T'$ is {\em weaker than} $\cc T$, if $\cc T'\subset\cc T$. 
\end{df}

Let $(\cc T_i)$ be a family of Grothendieck topologies. We observe that $\bigcap\cc T_i$ is a Grothendieck topology, and we denote by $\bigvee\cc T_i$ the intersection of all the Grothendieck topologies containing $\bigcup\cc T_i$.
\end{s}

%

\begin{s}
P.~415, Isomorphism (17.3.1). Recall briefly the setting. We have: 
$$
F\in\oo{PSh}(X,\A),\quad M\in\A,\quad U\in\C_X,
$$ 
and we claim 
%
\begin{equation}\label{1731}
\HOM_{\oo{PSh}(X,\A)}(M,F)(U)\simeq\Hom_\A(M,F(U)).
\end{equation} 
%
Here and in the sequel, we denote again by $M$ the constant presheaves over $X$ and $U$ attached to the object $M$ of $\A$. Note that, by \S\ref{s408} p.~\pageref{s408}, this isomorphism can be written 
$$
\HOM_\A(M,F)\simeq\Hom_\A(M,F(\ )).
$$ 
To prove \eqref{1731}, observe that we have  
$$
\HOM_{\oo{PSh}(X,\A)}(M,F)(U)\simeq\Hom_{\oo{PSh}(U,\A)}(\oo j_{U\to X*}M,\oo j_{U\to X*}F)
$$ 
$$
\simeq\Hom_{\oo{PSh}(U,\A)}(M,\oo j_{U\to X*}F), 
$$ 
the two isomorphisms following respectively from the definition of $\HOM_{\oo{PSh}(X,\A)}$ given in (17.1.14) p.~410 of the book, and from the definition of the functor $\oo j_{U\to X*}$, so that we must show 
$$
\Hom_{\oo{PSh}(U,\A)}(M,\oo j_{U\to X*}F)\simeq\Hom_\A(M,F(U)).
$$
We define maps 
$$
\begin{tikzcd}
\Hom_{\oo{PSh}(U,\A)}(M,\oo j_{U\to X*}F)\ar[yshift=0.7ex]{r}{\pp}&\Hom_\A(M,F(U))\ar[yshift=-0.7ex]{l}{\psi}
\end{tikzcd} 
$$ 
as follows: If $p:M\to\oo j_{U\to X*}F$ is a morphism in $\oo{PSh}(U,\A)$, given by morphisms $p(V\to U):M\to F(V)$ in $\A$, then we put $\pp(p):=p(U\xr{\id_U}U)$; if $a:M\to F(U)$ is a morphism in $\A$, then we put $\psi(a)(V\xr cU):=F(c)\circ a$; and we check that $\pp$ and $\psi$ are mutually inverse bijections. 
\end{s}

%

\begin{s} 
P.~418, proof of Lemma 17.4.2 (minor variant): Consider the natural morphisms 
$$
\colim\alpha\xr f\colim\alpha\circ\mu_u^{\op}\circ\lambda_u^{\op}\xr g\colim\alpha\circ\mu_u^{\op}\xr h\colim\alpha.
$$
We must show that $g\circ f$ is an isomorphism. The equality $h\circ g\circ f=\id_{\colim\alpha}$ is easily checked. Being a right adjoint, $\mu_u^{\op}$ is left exact, hence cofinal by Lemma 3.3.10 p.~84 of the book, and $h$ is an isomorphism. q.e.d.
\end{s}  

%

\begin{s}\label{1744i}
P.~419, proof of Proposition 17.4.4: 

First sentence of the proof: see \S\ref{fpl} p.~\pageref{fpl}. 

Step~(i), Line~4: The fact that $K^{\op}$ is cofinally small and filtrant results from \S\ref{poc} p.~\pageref{poc} together with Lemma 16.2.8 p.~398, Lemma 16.2.7 p.~398, and Proposition 3.2.1 (iii) p.~78 of the book. 

Step~(i), additional details about the chain of isomorphisms at the bottom of p.~419 of the book: The chain reads 
$$
\prod_i\ F^b(A_i)
\overset{(\text a)}{\simeq}\prod_i\ \col_{(B_i\to A_i)\in\cc{LI}_{A_i}}F(B_i)
\overset{(\text b)}{\simeq}\col_{(B_i\to A_i)_{i\in I}\in K}\ \prod_i\ F(B_i)
$$
$$
\overset{(\text c)}{\simeq}\col_{(B_i\to A_i)_{i\in I}\in K}\ F\left(``\bigsqcup_i\!"B_i\right)
\overset{(\text d)}{\simeq}\col_{(B\to A)\in\cc{LI}_A}F(B)
\overset{(\text e)}{\simeq}F^b(A), 
$$ 
and the isomorphisms can be justified as follows: 

\nn(a) definition of $F^b$, 

\nn(b) $\A$ satisfies IPC,

\nn(c) $F$ commutes with small projective limits, 

\nn(d) an inductive limit of local isomorphisms is a local isomorphism by Proposition 16.3.4 p.~401 of the book, 

\nn(e) definition of $F^b$. 
\end{s} 

%

\begin{s} 
P.~419, proof of Proposition 17.4.4, Step (ii). More details: The morphism $\ee_b(F^b)(A):F^b(A)\to F^{bb}(A)$ is obtained as the composition 
$$
F^b(A)\xr f\col_{(B\to A)\in\cc{LI}_A}F^b(A)\xr g\col_{(B\to A)\in\cc{LI}_A}F^b(B). 
$$ 
Moreover, $f$ is an isomorphism by Lemma 2.1.12 p.~41 of the book, and $g$ is an isomorphism by Lemma 17.4.2 p.~418 of the book.
\end{s}

%

\subsection{Proposition 17.4.4 (p.~420)}

We draw a few diagrams with the hope of helping the reader visualize the argument in Step (ii) of the proof of Proposition 17.4.4. 

An object of the category 
$$
M\big[J\to K\leftarrow M[I\to K\leftarrow K]\big]
$$ 
can be represented by a diagram 
$$
\begin{tikzcd}
B'\ar{d}[swap]{\beta}&B'\times_BA\ar{d}&A'\ar{l}[swap]{(u',\alpha)}\ar{r}{(v',\alpha)}\ar{d}{\alpha}&C'\times_CA\ar{d}&C'\ar{d}{\gamma}\\ 
B&A\ar[equal]{r}&A\ar[equal]{r}&A&C, 
\end{tikzcd}
$$ 
and it is clear that this category is equivalent to $\cc E^{\op}$. 

Recall that $D:=B\ ``\!\sqcup\!"\!\!_A\ C$, let $E$ be one of the objects $A,B,C,$ or $D$, and consider the ``obvious'' functors 
$$
\begin{tikzcd}
\cc E\ar{r}{p_E}\ar{rd}[swap]{q_E}&\cc{LI}_E\ar{d}{j_E}\\ 
{}&\C_X^\wedge
\end{tikzcd}
$$ 
($p_E$ is defined in the book, $j_E$ is the forgetful functor, and $q_E$ is the composition). We also define $r_E:\cc{LI}_E\to\cc E$ by mapping the object $E''\to E$ of $\cc{LI}_E$ to the object 
$$
\begin{tikzcd}
B\times_EE''\ar{d}&A\times_EE''\ar{l}\ar{r}\ar{d}&C\times_EE''\ar{d}\\ 
B&A\ar{r}\ar{l}&C
\end{tikzcd}
$$ 
of $\cc E$. One checks that $(p_E,r_E)$ is a pair of adjoint functors. In particular $p_E$ is cocofinal. We have
$$
F^b(D)
\overset{(\text a)}{\simeq}\col_{y\in\cc{LI}_D}F(j_D(y))
\overset{(\text b)}{\simeq}\col_{x\in\cc E}F(q_D(x))
$$
$$
\overset{(\text c)}{\simeq}\col_{x\in\cc E}F\left(q_B(x)``\!\!\!\bigsqcup_{q_A(x)}\!\!\!"q_C(x)\right)
\overset{(\text d)}{\simeq}\col_{x\in\cc E}(F(q_B(x))\times_{F(q_A(x))}F(q_C(x)))
$$
$$
\overset{(\text e)}{\simeq}\left(\col_{x\in\cc E}F(q_B(x))\right)\times_{\col_{x\in\cc E}F(q_A(x))}\left(\col_{x\in\cc E}F(q_C(x))\right)
$$ 
$$
\overset{(\text f)}{\simeq}\left(\col_{y\in\cc{LI}_B}F(j_B(y))\right)\times_{\col_{y\in\cc{LI}_A}F(j_A(y))}\left(\col_{y\in\cc{LI}_C}F(j_C(y))\right)
$$ 
$$
\overset{(\text g)}{\simeq}F^b(B)\times_{F^b(A)}F^b(C).
$$ 
Indeed, the isomorphisms can be justified as follows: 

\nn(a) definition of $F^b$, 

\nn(b) cocofinality of $p_D$,

\nn(c) definition of $p_E$, 

\nn(d) left exactness of $F$, 

\nn(e) exactness of filtrant inductive limits in $\A$, 

\nn(f) cocofinality of $p_D$, 

\nn(g) definition of $F^b$. 

%%

\subsection{Brief Comments}

\begin{s}\label{fau}
P.~421, first display: 
$$
F^a(U)\simeq\colim_{(U\to A)\in\cc{LI}_U}F(A).
$$ 
Lemma 16.2.8 p.~398 of the book, and its proof, show that $F^a$ does {\em not} depend on the universe such that $\C$ is a small category and $\A$ satisfies (17.4.1) p.~417 of the book.
\end{s}

%

\begin{s}P.~421, proof of Lemma 17.4.6 (i): The category $\cc{LI}_U$ is cofiltrant by Lemma 16.2.7 p.~398 of the book, small filtrant inductive limits are exact in $\A$ by Display (17.4.1) p.~417 of the book, exact functors preserve monomorphisms by Proposition~\ref{34i} p.~\pageref{34i}.
\end{s}

%

\begin{s} 
P.~422, the first sentence of the proof of Theorem 17.4.7 (iv) follows from Corollary~\ref{bre} p.~\pageref{bre}. 
\end{s}

%

\begin{s} 
P.~423, end of the proof of Theorem 17.4.9 (iv): the functor $(\ )^a$ is exact by Theorem 17.4.7 (iv) p.~421 of the book.
\end{s}

%

\begin{s} 
P.~424, proof of Theorem 17.5.2 (i). With the convention that a diagram of the form 
$$
\begin{tikzcd} 
\C_1\ar[xshift=-0.7ex]{d}[swap]{L}\\ \C_2\ar[xshift=0.7ex]{u}[swap]{R}
\end{tikzcd}
$$ 
means: ``$(L,R)$ is a pair of adjoint functors'', the proof of Theorem 17.5.2 (i) in the book can be visualized by the diagram 
$$
\begin{tikzcd} 
\oo{PSh}(Y,\A)\ar[xshift=-0.7ex]{d}[swap]{f^\dagger}\\ 
\oo{PSh}(X,\A)\ar[xshift=0.7ex]{u}[swap]{f_*}\ar[xshift=-0.7ex]{d}[swap]{(\ )^a}\\ 
\oo{Sh}(X,\A).\ar[xshift=0.7ex]{u}[swap]{\iota}
\end{tikzcd}
$$ 
\end{s}

%

\begin{s} 
P.~424, proof of Theorem 17.5.2 (iv). As already mentioned, there is a typo: ``The functor $f^\dagger$ is left exact'' should be ``The functor $f^\dagger$ is exact''. 
\end{s}

%

\begin{s}\label{1761}
P.~424, Definition 17.6.1. By Lemma 17.1.8 p.~409 of the book, a morphism 
$$
\begin{tikzcd} 
C\ar{rr}\ar{dr}&&B\ar{dl}\\ 
{}&A
\end{tikzcd}
$$ 
in $\C_A^\wedge$ is a local epimorphism if and only if $C\to B$ is a local epimorphism in $\C_X^\wedge$.
\end{s}

%

\begin{s}
P.~424, sentence following Definition 17.6.1: ``It is easily checked that we obtain a Grothendieck topology on $\C_A$''. The verification  of LE1, LE2, and LE3 is straightforward. Axiom LE4 follows from Parts (iii) and (ii) of Lemma 17.2.5 p.~413 of the book.
\end{s} 

%

\begin{s}\label{1761b} 
P.~424, Definition 17.6.1. Here is an observation which follows from \S\ref{1725ii} p.~\pageref{1725ii} and Lemma 17.2.5 (iii) p.~413 of the book: 

In the setting of Definition 17.6.1, let $B\to A$ be a morphism in $\C_X^\wedge$, let $u:C\to B$ be a morphism in $\C_X^\wedge$, and let $v:(C\to A)\to(B\to A)$ be the corresponding morphism in $\C_A^\wedge$. Then $u$ is a local epimorphism if and only if $v$ is a local epimorphism, $u$ is a local monomorphism if and only if $v$ is a local monomorphism, and $u$ is a local isomorphism if and only if $v$ is a local isomorphism. 
\end{s} 

%

\begin{s}
P.~425, proof of Proposition 17.6.3: 

Step (i): $\oo j_{A\to X}$ is weakly left exact by Lemma 17.2.5 (iii) p.~413 of the book, and $(\,\cdot\,)^a$ is exact by Theorem 17.4.7 (iv) p.~421 of the book.

Step (ii): ``$f$ factors as $X\xr{\oo j_{A\to X}}A\xr gY$'': see Definition 17.2.4 (ii) p.~413 of the book and Remark~\ref{272b} p.~\pageref{272b}. The isomorphism $f^{-1}\simeq\oo j_{A\to X}^{-1}\circ g^{-1}$ follows from Proposition 17.5.3 p.~424 of the book.
\end{s} 

% 

\begin{s} 
P.~425, Display (17.6.1): Putting $j:=\oo j_{A\to X}$, we have the adjunctions 
$$
\begin{tikzcd}
\oo{Sh}(A,\A)\ar[xshift=-4ex]{d}[swap]{j^{-1}}\ar[xshift=4ex]{d}{j^\ddagger}\\ 
\oo{Sh}(X,\A).\ar{u}{j_*}
\end{tikzcd}
$$ 

For the functor $j_*:\oo{Sh}(X,\A)\to\oo{Sh}(A,\A)$, see Proposition 17.5.1 p.~423 of the book. 

For the functor $j^{-1}:\oo{Sh}(A,\A)\to\oo{Sh}(X,\A)$, see last display of p.~423 of the book. 

For the functor $j^\ddagger:\oo{Sh}(A,\A)\to\oo{Sh}(X,\A)$, see Proposition 17.6.2 p.~425 of the book.
\end{s}

%

\begin{s} 
P.~426, proof of Proposition 17.6.7 (i). The isomorphism 
\begin{equation}\label{e406}
\fthat(V\times B)\simeq f^t(V)\times\fthat(B)
\end{equation} 
follows from Proposition~\ref{p406} p.~\pageref{p406}, and we have 
\begin{align*} 
%
\oo j_{B\to Y}^\ddagger\big(f_{B*}(G)(V)\big)&\simeq f_{B*}(G)(V\times B\to B)&\text{by (17.1.12) p. 409}\\ \\ 
%
&\simeq G\big(\fthat(V\times B)\to\fthat(B)\big)&\text{by (17.1.6) p. 408}\\ \\ 
%
&\simeq G\big(f^t(V)\times A\to A\big)&\text{by \eqref{e406}}, 
% 
\end{align*} 
as well as 
\begin{align*} 
%
f_*\big(\oo j_{A\to X}^\ddagger(G)(V)\big)&\simeq\oo j_{A\to X}^\ddagger(G)(f^t(V))\\ \\ 
%
&\simeq G\big(f^t(V)\times A\to A\big)&\text{by (17.1.12) p. 409}. 
% 
\end{align*}  
\end{s} 

% 

\begin{s}\label{1768}
P.~427, proof of Proposition 17.6.8, Step (i). The isomorphism 
$$
\oo j_{A\to X}^\ddagger(\oo j_{A\to X*}(G)(U))\simeq\oo j_{A\to X*}(G)(U\times A\to A)
$$ 
follows from (17.1.12) p.~409 of the book. The fact that $p:A\times U\to U$ is a local isomorphism follows from the fact that the obvious square 
$$
\begin{tikzcd} 
A\times U\ar{r}\ar{d}&U\ar{d}\\ 
A\ar{r}&\oo{pt}_X
\end{tikzcd}
$$ 
is cartesian and the bottom arrow is a local isomorphism by assumption. 
\end{s} 

%

\begin{s}
P.~427, proof of Proposition 17.6.8, Step (ii). Let $v:V\to A$ be a morphism in $\C_X^\wedge$. Here is a proof of the fact that 
\begin{equation}\label{1768ii}
\begin{tikzcd} 
V\ar{rr}{(\id_V,v)}\ar{dr}&&V\times A\ar{dl}\\ 
{}&A
\end{tikzcd}
\end{equation} 
is a local isomorphism in $\C_A^\wedge$. 

As $V\times A\to V$ is a local isomorphism in $\C_X^\wedge$ by \S\ref{1768}, and $V\to V\times A\to V$ is the identity of $V$, Lemma 16.2.4 (vii) p.~396 of the book implies that $V\to V\times A$ is a local isomorphism in $\C_X^\wedge$, and thus, by \S\ref{1761b} p.~\pageref{1761b}, that \eqref{1768ii} is a local isomorphism in $\C_A^\wedge$. 
\end{s} 

%

\begin{s}\label{a428}
P.~428, just after Definition 17.6.10: $((\ )_A,\Gamma_A(\ ))$ is a pair of adjoint functors: this follows from Theorem 17.5.2 (i) p.~424 of the book. 
\end{s} 

%

\begin{s}\label{429}
P.~429, top. By \S\ref{phistar} p.~\pageref{phistar} and Corollary~\ref{bre2} p.~\pageref{bre2}, the functor $\Gamma(A;\ )$ commutes with small projective limits. 
\end{s} 

%

\begin{s} 
P.~430, first sentence of the proof of Proposition 17.7.1 (i). Let us make a general observation. 

Let $X$ be a site. In this \S, for any $A$ in $\C_X^\wedge$, we denote the corresponding site by $A'$ instead of $A$. We also identify $\C_{A'}^\wedge$ to $(\C_X^\wedge)_A$ (see Lemma 17.1.8 p.~409 of the book). In particular, we get $\oo{pt}_{A'}\simeq(A\xr{\id_A}A)\in\C_{A'}^\wedge$. 

Let $A\to B$ be a local isomorphism in $\C_X^\wedge$, and let us write $\omega$ for ``the'' terminal object $\oo{pt}_{B'}\simeq(B\xr{\id_B}B)$ of $\C_{B'}^\wedge$. We claim that 
\begin{equation}\label{1771i1}
(A\to B)\to\omega 
\end{equation} 
is a local isomorphism in $\C_{B'}^\wedge$.

Proof: \eqref{1771i1} is a local epimorphism by \S\ref{1761} p.~\pageref{1761}. It remains to check that 
\begin{equation}\label{1771i2}
(A\to B)\to(A\to B)\times_\omega(A\to B)\simeq(A\times_BA\to B)
\end{equation} 
is a local epimorphism. But this follows again from \S\ref{1761} p.~\pageref{1761}. $\square$ 

Consider the morphism of presites $B'\to A'$ induced by $A\to B$ and note that the square
$$
\begin{tikzcd} 
X\ar{rr}{\oo j_{A\to X}}\ar[equal]{d}&&A'\\ 
X\ar{rr}[swap]{\oo j_{B\to X}}&&B'.\ar{u}
\end{tikzcd}
$$ 
commutes.
\end{s} 

%

\begin{s} 
P.~430, proof of Proposition 17.7.3. The third isomorphism follows, as indicated, from Proposition 17.6.7 (ii) p.~426 of the book. The fifth isomorphism follows from (17.6.2) (ii) p.~426 of the book. 
\end{s} 

%

\begin{s}\label{175i}
P.~431, Exercise 17.5 (i). Put $PX:=\oo{PSh}(X,\A),\ SX:=\oo{Sh}(X,\A)$, and define $PY$ and $SY$ similarly. Let
$$
\begin{tikzcd} 
SY\ar[xshift=0.7ex]{d}{\iota_Y}\\ 
PY\ar[xshift=-0.7ex]{u}{a_Y}\ar[xshift=-0.7ex]{d}[swap]{f^\dagger}\\ 
PX\ar[xshift=0.7ex]{u}[swap]{f_*}\ar[xshift=-0.7ex]{d}[swap]{a_X}\\ 
SX\ar[xshift=0.7ex]{u}[swap]{\iota_X}
\end{tikzcd}
$$ 
be the obvious diagram of adjoint functors. We must show 
$$
a_X\circ f^\dagger\circ \iota_Y\circ a_Y\simeq a_X\circ f^\dagger. 
$$ 
Let $F$ be in $SX$ and $G$ be in $PY$. We have (omitting most of the parenthesis) 
$$
\Hom_{SX}(a_Xf^\dagger\iota_Ya_YG,F)\simeq
\Hom_{PX}(f^\dagger\iota_Ya_YG,\iota_XF)\simeq
\Hom_{PY}(\iota_Ya_YG,f_*\iota_XF)
$$
$$
\overset{(\text a)}{\simeq}
\Hom_{PY}(\iota_Ya_YG,\iota_Ya_Yf_*\iota_XF)\simeq
\Hom_{SY}(a_Y\iota_Ya_YG,a_Yf_*\iota_XF)
$$
$$
\overset{(\text b)}{\simeq}
\Hom_{SY}(a_YG,a_Yf_*\iota_XF)\simeq
\Hom_{PY}(G,\iota_Ya_Yf_*\iota_XF)\overset{(\text c)}{\simeq}
\Hom_{PY}(G,f_*\iota_XF)
$$ 
$$ 
\simeq 
\Hom_{PX}(f^\dagger G,\iota_XF)\simeq 
\Hom_{SX}(a_Xf^\dagger G,F)
$$ 
where (a) and (c) follow from the fact that the presheaf $f_*\iota_XF$ is actually a sheaf (Proposition 17.5.1 p.~423 of the book), (b) follows from the isomorphism 
$$
a_Y\circ \iota_Y\circ a_Y\simeq a_Y,
$$ 
which holds by Lemma 17.4.6 (ii) p.~421 of the book, and the other isomorphisms hold by adjunction. 
\end{s} 

%

\begin{s} 
P.~431, Exercise 17.5 (ii). By \S\ref{1761b} p.~\pageref{1761b} we have, for $U$ in $\C_X$ and $U\to A$ in $\C_A$, an isomorphism 
$$
\cc{LI}_{U\to A}\simeq\cc{LI}_U.
$$
Exercise 17.5 (ii) follows immediately. 
\end{s} 

%%%

\section{About Chapter 18} 

\subsection{Brief Comments}

%

\begin{s}
P.~437, Theorem 18.1.6 (v). If $X$ is a site, if $R$ a ring, if $F$ and $G$ are complexes of $R$-modules, then the complex of abelian groups $\RHom_R(F,G)$ (see Corollary 14.3.2 p.~356 of the book) does {\em not} depend on the universe chosen to define it (the universe in question being subject to the obvious conditions). This follows from \S\ref{14112} p.~\pageref{14112} and \S\ref{fau} p.~\pageref{fau}. 
\end{s} 

%

\begin{s} 
P.~437, proof of Theorem 18.1.6 (v). We prove 
$$
\Hom_{\cc R}(\cc R_U,F)\simeq F(U).
$$ 
As 
$$
\Hom_{\cc R}(\cc R_U,F)\simeq\Hom_{\cc R}(\oo j_{U\to X*}^{-1}(\cc R|U),F)\simeq
\Hom_{\cc R|U}(\cc R|U,F|U), 
$$ 
we only need to verify 
$$
\Hom_{\cc R|U}(\cc R|U,F|U)\simeq F(U).
$$ 
We shall define maps 
$$
\begin{tikzcd}
\Hom_{\cc R|U}(\cc R|U,F|U)\ar[yshift=0.7ex]{r}{\pp}&F(U)\ar[yshift=-0.7ex]{l}{\psi}
\end{tikzcd}
$$ 
and leave it to the reader to check that they are mutually inverse. 

Definition of $\pp$: Let $\theta$ be in $\Hom_{\cc R|U}(\cc R|U,F|U)$. In particular, for each morphism $f:V\to U$ in $\C_X$ we have a map $\theta(f):\cc R(V)\to F(V)$, and we put $\pp(\theta):=\theta(\id_U)(1)$. 

Definition of $\psi$: Let $x$ be in $F(U)$. For each morphism $f:V\to U$ in $\C_X$ we define $\psi(x)(f):\cc R(V)\to F(V)$ by $\psi(x)(f)(\lambda):=\lambda\,F(f)(x)$.
\end{s}

%

\begin{s}\label{a438}
P.~438, end of Section 18.1: $\Gamma_A$ is left exact by \S\ref{a428} p.~\pageref{a428}. Moreover, $\Gamma(A;\ )$ commutes with small projective limits by \S\ref{429} p.~\pageref{429}, and is thus left exact by Proposition~\ref{333} p.~\pageref{333}. 
\end{s} 

% 

\begin{s}\label{homrr}
P.~438, bottom: One can add that we have $\HOM_{\cc R}(\cc R,F)\simeq F$ for all $F$ in $\oo{PSh}(\cc R)$. 
\end{s} 

% 

\begin{s}\label{neutral}
P.~439, after Definition 18.2.2: One can add that we have 
$$
\cc R\overset{\text{\tiny psh}}{\otimes}_{\cc R}F\simeq F
$$ 
for $F$ in $\oo{PSh}(\cc R)$ and 
$$
\cc R\otimes_{\cc R}F\simeq F
$$ 
for $F$ in $\Mod(\cc R^{\op})$, as well as 
$$F\overset{\text{\tiny psh}}{\otimes}_{\cc R}\cc R\simeq F
$$ 
for $F$ in $\oo{PSh}(\cc R^{\op})$ and 
$$
F\otimes_{\cc R}\cc R\simeq F
$$ 
for $F$ in $\Mod(\cc R^{\op})$. 
\end{s} 

% 

\begin{s}
P.~441. The proof of Proposition 18.2.5 uses Display (17.1.11) p.~409 of the book and \S\ref{175i} p.~\pageref{175i}. 
\end{s} 

% 

\begin{s} 
P.~441. In the notation of Remark 18.2.6 we have 
$$
\Hom_{\cc R_3}({}_3M_2\otimes_{\cc R_2}{}_2M_1,{}_3M_4)\simeq
\Hom_{\cc R_2}({}_2M_1,\HOM_{\cc R_3}({}_3M_2,{}_3M_4)),
$$
$$
\HOM_{\cc R_3}({}_3M_2\otimes_{\cc R_2}{}_2M_1,{}_3M_4)\simeq
\HOM_{\cc R_2}({}_2M_1,\HOM_{\cc R_3}({}_3M_2,{}_3M_4)),
$$ 
$$
\Hom_{\cc R_3^{\op}}({}_1M_2\otimes_{\cc R_2}{}_2M_3,{}_4M_3)\simeq
\Hom_{\cc R_2^{\op}}({}_1M_2,\HOM_{\cc R_3^{\op}}({}_2M_3,{}_4M_3)),
$$ 
$$
\HOM_{\cc R_3^{\op}}({}_1M_2\otimes_{\cc R_2}{}_2M_3,{}_4M_3)\simeq
\HOM_{\cc R_2^{\op}}({}_1M_2,\HOM_{\cc R_3^{\op}}({}_2M_3,{}_4M_3)).
$$ 
More generally, if $\cc{R,S,T}$ are $\cc O_X$-algebras, if $F$ is a $(\cc T\otimes_{\cc O_X}\cc R^{\op})$-module, if $G$ is an $(\cc R\otimes_{\cc O_X}\cc S)$-module, and if $H$ is an $(\cc S\otimes_{\cc O_X}\cc T)$-module, then we have 
$$ 
\Hom_{\cc S\otimes_{\cc O_X}\cc T}(F\otimes_{\cc R}G,H)\simeq
\Hom_{\cc R\otimes_{\cc O_X}\cc S}(G,\HOM_{\cc T}(F,H)), 
$$ 
\begin{equation}\label{HOM}
\HOM_{\cc S\otimes_{\cc O_X}\cc T}(F\otimes_{\cc R}G,H)\simeq
\HOM_{\cc R\otimes_{\cc O_X}\cc S}(G,\HOM_{\cc T}(F,H)). 
\end{equation}
\end{s} 

%

\begin{s}
P.~442, proof of Proposition 18.2.7. Here are additional details. 

Proof of (18.2.12): We must show 
\begin{equation}\label{18212}
F_A\simeq\cc R_A\otimes_{\cc R}F\simeq k_{XA}\otimes_{k_X}F. 
\end{equation} 
We have 
$$
F_A\overset{(\text a)}{\simeq}
\oo j_{A\to X}^{-1}(F|_A)\overset{(\text b)}{\simeq}
\oo j_{A\to X}^{-1}(\cc R|_A\otimes_{\cc R|_A}F|_A)\overset{(\text c)}{\simeq}
(\oo j_{A\to X}^{-1}(\cc R|_A))\otimes_{\cc R}F\overset{(\text d)}{\simeq} 
\cc R_A\otimes_{\cc R}F.
$$ 
Indeed, (a) and (d) hold by Definition 17.6.10 (i) and Display (17.6.5) p.~428 of the book, (b) follows from \S\ref{neutral}, (c) follows from (18.2.6) p.~441 of the book. The isomorphism $F_A\simeq k_{XA}\otimes_{k_X}F$ is a particular case of the isomorphism $F_A\simeq\cc R_A\otimes_{\cc R}F$ just proved. 

Proof of (18.2.13): We must show 
\begin{equation}\label{18213} 
\Gamma_A(F)\simeq\HOM_{\cc R}(\cc R_A,F)\simeq\HOM_{k_X}(k_{XA},F). 
\end{equation}
We have 
$$
\HOM_{\cc R}(\cc R_A,F)\overset{(\text a)}{\simeq}
\HOM_{\cc R}(\cc R\otimes_{k_X}k_{XA},F)
$$
$$
\overset{(\text b)}{\simeq}
\HOM_{k_X}(k_{XA},\HOM_{\cc R}(\cc R,F))\overset{(\text c)}{\simeq}
\HOM_{k_X}(k_{XA},F), 
$$ 
where (a) follows from \eqref{18212}, (b) follows from Display (18.2.4) p.~439 of the book (which is a particular case of \eqref{HOM}), and (c) follows from \S\ref{homrr}. Let us record the isomorphism 
%
\begin{equation}\label{record}
\HOM_{\cc R}(\cc R_A,F)\simeq\HOM_{k_X}(k_{XA},F). 
\end{equation} 
%
We also have for $G$ in $\Mod(\cc R)$ 
$$
\Hom_{\cc R}(G,\HOM_{k_X}(k_{XA},F))\overset{(\text a)}{\simeq}
\Hom_{\cc R}(G\otimes_{k_X}k_{XA},F)\overset{(\text b)}{\simeq}
\Hom_{\cc R}(\oo j_{A\to X}^{-1}\oo j_{A\to X*}G,F)
$$
$$
\overset{(\text c)}{\simeq}
\Hom_{\cc R}(G,\oo j_{A\to X}^\ddagger\oo j_{A\to X*}F)\overset{(\text d)}{\simeq}
\Hom_{\cc R}(G,\Gamma_A(F)), 
$$ 
where (a) follows from \eqref{HOM} with 
$$
(k_X;k_X,\cc R,k_X;k_{XA},G,F)
$$ 
instead of 
$$
(\cc O_X;\cc{R,S,T};F,G,H),
$$ 
(b) follows from \eqref{18212}, Definition 17.6.10 (i), and Display (17.6.5) p.~428 of the book, (c) follows by adjunction, and (d) by Definition 17.6.10 (ii) p.~428 of the book. 
\end{s} 

%%

\subsection{Lemma 18.5.3 (p. 447)} 

We give additional details about the proof of Lemma 18.5.3 of the book (stated below as Lemma~\ref{l1853} p.~\pageref{l1853}) with the hope of helping the reader. We start with a technical lemma.

\begin{lem}\label{techlem1}
Let $R$ be a ring, let $A$ be a right $R$-module, let $B$ be a left $R$-module, let $n$ be a positive integer, and let 
$$
(a_i)_{i=1}^n,\quad(b_i)_{i=1}^n
$$
be two families of elements belonging respectively to $A$ and $B$. Then Conditions \emph{(i)} and \emph{(ii)} below are equivalent:

\nn\emph{(i)} We have $\sum_{i=1}^n\,a_i\otimes b_i=0$ in $A\otimes_RB$. 

\nn\emph{(ii)} There are positive integers $\ell$ and $m$ with $\ell\ge n$, and there are three families 
$$
(a_i)_{i=n+1}^\ell,\quad(\lambda_{ij})_{1\le i\le\ell,1\le j\le m},\quad(b'_j)_{j=1}^m
$$ 
of elements belonging respectively to $A$, $R$, and $B$, such that, if we set $b_i=0$ for $n<i\le\ell$, we have:
%
\begin{equation}\label{lij1}
\sum_{j=1}^m\ \lambda_{ij}\,b'_j=b_i\quad(\forall\ 1\le i\le\ell),
\end{equation}
%
\begin{equation}\label{lij2}
\sum_{i=1}^\ell\ a_i\,\lambda_{ij}=0\quad(\forall\ 1\le j\le m).
\end{equation}
\end{lem} 

\begin{proof} 
Implication (ii)$\then$(i) is clear. To prove Implication (i)$\then$(ii), we assume (i), and we choose a set $I$ containing $\{1,\dots,\ell\}$, where $\ell$ is an integer $\ge n$ to be determined later, such that there is a family $(a_i)_{i\in I}$ which completes the family $(a_i)_{1\le i\le n}$ and generates $A$. We write $C$ for the kernel of the epimorphism 
$$
f:R^{\oplus I}\epi A,\quad(\mu_i)\mapsto\sum_{i\in I}a_i\,\mu_i.
$$ 
In particular we have exact sequences 
$$
C\xr gR^{\oplus I}\xr fA\to0,\qquad C\otimes_RB\xr{g'}B^{\oplus I}\xr{f'}A\otimes_RB\to0,
$$ 
with 
$$
g'\big((\mu_i)\otimes b)\big)=(\mu_i\,b),\quad f'((b''_i))=\sum_{i\in I}a_i\otimes b''_i.
$$
Put $b_i:=0$ for $i$ in $I\setminus\{1,\dots,\ell\}$. The family $(b_i)_{i\in I}$ is in $\Ker f'$, and thus in $\Ima g'$. The condition $(b_i)\in\Ima g'$ means that there is a positive integer $m$, a family 
$$
(\lambda_{ij})_{i\in I,1\le j\le m}
$$ 
of elements of $R$ such that 
$$
(\lambda_{ij})_i\in C\subset R^{\oplus I}
$$ 
for $1\le j\le m$, and a family $(b'_j)_{1\le j\le m}$ of elements of $B$, such that 
$$
(b_i)_i=g'
\left(\sum_{j=1}^m(\lambda_{ij})_i\otimes b'_j\right)=
\left(\sum_{j=1}^m\lambda_{ij}\,b'_j\right)_i.
$$ 
As $(\lambda_{ij})_i$ is in $R^{\oplus I}$ for all $j$, the set of those $i$ in $I$ for which there is a $j$ such that $\lambda_{ij}\neq0$ is finite, and we can arrange the notation so that this set is contained in $\{1,\dots,\ell\}$ with $\ell\ge n$, and we get \eqref{lij1}. As $(\lambda_{ij})_i$ is in $C$ for all $j$, we also have \eqref{lij2}. 
\end{proof} 

Here is another technical lemma:

\begin{lem}\label{techlem2}
Let $R$ be a ring, let $\pp:A'\to A$ be a morphism of right $R$-modules, let $B$ be a left $R$-module, and let $s$ be an element of $\Ker(A'\otimes_RB\to A\otimes_RB)$. Then there exist 

\nn$\bu$ a commutative diagram 
$$
\begin{tikzcd}
{}&F'\ar{d}{f}\ar[equal]{r}&F'\ar{dd}{0}\\ 
F''\ar{r}{\psi}\ar{d}[swap]{g}&F\ar{d}{h}\\ 
A'\ar{r}[swap]{\pp}&A\ar[equal]{r}&A
\end{tikzcd}
$$  
of right $R$-modules such that $F,F'$ and $F''$ are free of finite rank,  

\nn$\bu$ elements $t\in F'\otimes_RB$, $u\in F''\otimes_RB$ such that the commutative diagram 
$$
\begin{tikzcd}
{}&F'\otimes_RB\ni t\ar{d}{f_1}\\ 
u\in F''\otimes_RB\ar{r}{\psi_1}\ar{d}[swap]{g_1}&F\otimes_RB\ar{d}{h_1}\\ 
s\in A'\otimes_RB\ar{r}[swap]{\pp_1}&A\otimes_RB
\end{tikzcd}
$$  
satisfies $g_1(u)=s$ and $\psi_1(u)=f_1(t)$. 
\end{lem} 

\begin{proof} 
Write 
$$
s=\sum_{i=1}^na'_i\otimes b_i
$$ 
with $a'_i$ in $A'$ and $b_i$ in $B$, and put $a_i:=\pp(a_i')\in A$, so that we have 
$$
\sum_{i=1}^n\,a_i\otimes b_i=0.
$$ 
By Lemma~\ref{techlem1} p.~\pageref{techlem1} there are positive integers $\ell,m$ with $\ell\ge n$, and there are three families 
$$
(a_i)_{i=n+1}^\ell,\quad(\lambda_{ij})_{1\le i\le\ell,1\le j\le m},\quad(b'_j)_{j=1}^m
$$ 
of elements belonging respectively to $A$, $R$, and $B$, such that, if we set $b_i=0$ for $n<i\le\ell$, we get \eqref{lij1} and \eqref{lij2} p.~\pageref{lij1}. We have a commutative diagram of right $R$-modules 
%
\begin{equation}\label{d447}
\begin{tikzcd}
{}&R^m\ar{d}{f}\\ 
R^n\ar[hook]{r}{\psi}\ar{d}[swap]{g}&R^\ell\ar{d}{h}\\ 
A'\ar{r}[swap]{\pp}&A
\end{tikzcd}
\end{equation}
%
with 
$$
f(x)_i=\sum_{j=1}^m\lambda_{ij}\,x_j,\quad g(x)=\sum_{i=1}^na'_i\,x_i,\quad h(x)=\sum_{i=1}^\ell a_i\,x_i.
$$ 
In particular, \eqref{lij2} p.~\pageref{lij2} implies $h\circ f=0$. 
\end{proof} 

Let us turn to the proof of Lemma 18.5.3 p. 447. [As already pointed out, there are two typos in the proof: in (18.5.3) $M'|_U$ and $M|_U$ should be $M'(U)$ and $M(U)$, and, after the second display on p.~448, $s_1\in((\cc R^{\op})^{\oplus m}\otimes_{\cc R}P)(U)$ should be $s_1\in((\cc R^{\op})^{\oplus n}\otimes_{\cc R}P)(U)$.] 

For the reader's convenience we state (in a slightly different form) Lemma 18.5.3 (see Notation 17.6.13 p.~428 of the book): 

\begin{lem}[Lemma 18.5.3 p. 447]\label{l1853}
Let $P$ be an $\cc R$-module. Assume that for all $U$ in $\C_X$, all free right $\cc R$-module $F',F''$ of finite rank, and all $\cc R|_U$-linear morphism $u:F'|_U\to F''|_U$, the sequence 
$$
0\to\Ker(u)\otimes_{\cc R|_U}P|_U\to F'|_U\otimes_{\cc R(U)}P|_U\to F''|_U\otimes_{\cc R|_U}P|_U
$$ 
is exact. Then $P$ is a flat $\cc R$-module.
\end{lem} 

(Recall that the notation $?|_U$ is defined in Notation 17.6.13 (ii) p.~428 of the book.)

\begin{proof}
Consider a monomorphism $M'\mono M$ of right $\cc R$-modules and put 
$$ 
K:=\Ker(M'\otimes_{\cc R}P\to M\otimes_{\cc R}P).
$$ 
It suffices to prove $K\simeq0$. Let $K_0$ be the presheaf defined by 
$$
K_0(U):=\Ker\Big(M'(U)\otimes_{\cc R(U)}P(U)\to M(U)\otimes_{\cc R(U)}P(U)\Big),
$$ 
let $U$ be an object of $\C_X$, let $s$ be an element of $K_0(U)$, and let $\overline s$ be the image of $s$ in $K(U)$. We shall prove $\overline s=0$. By Definition 18.2.2 p.~439 and Theorem 17.4.7 (iv) p.~421 of the book, $K$ is the sheaf associated to $K_0$. Hence, as $U$ and $s$ are arbitrary, Equality $\overline s=0$ will imply that the natural morphism $K_0\to K$ vanishes. By (17.4.12) p.~421 of the book, this vanishing will entail $K\simeq0$, and, thus, the lemma. Let us record this observation: 
%
\begin{equation}\label{s=0il}
\text{Equality $\overline s=0$ implies the lemma.}
\end{equation}  
%  
By Lemma~\ref{techlem2} p.~\pageref{techlem2} there exist 

\nn$\bu$ a commutative diagram 
$$
\begin{tikzcd}
{}&F'(U)\ar{d}{f}\ar[equal]{r}&F'(U)\ar{dd}{0}\\ 
F''(U)\ar{r}{\psi}\ar{d}[swap]{g}&F(U)\ar{d}{h}\\ 
M'(U)\ar{r}[swap]{\pp}&M(U)\ar[equal]{r}&M(U)
\end{tikzcd}
$$  
of right $\cc R(U)$-modules such that $F,F'$ and $F''$ are free right $\cc R$-modules of finite rank,  

\nn$\bu$ elements $t\in F'(U)\otimes_{\cc R(U)}P(U)$, $u\in F''(U)\otimes_{\cc R(U)}P(U)$ such that the commutative diagram 
%
\begin{equation}\label{ff'f''}
\begin{tikzcd}
{}&F'(U)\otimes_{\cc R(U)}P(U)\ni t\ar{d}{f_1}\\ 
u\in F''(U)\otimes_{\cc R(U)}P(U)\ar{r}{\psi_1}\ar{d}[swap]{g_1}&F(U)\otimes_{\cc R(U)}P(U)\ar{d}{h_1}\\ 
s\in M'(U)\otimes_{\cc R(U)}P(U)\ar{r}[swap]{\pp_1}&M(U)\otimes_{\cc R(U)}P(U)
\end{tikzcd}
\end{equation} 
% 
satisfies $g_1(u)=s$ and $\psi_1(u)=f_1(t)$. The commutative diagram \eqref{d447} also induces the commutative diagram 
$$
\Delta:=\left\{
\begin{tikzcd}
N\ar{r}\ar{d}[swap]{k_2}&F'|_U\ar{d}{f_2}\\ 
F''|_U\ar{r}{\psi_2}\ar{d}[swap]{g_2}&F|_U\ar{d}{h_2}\\ 
M'|_U\ar{r}[swap]{\pp_2}&M|_U,
\end{tikzcd}
\right.
$$
the top square being cartesian. Then $\pp_2$ is a monomorphism by Proposition 17.6.6 p.~425 and Notation 17.6.13 p.~428 of the book (recall that $M'\to M$ is a monomorphism by assumption). This implies $g_2\circ k_2=0$. Hence $\Delta$ is a commutative diagram of \emph{complexes}. The condition that the top square is cartesian is equivalent to the exactness of 
$$ 
\Sigma:=\Big(0\to N\to F'|_U\oplus F''|_U\to F|_U\Big).
$$  
The sequence $\Sigma\otimes_{\cc R|_U}P|_U$ being exact thanks to the assumption in Lemma~\ref{l1853} p.~\pageref{l1853}, we see that the commutative diagram of complexes $\Delta\otimes_{\cc R|_U}P|_U$ has a cartesian top square, and that, by left exactness of $\Gamma(U;-)$ (see \S\ref{a438} p.~\pageref{a438}), the commutative diagram of complexes $\Gamma(U;\Delta\otimes_{\cc R|_U}P|_U)$, that is (see Notation 17.6.13 p.~428 of the book), 
$$
\begin{tikzcd}
(N\otimes_{\cc R|_U}P|_U)(U)\ar{r}\ar{d}[swap]{k_1}&(F'\otimes_{\cc R}P)(U)\ar{d}{f_1}\ni t\\ 
u\in(F''\otimes_{\cc R}P)(U)\ar{r}{\psi_1}\ar{d}[swap]{g_1}&(F\otimes_{\cc R}P)(U)\ar{d}{h_1}\\ 
\overline s\in\left(M'\otimes_{\cc R}P\right)(U)\ar{r}[swap]{\pp_1}&\left(M\otimes_{\cc R}P\right)(U) 
\end{tikzcd}
$$ 
(commutative diagram which completes \eqref{ff'f''}), has also a cartesian top square, and satisfies $g_1(u)=\overline s$ and 
%
\begin{equation}\label{p1uf1t}
\psi_1(u)=f_1(t).
\end{equation}
%
We have used the isomorphisms 
%
\begin{equation}\label{mupu}
\Gamma\left(U;M|_U\otimes_{\cc R|_U}P|_U\right)\simeq\Gamma\big(U;\left(M\otimes_{\cc R}P\right)|_U\big)\simeq\left(M\otimes_{\cc R}P\right)(U),
\end{equation}
% 
and similarly with $M'$ instead of $M$. Indeed, the first isomorphism in \eqref{mupu} is a particular case of (18.2.5) p.~441 of the book, and the second isomorphism in \eqref{mupu} results from the last two displays on p.~428 of the book. In other words, we have 
%
\begin{equation}\label{nrpu}
(N\otimes_{\cc R|_U}P|_U)(U)\simeq(F'\otimes_{\cc R}P)(U)\times_{(F\otimes_{\cc R}P)(U)}(F''\otimes_{\cc R}P)(U).
\end{equation}
% 
Note that \eqref{p1uf1t} implies 
$$
x:=(t,u)\in(F'\otimes_{\cc R}P)(U)\times_{(F\otimes_{\cc R}P)(U)}(F''\otimes_{\cc R}P)(U).
$$ 
If $y$ is the element of $(N\otimes_{\cc R|_U}P|_U)(U)$ corresponding to $x$ under Isomorphism~\eqref{nrpu}, then we get $k_1(y)=u$, and thus $\overline s=g_1(u)=g_1(k_1(y))=0$. By \eqref{s=0il}, this completes the proof. 
\end{proof}

%%

\subsection{Brief Comments} 

\begin{s}
P.~452, Part (i) (a) of the proof of Lemma 18.6.7. As already mentioned, $\cc O_U$ and $\cc O_V$ stand presumably for $\cc O_X|_U$ and $\cc O_Y|_V$ (and it would be better, in the penultimate display of the page, to write $\cc O_V$ instead of $\cc O_Y|_V$), and, a few lines before the penultimate display of the page, $f_W^{-1}:\cc O_U^{\oplus n}\xr u\cc O_U^{\oplus m}$ should be (I think) $f_W^{-1}:\cc O_W^{\oplus n}\to\cc O_W^{\oplus m}$. 

Also, one may refer to \eqref{ttau} p.~\pageref{ttau} and \S\ref{fdagger} p.~\pageref{fdagger} to describe the morphism of sites $f_W:W\to V$. More precisely, we define, in the notation \eqref{ttau}, the functor $(f_W)^\tau:((\C_Y)_V)^{\op}\to((\C_X)_W)^{\op}$ by
$$
(f_W)^\tau(V'\to V):=\big(f^\tau(V')\to f^\tau(V)\to W\big).
$$
Finally, let us rewrite explicitly one of the key equalities (see \S\ref{fdagger} p.~\pageref{fdagger}): 
$$
f^\dagger(\cc O_Y^{\oplus nm})(W)=\colim_{(f^\tau(V)\to W)\in((\C_Y)^{\op})_W}\cc O_Y^{\oplus nm}(V),
$$ 
where $f^\tau(V)\to W$ is a morphism in $(\C_X)^{\op}$ (corresponding to a morphism $W\to f^t(V)$ in $\C_X$).
\end{s}
\printindex
\end{document}
