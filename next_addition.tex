% ab-ag
% !TEX encoding = UTF-8 Unicode
\documentclass[12pt]{article}
\addtolength{\parskip}{.5\baselineskip}
\usepackage[a4paper]{geometry}
%\usepackage[a4paper,hmargin=3cm,vmargin=3.5cm]{geometry}
\usepackage{amssymb,amsmath}
\usepackage[T1]{fontenc} 
\usepackage[utf8]{inputenc}%\usepackage[latin1]{inputenc}
%\usepackage{tikz}
\usepackage{tikz-cd}
\usepackage{hyperref}
\usepackage{datetime}
%\pagestyle{empty}
\usepackage{amsthm}
\newtheorem{thm}{Theorem}
\newtheorem{lem}[thm]{Lemma}
\newtheorem{prop}[thm]{Proposition}
\newtheorem{cor}[thm]{Corollary}
\newtheorem{defn}[thm]{Definition} 
\theoremstyle{remark}
\newtheorem{cm}[thm]{Comment} 
\newtheorem{rk}[thm]{Remark} 
%\theoremstyle{definition}\newtheorem{defn}{Definition}
\newcommand{\bu}{\bullet}
\newcommand{\n}{\noindent} 
%
\newcommand{\cc}{\mathcal} 
\newcommand{\bb}{\mathbb} 
\newcommand{\A}{\mathcal A}
\newcommand{\B}{\mathcal B}
\newcommand{\C}{\mathcal C}
\newcommand{\F}{\mathcal F}
\newcommand{\G}{\mathcal G}
\newcommand{\J}{\mathcal J}
\newcommand{\M}{\mathcal M} 
\newcommand{\SSS}{\mathcal S}
\newcommand{\U}{\mathcal U}
\newcommand{\V}{\mathcal V}
\newcommand{\Set}{\textbf{Set}}
\newcommand{\Cat}{\textbf{Cat}}
\newcommand{\CCat}{\textbf{CCat}}
\newcommand{\e}{\varepsilon}
\newcommand{\epi}{\twoheadrightarrow}
\newcommand{\mono}{\rightarrowtail}
\newcommand{\m}{\rightarrowtail}
\newcommand{\op}{\text{op}}
\newcommand{\p}{\varphi} 
\newcommand{\pa}{\rightrightarrows} 
\newcommand{\pt}{\{\text{pt}\}} 
\newcommand{\xl}{\xleftarrow} 
\newcommand{\xr}{\xrightarrow} 
\newcommand{\be}{\begin{equation}} 
\newcommand{\ee}{\end{equation}} 
\newcommand{\cd}{commutative diagram} 
\newcommand{\ccd}{the comment containing Display} 
\newcommand{\nm}{natural morphism}
\newcommand{\pr}{Proposition} 
\newcommand{\sts}{t suffices to show} 
\newcommand{\rw}{[This is a rewriting of a previous comment. The new version below has already been incorporated into the main text.]} 
\newcommand{\cn}{(See (\ref{convnot}) p. \pageref{convnot} for an explanation of the notation.) }
%
% LIMITS
% old
\newcommand{\colim}{\operatornamewithlimits{\underset{\longrightarrow}{lim}}} 
\newcommand{\ilim}{\operatornamewithlimits{\underset{\longrightarrow}{lim}}} 
\newcommand{\plim}{\operatornamewithlimits{\underset{\longleftarrow}{lim}}} 
% new
\DeclareMathOperator*{\coli}{colim}
\DeclareMathOperator*{\co}{colim}
\DeclareMathOperator*{\icolim}{``\coli"}
\DeclareMathOperator*{\ic}{``\coli"}
% 
\DeclareMathOperator{\Ad}{Add} 
\DeclareMathOperator{\Coim}{Coim}
\DeclareMathOperator{\Coker}{Coker}
\DeclareMathOperator{\Ima}{Im} 
\DeclareMathOperator{\IM}{IM} 
\DeclareMathOperator{\hy}{h} 
\DeclareMathOperator{\ky}{k} 
\DeclareMathOperator{\id}{id}
\DeclareMathOperator{\Fct}{Fct}
\DeclareMathOperator{\Hom}{Hom}
\DeclareMathOperator{\h}{Hom}
\DeclareMathOperator{\Ind}{Ind}
\DeclareMathOperator{\Ker}{Ker}
\DeclareMathOperator{\Mod}{Mod} 
\DeclareMathOperator{\Mor}{Mor} 
\DeclareMathOperator{\Ob}{Ob}
%\DeclareMathOperator{\Set}{Set}
%\arrow[yshift=0.7ex]{r}\arrow[yshift=-0.7ex]{r} 
%
%%%%%%%%%%%%%%%%%%%%%%%%%%%%%%%%%%%%%%%%%%%%%%%%%%%%%%%%%%%
%  
%
\title{Next addition to \em{About ``Categories and Sheaves''}} 
\author{Pierre-Yves Gaillard} 
\date{\today, \currenttime} 
% 
\begin{document} 
% 
\maketitle  
%
\section{Typos and Details} 
% 
\n P. 18, Definition 1.2.16. [This is an old comment inserted here only because of the reference it contains.] The category $\C_{X'}$ is attached to the functor $F:\C\to\C'$ and to the object $X'\in\C'$. One often thinks of an object $(X,F(X)\to X')$ of $\C_{X'}$ as being an object of $X$ of $\C$ equipped with a morphism $F(X)\to X'$. This justifies the abusive but useful notation $X\in\C_{X'}$, which is an abbreviation for: ``firstly, $X$ is an object of $\C$, and, secondly, a morphism $F(X)\to X'$ is either supposed to be given, or obvious from the context''. For instance, if $X$ is an object of $\C$, then $X\in\C_X$ usually means $(X,\id_X)\in\C_X$. This abuse is especially useful when we have a functor $G:\C\to\C''$ and we consider an inductive limit (which may or may not exist in $\C''$) of the form 
%
\begin{equation}\label{convnot}
\coli_{X\in\C_{X'}}G(X).  
\end{equation}
% 
There is a similar remark for projective limits. 

%%

\n P. 188, after the second diagram: ``the set of isomorphisms classes of $\Delta$'' should be ``the set of isomorphisms classes of objects of $\Delta$''. 
% 
\section{Brief comments} %%%%%%%%%%%%%%%%%%%%%%%%%%%%%% 
%
%% 

\n$\bu$ P. 61, \pr\ 2.6.3 (i). [This is an old comment inserted here only because of the reference it contains.] Let $\C$ be a category and $A$ be in $\C^\wedge$. Consider the statements  
% 
\begin{equation}\label{263a}
\ic_{X\in\C_A}X\xrightarrow\sim A, 
\end{equation} 

\begin{equation}\label{263b}
\co_{X\in\C_A}\h_\C(Y,X)\xrightarrow\sim A(Y)\text{ for all }Y\in\C, 
\end{equation}

\begin{equation}\label{263} 
\h_{\C^\wedge}(A,B)\xrightarrow\sim\lim_{X\in\C_A}B(X)\text{ for all }B\in\C^\wedge. 
\end{equation} 

\n\cn Clearly, (\ref{263b}) implies (\ref{263a}) and (\ref{263}), and the proof of (\ref{263b}) is straightforward. %(See the comment containing Display (\ref{38}) p. \pageref{38} for the relationship between (\ref{263a}), (\ref{263b}), and (\ref{263}).) 

%% 

\n$\bu$ P. 63, Notation 2.7.2. \rw\ The formula 
$$
(\widehat F(A))(V)=\coli_{U\in\C_A}\Hom_{\C'}(V,F(U))
$$
may also be written as 
$$
\widehat F(A)=\icolim_{U\in\C_A}F(U).
$$
\cn It might be worth stating explicitly the isomorphism 
$$
\widehat F\circ\hy_\C\xr\sim\hy_{\C'}\circ F,
$$
as well as the following fact. 

If $F$ is fully faithful, then there an isomorphism $\widehat F(A)\circ F\xr\sim A$ functorial in $A\in\C^\wedge$. Indeed, we have 
$$
\widehat F(A)(F(X))=\coli_{Y\in\C_A}\Hom_{\C'}(F(X),F(Y))
$$
$$
\simeq\coli_{Y\in\C_A}\Hom_\C(X,Y)\xr\sim A(X),
$$
the last isomorphism following from \eqref{263b} p. \pageref{263b}. 

\begin{rk}\label{cof}
If $A'$ is in $\cc C'^\wedge$, then the natural functor $\p:\cc C_{A'\circ F}\to\cc C'_{A'}$ gives rise to a morphism $\psi:\widehat F(A'\circ F)\to A'$ functorial in $A'$. Moreover, if $\p$ is cofinal, then $\psi$ is an isomorphism. 
\end{rk} 

%% 

\n$\bu$ P. 80. \pr\ 3.2.4 and 3.2.6 can be combined as follows. 

\begin{prop}\label{comb} 
Let $\p:J\to I$ be fully faithful. Assume that $I$ is filtrant and cofinally small, and that for each $i$ in $I$ there is a morphism $i\to\p(j)$ for some $j$ in $J$. Then $\p$ is cofinal and $J$ is filtrant and cofinally small. 
\end{prop} 

\n{\em Proof.} In view of \pr\ 3.2.4 it suffices to show that $J$ is cofinally small. By \pr\ 3.2.6, there is a small full subcategory $S\subset I$ cofinal to $I$. For each $s$ in $S$ pick a morphism $s\to\p(j_s)$ with $j_s\in J$. Then, for each $j$ in $J$ there are morphisms $\p(j)\to s\to\p(j_s)$ with $s$ in $S$. As $\p$ is full there is a morphism $j\to j_s$, and we conclude by using again \pr\ 3.2.6. 

%% 

\n$\bu$ P. 147, Exercise 6.11. \rw\ Recall the setting: $\cc J$ is a full subcategory of a category $\cc C$, and $A$ is in $\Ind(\cc C)$. Let $F:\cc J\to\cc C$ be the inclusion, let $A$ be in $\Ind(\cc I)$, and consider the conditions 

\n(a) $A\simeq IF(B)$ for some $B$ in $\Ind(\cc J)$, 

\n(b) all morphism $X\to A$ in $\Ind(\cc C)$ with $X$ in $\cc C$ factorizes through some object of $\cc J$. 

\n Then (a) $\iff$ (b). 

\n{\em Proof of} $\implies$. Assume that we have $A\simeq IF(B)$ and $f:X\to IF(B)$ as above. Let $\gamma:I\to\cc J$ a functor such that $I$ is small and filtrant and $\co\gamma\simeq B$. Then $\co F\circ\gamma\simeq IF(B)$. By \pr\ 6.1.13 p. 134 there are functors $\alpha,\beta:J\pa\cc C$ and a functorial morphism $\p:\alpha\to\beta$ such that 

$J$ is small and filtrant, 

$\alpha$ is constant equal to $X$, 

$\beta(j)\in\cc J$ for all $j$, 

$\co\beta\simeq IF(B)$, 

$\co\p\simeq f$. 

\n Then $f$ factorizes as $X=\alpha(j)\xr{\p_j}\beta(j)\xr{c_j}IF(B)$, where $c_j$ is the coprojection. 

\n{\em Proof of} $\Longleftarrow$. The statement follows from Remark \ref{cof} p.~\pageref{cof} and \pr\ \ref{comb} p.~\pageref{comb}. 

%% 

\n$\bu$ P. 177, Definition 8.3.5. The following definitions and observations are implicit in the book. Let $\cc A$ be a subcategory of a pre-additive category $\cc B$, and let $\iota:\cc A\to \cc B$ be the inclusion. If $\cc A$ is pre-additive and $\iota$ is additive, we say that $\cc A$ is a {\em pre-additive subcategory} of $\cc B$. If in addition $\cc A$ and $\cc B$ are additive (resp. abelian), we say that $\cc A$ is {\em an additive (resp. abelian) subcategory} of $\cc B$. Now let $\cc A$ and $\cc B$ be categories. If $\cc B$ is pre-additive (resp. additive, abelian), then so is the category $\cc C:=\cc B^\cc A$ of functors from $\cc A$ to $\cc B$. Assume in addition that $\cc A$ is pre-additive. If $\cc B$ is pre-additive (resp. additive, abelian), then the full subcategory $\cc D:=\Ad(\cc A,\cc B)$ of $\cc C$ whose objects are the additive functors from $\cc A$ to $\cc B$ is a pre-additive (resp. additive, abelian) subcategory of $\cc C$. 

% 

\n$\bu$ P. 186, Definition 8.3.24 (definition of a Grothendieck category). The condition that small filtrant inductive limits are exact is not automatic. I know no entirely elementary proof of this important fact. Here is a proof using a little bit of sheaf theory. To show that there is an abelian category where small filtrant inductive limits exist but are not exact, it suffices to prove that there is an abelian category $\C$ where small filtrant {\em projective} limits exist but are not exact. It is even enough to show that small products are not exact in $\C$. Let $X$ be a topological space, and let $U_0\supset U_1\supset\cdots$ be a decreasing sequence of open subsets whose intersection is a non-open closed singleton $\{a\}$. We can take for $\C$ the category of small abelian sheaves on $X$. To see this, let $G$ be the abelian presheaf over $X$ such that $G(U)$ is $\mathbb Z$ if $a\in U$ and 0 otherwise, and, for each $n\in\mathbb N$, let $F_n$ be the abelian presheaf over $X$ such that $F_n(U)$ is $\mathbb Z$ if $U\subset U_n$ and 0 otherwise. These presheaves are easily seen to be sheaves. For each $n\in\mathbb N$ and each open set $U$ let $F_n(U)\to G(U)$ be the identity if $a\in U\subset U_n$ and 0 otherwise. This family of morphisms define, when $U$ varies, an epimorphism $\p_n:F_n\epi G$. Put 
$$
F:=\prod_{n\in\mathbb N}F_n,\quad H:=\prod_{n\in\mathbb N}G,\quad\p:=\prod_{n\in\mathbb N}\p_n:F\to H.
$$ 
It suffices to show that the morphism $\p(a):F(a)\to H(a)$ between the stalks at $a$ induced by $\p$ is not an epimorphism. This is clear because $\p(a)$ is the natural morphism 
$$
\bigoplus_{n\in\mathbb N}\mathbb Z\to\prod_{n\in\mathbb N}\mathbb Z.
$$ 

%

%\n$\bu$ P. 194, \pr\ 8.6.6 (c). It seems to me that one must assume that $\cc J$ is an abelian subcategory of $\cc C$. (?)
% 
\end{document}
